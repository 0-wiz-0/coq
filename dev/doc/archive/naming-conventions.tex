\documentclass[a4paper]{article}
\usepackage{fullpage}
\usepackage[utf8]{inputenc}
\usepackage[T1]{fontenc}
\usepackage{amsfonts}

\parindent=0pt
\parskip=10pt

%%%%%%%%%%%%%
% Macros
\newcommand\itemrule[3]{
\subsubsection{#1}
\begin{quote}
\begin{tt}
#3
\end{tt}
\end{quote}
\begin{quote}
Name: \texttt{#2}
\end{quote}}

\newcommand\formula[1]{\begin{tt}#1\end{tt}}
\newcommand\tactic[1]{\begin{tt}#1\end{tt}}
\newcommand\command[1]{\begin{tt}#1\end{tt}}
\newcommand\term[1]{\begin{tt}#1\end{tt}}
\newcommand\library[1]{\texttt{#1}}
\newcommand\name[1]{\texttt{#1}}

\newcommand\zero{\texttt{zero}}
\newcommand\op{\texttt{op}}
\newcommand\opPrime{\texttt{op'}}
\newcommand\opSecond{\texttt{op''}}
\newcommand\phimapping{\texttt{phi}}
\newcommand\D{\texttt{D}}
\newcommand\elt{\texttt{elt}}
\newcommand\rel{\texttt{rel}}
\newcommand\relp{\texttt{rel'}}

%%%%%%%%%%%%%

\begin{document}

\begin{center}
\begin{huge}
Proposed naming conventions for the Coq standard library
\end{huge}
\end{center}
\bigskip

The following document describes a proposition of canonical naming
schemes for the Coq standard library. Obviously and unfortunately, the
current state of the library is not as homogeneous as it would be if
it would systematically follow such a scheme. To tend in this
direction, we however recommend to follow the following suggestions.

\tableofcontents

\section{General conventions}

\subsection{Variable names}

\begin{itemize}

\item Variables are preferably quantified at the head of the
  statement, even if some premisses do not depend of one of them. For
  instance, one would state
\begin{quote}
\begin{tt}
  {forall x y z:D, x <= y -> x+z <= y+z}
\end{tt}
\end{quote}
and not
\begin{quote}
\begin{tt}
  {forall x y:D, x <= y -> forall z:D, x+z <= y+z}
\end{tt}
\end{quote}

\item Variables are preferably quantified (and named) in the order of
  ``importance'', then of appearance, from left to right, even if
  for the purpose of some tactics it would have been more convenient
  to have, say, the variables not occurring in the conclusion
  first. For instance, one would state
\begin{quote}
\begin{tt}
  {forall x y z:D, x+z <= y+z -> x <= y}
\end{tt}
\end{quote}
and not
\begin{quote}
\begin{tt}
  {forall z x y:D, x+z <= y+z -> x <= y}
\end{tt}
\end{quote}
nor
\begin{quote}
\begin{tt}
  {forall x y z:D, y+x <= z+x -> y <= z}
\end{tt}
\end{quote}

\item Choice of effective names is domain-dependent. For instance, on
  natural numbers, the convention is to use the variables $n$, $m$,
  $p$, $q$, $r$, $s$ in this order.

  On generic domains, the convention is to use the letters $x$, $y$,
  $z$, $t$. When more than three variables are needed, indexing variables

  It is conventional to use specific names for variables having a
  special meaning. For instance, $eps$ or $\epsilon$ can be used to
  denote a number intended to be as small as possible. Also, $q$ and
  $r$ can be used to denote a quotient and a rest. This is good
  practice.

\end{itemize}

\subsection{Disjunctive statements}

A disjunctive statement with a computational content will be suffixed
by \name{\_inf}. For instance, if

\begin{quote}
\begin{tt}
{forall x y, op x y = zero -> x = zero \/ y = zero}
\end{tt}
\end{quote}
has name \texttt{D\_integral}, then
\begin{quote}
\begin{tt}
{forall x y, op x y = zero -> \{x = zero\} + \{y = zero\}}
\end{tt}
\end{quote}
will have name \texttt{D\_integral\_inf}.

As an exception, decidability statements, such as
\begin{quote}
\begin{tt}
{forall x y, \{x = y\} + \{x <> y\}}
\end{tt}
\end{quote}
will have a named ended in \texttt{\_dec}. Idem for cotransitivity
lemmas which are inherently computational that are ended in
\texttt{\_cotrans}.

\subsection{Inductive types constructor names}

As a general rule, constructor names start with the name of the
inductive type being defined as in \texttt{Inductive Z := Z0 : Z |
  Zpos : Z -> Z | Zneg : Z -> Z} to the exception of very standard
types like \texttt{bool}, \texttt{nat}, \texttt{list}...

For inductive predicates, constructor names also start with the name
of the notion being defined with one or more suffixes separated with
\texttt{\_} for discriminating the different cases as e.g. in

\begin{verbatim}
Inductive even : nat -> Prop :=
  | even_O : even 0
  | even_S n : odd n -> even (S n)
with odd : nat -> Prop :=
  | odd_S n : even n -> odd (S n).
\end{verbatim}

As a general rule, inductive predicate names should be lowercase (to
the exception of notions referring to a proper name, e.g. \texttt{Bezout})
and multiple words must be separated by ``{\_}''.

As an exception, when extending libraries whose general rule is that
predicates names start with a capital letter, the convention of this
library should be kept and the separation between multiple words is
done by making the initial of each work a capital letter (if one of
these words is a proper name, then a ``{\_}'' is added to emphasize
that the capital letter is proper and not an application of the rule
for marking the change of word).

Inductive predicates that characterize the specification of a function
should be named after the function it specifies followed by
\texttt{\_spec} as in:

\begin{verbatim}
Inductive nth_spec : list A -> nat -> A -> Prop :=
  | nth_spec_O a l : nth_spec (a :: l) 0 a
  | nth_spec_S n a b l : nth_spec l n a -> nth_spec (b :: l) (S n) a.
\end{verbatim}

\section{Equational properties of operations}

\subsection{General conventions}

If the conclusion is in the other way than listed below, add suffix
\name{\_reverse} to the lemma name.

\subsection{Specific conventions}

\itemrule{Associativity of binary operator {\op} on domain {\D}}{Dop\_assoc}
{forall x y z:D, op x (op y z) = op (op x y) z}

  Remark: Symmetric form: \name{Dop\_assoc\_reverse}:
  \formula{forall x y z:D, op (op x y) z = op x (op y z)}

\itemrule{Commutativity of binary operator {\op} on domain {\D}}{Dop\_comm}
{forall x y:D, op x y = op y x}

  Remark: Avoid \formula{forall x y:D, op y x = op x y}, or at worst, call it
  \name{Dop\_comm\_reverse}

\itemrule{Left neutrality of element elt for binary operator {\op}}{Dop\_elt\_l}
{forall x:D, op elt x = x}

  Remark: In English, ``{\elt} is an identity for {\op}'' seems to be
  a more common terminology.

\itemrule{Right neutrality of element elt for binary operator {\op}}{Dop\_elt\_r}
{forall x:D, op x elt = x}

  Remark: By convention, if the identities are reminiscent to zero or one, they
  are written 1 and 0 in the name of the property.

\itemrule{Left absorption of element elt for binary operator {\op}}{Dop\_elt\_l}
{forall x:D, op elt x = elt}

  Remarks:
  \begin{itemize}
  \item In French school, this property is named "elt est absorbant pour op"
  \item English, the property seems generally named "elt is a zero of op"
  \item In the context of lattices, this a boundedness property, it may
     be called "elt is a bound on D", or referring to a (possibly
     arbitrarily oriented) order "elt is a least element of D" or "elt
     is a greatest element of D"
  \end{itemize}

\itemrule{Right absorption of element {\elt} for binary operator {\op}}{Dop\_elt\_l  [BAD ??]}
{forall x:D, op x elt = elt}

\itemrule{Left distributivity of binary operator {\op} over {\opPrime} on domain {\D}}{Dop\_op'\_distr\_l}
{forall x y z:D, op (op' x y) z = op' (op x z) (op y z)}

  Remark: Some authors say ``distribution''.

\itemrule{Right distributivity of binary operator {\op} over {\opPrime} on domain {\D}}{Dop\_op'\_distr\_r}
{forall x y z:D, op z (op' x y) = op' (op z x) (op z y)}

  Remark: Note the order of arguments.

\itemrule{Distributivity of unary operator {\op} over binary op' on domain {\D}}{Dop\_op'\_distr}
{forall x y:D, op (op' x y) = op' (op x) (op y)}

\itemrule{Distributivity of unary operator {\op} over binary op' on domain {\D}}{Dop\_op'\_distr}
{forall x y:D, op (op' x y) = op' (op x) (op y)}

  Remark: For a non commutative operation with inversion of arguments, as in
  \formula{forall x y z:D, op (op' x y) = op' (op y) (op y z)},
  we may probably still call the property distributivity since there
  is no ambiguity.

  Example: \formula{forall n m : Z, -(n+m) = (-n)+(-m)}.

  Example: \formula{forall l l' : list A, rev (l++l') = (rev l)++(rev l')}.

\itemrule{Left extrusion of unary operator {\op} over binary op' on domain {\D}}{Dop\_op'\_distr\_l}
{forall x y:D, op (op' x y) = op' (op x) y}

  Question: Call it left commutativity ?? left swap ?

\itemrule{Right extrusion of unary operator {\op} over binary op' on domain {\D}}{Dop\_op'\_distr\_r}
{forall x y:D, op (op' x y) = op' x (op y)}

\itemrule{Idempotency of binary operator {\op} on domain {\D}}{Dop\_idempotent}
{forall x:D, op x x = x}

\itemrule{Idempotency of unary operator {\op} on domain {\D}}{Dop\_idempotent}
{forall x:D, op (op x) = op x}

  Remark: This is actually idempotency of {\op} wrt to composition and
  identity.

\itemrule{Idempotency of element elt for binary operator {\op} on domain {\D}}{Dop\_elt\_idempotent}
{op elt elt = elt}

  Remark: Generally useless in CIC for concrete, computable operators

  Remark: The general definition is ``exists n, iter n op x = x''.

\itemrule{Nilpotency of element elt wrt a ring D with additive neutral
element {\zero} and multiplicative binary operator
{\op}}{Delt\_nilpotent}
{op elt elt = zero}

  Remark: We leave the ring structure of D implicit; the general definition is ``exists n, iter n op elt = zero''.

\itemrule{Zero-product property in a ring D with additive neutral
element {\zero} and multiplicative binary operator
{\op}}{D\_integral}
{forall x y, op x y = zero -> x = zero \/ y = zero}

  Remark: We leave the ring structure of D implicit; the Coq library
  uses either \texttt{\_is\_O} (for \texttt{nat}), \texttt{\_integral}
    (for \texttt{Z}, \texttt{Q} and \texttt{R}), \texttt{eq\_mul\_0} (for
    \texttt{NZ}).

  Remark: The French school says ``integrité''.

\itemrule{Nilpotency of binary operator {\op} wrt to its absorbing element
zero in D}{Dop\_nilpotent} {forall x, op x x = zero}

  Remark: Did not find this definition on the web, but it used in
  the Coq library (to characterize \name{xor}).

\itemrule{Involutivity of unary op on D}{Dop\_involutive}
{forall x:D, op (op x) = x}

\itemrule{Absorption law on the left for binary operator {\op} over binary operator {\op}' on the left}{Dop\_op'\_absorption\_l\_l}
{forall x y:D, op x (op' x y) = x}

\itemrule{Absorption law on the left for binary operator {\op} over binary operator {\op}' on the right}{Dop\_op'\_absorption\_l\_r}
{forall x y:D, op x (op' y x) = x}

  Remark: Similarly for \name{Dop\_op'\_absorption\_r\_l} and \name{Dop\_op'\_absorption\_r\_r}.

\itemrule{De Morgan law's for binary operators {\opPrime} and {\opSecond} wrt
to unary op on domain {\D}}{Dop'\_op''\_de\_morgan,
Dop''\_op'\_de\_morgan ?? \mbox{leaving the complementing operation
implicit})}
{forall x y:D, op (op' x y) = op'' (op x) (op y)\\
forall x y:D, op (op'' x y) = op' (op x) (op y)}

\itemrule{Left complementation of binary operator {\op} by means of unary {\opPrime} wrt neutral element {\elt} of {\op} on domain {\D}}{Dop\_op'\_opp\_l}
{forall x:D, op (op' x) x = elt}

Remark: If the name of the opposite function is reminiscent of the
notion of complement (e.g. if it is called \texttt{opp}), one can
simply say {Dop\_opp\_l}.

\itemrule{Right complementation of binary operator {\op} by means of unary {\op'} wrt neutral element {\elt} of {\op} on domain {\D}}{Dop\_opp\_r}
{forall x:D, op x (op' x) = elt}

Example: \formula{Radd\_opp\_l: forall r : R, - r + r = 0}

\itemrule{Associativity of binary operators {\op} and {\op'}}{Dop\_op'\_assoc}
{forall x y z, op x (op' y z) = op (op' x y) z}

Example: \formula{forall x y z, x + (y - z) = (x + y) - z}

\itemrule{Right extrusion of binary operator {\opPrime} over binary operator {\op}}{Dop\_op'\_extrusion\_r}
{forall x y z, op x (op' y z) = op' (op x y) z}

Remark: This requires {\op} and {\opPrime} to have their right and left
argument respectively and their return types identical.

Example: \formula{forall x y z, x + (y - z) = (x + y) - z}

Remark: Other less natural combinations are possible, such
as \formula{forall x y z, op x (op' y z) = op' y (op x z)}.

\itemrule{Left extrusion of binary operator {\opPrime} over binary operator {\op}}{Dop\_op'\_extrusion\_l}
{forall x y z, op (op' x y) z = op' x (op y z)}

Remark: Operations are not necessarily internal composition laws.  It
is only required that {\op} and {\opPrime} have their right and left
argument respectively and their return type identical.

Remark: When the type are heterogeneous, only one extrusion law is possible and it can simply be named {Dop\_op'\_extrusion}.

Example: \formula{app\_cons\_extrusion : forall a l l', (a :: l) ++ l' = a :: (l ++ l')}.

%======================================================================
%\section{Properties of elements}

%Remark: Not used in current library



%======================================================================
\section{Preservation and compatibility properties of operations}

\subsection{With respect to equality}

\itemrule{Injectivity of unary operator {\op}}{Dop\_inj}
{forall x y:D, op x = op y -> x = y}

\itemrule{Left regularity of binary operator {\op}}{Dop\_reg\_l, Dop\_inj\_l, or Dop\_cancel\_l}
{forall x y z:D, op z x = op z y -> x = y}

  Remark: Note the order of arguments.

  Remark: The Coq usage is to called it regularity but the English
  standard seems to be cancellation. The recommended form is not
  decided yet.

  Remark: Shall a property like $n^p \leq n^q \rightarrow p \leq q$
  (for $n\geq 1$) be called cancellation or should it be reserved for
  operators that have an inverse?

\itemrule{Right regularity of binary operator {\op}}{Dop\_reg\_r, Dop\_inj\_r, Dop\_cancel\_r}
{forall x y z:D, op x z = op y z -> x = y}

\subsection{With respect to a relation {\rel}}

\itemrule{Compatibility of unary operator {\op}}{Dop\_rel\_compat}
{forall x y:D, rel x y -> rel (op x) (op y)}

\itemrule{Left compatibility of binary operator {\op}}{Dop\_rel\_compat\_l}
{forall x y z:D, rel x y -> rel (op z x) (op z y)}

\itemrule{Right compatibility of binary operator {\op}}{Dop\_rel\_compat\_r}
{forall x y z:D, rel x y -> rel (op x z) (op y z)}

  Remark: For equality, use names of the form \name{Dop\_eq\_compat\_l} or
  \name{Dop\_eq\_compat\_r}
(\formula{forall x y z:D, y = x -> op y z = op x z} and
\formula{forall x y z:D, y = x -> op y z = op x z})

  Remark: Should we admit (or even prefer) the name
  \name{Dop\_rel\_monotone}, \name{Dop\_rel\_monotone\_l},
  \name{Dop\_rel\_monotone\_r} when {\rel} is an order ?

\itemrule{Left regularity of binary operator {\op}}{Dop\_rel\_reg\_l}
{forall x y z:D, rel (op z x) (op z y) -> rel x y}

\itemrule{Right regularity of binary operator {\op}}{Dop\_rel\_reg\_r}
{forall x y z:D, rel (op x z) (op y z) -> rel x y}

  Question: Would it be better to have \name{z} as first argument, since it
  is missing in the conclusion ?? (or admit we shall use the options
  ``\texttt{with p}''?)

\itemrule{Left distributivity of binary operator {\op} over {\opPrime} along relation {\rel} on domain {\D}}{Dop\_op'\_rel\_distr\_l}
{forall x y z:D, rel (op (op' x y) z) (op' (op x z) (op y z))}

  Example: standard property of (not necessarily distributive) lattices

  Remark: In a (non distributive) lattice, by swapping join and meet,
  one would like also,
\formula{forall x y z:D, rel (op' (op x z) (op y z)) (op (op' x y) z)}.
  How to name it with a symmetric name (use
 \name{Dop\_op'\_rel\_distr\_mon\_l} and
 \name{Dop\_op'\_rel\_distr\_anti\_l})?

\itemrule{Commutativity of binary operator {\op} along (equivalence) relation {\rel} on domain {\D}}{Dop\_op'\_rel\_comm}
{forall x y z:D, rel (op x y) (op y x)}

  Example:
\formula{forall l l':list A, Permutation (l++l') (l'++l)}

\itemrule{Irreducibility of binary operator {\op} on domain {\D}}{Dop\_irreducible}
{forall x y z:D, z = op x y -> z = x $\backslash/$ z = y}

  Question: What about the constructive version ? Call it \name{Dop\_irreducible\_inf} ?
\formula{forall x y z:D, z = op x y -> \{z = x\} + \{z = y\}}

\itemrule{Primality of binary operator {\op} along relation {\rel} on domain {\D}}{Dop\_rel\_prime}
{forall x y z:D, rel z (op x y) -> rel z x $\backslash/$ rel  z y}


%======================================================================
\section{Morphisms}

\itemrule{Morphism between structures {\D} and {\D'}}{\name{D'\_of\_D}}{D -> D'}

Remark: If the domains are one-letter long, one can used \texttt{IDD'} as for
\name{INR} or \name{INZ}.

\itemrule{Morphism {\phimapping} mapping unary operators {\op} to {\op'}}{phi\_op\_op', phi\_op\_op'\_morphism}
{forall x:D, phi (op x) = op' (phi x)}

Remark: If the operators have the same name in both domains, one use
\texttt{D'\_of\_D\_op} or \texttt{IDD'\_op}.

Example: \formula{Z\_of\_nat\_mult: forall n m : nat, Z\_of\_nat (n * m) = (Z\_of\_nat n * Z\_of\_nat m)\%Z}.

Remark: If the operators have different names on distinct domains, one
can use \texttt{op\_op'}.

\itemrule{Morphism {\phimapping} mapping binary operators {\op} to
{\op'}}{phi\_op\_op', phi\_op\_op'\_morphism}  {forall
x y:D, phi (op x y) = op' (phi x) (phi y)}

Remark: If the operators have the same name in both domains, one use
\texttt{D'\_of\_D\_op} or \texttt{IDD'\_op}.

Remark: If the operators have different names on distinct domains, one
can use \texttt{op\_op'}.

\itemrule{Morphism {\phimapping} mapping binary operator {\op} to
binary relation {\rel}}{phi\_op\_rel, phi\_op\_rel\_morphism}
{forall x y:D, phi (op x y) <-> rel (phi x) (phi y)}

Remark: If the operator and the relation have similar name, one uses
\texttt{phi\_op}.

Question: How to name each direction? (add \_elim for -> and \_intro
for <- ?? -- as done in Bool.v ??)

Example: \formula{eq\_true\_neg: \~{} eq\_true b <-> eq\_true (negb b)}.

%======================================================================
\section{Preservation and compatibility properties of operations wrt order}

\itemrule{Compatibility of binary operator {\op} wrt (strict order) {\rel} and (large order) {\rel'}}{Dop\_rel\_rel'\_compat}
{forall x y z t:D, rel x y -> rel' z t -> rel (op x z) (op y t)}

\itemrule{Compatibility of binary operator {\op} wrt (large order) {\relp} and (strict order) {\rel}}{Dop\_rel'\_rel\_compat}
{forall x y z t:D, rel' x y -> rel z t -> rel (op x z) (op y t)}


%======================================================================
\section{Properties of relations}

\itemrule{Reflexivity of relation {\rel} on domain {\D}}{Drel\_refl}
{forall x:D, rel x x}

\itemrule{Symmetry of relation {\rel} on domain {\D}}{Drel\_sym}
{forall x y:D, rel x y -> rel y x}

\itemrule{Transitivity of relation {\rel} on domain {\D}}{Drel\_trans}
{forall x y z:D, rel x y -> rel y z -> rel x z}

\itemrule{Antisymmetry of relation {\rel} on domain {\D}}{Drel\_antisym}
{forall x y:D, rel x y -> rel y x -> x = y}

\itemrule{Irreflexivity of relation {\rel} on domain {\D}}{Drel\_irrefl}
{forall x:D, \~{} rel x x}

\itemrule{Asymmetry of relation {\rel} on domain {\D}}{Drel\_asym}
{forall x y:D, rel x y -> \~{} rel y x}

\itemrule{Cotransitivity of relation {\rel} on domain {\D}}{Drel\_cotrans}
{forall x y z:D, rel x y -> \{rel z y\} + \{rel x z\}}

\itemrule{Linearity of relation {\rel} on domain {\D}}{Drel\_trichotomy}
{forall x y:D, \{rel x y\} + \{x = y\} + \{rel y x\}}

  Questions: Or call it \name{Drel\_total}, or \name{Drel\_linear}, or
  \name{Drel\_connected}? Use
  $\backslash/$ ? or use a ternary sumbool, or a ternary disjunction,
  for nicer elimination.

\itemrule{Informative decidability of relation {\rel} on domain {\D}}{Drel\_dec (or Drel\_dect, Drel\_dec\_inf ?)}
{forall x y:D, \{rel x y\} + \{\~{} rel x y\}}

  Remark: If equality: \name{D\_eq\_dec} or \name{D\_dec} (not like
  \name{eq\_nat\_dec})

\itemrule{Non informative decidability of relation {\rel} on domain {\D}}{Drel\_dec\_prop (or Drel\_dec)}
{forall x y:D, rel x y $\backslash/$ \~{} rel x y}

\itemrule{Inclusion of relation {\rel} in relation {\rel}' on domain {\D}}{Drel\_rel'\_incl (or Drel\_incl\_rel')}
{forall x y:D, rel x y -> rel' x y}

  Remark: Use \name{Drel\_rel'\_weak} for a strict inclusion ??

%======================================================================
\section{Relations between properties}

\itemrule{Equivalence of properties \texttt{P} and \texttt{Q}}{P\_Q\_iff}
{forall x1 .. xn, P <-> Q}

  Remark: Alternatively use \name{P\_iff\_Q} if it is too difficult to
  recover what pertains to \texttt{P} and what pertains to \texttt{Q}
  in their concatenation (as e.g. in
  \texttt{Godel\_Dummett\_iff\_right\_distr\_implication\_over\_disjunction}).

%======================================================================
\section{Arithmetical conventions}

\begin{minipage}{6in}
\renewcommand{\thefootnote}{\thempfootnote}  % For footnotes...
\begin{tabular}{lll}
Zero on domain {\D} & D0 & (notation \verb=0=)\\
One on domain {\D} & D1 (if explicitly defined) & (notation \verb=1=)\\
Successor on domain {\D} & Dsucc\\
Predessor on domain {\D} & Dpred\\
Addition on domain {\D} & Dadd/Dplus\footnote{Coq historically uses \texttt{plus} and \texttt{mult} for addition and multiplication which are inconsistent notations, the recommendation is to use \texttt{add} and \texttt{mul} except in existng libraries that already use \texttt{plus} and \texttt{mult}}
 & (infix notation \verb=+= [50,L])\\
Multiplication on domain {\D} & Dmul/Dmult\footnotemark[\value{footnote}] & (infix notation \verb=*= [40,L]))\\
Soustraction on domain {\D} & Dminus & (infix notation \verb=-= [50,L])\\
Opposite on domain {\D} & Dopp (if any) & (prefix notation \verb=-= [35,R]))\\
Inverse on domain {\D} & Dinv (if any) & (prefix notation \verb=/= [35,R]))\\
Power on domain {\D} & Dpower & (infix notation \verb=^= [30,R])\\
Minimal element on domain {\D} & Dmin\\
Maximal element on domain {\D} & Dmax\\
Large less than order on {\D} & Dle & (infix notations \verb!<=! and \verb!>=! [70,N]))\\
Strict less than order on {\D} & Dlt & (infix notations \verb=<= and \verb=>= [70,N]))\\
\end{tabular}
\bigskip
\end{minipage}

\bigskip

The status of \verb!>=! and \verb!>! is undecided yet. It will eithet
be accepted only as parsing notations or may also accepted as a {\em
  definition} for the \verb!<=! and \verb!<! (like in \texttt{nat}) or
even as a different definition (like it is the case in \texttt{Z}).

\bigskip

Exception: Peano Arithmetic which is used for pedagogical purpose:

\begin{itemize}
\item domain name is implicit
\item \term{0} (digit $0$) is \term{O} (the 15th letter of the alphabet)
\item \term{succ} is \verb!S! (but \term{succ} can be used in theorems)
\end{itemize}

\end{document}
