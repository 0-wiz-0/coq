\achapter{Proof of imperative programs}
\aauthor{Jean-Christophe Filli�tre}
\label{Addoc-programs}
\index{Imperative programs!proof of}
\comindex{Correctness}

%%%%%%%%%%%%%%%%
% Introduction %
%%%%%%%%%%%%%%%%

This chapter describes a new tactic
to prove the correctness and termination of imperative
programs annotated in a Floyd-Hoare logic style.
The theoretical fundations of this tactic are described 
in~\cite{Filliatre99,Filliatre00a}.
This tactic is provided in the \Coq\ module \texttt{Correctness},
which does not belong to the initial state of \Coq. So you must import
it when necessary, with the following command:
$$
\mbox{\texttt{Require Correctness.}}
$$


%%%%%%%%%%%%%%%%%%%%%
% comment �a marche %
%%%%%%%%%%%%%%%%%%%%%

\asection{How it works}

Before going on into details and syntax, let us give a quick overview
of how that tactic works. 
Its behavior is the following: you give a
program annotated with logical assertions and the tactic will generate
a bundle of subgoals, called \emph{proof obligations}. Then, if you
prove all those proof obligations, you will establish the correctness and the
termination of the program.
The implementation currently supports traditional imperative programs
with references and arrays on arbitrary purely functional datatypes,
local variables, functions with call-by-value and call-by-variable
arguments, and recursive functions.

Although it behaves as an implementation of Floyd-Hoare logic, it is not.
The idea of the underlying mechanism is to translate the imperative
program into a partial proof of a proposition of the kind
$$
\forall \vec{x}. P(\vec{x}) 
  \Rightarrow \exists(\vec{y},v). Q(\vec{x},\vec{y},v)
$$
where $P$ and $Q$ stand for the pre- and post-conditions of the
program, $\vec{x}$ and $\vec{y}$ the variables used and modified by
the program and $v$ its result.
Then this partial proof is given to the tactic \texttt{Refine}
(see~\ref{Refine}, page~\pageref{Refine}), which effect is to generate as many
subgoals as holes in the partial proof term.

\medskip

The syntax to invoke the tactic is the following:
$$
\mbox{\tt Correctness} ~~ ident ~~ annotated\_program.
$$
Notice that this is not exactly a \emph{tactic}, since it does not
apply to a goal. To be more rigorous, it is the combination of a
vernacular command (which creates the goal from the annotated program)
and a tactic (which partially solves it, leaving some proof
obligations to the user).

Although \texttt{Correctness} is not a tactic, the following syntax is
available:  
$$
\mbox{\tt Correctness} ~~ ident ~~ annotated\_program ~ ; ~ tactic.
$$
In that case, the given tactic is applied on any proof
obligation generated by the first command.


%%%%%%%%%%%%%%%%%%%%%%%%%%%%%%%%%
% Syntaxe de programmes annotes %
%%%%%%%%%%%%%%%%%%%%%%%%%%%%%%%%%

\asection{Syntax of annotated programs}

\asubsection{Programs}

The syntax of programs is given in figure~\ref{fig:ProgramsSyntax}.
Basically, the programming language is a purely functional kernel
with an addition of references and arrays on purely functional values.
If you do not consider the logical assertions, the syntax coincide
with \ocaml\ syntax, except for elements of arrays which are written
$t[i]$. In particular, the dereference of a mutable variable $x$ is
written $!x$ and assignment is written \verb!:=!
(for instance, the increment of the variable $x$ will
be written \verb@x := !x + 1@).
Actually, that syntax does not really matters, since it would
be extracted later to real concrete syntax, in different programming
languages. 

\begin{figure}[htbp]
  \begin{center}
    \leavevmode
$$
\begin{array}{lrl}
    prog & ::= & \verb!{! ~ predicate ~ \verb!}! *
                 ~ statement ~ [ \verb!{! ~ predicate ~ \verb!}! ] \\

    & & \\[0.1em]

    statement
         & ::= & expression \\
         &   | & identifier ~ \verb!:=! ~ prog \\
         &   | & identifier ~ \verb![! ~ expression ~ \verb!]!
                 ~ \verb!:=! ~ prog \\
         &   | & \verb!let! ~ identifier ~ \verb!=! ~ \verb!ref! ~
                  prog ~ \verb!in! ~ prog \\
         &   | & \verb!if! ~ prog ~ \verb!then! ~ prog 
                 ~ [ \verb!else! ~ prog ] \\
         &   | & \verb!while! ~ prog ~ \verb!do! 
                 ~ loop\_annot ~ block ~ \verb!done! \\
         &   | & \verb!begin! ~ block ~ \verb!end! \\
         &   | & \verb!let! ~ identifier ~ \verb!=! ~ prog ~ 
                 \verb!in! ~ prog \\
         &   | & \verb!fun! ~ binders ~ \verb!->! ~ prog \\
         &   | & \verb!let! ~ \verb!rec! ~ identifier ~ binder ~ \verb!:! 
                 ~ value\_type \\
         &     & ~~ \verb!{! ~ \verb!variant! ~ wf\_arg ~ \verb!}! 
                 ~ \verb!=! ~ prog ~ [ \verb!in! ~ prog ] \\
         &   | & \verb!(! ~ prog ~~ prog ~ \verb!)! \\

    & & \\[1em]

    expression
         & ::= & identifier \\
         &   | & \verb@!@ ~ identifier \\
         &   | & identifier ~ \verb![! ~ expression ~ \verb!]! \\
         &   | & integer \\
         &   | & \verb!(! ~ expression+  \verb!)! \\

    & & \\[1em]

    block & ::= & block\_statement ~ [ ~ \verb!;! ~ block ~ ] \\

    & & \\[0.1em]

    block\_statement
          & ::= & prog \\
          &   | & \verb!label! ~ identifier \\
          &   | & \verb!assert! ~ \verb!{! ~ predicate ~ \verb!}! \\

    & & \\[1em]

    binders & ::= & \verb!(! ~ identifier,\dots,identifier ~ \verb!:!
                    ~ value\_type ~ \verb!)! + \\

    & & \\[1em]

    loop\_annot 
          & ::= & \verb!{! ~ \verb!invariant! ~ predicate ~
                  \verb!variant! ~ wf\_arg ~ \verb!}! \\
    & & \\[0.1em]

    wf\_arg & ::= & cic\_term ~ [ \verb!for! ~ cic\_term ] \\

    & & \\[0.1em]

    predicate & ::= & cci\_term ~ [ \verb!as! ~ identifier ] \\

  \end{array}
$$
      \caption{Syntax of annotated programs}
    \label{fig:ProgramsSyntax}
  \end{center}
\end{figure}


\paragraph{Syntactic sugar.}

\begin{itemize}
  \item \textbf{Boolean expressions:} 

    Boolean expressions appearing in programs (and in particular in
    \kw{if} and \kw{while} tests) are arbitrary programs of type \texttt{bool}.
    In order to make programs more readable, some syntactic sugar is
    provided for the usual logical connectives and the usual order
    relations over type \texttt{Z}, with the following syntax:
    $$
    \begin{array}{lrl}
      prog
      & ::= & prog ~ \verb!and! ~ prog \\
      &   | & prog ~ \verb!or!  ~ prog \\
      &   | & \verb!not! ~ prog \\
      &   | & expression ~ order\_relation ~ expression \\[1em]
      order\_relation
      & ::= & \verb!>! ~|~ \verb!>=! ~|~ \verb!<! ~|~ \verb!<=! 
      ~|~ \verb!=! ~|~ \verb!<>! \\
    \end{array}
    $$
    where the order relations have the strongest precedences,
    \verb!not! has a stronger precedence than \verb!and!,
    and \verb!and! a stronger precedence than \verb!or!.
    
    Order relations in other types, like \texttt{lt}, \texttt{le}, \dots in
    type \texttt{nat}, should be explicited as described in the
    paragraph about \emph{Boolean expressions}, page~\pageref{prog:booleans}.

  \item \textbf{Arithmetical expressions:}

    Some syntactic sugar is provided for the usual arithmetic operator
    over type \texttt{Z}, with the following syntax:
    $$
    \begin{array}{lrl}
      prog
      & ::= & prog ~ \verb!*! ~ prog \\
      &   | & prog ~ \verb!+! ~ prog \\
      &   | & prog ~ \verb!-! ~ prog \\
      &   | & \verb!-! ~ prog 
    \end{array}
    $$
    where the unary operator \texttt{-} has the strongest precedence,
    and \texttt{*} a stronger precedence than \texttt{+} and \texttt{-}.

    Operations in other arithmetical types (such as type \texttt{nat})
    must be explicitly written as applications, like 
    \texttt{(plus~a~b)}, \texttt{(pred~a)}, etc.

  \item \texttt{if $b$ then $p$} is a shortcut for
    \texttt{if $b$ then $p$ else tt}, where \texttt{tt} is the
    constant of type \texttt{unit};

  \item Values in type \texttt{Z}
    may be directly written as integers : 0,1,12546,\dots
    Negative integers are not recognized and must be written
    as \texttt{(Zinv $x$)};

  \item Multiple application may be written $(f~a_1~\dots~a_n)$,
    which must be understood as left-associa\-tive
    i.e. as $(\dots((f~a_1)~a_2)\dots~a_n)$.
\end{itemize}


\paragraph{Restrictions.}

You can notice some restrictions with respect to real ML programs:
\begin{enumerate}
  \item Binders in functions (recursive or not) are explicitly typed,
    and the type of the result of a recursive function is also given.
    This is due to the lack of type inference.

  \item Full expressions are not allowed on left-hand side of assignment, but
    only variables. Therefore, you can not write
\begin{verbatim}
    (if b then x else y) := 0
\end{verbatim}
    But, in most cases, you can rewrite
    them into acceptable programs. For instance, the previous program
    may be rewritten into the following one:
\begin{verbatim}
    if b then x := 0 else y := 0
\end{verbatim}
\end{enumerate}



%%%%%%%%%%%%%%%
% Type system %
%%%%%%%%%%%%%%%

\asubsection{Typing}

The types of annotated programs are split into two kinds: the types of
\emph{values} and the types of \emph{computations}. Those two types
families are recursively mutually defined with the following concrete syntax:
$$
\begin{array}{lrl}
  value\_type
    & ::= & cic\_term \\
    &   | & {cic\_term} ~ \verb!ref! \\
    &   | & \verb!array! ~ cic\_term ~ \verb!of! ~ cic\_term \\
    &   | & \verb!fun! ~ \verb!(! ~ x \verb!:! value\_type ~ \verb!)!\!+
            ~ computation\_type \\
    &     & \\
  computation\_type
    & ::= & \verb!returns! ~ identifier \verb!:! value\_type \\
    &     & [\verb!reads! ~ identifier,\ldots,identifier] 
            ~ [\verb!writes! ~ identifier,\ldots,identifier] \\
    &     & [\verb!pre! ~ predicate] ~ [\verb!post! ~ predicate] \\
    &     & \verb!end! \\
    &     & \\
  predicate
    & ::= & cic\_term \\
    &     & \\
\end{array}
$$

The typing is mostly the one of ML, without polymorphism.
The user should notice that:
\begin{itemize}
  \item Arrays are indexed over the type \texttt{Z} of binary integers 
    (defined in the module \texttt{ZArith});

  \item Expressions must have purely functional types, and can not be 
    references or arrays (but, of course, you can pass mutables to
    functions as call-by-variable arguments);

  \item There is no partial application.
\end{itemize}


%%%%%%%%%%%%%%%%%%
% Specifications %
%%%%%%%%%%%%%%%%%%

\asubsection{Specification}

The specification part of programs is made of different kind of
annotations, which are terms of sort \Prop\ in the Calculus of Inductive
Constructions.

Those annotations can refer to the values of the variables
directly by their names. \emph{There is no dereference operator ``!'' in
annotations}. Annotations are read with the \Coq\ parser, so you can
use all the \Coq\ syntax to write annotations. For instance, if $x$
and $y$ are references over integers (in type \texttt{Z}), you can write the
following annotation
$$
\mbox{\texttt{\{ `0 < x <= x+y` \}}}
$$
In a post-condition, if necessary, you can refer to the value of the variable
$x$ \emph{before} the evaluation with the notation $x@$.
Actually, it is possible to refer to the value of a variable at any 
moment of the evaluation with the notation $x@l$,
provided that $l$ is a \emph{label} previously inserted in your program
(see below the paragraph about labels).

You have the possibility to give some names to the annotations, with
the syntax
$$
\texttt{\{ \emph{annotation} as \emph{identifier} \}}
$$
and then the annotation will be given this name in the proof
obligations.
Otherwise, fresh names are given automatically, of the kind
\texttt{Post3}, \texttt{Pre12}, \texttt{Test4}, etc.
You are encouraged to give explicit names, in order not to have to
modify your proof script when your proof obligations change (for
instance, if you modify a part of the program).


\asubsubsection{Pre- and post-conditions}

Each program, and each of its sub-programs, may be annotated by a
pre-condition and/or a post-condition.
The pre-condition is an annotation about the values of variables
\emph{before} the evaluation, and the post-condition is an annotation
about the values of variables \emph{before} and \emph{after} the
evaluation. Example:
$$
\mbox{\texttt{\{ `0 < x` \} x := (Zplus !x !x) \{ `x@ < x` \}}}
$$
Moreover, you can assert some properties of the result of the evaluation
in the post-condition, by referring to it through the name \emph{result}.
Example:
$$
\mbox{\texttt{(Zs (Zplus !x !x)) \{ (Zodd result) \}}}
$$


\asubsubsection{Loops invariants and variants}

Loop invariants and variants are introduced just after the \kw{do}
keyword, with the following syntax:
$$
\begin{array}{l}
  \kw{while} ~ B ~ \kw{do} \\
  ~~~ \{ ~ \kw{invariant} ~ I ~~~ \kw{variant} ~ \phi ~ \kw{for} ~ R ~
      \} \\
  ~~~ block \\
  \kw{done}
\end{array}
$$

The invariant $I$ is an annotation about the values of variables
when the loop is entered, since $B$ has no side effects ($B$ is a
purely functional expression). Of course, $I$ may refer to values of
variables at any moment before the entering of the loop.

The variant $\phi$ must be given in order to establish the termination of
the loop. The relation $R$ must be a term of type $A\rightarrow
A\rightarrow\Prop$, where $\phi$ is of type $A$.
When $R$ is not specified, then $\phi$ is assumed to be of type
\texttt{Z} and the usual order relation on natural number is used.


\asubsubsection{Recursive functions}

The termination of a recursive function is justified in the same way as
loops, using a variant. This variant is introduced with the following syntax
$$
\kw{let} ~ \kw{rec} ~ f ~ (x_1:V_1)\dots(x_n:V_n) : V
  ~ \{ ~ \kw{variant} ~ \phi ~ \kw{for} ~ R ~ \} = prog
$$
and is interpreted as for loops. Of course, the variant may refer to
the bound variables $x_i$.
The specification of a recursive function is the one of its body, $prog$.
Example:
$$
\kw{let} ~ \kw{rec} ~ fact ~ (x:Z) : Z ~ \{ ~ \kw{variant} ~ x \} =
  \{ ~ x\ge0 ~ \} ~ \dots ~ \{ ~ result=x! ~ \}
$$


\asubsubsection{Assertions inside blocks}

Assertions may be inserted inside blocks, with the following syntax
$$
\kw{begin} ~ block\_statements \dots; ~ \kw{assert} ~ \{ ~ P ~ \};
  ~ block\_statements \dots ~ \kw{end}
$$
The annotation $P$ may refer to the values of variables at any labels
known at this moment of evaluation.


\asubsubsection{Inserting labels in your program}

In order to refer to the values of variables at any moment of
evaluation of the program, you may put some \emph{labels} inside your
programs. Actually, it is only necessary to insert them inside blocks,
since this is the only place where side effects can appear. The syntax
to insert a label is the following:
$$
\kw{begin} ~ block\_statements \dots; ~ \kw{label} ~ L;
  ~ block\_statements \dots ~ \kw{end}
$$
Then it is possible to refer to the value of the variable $x$ at step
$L$ with the notation $x@L$.

There is a special label $0$ which is automatically inserted at the
beginning of the program. Therefore, $x@0$ will always refer to the
initial value of the variable $x$.

\medskip

Notice that this mechanism allows the user to get rid of the so-called
\emph{auxiliary variables}, which are usually widely used in
traditional frameworks to refer to previous values of variables.


%%%%%%%%%%%%
% bool�ens %
%%%%%%%%%%%%

\asubsubsection{Boolean expressions}\label{prog:booleans}

As explained above, boolean expressions appearing in \kw{if} and
\kw{while} tests are arbitrary programs of type \texttt{bool}. 
Actually, there is a little restriction: a test can not do some side
effects. 
Usually, a test if annotated in such a way:
$$
  B ~ \{ \myifthenelse{result}{T}{F} \}
$$
(The \textsf{if then else} construction in the annotation is the one
of \Coq\ !)
Here $T$ and $F$ are the two propositions you want to get in the two
branches of the test.
If you do not annotate a test, then $T$ and $F$ automatically become
$B=\kw{true}$ and $B=\kw{false}$, which is the usual annotation in
Floyd-Hoare logic.

But you should take advantages of the fact that $T$ and $F$ may be
arbitrary propositions, or you can even annotate $B$ with any other
kind of proposition (usually depending on $result$).

As explained in the paragraph about the syntax of boolean expression,
some syntactic sugar is provided for usual order relations over type
\texttt{Z}. When you write $\kw{if} ~ x<y ~ \dots$ in your program,
it is only a shortcut for $\kw{if} ~ (\texttt{Z\_lt\_ge\_bool}~x~y) ~ \dots$,
where \texttt{Z\_lt\_ge\_bool} is the proof of
$\forall x,y:\texttt{Z}. \exists b:\texttt{bool}.
 (\myifthenelse{b}{x<y}{x\ge y})$
i.e. of a program returning a boolean with the expected post-condition.
But you can use any other functional expression of such a type.
In particular, the \texttt{Correctness} standard library comes
with a bunch of decidability theorems on type \texttt{nat}:
$$
\begin{array}{ll}
  zerop\_bool & : \forall n:\kw{nat}.\exists b:\texttt{bool}.
    \myifthenelse{b}{n=0}{0<n} \\
  nat\_eq\_bool & : \forall n,m:\kw{nat}.\exists b:\texttt{bool}.
    \myifthenelse{b}{n=m}{n\not=m} \\
  le\_lt\_bool & : \forall n,m:\kw{nat}.\exists b:\texttt{bool}.
    \myifthenelse{b}{n\le m}{m<n} \\
  lt\_le\_bool & : \forall n,m:\kw{nat}.\exists b:\texttt{bool}.
    \myifthenelse{b}{n<m}{m\le n} \\
\end{array}
$$
which you can combine with the logical connectives.

It is often the case that you have a decidability theorem over some
type, as for instance a theorem of decidability of equality over some
type $S$:
$$
S\_dec : (x,y:S)\sumbool{x=y}{\neg x=y}
$$
Then you can build a test function corresponding to $S\_dec$ using the
operator \texttt{bool\_of\_sumbool} provided with the
\texttt{Prorgams} module, in such a way:
$$
\texttt{Definition} ~ S\_bool ~ \texttt{:=} 
  [x,y:S](\texttt{bool\_of\_sumbool} ~ ? ~ ? ~ (S\_dec ~ x ~ y))
$$
Then you can use the test function $S\_bool$ in your programs,
and you will get the hypothesis $x=y$ and $\neg x=y$ in the corresponding
branches. 
Of course, you can do the same for any function returning some result
in the constructive sum $\sumbool{A}{B}$.


%%%%%%%%%%%%%%%%%%%%%%%%%%%%%%%%%
% variables locales et globales %
%%%%%%%%%%%%%%%%%%%%%%%%%%%%%%%%%

\asection{Local and global variables}

\asubsection{Global variables}
\comindex{Global Variable}

You can declare a new global variable with the following command
$$
\mbox{\texttt{Global Variable}} ~~ x ~ : ~ value\_type.
$$
where $x$ may be a reference, an array or a function.
\Example
\begin{verbatim}
Parameter N : Z.
Global Variable x : Z ref.
Correctness foo { `x < N` } begin x := (Zmult 2 !x) end { `x < 2*N` }.
\end{verbatim}

\comindex{Show Programs}
Each time you complete a correctness proof, the corresponding program is
added to the programs environment. You can list the current programs
environment with the command
$$
\mbox{\texttt{Show Programs.}}
$$


\asubsection{Local variables}

Local variables are introduced with the following syntax
$$
\mbox{\texttt{let}} ~ x ~ = ~ \mbox{\texttt{ref}} ~ e_1 
~ \mbox{\texttt{in}} ~ e_2
$$
where the scope of $x$ is exactly the program $e_2$.
Notice that, as usual in ML, local variables must be
initialized (here with $e_1$).

When specifying a program including local variables, you have to take
care about their scopes. Indeed, the following two programs are not
annotated in the same way:
\begin{itemize}
  \item $\kw{let} ~ x = e_1 ~ \kw{in} ~ e_2 ~ \spec{Q}$ \\
    The post-condition $Q$ applies to $e_2$, and therefore $x$ may
    appear in $Q$;

  \item $(\kw{let} ~ x = e_1 ~ \kw{in} ~ e_2) ~ \spec{Q}$ \\
    The post-condition $Q$ applies to the whole program, and therefore
    the local variable $x$ may \emph{not} appear in $Q$ (it is beyond
    its scope).
\end{itemize}


%%%%%%%%%%%%%%%%%
% Function call %
%%%%%%%%%%%%%%%%%

\asection{Function call}

Still following the syntax of ML, the function application is written
$(f ~ a_1 ~ \ldots ~ a_n)$, where $f$ is a function and the $a_i$'s
its arguments. Notice that $f$ and the $a_i$'s may be annotated
programs themselves.

In the general case, $f$ is a function already specified (either with
\texttt{Global Variable} or with a proof of correctness) and has a
pre-condition $P_f$ and a post-condition $Q_f$.

As expected, a proof obligation is generated, which correspond to
$P_f$ applied to the values of the arguments, once they are evaluated.

Regarding the post-condition of $f$, there are different possible cases:
\begin{itemize}
  \item either you did not annotate the function call, writing directly
    $$(f ~ a_1 ~ \ldots ~ a_n)$$
    and then the post-condition of $f$ is added automatically
    \emph{if possible}: indeed, if some arguments of $f$ make side-effects
    this is not always possible. In that case, you have to put a 
    post-condition to the function call by yourself;

  \item or you annotated it with a post-condition, say $Q$:
    $$(f ~ a_1 ~ \ldots ~ a_n) ~ \spec{Q}$$
    then you will have to prove that $Q$ holds under the hypothesis that
    the post-condition $Q_f$ holds (where both are
    instantiated by the results of the evaluation of the $a_i$). 
    Of course, if $Q$ is exactly the post-condition of $f$ then
    the corresponding proof obligation will be automatically
    discharged.
\end{itemize}


%%%%%%%%%%%%
% Libraries %
%%%%%%%%%%%%

\asection{Libraries}
\index{Imperative programs!libraries}

The tactic comes with some libraries, useful to write programs and
specifications.
The first set of libraries is automatically loaded with the module
\texttt{Correctness}. Among them, you can find the modules:
\begin{description}
  \item[ProgWf]: this module defines a family of relations $Zwf$ on type
    \texttt{Z} by
    $$(Zwf ~ c) = \lambda x,y. c\le x \land c \le y \land x < y$$
    and establishes that this relation is well-founded for all $c$
    (lemma \texttt{Zwf\_well\_founded}). This lemma is automatically
    used by the tactic \texttt{Correctness} when necessary.
    When no relation is given for the variant of a loop or a recursive
    function, then $(Zwf~0)$ is used \emph{i.e.} the usual order
    relation on positive integers.

  \item[Arrays]: this module defines an abstract type \texttt{array}
    for arrays, with the corresponding operations \texttt{new},
    \texttt{access} and \texttt{store}. Access in a array $t$ at index
    $i$ may be written \texttt{\#$t$[$i$]} in \Coq, and in particular
    inside specifications.
    This module also provides some axioms to manipulate arrays
    expression, among which \texttt{store\_def\_1} and
    \texttt{store\_def\_2} allow you to simplify expressions of the
    kind \texttt{(access (store $t$ $i$ $v$) $j$)}.
\end{description}

\noindent Other useful modules, which are not automatically loaded, are the
following:
\begin{description}
  \item[Exchange]: this module defines a predicate \texttt{(exchange
      $t$ $t'$ $i$ $j$)} which means that elements of indexes $i$ and
    $j$ are swapped in arrays $t$ and $t'$, and other left unchanged.
    This modules also provides some lemmas to establish this property
    or conversely to get some consequences of this property.
    
  \item[Permut]: this module defines the notion of permutation between
    two arrays, on a segment of the arrays (\texttt{sub\_permut}) or
    on the whole array (\texttt{permut}). 
    Permutations are inductively defined as the smallest equivalence
    relation containing the transpositions (defined in the module
    \texttt{Exchange}).

  \item[Sorted]: this  module defines the property  for an array to be
    sorted, on the whole  array  (\texttt{sorted\_array}) or on a  segment
    (\texttt{sub\_sorted\_array}). It  also   provides  a few   lemmas  to
    establish this property.
\end{description}



%%%%%%%%%%%%%%
% Extraction %
%%%%%%%%%%%%%%

\asection{Extraction}
\index{Imperative programs!extraction of}

Once a program is proved, one usually wants to run it, and that's why
this implementation comes with an extraction mechanism.
For the moment, there is only extraction to \ocaml\ code.
This functionality is provided by the following module:
$$
\mbox{\texttt{Require ProgramsExtraction.}}
$$
This extraction facility extends the extraction of functional programs
(see chapter~\ref{Extraction}), on which it is based.
Indeed, the extraction of functional terms (\Coq\ objects) is first
performed by the module \texttt{Extraction}, and the extraction of
imperative programs comes after.
Therefore, we have kept the same syntax as for functional terms:
$$
\mbox{\tt Write Caml File "\str" [ \ident$_1$ \dots\ \ident$_n$ ].} 
$$
where \str\ is the name given to the file to be produced (the suffix
{\tt .ml} is added if necessary), and \ident$_1$ \dots\ \ident$_n$ the
names of the objects to be extracted. 
That list may contain functional and imperative objects, and does not need
to be exhaustive.

Of course, you can use the extraction facilities described in
chapter~\ref{Extraction}, namely the \texttt{ML Import},
\texttt{Link} and \texttt{Extract} commands.


\paragraph{The case of integers}
There is no use of the \ocaml\ native integers: indeed, it would not be safe
to use the machine integers while the correctness proof is done
with unbounded integers (\texttt{nat}, \texttt{Z} or whatever type).
But since \ocaml\ arrays are indexed over the type \texttt{int}
(the machine integers) arrays indexes are converted from \texttt{Z}
to \texttt{int} when necessary (the conversion is very fast: due
to the binary representation of integers in type \texttt{Z}, it
will never exceed thirty elementary steps).

And it is also safe, since the size of a \ocaml\ array can not be greater
than the maximum integer ($2^{30}-1$) and since the correctness proof
establishes that indexes are always within the bounds of arrays
(Therefore, indexes will never be greater than the maximum integer,
and the conversion will never produce an overflow).


%%%%%%%%%%%%
% Examples %
%%%%%%%%%%%%

\asection{Examples}\label{prog:examples}

\asubsection{Computation of $X^n$}

As a first example, we prove the correctness of a program computing
$X^n$ using the following equations:
$$
\left\{
\begin{array}{lcl}
X^{2n}   & = & (X^n)^2 \\
X^{2n+1} & = & X \times (X^n)^2
\end{array}
\right.
$$
If $x$ and $n$ are variables containing the input and $y$ a variable that will
contain the result ($x^n$), such a program may be the following one:
\begin{center}
  \begin{minipage}{8cm}
  \begin{tabbing}
    AA\=\kill
    $m$ := $!x$ ; $y$ := $1$ ; \\
    \kw{while} $!n \not= 0$ \kw{do} \\
    \> \kw{if} $(odd ~ !n)$ \kw{then} $y$ := $!y \times !m$ ;\\
    \> $m$ := $!m \times !m$ ; \\
    \> $n$ := $!n / 2$ \\
    \kw{done} \\
  \end{tabbing}
  \end{minipage}
\end{center}


\paragraph{Specification part.}

Here we choose to use the binary integers of \texttt{ZArith}.
The exponentiation $X^n$ is defined, for $n \ge 0$, in the module 
\texttt{Zpower}:
\begin{coq_example*}
Require ZArith.
Require Zpower.
\end{coq_example*}

In particular, the module \texttt{ZArith} loads a module \texttt{Zmisc}
which contains the definitions of the predicate \texttt{Zeven} and
\texttt{Zodd}, and the function \texttt{Zdiv2}.
This module \texttt{ProgBool} also contains a test function
\texttt{Zeven\_odd\_bool} of type 
$\forall n. \exists b. \myifthenelse{b}{(Zeven~n)}{(Zodd~n)}$
derived from the proof \texttt{Zeven\_odd\_dec},
as explained in section~\ref{prog:booleans}:

\begin{coq_eval}
Require Ring.
\end{coq_eval}


\paragraph{Correctness part.}

Then we come to the correctness proof. We first import the \Coq\
module \texttt{Correctness}:
\begin{coq_example*}
Require Correctness.
\end{coq_example*}
\begin{coq_eval}
Definition Zsquare := [n:Z](Zmult n n).
Definition Zdouble := [n:Z]`2*n`.
\end{coq_eval}

Then we introduce all the variables needed by the program:
\begin{coq_example*}
Parameter x : Z.
Global Variable n,m,y : Z ref.
\end{coq_example*}

At last, we can give the annotated program:
\begin{coq_example}
Correctness i_exp
  { `n >= 0` }
  begin
     m := x; y := 1;
     while !n > 0 do
       { invariant (Zpower x n@0)=(Zmult y (Zpower m n)) /\ `n >= 0`
         variant n }
       (if not (Zeven_odd_bool !n) then y := (Zmult !y !m))
          { (Zpower x n@0) = (Zmult y (Zpower m (Zdouble (Zdiv2 n)))) };
       m := (Zsquare !m);
       n := (Zdiv2 !n)
     done
   end
   { y=(Zpower x n@0) }
.
\end{coq_example}

The proof obligations require some lemmas involving \texttt{Zpower} and
\texttt{Zdiv2}. You can find the whole proof in the  \Coq\ standard
library (see below).
Let us make some quick remarks about this program and the way it was
written: 
\begin{itemize}
  \item The name \verb!n@0! is used to refer to the initial value of
    the variable \verb!n!, as well inside the loop invariant as in
    the post-condition;

  \item Purely functional expressions are allowed anywhere in the
    program and they can use any purely informative \Coq\ constants;
    That is why we can use \texttt{Zmult}, \texttt{Zsquare} and
    \texttt{Zdiv2} in the programs even if they are not other functions
    previously introduced as programs.
\end{itemize}


\asubsection{A recursive program}

To give an example of a recursive program, let us rewrite the previous
program into a recursive one. We obtain the following program:
\begin{center}
  \begin{minipage}{8cm}
  \begin{tabbing}
    AA\=AA\=AA\=\kill
    \kw{let} \kw{rec} $exp$ $x$ $n$ = \\
    \> \kw{if} $n = 0$ \kw{then} \\
    \> \> 1 \\
    \> \kw{else} \\
    \> \> \kw{let} $y$ = $(exp ~ x ~ (n/2))$ \kw{in} \\
    \> \> \kw{if} $(even ~ n)$ \kw{then} $y\times y$ 
                               \kw{else} $x\times (y\times y)$ \\
  \end{tabbing}
  \end{minipage}
\end{center}

This recursive program, once it is annotated, is given to the
tactic \texttt{Correctness}:
\begin{coq_eval}
Reset n.
\end{coq_eval}
\begin{coq_example}
Correctness r_exp
  let rec exp (x:Z) (n:Z) : Z { variant n } =
    { `n >= 0` }
    (if n = 0 then
       1
     else
       let y = (exp x (Zdiv2 n)) in
       (if (Zeven_odd_bool n) then
          (Zmult y y)
        else
          (Zmult x (Zmult y y))) { result=(Zpower x n) }
    ) 
    { result=(Zpower x n) }
.
\end{coq_example}

You can notice that the specification is simpler in the recursive case:
we only have to give the pre- and post-conditions --- which are the same
as for the imperative version --- but there is no annotation
corresponding to the loop invariant.
The other two annotations in the recursive program are added for the
recursive call and for the test inside the \textsf{let in} construct
(it can not be done automatically in general,
so the user has to add it by himself).


\asubsection{Other examples}

You will find some other examples with the distribution of the system
\Coq, in the sub-directory \verb!contrib/correctness! of the
\Coq\ standard library. Those examples are mostly programs to compute
the factorial and the exponentiation in various ways (on types \texttt{nat}
or \texttt{Z}, in imperative way or recursively, with global
variables or as functions, \dots).

There are also some bigger correctness developments in the 
\Coq\ contributions, which are available on the web page
\verb!coq.inria.fr/contribs!.
for the moment, you can find:
\begin{itemize}
  \item A proof of \emph{insertion sort} by Nicolas Magaud, ENS Lyon;
  \item Proofs of \emph{quicksort}, \emph{heapsort} and \emph{find} by
    the author.
\end{itemize}
These examples are fully detailed in~\cite{FilliatreMagaud99,Filliatre99c}.


%%%%%%%%%%%%%%%
% BUG REPORTS %
%%%%%%%%%%%%%%%

\asection{Bugs}

\begin{itemize}
  \item There is no discharge mechanism for programs; so you
    \emph{cannot} do a program's proof inside a section (actually,
    you can do it, but your program will not exist anymore after having
    closed the section).
\end{itemize}

Surely there are still many bugs in this implementation.
Please send bug reports to \textsf{Jean-Christophe.Filliatre$@$lri.fr}.
Don't forget to send the version of \Coq\ used (given by
\texttt{coqtop -v}) and a script producing the bug.


%%% Local Variables: 
%%% mode: latex
%%% TeX-master: "Reference-Manual"
%%% End: 
