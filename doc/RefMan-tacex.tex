\chapter{Detailed examples of tactics}
\label{Tactics-examples}

This chapter presents detailed examples of certain tactics, to
illustrate their behavior.

\section{\tt Refine}
\tacindex{Refine}
\label{Refine-example}

This tactic applies to any goal. It behaves like {\tt Exact} with a
big difference : the user can leave some holes (denoted by \texttt{?} or 
{\tt (?::}{\it type}{\tt )}) in the term. 
{\tt Refine} will generate as many
subgoals as they are holes in the term. The type of holes must be
either synthesized by the system or declared by an
explicit cast like \verb|(?::nat->Prop)|. This low-level
tactic can be useful to advanced users.

%\firstexample
\Example

\begin{coq_example*}
Inductive Option : Set :=
  | Fail : Option
  | Ok : bool -> Option.
\end{coq_example}
\begin{coq_example}
Definition get : forall x:Option, x <> Fail -> bool.
refine
 (fun x:Option =>
    match x return x <> Fail -> bool with
    | Fail => _
    | Ok b => fun _ => b
    end).
intros; absurd (Fail = Fail); trivial.
\end{coq_example}
\begin{coq_example*}
Defined.
\end{coq_example*}

% \example{Using Refine to build a poor-man's ``Cases'' tactic}

% \texttt{Refine} is actually the only way for the user to do
% a proof with the same structure as a {\tt Cases} definition. Actually,
% the tactics \texttt{Case} (see \ref{Case}) and \texttt{Elim} (see
% \ref{Elim}) only allow one step of elementary induction. 

% \begin{coq_example*}
% Require Bool.
% Require Arith.
% \end{coq_example*}
% %\begin{coq_eval}
% %Abort.
% %\end{coq_eval}
% \begin{coq_example}
% Definition one_two_or_five := [x:nat]
%   Cases x of
%     (1) => true
%   | (2) => true
%   | (5) => true
%   | _ => false
%   end.
% Goal (x:nat)(Is_true (one_two_or_five x)) -> x=(1)\/x=(2)\/x=(5).
% \end{coq_example}

% A traditional script would be the following:

% \begin{coq_example*}
% Destruct x.
% Tauto.
% Destruct n.
% Auto.
% Destruct n0.
% Auto.
% Destruct n1.
% Tauto.
% Destruct n2.
% Tauto.
% Destruct n3.
% Auto.
% Intros; Inversion H.
% \end{coq_example*}

% With the tactic \texttt{Refine}, it becomes quite shorter:

% \begin{coq_example*}
% Restart.
% \end{coq_example*}
% \begin{coq_example}
% Refine [x:nat]
%   <[y:nat](Is_true (one_two_or_five y))->(y=(1)\/y=(2)\/y=(5))>
%   Cases x of
%     (1) => [H]?
%   | (2) => [H]?
%   | (5) => [H]?
%   | n => [H](False_ind ? H)
%   end; Auto.
% \end{coq_example}
% \begin{coq_eval}
% Abort.
% \end{coq_eval}

\section{\tt EApply}
\tacindex{EApply}
\label{EApply-example}
\Example
Assume we have a relation on {\tt nat} which is transitive:

\begin{coq_example*}
Variable R : nat -> nat -> Prop.
Hypothesis Rtrans : forall x y z:nat, R x y -> R y z -> R x z.
Variables n m p : nat.
Hypothesis Rnm : R n m.
Hypothesis Rmp : R m p.
\end{coq_example*}

Consider the goal {\tt (R n p)} provable using the transitivity of
{\tt R}:

\begin{coq_example*}
Goal R n p.
\end{coq_example*}

The direct application of {\tt Rtrans} with {\tt Apply} fails because
no value for {\tt y} in {\tt Rtrans} is found by {\tt Apply}:

\begin{coq_eval}
Set Printing Depth 50.
\end{coq_eval}
\begin{coq_example}

(********** The following is not correct and should produce **********)
(**** Error: generated subgoal (R n ?17) has metavariables in it *****)
(* Just to adjust the prompt: *) apply Rtrans.
\end{coq_example}

A solution is to rather apply {\tt (Rtrans n m p)}.

\begin{coq_example}
apply (Rtrans n m p).
\end{coq_example}

\begin{coq_eval}
Undo.
\end{coq_eval}

More elegantly, {\tt Apply Rtrans with y:=m} allows to only mention
the unknown {\tt m}:

\begin{coq_example}

  apply Rtrans with (y := m).
\end{coq_example}

\begin{coq_eval}
Undo.
\end{coq_eval}

Another solution is to mention the proof of {\tt (R x y)} in {\tt
Rtrans}...

\begin{coq_example}

  apply Rtrans with (1 := Rnm).
\end{coq_example}

\begin{coq_eval}
Undo.
\end{coq_eval}

... or the proof of {\tt (R y z)}:

\begin{coq_example}

  apply Rtrans with (2 := Rmp).
\end{coq_example}

\begin{coq_eval}
Undo.
\end{coq_eval}

On the opposite, one can use {\tt EApply} which postpone the problem
of finding {\tt m}. Then one can apply the hypotheses {\tt Rnm} and {\tt
Rmp}. This instantiates the existential variable and completes the proof.

\begin{coq_example}

  eapply Rtrans.
apply Rnm.
apply Rmp.
\end{coq_example}

\begin{coq_eval}
Reset R.
\end{coq_eval}

\section{{\tt Scheme}}
\comindex{Scheme}
\label{Scheme-examples}

\firstexample
\example{Induction scheme for \texttt{tree} and \texttt{forest}}

The definition of principle of mutual induction for {\tt tree} and
{\tt forest} over the sort {\tt Set} is defined by the command:

\begin{coq_eval}
Reset Initial.
Variables A B : 
              Set.
Inductive tree : Set :=
    node : A -> forest -> tree
with forest : Set :=
  | leaf : B -> forest
  | cons : tree -> forest -> forest.
\end{coq_eval}

\begin{coq_example*}
Scheme tree_forest_rec := Induction for tree
  Sort Set
  with forest_tree_rec := Induction for forest Sort Set.
\end{coq_example*}

You may now look at the type of {\tt tree\_forest\_rec}:

\begin{coq_example}
Check tree_forest_rec.
\end{coq_example}

This principle involves two different predicates for {\tt trees} and
{\tt forests}; it also has three premises each one corresponding to a
constructor of one of the inductive definitions.

The principle {\tt tree\_forest\_rec} shares exactly the same
premises, only the conclusion now refers to the property of forests.

\begin{coq_example}
Check forest_tree_rec.
\end{coq_example}

\example{Predicates {\tt odd} and {\tt even} on naturals}

Let {\tt odd} and {\tt even} be inductively defined as:

\begin{coq_eval}
Reset Initial.
\end{coq_eval}

\begin{coq_example*}
Inductive odd : nat -> Prop :=
    oddS : forall n:nat, even n -> odd (S n)
with even : nat -> Prop :=
  | evenO : even 0%N
  | evenS : forall n:nat, odd n -> even (S n).
\end{coq_example*}

The following command generates a powerful elimination
principle:

\begin{coq_example*}
Scheme odd_even := Minimality for   odd Sort Prop
  with even_odd := Minimality for even Sort Prop.
\end{coq_example*}

The type of {\tt odd\_even} for instance will be:

\begin{coq_example}
Check odd_even.
\end{coq_example}

The type of {\tt even\_odd} shares the same premises but the
conclusion is {\tt (n:nat)(even n)->(Q n)}.

\section{{\tt Functional Scheme} and {\tt Functional Induction}}
\comindex{FunScheme}
\label{FunScheme-examples}

\firstexample
\example{Induction scheme for \texttt{div2}}

We define the function \texttt{div2} as follows:

\begin{coq_eval}
Reset Initial.
\end{coq_eval}

\begin{coq_example*}
Require Import Arith.
Fixpoint div2 (n:nat) : nat :=
  match n with
  | O => 0%N
  | S n0 => match n0 with
            | O => 0%N
            | S n' => S (div2 n')
            end
  end.
\end{coq_example*}

The definition of a principle of induction corresponding to the
recursive structure of \texttt{div2} is defined by the command:

\begin{coq_example}
Functional Scheme div2_ind := Induction for div2.
\end{coq_example}

You may now look at the type of {\tt div2\_ind}:

\begin{coq_example}
Check div2_ind.
\end{coq_example}

We can now prove the following lemma using this principle:


\begin{coq_example*}
Lemma div2_le' : forall n:nat, (div2 n <= n)%N.
intro n.
 pattern n.
\end{coq_example*}


\begin{coq_example}
apply div2_ind; intros.
\end{coq_example}

\begin{coq_example*}
auto with arith.
auto with arith.
simpl; auto with arith.
Qed.
\end{coq_example*}

Since \texttt{div2} is not mutually recursive, we can use
directly the \texttt{Functional Induction} tactic instead of
building the principle:

\begin{coq_example*}
Reset div2_ind.
Lemma div2_le : forall n:nat, (div2 n <= n)%N.
intro n.
\end{coq_example*}

\begin{coq_example}
functional induction div2 n.
\end{coq_example}

\begin{coq_example*}
auto with arith.
auto with arith.
auto with arith.
Qed.
\end{coq_example*}

\paragraph{remark:} \texttt{Functional Induction} makes no use of
a induction principle, so be warned that each time it is used, a
term mimicking the structure of \texttt{div2} (roughly the size
of {\tt div2\_ind}) is built. Using \texttt{Functional Scheme} is
generally faster and less memory consuming.  On the other hand
\texttt{Functional Induction} performs some extra simplification
steps that \texttt{Functional Scheme} doesn't.



\example{Induction scheme for \texttt{tree\_size}}

\begin{coq_eval}
Reset Initial.
\end{coq_eval}

We define trees by the following mutual inductive type:

\begin{coq_example*}
Variable A : Set.
Inductive tree : Set :=
    node : A -> forest -> tree
with forest : Set :=
  | empty : forest
  | cons : tree -> forest -> forest.
\end{coq_example*}

We define the function \texttt{tree\_size} that computes the size
of a tree or a forest.

\begin{coq_example*}                
Fixpoint tree_size (t:tree) : nat :=
  match t with
  | node A f => S (forest_size f)
  end
 with forest_size (f:forest) : nat :=
  match f with
  | empty => 0%N
  | cons t f' => (tree_size t + forest_size f')%N
  end.
\end{coq_example*}

The definition of principle of mutual induction following the
recursive structure of \texttt{tree\_size} is defined by the
command:

\begin{coq_example*}
Functional Scheme treeInd := Induction for tree_size
 with tree_size forest_size.
\end{coq_example*}

You may now look at the type of {\tt treeInd}:

\begin{coq_example}
Check treeInd.
\end{coq_example}



\section{{\tt Inversion}}
\tacindex{Inversion}
\label{Inversion-examples}

\subsection*{Generalities about inversion}

When working with (co)inductive predicates, we are very often faced to
some of these situations:
\begin{itemize}
\item we have an inconsistent instance of an inductive predicate in the
  local context of hypotheses. Thus, the current goal can be trivially
  proved by absurdity. 
\item we have a hypothesis that is an instance of an inductive
  predicate, and the instance has some variables whose constraints we
  would like to derive.
\end{itemize}

The inversion tactics are very useful to simplify the work in these
cases. Inversion tools can be classified in three groups:

\begin{enumerate}
\item tactics for inverting an instance without stocking the inversion
  lemma in the context; this includes the tactics
  (\texttt{Dependent})  \texttt{Inversion} and
 (\texttt{Dependent}) \texttt{Inversion\_clear}.
\item commands for generating and stocking in the context the inversion
  lemma corresponding to an instance; this includes \texttt{Derive}
  (\texttt{Dependent}) \texttt{Inversion} and \texttt{Derive}
  (\texttt{Dependent}) \texttt{Inversion\_clear}.
\item tactics for inverting an instance using an already defined
  inversion lemma; this includes the tactic \texttt{Inversion \ldots using}.
\end{enumerate}

As inversion proofs may be large in size, we recommend the user to
stock the lemmas whenever the same instance needs to be inverted
several times.

\firstexample
\example{Non-dependent inversion}

Let's consider the relation \texttt{Le} over natural numbers and the
following variables:

\begin{coq_eval}
Reset Initial.
\end{coq_eval}

\begin{coq_example*}
Inductive Le : nat -> nat -> Set :=
  | LeO : forall n:nat, Le 0%N n
  | LeS : forall n m:nat, Le n m -> Le (S n) (S m).
Variable P : nat -> nat -> Prop.
Variable Q : forall n m:nat, Le n m -> Prop.
\end{coq_example*}

For example, consider the goal:

\begin{coq_eval}
Lemma ex : forall n m:nat, Le (S n) m -> P n m.
intros.
\end{coq_eval}

\begin{coq_example}
Show.
\end{coq_example}

To prove the goal we may need to reason by cases on \texttt{H} and to 
 derive that \texttt{m}  is necessarily of
the form $(S~m_0)$ for certain $m_0$ and that $(Le~n~m_0)$.  
Deriving these conditions corresponds to prove that the
only possible constructor of \texttt{(Le (S n) m)}  is
\texttt{LeS} and that we can invert the 
\texttt{->} in the type  of \texttt{LeS}.  
This inversion is possible because \texttt{Le} is the smallest set closed by
the constructors \texttt{LeO} and \texttt{LeS}.

\begin{coq_example}
inversion_clear H.
\end{coq_example}

Note that \texttt{m} has been substituted in the goal for \texttt{(S m0)}
and that the hypothesis \texttt{(Le n m0)} has been added to the
context.

Sometimes it is
interesting to have the equality \texttt{m=(S m0)} in the
context to use it after. In that case we can use  \texttt{Inversion} that
does not clear the equalities:

\begin{coq_example*}
Undo.
\end{coq_example*}

\begin{coq_example}
inversion H.
\end{coq_example}

\begin{coq_eval}
Undo.
\end{coq_eval}

\example{Dependent Inversion}

Let us consider the following goal:

\begin{coq_eval}
Lemma ex_dep : forall (n m:nat) (H:Le (S n) m), Q (S n) m H.
intros.
\end{coq_eval}

\begin{coq_example}
Show.
\end{coq_example}

As \texttt{H} occurs in the goal, we may want to reason by cases on its
structure and so, we would like  inversion tactics to
substitute \texttt{H} by the corresponding term in constructor form. 
Neither \texttt{Inversion} nor  {\tt Inversion\_clear} make such a
substitution. 
To have such a behavior we use the dependent inversion tactics:

\begin{coq_example}
dependent inversion_clear H.
\end{coq_example}

Note that \texttt{H} has been substituted by \texttt{(LeS n m0 l)} and
\texttt{m} by \texttt{(S m0)}.

\example{using already defined inversion  lemmas}

\begin{coq_eval}
Abort.
\end{coq_eval}

For example, to generate the inversion lemma for the instance
\texttt{(Le (S n) m)} and the sort \texttt{Prop} we do:

\begin{coq_example*}
Derive Inversion_clear leminv with (forall n m:nat, Le (S n) m) Sort
 Prop.
\end{coq_example*}

\begin{coq_example}
Check leminv.
\end{coq_example}

Then we can use the proven inversion lemma:

\begin{coq_example}
Show.
\end{coq_example}

\begin{coq_example}
inversion H using leminv.
\end{coq_example}

\begin{coq_eval}
Reset Initial.
\end{coq_eval}

\section{\tt AutoRewrite}
\label{AutoRewrite-example}

Here are two examples of {\tt AutoRewrite} use. The first one ({\em Ackermann
function}) shows actually a quite basic use where there is no conditional
rewriting. The second one ({\em Mac Carthy function}) involves conditional
rewritings and shows how to deal with them using the optional tactic of the
{\tt Hint~Rewrite} command.

\firstexample
\example{Ackermann function}
%Here is a basic use of {\tt AutoRewrite} with the Ackermann function:

\begin{coq_example*}
Require Import Arith.
Variable Ack : 
           nat -> nat -> nat.
Axiom Ack0 : 
        forall m:nat, Ack 0 m = S m.
Axiom Ack1 : forall n:nat, Ack (S n) 0 = Ack n 1.
Axiom Ack2 : forall n m:nat, Ack (S n) (S m) = Ack n (Ack (S n) m).
\end{coq_example*}

\begin{coq_example}
Hint Rewrite [ Ack0 Ack1 Ack2 ] in base0.
Lemma ResAck0 : 
 Ack 3 2 = 29%N.
autorewrite [ base0 ] using try reflexivity.
\end{coq_example}

\begin{coq_eval}
Reset Initial.
\end{coq_eval}

\example{Mac Carthy function}
%The Mac Carthy function shows a more complex case:

\begin{coq_example*}
Require Import Omega.
Variable g :   
           nat -> nat -> nat.
Axiom g0 : 
        forall m:nat, g 0 m = m.
Axiom
  g1 :
    forall n m:nat,
      (n > 0)%N -> (m > 100)%N -> g n m = g (pred n) (m - 10).
Axiom
  g2 :
    forall n m:nat,
      (n > 0)%N -> (m <= 100)%N -> g n m = g (S n) (m + 11).
\end{coq_example*}

\begin{coq_example}
Hint Rewrite [ g0 g1 g2 ] in base1 using omega.
Lemma Resg0 : 
 g 1 110 = 100%N.
autorewrite [ base1 ] using reflexivity || simpl.
\end{coq_example}

\begin{coq_eval}
Abort.
\end{coq_eval}

\begin{coq_example}
Lemma Resg1 : g 1 95 = 91%N.
autorewrite [ base1 ] using reflexivity || simpl.
\end{coq_example}

\begin{coq_eval}
Reset Initial.
\end{coq_eval}

\section{\tt Quote}
\tacindex{Quote}
\label{Quote-examples}

The tactic \texttt{Quote} allows to use Barendregt's so-called
2-level approach without writing any ML code. Suppose you have a
language \texttt{L} of 
'abstract terms' and a type \texttt{A} of 'concrete terms' 
and a function \texttt{f : L -> A}. If \texttt{L} is a simple
inductive datatype and \texttt{f} a simple fixpoint, \texttt{Quote f}
will replace the head of current goal by a convertible term of the form 
\texttt{(f t)}. \texttt{L} must have a constructor of type: \texttt{A
  -> L}. 

Here is an example:

\begin{coq_example}
Require Import Quote.
Parameters A B C : Prop.
Inductive formula : Type :=
  | f_and : formula -> formula -> formula (* binary constructor *)
  | f_or : formula -> formula -> formula
  | f_not : formula -> formula (* unary constructor *)
  | f_true : formula (* 0-ary constructor *)
  | f_const : Prop -> formula (* contructor for constants *).
Fixpoint interp_f (f:
                   formula) : Prop :=
  match f with
  | f_and f1 f2 => interp_f f1 /\ interp_f f2
  | f_or f1 f2 => interp_f f1 \/ interp_f f2
  | f_not f1 => ~ interp_f f1
  | f_true => True
  | f_const c => c
  end.
Goal A /\ (A \/ True) /\ ~ B /\ (A <-> A).
quote interp_f.
\end{coq_example}

The algorithm to perform this inversion is: try to match the
term with right-hand sides expression of \texttt{f}. If there is a
match, apply the corresponding left-hand side and call yourself
recursively on sub-terms. If there is no match, we are at a leaf:
return the corresponding constructor (here \texttt{f\_const}) applied
to the term. 

\begin{ErrMsgs}
\item \errindex{Quote: not a simple fixpoint} \\
  Happens when \texttt{Quote} is not able to perform inversion properly.
\end{ErrMsgs}

\subsection{Introducing variables map}

The normal use of \texttt{Quote} is to make proofs by reflection: one
defines a function \texttt{simplify : formula -> formula} and proves a 
theorem \texttt{simplify\_ok: (f:formula)(interp\_f (simplify f)) ->
  (interp\_f f)}. Then, one can simplify formulas by doing:

\begin{quotation}
\begin{verbatim}
   Quote interp_f.
   Apply simplify_ok.
   Compute.
\end{verbatim}
\end{quotation}

But there is a problem with leafs: in the example above one cannot
write a function that implements, for example, the logical simplifications 
$A \wedge A \ra A$ or $A \wedge \neg A \ra \texttt{False}$. This is
because the \Prop{} is impredicative.

It is better to use that type of formulas:

\begin{coq_eval}
Reset formula.
\end{coq_eval}
\begin{coq_example}
Inductive formula : Set :=
  | f_and : formula -> formula -> formula
  | f_or : formula -> formula -> formula
  | f_not : formula -> formula
  | f_true : formula
  | f_atom : index -> formula.
\end{coq_example*}

\texttt{index} is defined in module \texttt{Quote}. Equality on that
type is decidable so we are able to simplify $A \wedge A$ into $A$ at
the abstract level. 

When there are variables, there are bindings, and \texttt{Quote}
provides also a type \texttt{(varmap A)} of bindings from
\texttt{index} to any set \texttt{A}, and a function
\texttt{varmap\_find} to search in such maps. The interpretation
function has now another argument, a variables map:

\begin{coq_example}
Fixpoint interp_f (vm:
                    varmap Prop) (f:formula) {struct f} : Prop :=
  match f with
  | f_and f1 f2 => interp_f vm f1 /\ interp_f vm f2
  | f_or f1 f2 => interp_f vm f1 \/ interp_f vm f2
  | f_not f1 => ~ interp_f vm f1
  | f_true => True
  | f_atom i => varmap_find True i vm
  end.
\end{coq_example}

\noindent\texttt{Quote} handles this second case properly:

\begin{coq_example}
Goal A /\ (B \/ A) /\ (A \/ ~ B).
quote interp_f.
\end{coq_example}

It builds \texttt{vm} and \texttt{t} such that \texttt{(f vm t)} is
convertible with the conclusion of current goal.

\subsection{Combining variables and constants}

One can have both variables and constants in abstracts terms; that is
the case, for example, for the \texttt{Ring} tactic (chapter
\ref{Ring}). Then one must provide to Quote a list of
\emph{constructors of constants}. For example, if the list is
\texttt{[O S]} then closed natural numbers will be considered as
constants and other terms as variables. 

Example: 

\begin{coq_eval}
Reset formula.
\end{coq_eval}
\begin{coq_example*}
Inductive formula : Type :=
  | f_and : formula -> formula -> formula
  | f_or : formula -> formula -> formula
  | f_not : formula -> formula
  | f_true : formula
  | f_const : Prop -> formula (* constructor for constants *)
  | f_atom : index -> formula.
Fixpoint interp_f
 (vm:            (* constructor for variables *)
  varmap Prop) (f:formula) {struct f} : Prop :=
  match f with
  | f_and f1 f2 => interp_f vm f1 /\ interp_f vm f2
  | f_or f1 f2 => interp_f vm f1 \/ interp_f vm f2
  | f_not f1 => ~ interp_f vm f1
  | f_true => True
  | f_const c => c
  | f_atom i => varmap_find True i vm
  end.
Goal 
A /\ (A \/ True) /\ ~ B /\ (C <-> C).
\end{coq_example*}

\begin{coq_example}
quote interp_f [ A B ].
Undo.
  quote interp_f [ B C iff ].
\end{coq_example}

\Warning Since function inversion
is undecidable in general case, don't expect miracles from it!

% \SeeAlso file \texttt{theories/DEMOS/DemoQuote.v}

\SeeAlso comments of source file \texttt{tactics/contrib/polynom/quote.ml}

\SeeAlso the tactic \texttt{Ring} (chapter \ref{Ring})


%%% Local Variables: 
%%% mode: latex
%%% TeX-master: "Reference-Manual"
%%% End: 
