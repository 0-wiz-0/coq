\documentclass{article}
\usepackage{fullpage}
\usepackage{hevea}
\usepackage{url}

%%%%%%%%%%%%%%%%%%%%%%%%%%%%%%%%%%%%%%%%%%
% MACROS FOR THE REFERENCE MANUAL OF COQ %
%%%%%%%%%%%%%%%%%%%%%%%%%%%%%%%%%%%%%%%%%%

\newcommand{\coqversion}{7.3}

% For commentaries (define \com as {} for the release manual)
%\newcommand{\com}[1]{{\it(* #1 *)}}
\newcommand{\com}[1]{}

%%%%%%%%%%%%%%%%%%%%%%%
% Formatting commands %
%%%%%%%%%%%%%%%%%%%%%%%

\newcommand{\ErrMsg}{\medskip \noindent {\bf Error message: }}
\newcommand{\ErrMsgx}{\medskip \noindent {\bf Error messages: }}
\newcommand{\variant}{\medskip \noindent {\bf Variant: }}
\newcommand{\variants}{\medskip \noindent {\bf Variants: }}
\newcommand{\SeeAlso}{\medskip \noindent {\bf See also: }}
\newcommand{\Rem}{\medskip \noindent {\bf Remark: }}
\newcommand{\Rems}{\medskip \noindent {\bf Remarks: }}
\newcommand{\Example}{\medskip \noindent {\bf Example: }}
\newcommand{\Warning}{\medskip \noindent {\bf Warning: }}
\newcommand{\Warns}{\medskip \noindent {\bf Warnings: }}
\newcounter{ex}
\newcommand{\firstexample}{\setcounter{ex}{1}}
\newcommand{\example}[1]{
\medskip \noindent \textbf{Example \arabic{ex}: }\textit{#1}
\addtocounter{ex}{1}}

\newenvironment{Variant}{\variant\begin{enumerate}}{\end{enumerate}}
\newenvironment{Variants}{\variants\begin{enumerate}}{\end{enumerate}}
\newenvironment{ErrMsgs}{\ErrMsgx\begin{enumerate}}{\end{enumerate}}
\newenvironment{Remarks}{\Rems\begin{enumerate}}{\end{enumerate}}
\newenvironment{Warnings}{\Warns\begin{enumerate}}{\end{enumerate}}
\newenvironment{Examples}{\medskip\noindent{\bf Examples:}
\begin{enumerate}}{\end{enumerate}}

\newcommand{\rr}{\raggedright}

\newcommand{\tinyskip}{\rule{0mm}{4mm}}

\newcommand{\bd}{\noindent\bf}
\newcommand{\sbd}{\vspace{8pt}\noindent\bf}
\newcommand{\sdoll}[1]{\begin{small}$ #1~ $\end{small}}
\newcommand{\sdollnb}[1]{\begin{small}$ #1 $\end{small}}
\newcommand{\kw}[1]{\textsf{#1}}
\newcommand{\spec}[1]{\{\,#1\,\}}

% Building regular expressions
\newcommand{\zeroone}[1]{{\sl [}#1{\sl ]}}
%\newcommand{\zeroonemany}[1]{$\{$#1$\}$*}
%\newcommand{\onemany}[1]{$\{$#1$\}$+}
\newcommand{\nelist}[2]{{#1} {\tt #2} {\ldots} {\tt #2} {#1}}
\newcommand{\sequence}[2]{{\sl [}{#1} {\tt #2} {\ldots} {\tt #2} {#1}{\sl ]}}
\newcommand{\nelistwithoutblank}[2]{#1{\tt #2}\ldots{\tt #2}#1}
\newcommand{\sequencewithoutblank}[2]{$[$#1{\tt #2}\ldots{\tt #2}#1$]$}

% Used for RefMan-gal
\newcommand{\ml}[1]{\hbox{\tt{#1}}}
\newcommand{\op}{\,|\,}

%%%%%%%%%%%%%%%%%%%%%%%%
% Trademarks and so on %
%%%%%%%%%%%%%%%%%%%%%%%%

\newcommand{\Coq}{{\sf Coq}}
\newcommand{\ocaml}{{\sf Objective Caml}}
\newcommand{\camlpppp}{{\sf Camlp4}}
\newcommand{\emacs}{{\sf GNU Emacs}}
\newcommand{\gallina}{\textsf{Gallina}}
\newcommand{\CIC}{\mbox{\sc Cic}}
\newcommand{\FW}{\mbox{$F_{\omega}$}}
\newcommand{\bn}{{\sf BNF}}

%%%%%%%%%%%%%%%%%%%
% Name of tactics %
%%%%%%%%%%%%%%%%%%%

\newcommand{\Natural}{\mbox{\tt Natural}}

%%%%%%%%%%%%%%%%%
% \rm\sl series %
%%%%%%%%%%%%%%%%%

\newcommand{\Fwterm}{\textrm{\textsl{Fwterm}}}
\newcommand{\Index}{\textrm{\textsl{index}}}
\newcommand{\abbrev}{\textrm{\textsl{abbreviation}}}
\newcommand{\annotation}{\textrm{\textsl{annotation}}}
\newcommand{\atomictac}{\textrm{\textsl{atomic\_tactic}}}
\newcommand{\binders}{\textrm{\textsl{bindings}}}
\newcommand{\binder}{\textrm{\textsl{binding}}}
\newcommand{\bindinglist}{\textrm{\textsl{bindings\_list}}}
\newcommand{\cast}{\textrm{\textsl{cast}}}
\newcommand{\cofixpointbody}{\textrm{\textsl{cofix\_body}}}
\newcommand{\coinductivebody}{\textrm{\textsl{coind\_body}}}
\newcommand{\commandtac}{\textrm{\textsl{tactic\_invocation}}}
\newcommand{\constructor}{\textrm{\textsl{constructor}}}
\newcommand{\convtactic}{\textrm{\textsl{conv\_tactic}}}
\newcommand{\declarationkeyword}{\textrm{\textsl{declaration\_keyword}}}
\newcommand{\declaration}{\textrm{\textsl{declaration}}}
\newcommand{\definition}{\textrm{\textsl{definition}}}
\newcommand{\digit}{\textrm{\textsl{digit}}}
\newcommand{\eqn}{\textrm{\textsl{equation}}}
\newcommand{\exteqn}{\textrm{\textsl{ext\_eqn}}}
\newcommand{\field}{\textrm{\textsl{field}}}
\newcommand{\firstletter}{\textrm{\textsl{first\_letter}}}
\newcommand{\fixpg}{\textrm{\textsl{fix\_pgm}}}
\newcommand{\fixpointbody}{\textrm{\textsl{fix\_body}}}
\newcommand{\fixpoint}{\textrm{\textsl{fixpoint}}}
\newcommand{\flag}{\textrm{\textsl{flag}}}
\newcommand{\form}{\textrm{\textsl{form}}}
\newcommand{\gensymbol}{\textrm{\textsl{symbol}}}
\newcommand{\localassums}{\textrm{\textsl{local\_assums}}}
\newcommand{\localdef}{\textrm{\textsl{local\_def}}}
\newcommand{\localdecls}{\textrm{\textsl{local\_decls}}}
\newcommand{\ident}{\textrm{\textsl{ident}}}
\newcommand{\accessident}{\textrm{\textsl{access\_ident}}}
\newcommand{\inductivebody}{\textrm{\textsl{ind\_body}}}
\newcommand{\inductive}{\textrm{\textsl{inductive}}}
\newcommand{\naturalnumber}{\textrm{\textsl{natural}}}
\newcommand{\integer}{\textrm{\textsl{integer}}}
\newcommand{\multpattern}{\textrm{\textsl{mult\_pattern}}}
\newcommand{\mutualcoinductive}{\textrm{\textsl{mutual\_coinductive}}}
\newcommand{\mutualinductive}{\textrm{\textsl{mutual\_inductive}}}
\newcommand{\nestedpattern}{\textrm{\textsl{nested\_pattern}}}
\newcommand{\num}{\textrm{\textsl{num}}}
\newcommand{\params}{\textrm{\textsl{params}}}
\newcommand{\pattern}{\textrm{\textsl{pattern}}}
\newcommand{\pat}{\textrm{\textsl{pat}}}
\newcommand{\pgs}{\textrm{\textsl{pgms}}}
\newcommand{\pg}{\textrm{\textsl{pgm}}}
\newcommand{\proof}{\textrm{\textsl{proof}}}
\newcommand{\record}{\textrm{\textsl{record}}}
\newcommand{\rewrule}{\textrm{\textsl{rewriting\_rule}}}
\newcommand{\sentence}{\textrm{\textsl{sentence}}}
\newcommand{\simplepattern}{\textrm{\textsl{simple\_pattern}}}
\newcommand{\sort}{\textrm{\textsl{sort}}}
\newcommand{\specif}{\textrm{\textsl{specif}}}
\newcommand{\statement}{\textrm{\textsl{statement}}}
\newcommand{\str}{\textrm{\textsl{string}}}
\newcommand{\subsequentletter}{\textrm{\textsl{subsequent\_letter}}}
\newcommand{\switch}{\textrm{\textsl{switch}}}
\newcommand{\tac}{\textrm{\textsl{tactic}}}
\newcommand{\terms}{\textrm{\textsl{terms}}}
\newcommand{\term}{\textrm{\textsl{term}}}
\newcommand{\module}{\textrm{\textsl{module}}}
\newcommand{\modexpr}{\textrm{\textsl{module\_expression}}}
\newcommand{\modtype}{\textrm{\textsl{module\_type}}}
\newcommand{\onemodbinding}{\textrm{\textsl{module\_binding}}}
\newcommand{\modbindings}{\textrm{\textsl{module\_bindings}}}
\newcommand{\qualid}{\textrm{\textsl{qualid}}}
\newcommand{\class}{\textrm{\textsl{class}}}
\newcommand{\dirpath}{\textrm{\textsl{dirpath}}}
\newcommand{\typedidents}{\textrm{\textsl{typed\_idents}}}
\newcommand{\type}{\textrm{\textsl{type}}}
\newcommand{\vref}{\textrm{\textsl{ref}}}
\newcommand{\zarithformula}{\textrm{\textsl{zarith\_formula}}}
\newcommand{\zarith}{\textrm{\textsl{zarith}}}

%%%%%%%%%%%%%%%%%%%%%%%%%%%%%%%%%%%%%%%%%%%%%%%%%%%%%%%
% \mbox{\sf } series for roman text in maths formulas %
%%%%%%%%%%%%%%%%%%%%%%%%%%%%%%%%%%%%%%%%%%%%%%%%%%%%%%%

\newcommand{\alors}{\mbox{\textsf{then}}}
\newcommand{\alter}{\mbox{\textsf{alter}}}
\newcommand{\bool}{\mbox{\textsf{bool}}}
\newcommand{\conc}{\mbox{\textsf{conc}}}
\newcommand{\cons}{\mbox{\textsf{cons}}}
\newcommand{\consf}{\mbox{\textsf{consf}}}
\newcommand{\emptyf}{\mbox{\textsf{emptyf}}}
\newcommand{\EqSt}{\mbox{\textsf{EqSt}}}
\newcommand{\false}{\mbox{\textsf{false}}}
\newcommand{\filter}{\mbox{\textsf{filter}}}
\newcommand{\forest}{\mbox{\textsf{forest}}}
\newcommand{\from}{\mbox{\textsf{from}}}
\newcommand{\hd}{\mbox{\textsf{hd}}}
\newcommand{\Length}{\mbox{\textsf{Length}}}
\newcommand{\length}{\mbox{\textsf{length}}}
\newcommand{\LengthA}{\mbox {\textsf{Length\_A}}}
\newcommand{\List}{\mbox{\textsf{List}}}
\newcommand{\ListA}{\mbox{\textsf{List\_A}}}
\newcommand{\LNil}{\mbox{\textsf{Lnil}}}
\newcommand{\LCons}{\mbox{\textsf{Lcons}}}
\newcommand{\nat}{\mbox{\textsf{nat}}}
\newcommand{\nO}{\mbox{\textsf{O}}}
\newcommand{\nS}{\mbox{\textsf{S}}}
\newcommand{\node}{\mbox{\textsf{node}}}
\newcommand{\Nil}{\mbox{\textsf{nil}}}
\newcommand{\Prop}{\mbox{\textsf{Prop}}}
\newcommand{\Set}{\mbox{\textsf{Set}}}
\newcommand{\si}{\mbox{\textsf{if}}}
\newcommand{\sinon}{\mbox{\textsf{else}}}
\newcommand{\Str}{\mbox{\textsf{Stream}}}
\newcommand{\tl}{\mbox{\textsf{tl}}}
\newcommand{\tree}{\mbox{\textsf{tree}}}
\newcommand{\true}{\mbox{\textsf{true}}}
\newcommand{\Type}{\mbox{\textsf{Type}}}
\newcommand{\unfold}{\mbox{\textsf{unfold}}}
\newcommand{\zeros}{\mbox{\textsf{zeros}}}

%%%%%%%%%
% Misc. %
%%%%%%%%%
\newcommand{\T}{\texttt{T}}
\newcommand{\U}{\texttt{U}}
\newcommand{\real}{\textsf{Real}}
\newcommand{\Spec}{\textit{Spec}}
\newcommand{\Data}{\textit{Data}}
\newcommand{\In} {{\textbf{in }}}
\newcommand{\AND} {{\textbf{and}}}
\newcommand{\If}{{\textbf{if }}}
\newcommand{\Else}{{\textbf{else }}}
\newcommand{\Then} {{\textbf{then }}}
\newcommand{\Let}{{\textbf{let }}}
\newcommand{\Where}{{\textbf{where rec }}}
\newcommand{\Function}{{\textbf{function }}}
\newcommand{\Rec}{{\textbf{rec }}}
\newcommand{\cn}{\centering}

%%%%%%%%%%%%%%%%%%%%%%%%%%%%%
% Math commands and symbols %
%%%%%%%%%%%%%%%%%%%%%%%%%%%%%

\newcommand{\la}{\leftarrow}
\newcommand{\ra}{\rightarrow}
\newcommand{\Ra}{\Rightarrow}
\newcommand{\rt}{\Rightarrow}
\newcommand{\lla}{\longleftarrow}
\newcommand{\lra}{\longrightarrow}
\newcommand{\Llra}{\Longleftrightarrow}
\newcommand{\mt}{\mapsto}
\newcommand{\ov}{\overrightarrow}
\newcommand{\wh}{\widehat}
\newcommand{\up}{\uparrow}
\newcommand{\dw}{\downarrow}
\newcommand{\nr}{\nearrow}
\newcommand{\se}{\searrow}
\newcommand{\sw}{\swarrow}
\newcommand{\nw}{\nwarrow}

\newcommand{\vm}[1]{\vspace{#1em}}
\newcommand{\vx}[1]{\vspace{#1ex}}
\newcommand{\hm}[1]{\hspace{#1em}}
\newcommand{\hx}[1]{\hspace{#1ex}}
\newcommand{\sm}{\mbox{ }}
\newcommand{\mx}{\mbox}

\newcommand{\nq}{\neq}
\newcommand{\eq}{\equiv}
\newcommand{\fa}{\forall}
\newcommand{\ex}{\exists}
\newcommand{\impl}{\rightarrow}
\newcommand{\Or}{\vee}
\newcommand{\And}{\wedge}
\newcommand{\ms}{\models}
\newcommand{\bw}{\bigwedge}
\newcommand{\ts}{\times}
\newcommand{\cc}{\circ}
\newcommand{\es}{\emptyset}
\newcommand{\bs}{\backslash}
\newcommand{\vd}{\vdash}
\newcommand{\lan}{{\langle }}
\newcommand{\ran}{{\rangle }}

\newcommand{\al}{\alpha}
\newcommand{\bt}{\beta}
\newcommand{\io}{\iota}
\newcommand{\lb}{\lambda}
\newcommand{\sg}{\sigma}
\newcommand{\sa}{\Sigma}
\newcommand{\om}{\Omega}
\newcommand{\tu}{\tau}

%%%%%%%%%%%%%%%%%%%%%%%%%
% Custom maths commands %
%%%%%%%%%%%%%%%%%%%%%%%%%

\newcommand{\sumbool}[2]{\{#1\}+\{#2\}}
\newcommand{\myifthenelse}[3]{\kw{if} ~ #1 ~\kw{then} ~ #2 ~ \kw{else} ~ #3}
\newcommand{\fun}[2]{\item[]{\tt {#1}}. \quad\\ #2}
\newcommand{\WF}[2]{\ensuremath{{\cal W\!F}(#1)[#2]}}
\newcommand{\WFE}[1]{\WF{E}{#1}}
\newcommand{\WT}[4]{\ensuremath{#1[#2] \vdash #3 : #4}}
\newcommand{\WTE}[3]{\WT{E}{#1}{#2}{#3}}
\newcommand{\WTEG}[2]{\WTE{\Gamma}{#1}{#2}}

\newcommand{\WTM}[3]{\WT{#1}{}{#2}{#3}}
\newcommand{\WFT}[2]{\ensuremath{#1[] \vdash {\cal W\!F}(#2)}}
\newcommand{\WS}[3]{\ensuremath{#1[] \vdash #2 <: #3}}
\newcommand{\WSE}[2]{\WS{E}{#1}{#2}}

\newcommand{\WTRED}[5]{\mbox{$#1[#2] \vdash #3 #4 #5$}}
\newcommand{\WTERED}[4]{\mbox{$E[#1] \vdash #2 #3 #4$}}
\newcommand{\WTELECONV}[3]{\WTERED{#1}{#2}{\leconvert}{#3}}
\newcommand{\WTEGRED}[3]{\WTERED{\Gamma}{#1}{#2}{#3}}
\newcommand{\WTECONV}[3]{\WTERED{#1}{#2}{\convert}{#3}}
\newcommand{\WTEGCONV}[2]{\WTERED{\Gamma}{#1}{\convert}{#2}}
\newcommand{\WTEGLECONV}[2]{\WTERED{\Gamma}{#1}{\leconvert}{#2}}

\newcommand{\lab}[1]{\mathit{labels}(#1)}
\newcommand{\dom}[1]{\mathit{dom}(#1)}

\newcommand{\CI}[2]{\mbox{$\{#1\}^{#2}$}}
\newcommand{\CIP}[3]{\mbox{$\{#1\}_{#2}^{#3}$}}
\newcommand{\CIPV}[1]{\CIP{#1}{I_1.. I_k}{P_1.. P_k}}
\newcommand{\CIPI}[1]{\CIP{#1}{I}{P}}
\newcommand{\CIF}[1]{\mbox{$\{#1\}_{f_1.. f_n}$}}
\newcommand{\NInd}[3]{\mbox{{\sf Ind}$(#1)(\begin{array}[t]{l}#2:=#3
                                              \,)\end{array}$}}
\newcommand{\Ind}[4]{\mbox{{\sf Ind}$(#1)[#2](\begin{array}[t]{l}#3:=#4
                                                 \,)\end{array}$}}
\newcommand{\Indp}[5]{\mbox{{\sf Ind}$_{#5}(#1)[#2](\begin{array}[t]{l}#3:=#4
                                                 \,)\end{array}$}}
\newcommand{\Def}[4]{\mbox{{\sf Def}$(#1)(#2:=#3:#4)$}}
\newcommand{\Assum}[3]{\mbox{{\sf Assum}$(#1)(#2:#3)$}}
\newcommand{\Match}[3]{\mbox{$<\!#1\!>\!{\mbox{\tt Match}}~#2~{\mbox{\tt with}}~#3~{\mbox{\tt end}}$}}
\newcommand{\Case}[3]{\mbox{$<\!#1\!>\!{\mbox{\tt Cases}}~#2~{\mbox{\tt of}}~#3~{\mbox{\tt end}}$}}
\newcommand{\Fix}[2]{\mbox{\tt Fix}~#1\{#2\}}
\newcommand{\CoFix}[2]{\mbox{\tt CoFix}~#1\{#2\}}
\newcommand{\With}[2]{\mbox{\tt ~with~}}
\newcommand{\subst}[3]{#1\{#2/#3\}}
\newcommand{\substs}[4]{#1\{(#2/#3)_{#4}\}}
\newcommand{\Sort}{\mbox{$\cal S$}}
\newcommand{\convert}{=_{\beta\delta\iota\zeta}}
\newcommand{\leconvert}{\leq_{\beta\delta\iota\zeta}}
\newcommand{\NN}{\mbox{I\hspace{-7pt}N}}
\newcommand{\inference}[1]{$${#1}$$}
\newcommand{\compat}[2]{\mbox{$[#1|#2]$}}
\newcommand{\tristackrel}[3]{\mathrel{\mathop{#2}\limits_{#3}^{#1}}}

\newcommand{\Impl}{{\it Impl}}
\newcommand{\Mod}[3]{{\sf Mod}({#1}:{#2}:={#3})}
\newcommand{\ModType}[2]{{\sf ModType}({#1}:={#2})}
\newcommand{\ModS}[2]{{\sf ModS}({#1}:{#2})}
\newcommand{\ModSEq}[3]{{\sf ModSEq}({#1}:{#2}=={#3})}
\newcommand{\functor}[3]{\ensuremath{{\sf Functor}[#1:#2]\;#3}}
\newcommand{\funsig}[3]{\ensuremath{{\sf Funsig}(#1:#2)\;#3}}
\newcommand{\sig}[1]{\ensuremath{{\sf Sig}~#1~{\sf End}}}
\newcommand{\struct}[1]{\ensuremath{{\sf Struct}~#1~{\sf End}}}


\newbox\tempa
\newbox\tempb
\newdimen\tempc
\newcommand{\mud}[1]{\hfil $\displaystyle{\mathstrut #1}$\hfil}
\newcommand{\rig}[1]{\hfil $\displaystyle{#1}$}
\newcommand{\irulehelp}[3]{\setbox\tempa=\hbox{$\displaystyle{\mathstrut #2}$}%
                        \setbox\tempb=\vbox{\halign{##\cr
        \mud{#1}\cr
        \noalign{\vskip\the\lineskip}
        \noalign{\hrule height 0pt}
        \rig{\vbox to 0pt{\vss\hbox to 0pt{${\; #3}$\hss}\vss}}\cr
        \noalign{\hrule}
        \noalign{\vskip\the\lineskip}
        \mud{\copy\tempa}\cr}}
                      \tempc=\wd\tempb
                      \advance\tempc by \wd\tempa
                      \divide\tempc by 2 }
\newcommand{\irule}[3]{{\irulehelp{#1}{#2}{#3}
                     \hbox to \wd\tempa{\hss \box\tempb \hss}}}

\newcommand{\sverb}[1]{\tt #1}
\newcommand{\mover}[2]{{#1\over #2}}
\newcommand{\jd}[2]{#1 \vdash #2}
\newcommand{\mathline}[1]{\[#1\]}
\newcommand{\zrule}[2]{#2: #1}
\newcommand{\orule}[3]{#3: {\mover{#1}{#2}}}
\newcommand{\trule}[4]{#4: \mover{#1  \qquad #2} {#3}}
\newcommand{\thrule}[5]{#5: {\mover{#1  \qquad #2 \qquad #3}{#4}}}


% $Id$ 


%%% Local Variables: 
%%% mode: latex
%%% TeX-master: "Reference-Manual"
%%% End: 


\newcommand{\coqtt}[1]{{\tt #1}}
\newcommand{\coqimp}{{$->$}}
\newcommand{\coqequiv}{{$<->$}}

\newcommand{\question}[1]{
  \subsection*{\aname{no:number}{Question:~\sf{#1}}}
  \addcontentsline{toc}{subsection}{\ahrefloc{no:number}{#1}}
  \addcontentsline{htoc}{subsection}{\ahrefloc{no:number}{#1}}}
%{\subsubsection{Question:~\sf{#1}}}}
\newcommand{\answer}{{\bf Answer:~}}


\newcommand\vfile[2]{\ahref{#1}{\tt {#2}.v}}

\begin{document}
\urldef{\refman}{\url}{http://coq.inria.fr/doc/main.html}
\urldef{\InitWf}{\url}
  {http://coq.inria.fr/library/Coq.Init.Wf.html}
\urldef{\LogicBerardi}{\url}
  {http://coq.inria.fr/library/Coq.Logic.Berardi.html}
\urldef{\LogicClassical}{\url}
  {http://coq.inria.fr/library/Coq.Logic.Classical.html}
\urldef{\LogicClassicalFacts}{\url}
  {http://coq.inria.fr/library/Coq.Logic.ClassicalFacts.html}
\urldef{\LogicProofIrrelevance}{\url}
  {http://coq.inria.fr/library/Coq.Logic.ProofIrrelevance.html}

\begin{center}
\begin{Large}
Frequently Asked Questions
\end{Large}

\medskip
{\Coq} version 7
\end{center}

\tableofcontents

\section{General}

\question{What is {\Coq}?}

\answer {\Coq} is a proof assistant with two main
application fields: proving program correctness and formalizing mathematical
theories. {\Coq} offers also automatic extraction of zero-defect programs
from a proof of their specification.

\question{What are the main applications done in {\Coq}?}

\answer Consequent developments include Java certification, safety of
cryptographic protocols, medical robotics, constructive and classical
analysis, algebraic topology, ...

\question{What is the logic of {\Coq}?}

\answer {\Coq} is based on an axiom-free type theory called
the Calculus of Inductive Constructions (see Coquand \cite{CoHu86}
and Coquand--Paulin-Mohring \cite{CoPa89}). It includes higher-order
functions and predicates, inductive and co-inductive datatypes and
predicates and a stratified hierarchy of sets.

\question{Is {\Coq}'s logic constructive?}

\answer {\Coq} theory is constructive, but classical
reasoning on proposition can be loaded as an external module.

\question{Is {\Coq} about proofs or provability?}

\answer {\Coq} is definitely about proofs. A statement
cannot be accepted without a proof expressed in the Calculus of
Inductive Constructions. The correctness of a proof relies on a
relatively small proof-checker at the kernel of {\Coq} (8000 lines of ML
code). Even {\Coq} decision procedures are producing proofs which are
double-checked by the kernel.

\question{Can I define non-terminating programs in {\Coq}?}

\answer No, all programs in {\Coq} are terminating. To reason on
non-terminating programs, you must turn your program into a
terminating one by explicitly giving a bound on the number of
iterations of the program.

\question{Is {\Coq} a theorem prover?}

\answer {\Coq} comes with a few decision procedures (on propositional
calculus, Presburger arithmetic, ring and field simplification,
resolution, ...) but the main style for proving theorems is by using
LCF-style tactics.

\question{How is equational reasoning working in {\Coq}?}

\answer {\Coq} is reasoning modulo sequential evaluation of
(not necessarily closed) programs (it is decidable because {\Coq}
programs are terminating). This equality is called {\em conversion}.
Besides conversion, equations are to be treated by hand or using
specialized tactics (e.g. simplifications on fields).

\section{Equality}

%\subsection{Equality on Terms}

\question{How can I prove that 2 terms in an inductive sets are equal? Or different?}

\answer Cf "Decide Equality" and "Discriminate" in the \ahref{\refman}{Reference Manual}.

\question{Why is the proof of \coqtt{0+n=n} on natural numbers
trivial but the proof of \coqtt{n+0=n} is not?}

\answer Since \coqtt{+} (\coqtt{plus}) on natural numbers is defined by analysis on its first argument

\begin{coq_example}
Print plus.
\end{coq_example}

the expression \coqtt{0+n} evaluates to \coqtt{n}. As {\Coq} reasons
modulo evaluation of expressions, \coqtt{0+n} and \coqtt{n} are
considered equal and the theorem \coqtt{0+n=n} is an instance of the
reflexivity of equality. On the other side, \coqtt{n+0} does not
evaluate to \coqtt{n} and a proof by induction on \coqtt{n} is
necessary to trigger the evaluation of \coqtt{+}.

\question{I have two proofs of the same proposition. Can I prove they are equal?}
\label{proof-irrelevance}

\answer In the base {\Coq} system, the answer is no. However, if
classical logic is set, the answer is yes for propositions in {\Prop}.

More precisely, the equality of all proofs of a given proposition is called
{\em proof-irrelevance}. Proof-irrelevance (in {\Prop}) can be assumed
  without leading to contradiction in {\Coq}.

  That classical logic (what can be assumed in {\Coq} by requiring the file
  \vfile{\LogicClassical}{Classical}) implies proof-irrelevance is shown in files
  \vfile{\LogicProofIrrelevance}{ProofIrrelevance} and \vfile{\LogicBerardi}{Berardi}.

  Alternatively, propositional extensionality (i.e.  \coqtt{(A
  \coqequiv B) \coqimp A=B}, defined in \vfile{\LogicClassicalFacts}{ClassicalFacts}),
  also implies proof-irrelevance.

  It is an ongoing work of research to natively include proof
  irrelevance in {\Coq}.

\question{I have two proofs of an equality statement. Can I prove they are 
equal?}

\answer Yes, if equality is decidable on the domain considered (which
is the case for {\tt nat}, {\tt bool}, etc): see {\Coq} file
\verb=Eqdep_dec.v=). No otherwise, unless
assuming Streicher's axiom $K$ (see \cite{HofStr98}) or a more general
assumption such as proof-irrelevance (see \ref{proof-irrelevance}) or
classical logic.

\question{Can I prove that the second components of equal dependent
pairs are equal?}

\answer The answer is the same as for proofs of equality
statements. It is yes if equality on the domain of the first component
is decidable, no in the general case. More precisely it is a
consequence of Streicher's axiom $K$.

\question{What is Streicher's axiom $K$}
\label{Streicher}

\answer Streicher's axiom $K$ \cite{HofStr98} states that the only proof of a reflexivity statement is a reflexivity proof:

\begin{coq_example*}
Axiom K: (A:Set)(x:A)(p:x=x)p==(refl_equal ? x).
\end{coq_example*}

In the general case, axiom $K$ is an independent statement of the
Calculus of Inductive Constructions.
However, it is true on decidable domains. It is also
trivially a consequence of proof-irrelevance (see
\ref{proof-irrelevance}) hence of classical logic.

Axiom $K$ is equivalent to {\em Uniqueness of Identity Proofs} which
states that

\begin{coq_example*}
Axiom UIP: (A:Set)(x,y:A)(p1,p2:x=y)p1==p2.
\end{coq_example*}

Axiom $K$ is also equivalent to the following statements.

\begin{coq_example*}
Axiom eq_rec_eq : (a:A)(P:A->Set)(p:(P p))(h:a=a) x=(eq_rec A a P p a h).
Inductive eq_dep [p:U;x:(P p)] : (q:U)(P q)->Prop :=
   eq_dep_intro : (eq_dep p x p x).
Axiom eq_dep_eq : (a:A)(P:A->Set)(p1,p2:(P a))(eq_dep a x a y)->x=y.
\end{coq_example*}

All of these statements can be found in file \verb=Eqdep.v=.

% \subsection{Equality on types}

\question{How to prove that 2 sets are differents?}

\answer You need to find a property true on one set and false on the
other one. As an example we show how to prove that {\tt bool} and {\tt
nat} are discriminable. As discrimination property we take the
property to have no more than 2 elements.

\begin{coq_example*}
Theorem nat_bool_discr : ~bool==nat.
Proof.
  Pose discr := [X:Set]~(a,b:X)~(x:X)~x=a->~x=b->False.
  Intro Heq; Assert H : (discr bool).
    Intro H; Apply (H true false); NewDestruct x; Auto.
  Rewrite Heq in H; Apply H; Clear H.
  NewDestruct a; NewDestruct b; Intro H0; EAuto.
  NewDestruct n; [Apply (H0 (S (S O))); Discriminate | EAuto].
Qed.
\end{coq_example*}

\question{How to exploit equalities on sets}

\answer To extract information from an equality on sets, you need to find a
predicate satisfied by the elements of the sets. As an example we show
the following theorem.

\begin{coq_example*}
Theorem le_le: (m,n:nat){x:nat | (le x m)} = {x:nat | (le x n)} -> m = n.

Inductive discr_int_n [n:nat;X:Set] : Prop :=
  intro : (fst:X->nat)(snd:(x:X)(le (fst x) n))
            ((x,y:X)(exist ? [z](le z n) (fst x) (snd x))=
                (exist ? [z](le z n) (fst y) (snd y))->x=y)
          -> (discr_int_n n X).

Proof.
Intros.
Assert Hdiscr : (discr_int_n m {x:nat | (le x m)}).
Split with (proj1_sig nat [x:?](le x m)) (proj2_sig nat [x:?](le x m)).
NewDestruct x; NewDestruct y.
Exact [x:?]x.
Rewrite H in Hdiscr.
Elim Hdiscr.
Assert H1:=(snd (exist nat [z](le z (S n)) (S n) (le_n (S n)))).
...
\end{coq_example*}

\section{Axioms}

\question{Can I identify two functions that have the same behaviour (extensionality)?}

\answer Extensionality of functions is an axiom in, say set theory,
but from a programming point of view, extensionality cannot be {\em a
priori} accepted since it would identify, say programs as mergesort
and quicksort.

%\begin{coq_example*}
%  Axiom extensionality : (A,B:Set)(f,g:(A->B))(x:A)(f x)=(g x)->f=g.
%\end{coq_example*}

Let {\tt A}, {\tt B} be types. To deal with extensionality on 
\verb=A->B=, the recommended approach is to define his
own extensional equality on \verb=A->B=.

\begin{coq_example*}
  Definition ext_eq [f,g:A->B] := (x:A)(f x)=(g x).
\end{coq_example*}

and to reason on \verb=A->B= as a setoid (see the Chapter on
Setoids in the Reference Manual).


\question{Is {\Type} impredicative (that is, letting
\coqtt{T:=(X:Type)X->X}, can I apply an expression \coqtt{t} of type \coqtt{T} to \coqtt{T} itself)?}

\answer No, {\Type} is stratified. This is hidden for the
user, but {\Coq} internally maintains a set of constraints ensuring
stratification.

  If {\Type} were impredicative then Girard's systems U- and U could be
encoded in {\Coq} and it is known from Girard, Coquand, Hurkens and
Miquel that systems U- and U are inconsistent [cf Girard 1972, Coquand
1991, Hurkens 1993, Miquel 2001].

  For instance, when the user see {\tt (X:Type)X->X : Type}, each
occurrence of {\Type} is implicitly bound to a different level, say
$\alpha$ and $\beta$ and the actual statement is {\tt
(X:Type($\alpha$))X->X : Type($\beta$)} with the constraint
$\alpha<\beta$.

  When a statement violates a constraint, the message {\tt Universe
inconsistency} appears. Example: {\tt [x:=Type][y:(X:Type)X \coqimp X](z x x)}.

\section{Inductive types}

\question{What is a "large inductive definition"?}

\answer An inductive definition in {\Prop} pr {\Set} is called large
if its constructors embed sets or propositions. As an example, here is
a large inductive type:

\begin{coq_example*}
  Inductive sigST [P:Set->Set] : Set
      := existST : (X:Set)(P X) -> (sigST P).
\end{coq_example*}

   Large inductive definition have restricted elimination scheme to
prevent inconsistencies. Especially, projecting the set or the
proposition content of a large inductive definition is forbidden. If
it were allowed, we could be able to encode
e.g. Burali-Forti paradox \cite{Gir70,Coq85}.

\question{Why Injection does not work on impredicative types?}
 E.g. in this case:

\begin{coq_example*}
  Inductive I : Set := intro : (k:Set)k -> I.
  Lemma eq_jdef : (x,y:nat) (intro ? x)=(intro ? y) -> x=y.
  Intros x y H; Injection H.
\end{coq_example*}

\answer Injectivity of constructors is restricted to predicative types. If
injectivity on large inductive types were not restricted, we would be
allowed to derive an inconsistency (e.g. following the lines of
Burali-Forti paradox). The question remains open whether injectivity
is consistent on some large inductive types not expressive enough to
encode known paradoxes (such as type I above).

\section{Recursion}

\question{Why can't I define a non terminating program?}

\answer Because otherwise the decidability of the type-checking
algorithm (which involves evaluation of programs) is not ensured.  On
another side, if non terminating proofs were allowed, we could get a
proof of {\tt False}:

\begin{coq_example*}
(* This is fortunately not allowed! *)
Fixpoint InfiniteProof [n:nat] : False := (InfiniteProof n).
Theorem Paradox : False.
Proof (InfiniteProof O).
\end{coq_example*}

\question{Why only structurally well-founded loops are allowed?}

\answer The structural order on inductive types is a simple and
powerful notion of termination. The consistency of the Calculus of
Inductive Constructions relies on it and another consistency proof
would have to be made for stronger termination arguments (such
as the termination of the evaluation of CIC programs themselves!).

In spite of this, all non-pathological termination orders can be mapped
to a structural order. Tools to do this are provided in the file 
\vfile{\InitWf}{Wf} of the standard library of {\Coq}.

\question{How to define loops based on non structurally smaller
recursive calls?}

\answer The procedure is as follows (we consider the definition of {\tt
mergesort} as an example).

\begin{itemize}

\item Define the termination order, say {\tt R} on the type {\tt A} of
the arguments of the loop.

\begin{coq_example*}
Definition R [a,b:(list nat)] := (length a) < (length b).
\end{coq_example*}

\item Prove that this order is well-founded (in fact that all elements in {\tt A} are accessible along {\tt R}).

\begin{coq_example*}
Lemma Rwf : (well_founded R).
\end{coq_example*}

\item Define the step function (which needs proofs that recursive
calls are on smaller arguments).

\begin{coq_example*}
Definition split :
   (l:(list nat)){l1:(list nat) | (R l1 l)} * {l2:(list nat) | (R l2 l)}
   := (* ... *) .
Definition concat : (l1,l2:(list nat))(list nat) := (* ... *) .
Definition merge_step [l:(list nat)][f:(l':(list nat))(R l' l)->(list nat)] :=
  let (lH1,lH2) = (split l) in
  let (l1,H1) = lH1 in
  let (l2,H2) = lH2 in
  (concat (f l1 H1) (f l2 H2)).
\end{coq_example*}

\item Define the recursive function by fixpoint on the step function.

\begin{coq_example*}
Definition merge := (fix (list nat) R Rwf [_](list nat) merge_step).
\end{coq_example*}

\end{itemize}

\question{What is behind these accessibility and well-foundedness proofs?}

\answer Well-foundedness of some relation {\tt R} on some type {\tt A}
is defined as the accessibility of all elements of {\tt A} along {\tt R}.

\begin{coq_example}
Print well_founded.
Print Acc.
\end{coq_example}

The structure of the accessibility predicate is a well-founded tree
branching at each node {\tt x} in {\tt A} along all the nodes {\tt x'}
less than {\tt x} along {\tt R}. Any sequence of elements of {\tt A}
decreasing along the order {\tt R} are branches in the accessibility
tree. Hence any decreasing along {\tt R} is mapped into a structural
decreasing in the accessibility tree of {\tt R}. This is emphasized in the definition of {\tt fix} which recurs not on its argument {\tt x:A} but on the accessibility of this argument along {\tt R}.

See file \vfile{\InitWf}{Wf}.

\section{Elimination, Case Analysis and Induction}

%\subsection{Parametric elimination}

\question{What is parametric elimination?}

\answer Parametric elimination is a 
generalisation of the substitutivity of equality. Only inductive types
with non-constant parameters are concerned with this kind of dependent
elimination. Typical examples of inductive types enjoying parametric
elimination in {\Coq} standard library are the equality predicate
\verb=eq= and the less than predicate on natural numbers \verb=le=.

\begin{coq_example}
Print eq.
Print le.
\end{coq_example}

%\subsection{Dependent elimination}

\question{What means dependent elimination?}

\answer Dependent elimination is a generalisation of the
standard Peano's induction principle on natural numbers.  This
elimination asserts that to prove a property P on say the set {\tt nat} of
natural numbers, it is enough to prove the property on O and to prove
it on successor numbers. Combined with fixpoints, this directly
provides with usual Peano's induction scheme.

Non dependent elimination is when elimination is used to build an
object whose type does not depend on the eliminated object. An example
of non dependent elimination is recursion, or the simpler form of
{\coqtt if $\ldots$ then $\ldots$ else}.

The relevant tactics to deal with this kind of elimination are {\tt
Elim}, {\tt Induction}, {\tt Destruct} and {\tt Case}, {\tt
NewInduction} and {\tt NewDestruct}.


% \question{Does the elimination in Prop follows a
%   different rule than the elimination in Set?}

% \answer There are two kinds of dependent case analysis. The result type of a
% case analysis may depend on the matched object (this is what is
% usually intended by dependent) but may also depend on the non-constant
% arguments of the type of the matched object without being dependent of
% the object itself.

%   This is typically what happens in a proof using equality. The result
% does not depend on the proof of the equality itself but crucially
% depends on the second compared argument of the equality.

%   It happens that proofs using elimination (i.e. case analysis or
% induction) on objects at the Set level are usually dependent on the
% matched term. This is because propositions precisely depend on the
% object on which an induction or case analysis is applied. Conversely,
% it happens that proofs using elimination on objects at the Prop level
% (typically elimination on equality proofs) are usually not dependent
% on the matched object. This is because proofs usually stay at the
% proof level and a few developments are stating things about proofs.
% But, again, to be not dependent on the matched objet does not mean not
% dependent at all. For equality, dependency on the second compared
% argument is precisely what provides with the substitutivity schema of
% equality.

\question{Why is dependent elimination in Prop not
available by default?}

\answer 
This is just because most of the time it is not needed. To derive a
dependent elimination principle in Prop, use the command {\tt Scheme} and
apply the elimination scheme using the \verb=using= option of
\verb=Elim=.



%\subsection{Induction}

\question{How to perform double induction?}

\answer To reason by induction on pairs, just reason by induction on the first component then by case analysis on the second component. Here is an example:

\begin{coq_eval}
Require Arith.
\end{coq_eval}

\begin{coq_example}
Infix "<" lt (at level 5, no associativity).
Infix "<=" le (at level 5, no associativity).
Lemma le_or_lt : (n,n0:nat) n0<n \/ n<=n0.
NewInduction n; NewDestruct n0; Auto with arith.
NewDestruct (IHn n0); Auto with arith.
\end{coq_example}

\question{How to define a function by double recursion?}

\answer The same trick applies, you can even use the patterm-matching
compilation algorithm to do the work for you. Here is an example:

\begin{coq_example}
Fixpoint minus [n:nat] : nat -> nat := 
  [m:nat]Cases n m of
            O    _     =>  O
         | (S k) O     => (S k)
         | (S k) (S l) => (minus k l)
        end. 
Print minus.
\end{coq_example}


\section{Coinductive types}

\question{I have a cofixpoint t:=F(t) and I want to prove t=F(t). What is
the simplest way?}

\answer Just case-expand F(t) then complete by a trivial case analysis.
Here is what it gives on e.g. the type of streams on naturals

\begin{coq_example}
  Set Implicit Arguments.
  CoInductive Stream [A : Set]  : Set := Cons : A->(Stream A)->(Stream A).
  CoFixpoint nats : nat -> (Stream nat) := [n](Cons n (nats (S n))).

  Lemma Stream_unfold : (n:nat)(nats n)= (Cons n (nats (S n))).
  Proof.
    Intro; Change (nats n) = (Cases (nats n) of (Cons x s) => (Cons x s) end).
    Case (nats n); Reflexivity.
  Qed.
\end{coq_example}

\section{Miscellaneous}

\question{I copy-paste a term and {\Coq} says it is not convertible
  to the original term. Sometimes it even says the copied term is not
well-typed.}

\answer This is probably due to invisible implicit information (implicit
arguments, coercions and Cases annotations) in the printed term, which
is not re-synthetized from the copied-pasted term in the same way as
it is in the original term.

  Consider for instance {\tt (eqT Type True True)}. This term is
printed as {\tt True==True} and re-parsed as {\tt (eqT Prop True
True)}. The two terms are not convertible (hence they fool tactics
like {\tt Pattern}.

  There is currently no satisfactory answer to the problem.
 
  Due to coercions, one may even face type-checking errors. In some
rare cases, the criterion to hide coercions is a bit too loosy, which
may result in a typing error message if the parser is not able to find
again the missing coercion.

\question{I have a problem of dependent elimination on
proofs, how to solve it?}

\begin{coq_eval}
Reset Initial.
\end{coq_eval}

\begin{coq_example*}
Inductive Def1 : Set
   := c1 : Def1.

Inductive DefProp : Def1 -> Set
   := c2 : (d:Def1)(DefProp d).

Inductive Comb : Set
   := c3 : (d:Def1)(DefProp d) -> Comb.

Lemma eq_comb
   : (d1,d1':Def1; d2:(DefProp d1); d2':(DefProp d1')) d1=d1' ->
     (c3 d1 d2)=(c3 d1' d2').
\end{coq_example*}

\answer You need to derive the dependent elimination
scheme for DefProp by hand using {\coqtt Scheme}.

\begin{coq_example*}
Scheme DefProp_elim := Induction for DefProp Sort Prop.

Lemma eq_comb
   : (d1,d1':Def1)d1=d1'->(d2:(DefProp d1); d2':(DefProp d1'))
     (c3 d1 d2)=(c3 d1' d2').
Intros.
NewDestruct H.
NewDestruct d2 using DefProp_elim.
NewDestruct d2' using DefProp_elim.
Reflexivity.
Qed.
\end{coq_example*}

\question{And what if I want to prove the following?}

\begin{coq_example*}
Inductive natProp : nat -> Prop
   := p0 : (natProp O)
   |  pS : (n:nat)(natProp n) -> (natProp (S n)).

Inductive package : Set
   := pack : (n:nat)(natProp n) -> package.

Lemma eq_pack
   : (n,n':nat)
     n=n' ->
     (np:(natProp n); np':(natProp n'))
     (pack n np)=(pack n' np').
\end{coq_example*}

\answer

\begin{coq_eval}
Reset eq_pack.
\end{coq_eval}
\begin{coq_example*}
Scheme natProp_elim := Induction for natProp Sort Prop.

Definition pack_S : package -> package. 
NewDestruct 1.
Apply (pack (S n)).
Apply pS; Assumption.
Defined.

Lemma eq_pack
   : (n,n':nat)
     n=n' ->
     (np:(natProp n); np':(natProp n'))
     (pack n np)=(pack n' np').
Intros n n' Heq np np'.
Generalize Dependent n'.
NewInduction np using natProp_elim.
NewInduction np' using natProp_elim; Intros; Auto. 
Discriminate Heq.
NewInduction np' using natProp_elim; Intros; Auto. 
Discriminate Heq.
Change (pack_S (pack n  np))=(pack_S (pack n0 np')).
Apply (f_equal package).
Apply IHnp.
Auto.
Qed.
\end{coq_example*}

\question{Why does {\Coq} tell me that \texttt{\{x:A|(P x)\}} is not convertible with \texttt{(sig A P)}?}

\answer
This is because \texttt{\{x:A|(P x)\}} is a notation for \texttt{(sig A [x:A](p x))}. Since {\Coq} does not reason up to $\eta$-conversion, this is different from
\texttt{(sig A P)}.

\bibliographystyle{plain}
\bibliography{biblio}

\end{document}
