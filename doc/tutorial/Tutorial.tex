\documentclass[11pt,a4paper]{book}
\usepackage[T1]{fontenc}
\usepackage[utf8]{inputenc}
\usepackage{textcomp}
\usepackage{pslatex}
\usepackage{hyperref}

\input{../common/version.tex}
%%%%%%%%%%%%%%%%%%%%%%%%%%%%%%%%%%%%%%%%%%
% MACROS FOR THE REFERENCE MANUAL OF COQ %
%%%%%%%%%%%%%%%%%%%%%%%%%%%%%%%%%%%%%%%%%%

\newcommand{\coqversion}{7.3}

% For commentaries (define \com as {} for the release manual)
%\newcommand{\com}[1]{{\it(* #1 *)}}
\newcommand{\com}[1]{}

%%%%%%%%%%%%%%%%%%%%%%%
% Formatting commands %
%%%%%%%%%%%%%%%%%%%%%%%

\newcommand{\ErrMsg}{\medskip \noindent {\bf Error message: }}
\newcommand{\ErrMsgx}{\medskip \noindent {\bf Error messages: }}
\newcommand{\variant}{\medskip \noindent {\bf Variant: }}
\newcommand{\variants}{\medskip \noindent {\bf Variants: }}
\newcommand{\SeeAlso}{\medskip \noindent {\bf See also: }}
\newcommand{\Rem}{\medskip \noindent {\bf Remark: }}
\newcommand{\Rems}{\medskip \noindent {\bf Remarks: }}
\newcommand{\Example}{\medskip \noindent {\bf Example: }}
\newcommand{\Warning}{\medskip \noindent {\bf Warning: }}
\newcommand{\Warns}{\medskip \noindent {\bf Warnings: }}
\newcounter{ex}
\newcommand{\firstexample}{\setcounter{ex}{1}}
\newcommand{\example}[1]{
\medskip \noindent \textbf{Example \arabic{ex}: }\textit{#1}
\addtocounter{ex}{1}}

\newenvironment{Variant}{\variant\begin{enumerate}}{\end{enumerate}}
\newenvironment{Variants}{\variants\begin{enumerate}}{\end{enumerate}}
\newenvironment{ErrMsgs}{\ErrMsgx\begin{enumerate}}{\end{enumerate}}
\newenvironment{Remarks}{\Rems\begin{enumerate}}{\end{enumerate}}
\newenvironment{Warnings}{\Warns\begin{enumerate}}{\end{enumerate}}
\newenvironment{Examples}{\medskip\noindent{\bf Examples:}
\begin{enumerate}}{\end{enumerate}}

\newcommand{\rr}{\raggedright}

\newcommand{\tinyskip}{\rule{0mm}{4mm}}

\newcommand{\bd}{\noindent\bf}
\newcommand{\sbd}{\vspace{8pt}\noindent\bf}
\newcommand{\sdoll}[1]{\begin{small}$ #1~ $\end{small}}
\newcommand{\sdollnb}[1]{\begin{small}$ #1 $\end{small}}
\newcommand{\kw}[1]{\textsf{#1}}
\newcommand{\spec}[1]{\{\,#1\,\}}

% Building regular expressions
\newcommand{\zeroone}[1]{{\sl [}#1{\sl ]}}
%\newcommand{\zeroonemany}[1]{$\{$#1$\}$*}
%\newcommand{\onemany}[1]{$\{$#1$\}$+}
\newcommand{\nelist}[2]{{#1} {\tt #2} {\ldots} {\tt #2} {#1}}
\newcommand{\sequence}[2]{{\sl [}{#1} {\tt #2} {\ldots} {\tt #2} {#1}{\sl ]}}
\newcommand{\nelistwithoutblank}[2]{#1{\tt #2}\ldots{\tt #2}#1}
\newcommand{\sequencewithoutblank}[2]{$[$#1{\tt #2}\ldots{\tt #2}#1$]$}

% Used for RefMan-gal
\newcommand{\ml}[1]{\hbox{\tt{#1}}}
\newcommand{\op}{\,|\,}

%%%%%%%%%%%%%%%%%%%%%%%%
% Trademarks and so on %
%%%%%%%%%%%%%%%%%%%%%%%%

\newcommand{\Coq}{{\sf Coq}}
\newcommand{\ocaml}{{\sf Objective Caml}}
\newcommand{\camlpppp}{{\sf Camlp4}}
\newcommand{\emacs}{{\sf GNU Emacs}}
\newcommand{\gallina}{\textsf{Gallina}}
\newcommand{\CIC}{\mbox{\sc Cic}}
\newcommand{\FW}{\mbox{$F_{\omega}$}}
\newcommand{\bn}{{\sf BNF}}

%%%%%%%%%%%%%%%%%%%
% Name of tactics %
%%%%%%%%%%%%%%%%%%%

\newcommand{\Natural}{\mbox{\tt Natural}}

%%%%%%%%%%%%%%%%%
% \rm\sl series %
%%%%%%%%%%%%%%%%%

\newcommand{\Fwterm}{\textrm{\textsl{Fwterm}}}
\newcommand{\Index}{\textrm{\textsl{index}}}
\newcommand{\abbrev}{\textrm{\textsl{abbreviation}}}
\newcommand{\annotation}{\textrm{\textsl{annotation}}}
\newcommand{\atomictac}{\textrm{\textsl{atomic\_tactic}}}
\newcommand{\binders}{\textrm{\textsl{bindings}}}
\newcommand{\binder}{\textrm{\textsl{binding}}}
\newcommand{\bindinglist}{\textrm{\textsl{bindings\_list}}}
\newcommand{\cast}{\textrm{\textsl{cast}}}
\newcommand{\cofixpointbody}{\textrm{\textsl{cofix\_body}}}
\newcommand{\coinductivebody}{\textrm{\textsl{coind\_body}}}
\newcommand{\commandtac}{\textrm{\textsl{tactic\_invocation}}}
\newcommand{\constructor}{\textrm{\textsl{constructor}}}
\newcommand{\convtactic}{\textrm{\textsl{conv\_tactic}}}
\newcommand{\declarationkeyword}{\textrm{\textsl{declaration\_keyword}}}
\newcommand{\declaration}{\textrm{\textsl{declaration}}}
\newcommand{\definition}{\textrm{\textsl{definition}}}
\newcommand{\digit}{\textrm{\textsl{digit}}}
\newcommand{\eqn}{\textrm{\textsl{equation}}}
\newcommand{\exteqn}{\textrm{\textsl{ext\_eqn}}}
\newcommand{\field}{\textrm{\textsl{field}}}
\newcommand{\firstletter}{\textrm{\textsl{first\_letter}}}
\newcommand{\fixpg}{\textrm{\textsl{fix\_pgm}}}
\newcommand{\fixpointbody}{\textrm{\textsl{fix\_body}}}
\newcommand{\fixpoint}{\textrm{\textsl{fixpoint}}}
\newcommand{\flag}{\textrm{\textsl{flag}}}
\newcommand{\form}{\textrm{\textsl{form}}}
\newcommand{\gensymbol}{\textrm{\textsl{symbol}}}
\newcommand{\localassums}{\textrm{\textsl{local\_assums}}}
\newcommand{\localdef}{\textrm{\textsl{local\_def}}}
\newcommand{\localdecls}{\textrm{\textsl{local\_decls}}}
\newcommand{\ident}{\textrm{\textsl{ident}}}
\newcommand{\accessident}{\textrm{\textsl{access\_ident}}}
\newcommand{\inductivebody}{\textrm{\textsl{ind\_body}}}
\newcommand{\inductive}{\textrm{\textsl{inductive}}}
\newcommand{\naturalnumber}{\textrm{\textsl{natural}}}
\newcommand{\integer}{\textrm{\textsl{integer}}}
\newcommand{\multpattern}{\textrm{\textsl{mult\_pattern}}}
\newcommand{\mutualcoinductive}{\textrm{\textsl{mutual\_coinductive}}}
\newcommand{\mutualinductive}{\textrm{\textsl{mutual\_inductive}}}
\newcommand{\nestedpattern}{\textrm{\textsl{nested\_pattern}}}
\newcommand{\num}{\textrm{\textsl{num}}}
\newcommand{\params}{\textrm{\textsl{params}}}
\newcommand{\pattern}{\textrm{\textsl{pattern}}}
\newcommand{\pat}{\textrm{\textsl{pat}}}
\newcommand{\pgs}{\textrm{\textsl{pgms}}}
\newcommand{\pg}{\textrm{\textsl{pgm}}}
\newcommand{\proof}{\textrm{\textsl{proof}}}
\newcommand{\record}{\textrm{\textsl{record}}}
\newcommand{\rewrule}{\textrm{\textsl{rewriting\_rule}}}
\newcommand{\sentence}{\textrm{\textsl{sentence}}}
\newcommand{\simplepattern}{\textrm{\textsl{simple\_pattern}}}
\newcommand{\sort}{\textrm{\textsl{sort}}}
\newcommand{\specif}{\textrm{\textsl{specif}}}
\newcommand{\statement}{\textrm{\textsl{statement}}}
\newcommand{\str}{\textrm{\textsl{string}}}
\newcommand{\subsequentletter}{\textrm{\textsl{subsequent\_letter}}}
\newcommand{\switch}{\textrm{\textsl{switch}}}
\newcommand{\tac}{\textrm{\textsl{tactic}}}
\newcommand{\terms}{\textrm{\textsl{terms}}}
\newcommand{\term}{\textrm{\textsl{term}}}
\newcommand{\module}{\textrm{\textsl{module}}}
\newcommand{\modexpr}{\textrm{\textsl{module\_expression}}}
\newcommand{\modtype}{\textrm{\textsl{module\_type}}}
\newcommand{\onemodbinding}{\textrm{\textsl{module\_binding}}}
\newcommand{\modbindings}{\textrm{\textsl{module\_bindings}}}
\newcommand{\qualid}{\textrm{\textsl{qualid}}}
\newcommand{\class}{\textrm{\textsl{class}}}
\newcommand{\dirpath}{\textrm{\textsl{dirpath}}}
\newcommand{\typedidents}{\textrm{\textsl{typed\_idents}}}
\newcommand{\type}{\textrm{\textsl{type}}}
\newcommand{\vref}{\textrm{\textsl{ref}}}
\newcommand{\zarithformula}{\textrm{\textsl{zarith\_formula}}}
\newcommand{\zarith}{\textrm{\textsl{zarith}}}

%%%%%%%%%%%%%%%%%%%%%%%%%%%%%%%%%%%%%%%%%%%%%%%%%%%%%%%
% \mbox{\sf } series for roman text in maths formulas %
%%%%%%%%%%%%%%%%%%%%%%%%%%%%%%%%%%%%%%%%%%%%%%%%%%%%%%%

\newcommand{\alors}{\mbox{\textsf{then}}}
\newcommand{\alter}{\mbox{\textsf{alter}}}
\newcommand{\bool}{\mbox{\textsf{bool}}}
\newcommand{\conc}{\mbox{\textsf{conc}}}
\newcommand{\cons}{\mbox{\textsf{cons}}}
\newcommand{\consf}{\mbox{\textsf{consf}}}
\newcommand{\emptyf}{\mbox{\textsf{emptyf}}}
\newcommand{\EqSt}{\mbox{\textsf{EqSt}}}
\newcommand{\false}{\mbox{\textsf{false}}}
\newcommand{\filter}{\mbox{\textsf{filter}}}
\newcommand{\forest}{\mbox{\textsf{forest}}}
\newcommand{\from}{\mbox{\textsf{from}}}
\newcommand{\hd}{\mbox{\textsf{hd}}}
\newcommand{\Length}{\mbox{\textsf{Length}}}
\newcommand{\length}{\mbox{\textsf{length}}}
\newcommand{\LengthA}{\mbox {\textsf{Length\_A}}}
\newcommand{\List}{\mbox{\textsf{List}}}
\newcommand{\ListA}{\mbox{\textsf{List\_A}}}
\newcommand{\LNil}{\mbox{\textsf{Lnil}}}
\newcommand{\LCons}{\mbox{\textsf{Lcons}}}
\newcommand{\nat}{\mbox{\textsf{nat}}}
\newcommand{\nO}{\mbox{\textsf{O}}}
\newcommand{\nS}{\mbox{\textsf{S}}}
\newcommand{\node}{\mbox{\textsf{node}}}
\newcommand{\Nil}{\mbox{\textsf{nil}}}
\newcommand{\Prop}{\mbox{\textsf{Prop}}}
\newcommand{\Set}{\mbox{\textsf{Set}}}
\newcommand{\si}{\mbox{\textsf{if}}}
\newcommand{\sinon}{\mbox{\textsf{else}}}
\newcommand{\Str}{\mbox{\textsf{Stream}}}
\newcommand{\tl}{\mbox{\textsf{tl}}}
\newcommand{\tree}{\mbox{\textsf{tree}}}
\newcommand{\true}{\mbox{\textsf{true}}}
\newcommand{\Type}{\mbox{\textsf{Type}}}
\newcommand{\unfold}{\mbox{\textsf{unfold}}}
\newcommand{\zeros}{\mbox{\textsf{zeros}}}

%%%%%%%%%
% Misc. %
%%%%%%%%%
\newcommand{\T}{\texttt{T}}
\newcommand{\U}{\texttt{U}}
\newcommand{\real}{\textsf{Real}}
\newcommand{\Spec}{\textit{Spec}}
\newcommand{\Data}{\textit{Data}}
\newcommand{\In} {{\textbf{in }}}
\newcommand{\AND} {{\textbf{and}}}
\newcommand{\If}{{\textbf{if }}}
\newcommand{\Else}{{\textbf{else }}}
\newcommand{\Then} {{\textbf{then }}}
\newcommand{\Let}{{\textbf{let }}}
\newcommand{\Where}{{\textbf{where rec }}}
\newcommand{\Function}{{\textbf{function }}}
\newcommand{\Rec}{{\textbf{rec }}}
\newcommand{\cn}{\centering}

%%%%%%%%%%%%%%%%%%%%%%%%%%%%%
% Math commands and symbols %
%%%%%%%%%%%%%%%%%%%%%%%%%%%%%

\newcommand{\la}{\leftarrow}
\newcommand{\ra}{\rightarrow}
\newcommand{\Ra}{\Rightarrow}
\newcommand{\rt}{\Rightarrow}
\newcommand{\lla}{\longleftarrow}
\newcommand{\lra}{\longrightarrow}
\newcommand{\Llra}{\Longleftrightarrow}
\newcommand{\mt}{\mapsto}
\newcommand{\ov}{\overrightarrow}
\newcommand{\wh}{\widehat}
\newcommand{\up}{\uparrow}
\newcommand{\dw}{\downarrow}
\newcommand{\nr}{\nearrow}
\newcommand{\se}{\searrow}
\newcommand{\sw}{\swarrow}
\newcommand{\nw}{\nwarrow}

\newcommand{\vm}[1]{\vspace{#1em}}
\newcommand{\vx}[1]{\vspace{#1ex}}
\newcommand{\hm}[1]{\hspace{#1em}}
\newcommand{\hx}[1]{\hspace{#1ex}}
\newcommand{\sm}{\mbox{ }}
\newcommand{\mx}{\mbox}

\newcommand{\nq}{\neq}
\newcommand{\eq}{\equiv}
\newcommand{\fa}{\forall}
\newcommand{\ex}{\exists}
\newcommand{\impl}{\rightarrow}
\newcommand{\Or}{\vee}
\newcommand{\And}{\wedge}
\newcommand{\ms}{\models}
\newcommand{\bw}{\bigwedge}
\newcommand{\ts}{\times}
\newcommand{\cc}{\circ}
\newcommand{\es}{\emptyset}
\newcommand{\bs}{\backslash}
\newcommand{\vd}{\vdash}
\newcommand{\lan}{{\langle }}
\newcommand{\ran}{{\rangle }}

\newcommand{\al}{\alpha}
\newcommand{\bt}{\beta}
\newcommand{\io}{\iota}
\newcommand{\lb}{\lambda}
\newcommand{\sg}{\sigma}
\newcommand{\sa}{\Sigma}
\newcommand{\om}{\Omega}
\newcommand{\tu}{\tau}

%%%%%%%%%%%%%%%%%%%%%%%%%
% Custom maths commands %
%%%%%%%%%%%%%%%%%%%%%%%%%

\newcommand{\sumbool}[2]{\{#1\}+\{#2\}}
\newcommand{\myifthenelse}[3]{\kw{if} ~ #1 ~\kw{then} ~ #2 ~ \kw{else} ~ #3}
\newcommand{\fun}[2]{\item[]{\tt {#1}}. \quad\\ #2}
\newcommand{\WF}[2]{\ensuremath{{\cal W\!F}(#1)[#2]}}
\newcommand{\WFE}[1]{\WF{E}{#1}}
\newcommand{\WT}[4]{\ensuremath{#1[#2] \vdash #3 : #4}}
\newcommand{\WTE}[3]{\WT{E}{#1}{#2}{#3}}
\newcommand{\WTEG}[2]{\WTE{\Gamma}{#1}{#2}}

\newcommand{\WTM}[3]{\WT{#1}{}{#2}{#3}}
\newcommand{\WFT}[2]{\ensuremath{#1[] \vdash {\cal W\!F}(#2)}}
\newcommand{\WS}[3]{\ensuremath{#1[] \vdash #2 <: #3}}
\newcommand{\WSE}[2]{\WS{E}{#1}{#2}}

\newcommand{\WTRED}[5]{\mbox{$#1[#2] \vdash #3 #4 #5$}}
\newcommand{\WTERED}[4]{\mbox{$E[#1] \vdash #2 #3 #4$}}
\newcommand{\WTELECONV}[3]{\WTERED{#1}{#2}{\leconvert}{#3}}
\newcommand{\WTEGRED}[3]{\WTERED{\Gamma}{#1}{#2}{#3}}
\newcommand{\WTECONV}[3]{\WTERED{#1}{#2}{\convert}{#3}}
\newcommand{\WTEGCONV}[2]{\WTERED{\Gamma}{#1}{\convert}{#2}}
\newcommand{\WTEGLECONV}[2]{\WTERED{\Gamma}{#1}{\leconvert}{#2}}

\newcommand{\lab}[1]{\mathit{labels}(#1)}
\newcommand{\dom}[1]{\mathit{dom}(#1)}

\newcommand{\CI}[2]{\mbox{$\{#1\}^{#2}$}}
\newcommand{\CIP}[3]{\mbox{$\{#1\}_{#2}^{#3}$}}
\newcommand{\CIPV}[1]{\CIP{#1}{I_1.. I_k}{P_1.. P_k}}
\newcommand{\CIPI}[1]{\CIP{#1}{I}{P}}
\newcommand{\CIF}[1]{\mbox{$\{#1\}_{f_1.. f_n}$}}
\newcommand{\NInd}[3]{\mbox{{\sf Ind}$(#1)(\begin{array}[t]{l}#2:=#3
                                              \,)\end{array}$}}
\newcommand{\Ind}[4]{\mbox{{\sf Ind}$(#1)[#2](\begin{array}[t]{l}#3:=#4
                                                 \,)\end{array}$}}
\newcommand{\Indp}[5]{\mbox{{\sf Ind}$_{#5}(#1)[#2](\begin{array}[t]{l}#3:=#4
                                                 \,)\end{array}$}}
\newcommand{\Def}[4]{\mbox{{\sf Def}$(#1)(#2:=#3:#4)$}}
\newcommand{\Assum}[3]{\mbox{{\sf Assum}$(#1)(#2:#3)$}}
\newcommand{\Match}[3]{\mbox{$<\!#1\!>\!{\mbox{\tt Match}}~#2~{\mbox{\tt with}}~#3~{\mbox{\tt end}}$}}
\newcommand{\Case}[3]{\mbox{$<\!#1\!>\!{\mbox{\tt Cases}}~#2~{\mbox{\tt of}}~#3~{\mbox{\tt end}}$}}
\newcommand{\Fix}[2]{\mbox{\tt Fix}~#1\{#2\}}
\newcommand{\CoFix}[2]{\mbox{\tt CoFix}~#1\{#2\}}
\newcommand{\With}[2]{\mbox{\tt ~with~}}
\newcommand{\subst}[3]{#1\{#2/#3\}}
\newcommand{\substs}[4]{#1\{(#2/#3)_{#4}\}}
\newcommand{\Sort}{\mbox{$\cal S$}}
\newcommand{\convert}{=_{\beta\delta\iota\zeta}}
\newcommand{\leconvert}{\leq_{\beta\delta\iota\zeta}}
\newcommand{\NN}{\mbox{I\hspace{-7pt}N}}
\newcommand{\inference}[1]{$${#1}$$}
\newcommand{\compat}[2]{\mbox{$[#1|#2]$}}
\newcommand{\tristackrel}[3]{\mathrel{\mathop{#2}\limits_{#3}^{#1}}}

\newcommand{\Impl}{{\it Impl}}
\newcommand{\Mod}[3]{{\sf Mod}({#1}:{#2}:={#3})}
\newcommand{\ModType}[2]{{\sf ModType}({#1}:={#2})}
\newcommand{\ModS}[2]{{\sf ModS}({#1}:{#2})}
\newcommand{\ModSEq}[3]{{\sf ModSEq}({#1}:{#2}=={#3})}
\newcommand{\functor}[3]{\ensuremath{{\sf Functor}[#1:#2]\;#3}}
\newcommand{\funsig}[3]{\ensuremath{{\sf Funsig}(#1:#2)\;#3}}
\newcommand{\sig}[1]{\ensuremath{{\sf Sig}~#1~{\sf End}}}
\newcommand{\struct}[1]{\ensuremath{{\sf Struct}~#1~{\sf End}}}


\newbox\tempa
\newbox\tempb
\newdimen\tempc
\newcommand{\mud}[1]{\hfil $\displaystyle{\mathstrut #1}$\hfil}
\newcommand{\rig}[1]{\hfil $\displaystyle{#1}$}
\newcommand{\irulehelp}[3]{\setbox\tempa=\hbox{$\displaystyle{\mathstrut #2}$}%
                        \setbox\tempb=\vbox{\halign{##\cr
        \mud{#1}\cr
        \noalign{\vskip\the\lineskip}
        \noalign{\hrule height 0pt}
        \rig{\vbox to 0pt{\vss\hbox to 0pt{${\; #3}$\hss}\vss}}\cr
        \noalign{\hrule}
        \noalign{\vskip\the\lineskip}
        \mud{\copy\tempa}\cr}}
                      \tempc=\wd\tempb
                      \advance\tempc by \wd\tempa
                      \divide\tempc by 2 }
\newcommand{\irule}[3]{{\irulehelp{#1}{#2}{#3}
                     \hbox to \wd\tempa{\hss \box\tempb \hss}}}

\newcommand{\sverb}[1]{\tt #1}
\newcommand{\mover}[2]{{#1\over #2}}
\newcommand{\jd}[2]{#1 \vdash #2}
\newcommand{\mathline}[1]{\[#1\]}
\newcommand{\zrule}[2]{#2: #1}
\newcommand{\orule}[3]{#3: {\mover{#1}{#2}}}
\newcommand{\trule}[4]{#4: \mover{#1  \qquad #2} {#3}}
\newcommand{\thrule}[5]{#5: {\mover{#1  \qquad #2 \qquad #3}{#4}}}


% $Id$ 


%%% Local Variables: 
%%% mode: latex
%%% TeX-master: "Reference-Manual"
%%% End: 

%%%%%%%%%%%%%%%%%%%%%%%%%%%%%%%%
% File title.tex
% Page formatting commands
% Macro \coverpage
%%%%%%%%%%%%%%%%%%%%%%%%%%%%%%%%

%\setlength{\marginparwidth}{0pt}
%\setlength{\oddsidemargin}{0pt}
%\setlength{\evensidemargin}{0pt}
%\setlength{\marginparsep}{0pt}
%\setlength{\topmargin}{0pt}
%\setlength{\textwidth}{16.9cm}
%\setlength{\textheight}{22cm}
%\usepackage{fullpage}

%\newcommand{\printingdate}{\today}
%\newcommand{\isdraft}{\Large\bf\today\\[20pt]}
%\newcommand{\isdraft}{\vspace{20pt}}

\newcommand{\coverpage}[3]{
\thispagestyle{empty}
\begin{center}
\bfseries % for the rest of this page, until \end{center}
\Huge
The Coq Proof Assistant\\[12pt]
#1\\[20pt]
\Large\today\\[20pt]
Version \coqversion%\footnote[1]{This research was partly supported by IST working group ``Types''}

\vspace{0pt plus .5fill}
#2
\par\vfill
The Coq Development Team

\vspace*{15pt}
\end{center}
\newpage

\thispagestyle{empty}
\hbox{}\vfill % without \hbox \vfill does not work at the top of the page
\begin{flushleft}
%BEGIN LATEX
V\coqversion, \today
\par\vspace{20pt}
%END LATEX
\copyright INRIA 1999-2004 ({\Coq} versions 7.x)

\copyright INRIA 2004-2010 ({\Coq} versions 8.x)

#3
\end{flushleft}
} % end of \coverpage definition


% \newcommand{\shorttitle}[1]{
% \begin{center}
% \begin{huge}
% \begin{bf}
% The Coq Proof Assistant\\
% \vspace{10pt}
%     #1\\
% \end{bf}
% \end{huge}
% \end{center}
% \vspace{5pt}
% }

% Local Variables: 
% mode: LaTeX
% TeX-master: ""
% End: 

% $Id$ 


%\makeindex

\begin{document}
\coverpage{A Tutorial}{Gérard Huet, Gilles Kahn and Christine Paulin-Mohring}{}

%\tableofcontents

\chapter*{Getting started}

\Coq{} is a Proof Assistant for a Logical Framework known as the Calculus
of Inductive Constructions. It allows the interactive construction of
formal proofs, and also the manipulation of functional programs 
consistently with their specifications. It runs as a computer program
on many architectures.

It is available with a variety of user interfaces. The present
document does not attempt to present a comprehensive view of all the
possibilities of \Coq, but rather to present in the most elementary
manner a tutorial on the basic specification language, called Gallina,
in which formal axiomatisations may be developed, and on the main
proof tools.  For more advanced information, the reader could refer to
the \Coq{} Reference Manual or the \textit{Coq'Art}, a book by Y.
Bertot and P. Castéran on practical uses of the \Coq{} system.

Instructions on installation procedures, as well as more comprehensive
documentation, may be found in the standard distribution of \Coq,
which may be obtained from \Coq{} web site
\url{https://coq.inria.fr/}\footnote{You can report any bug you find
while using \Coq{} at \url{https://coq.inria.fr/bugs}. Make sure to
always provide a way to reproduce it and to specify the exact version
you used. You can get this information by running \texttt{coqc -v}}.
\Coq{} is distributed together with a graphical user interface called
CoqIDE. Alternative interfaces exist such as
Proof General\footnote{See \url{https://proofgeneral.github.io/}.}.

In the following examples, lines preceded by the prompt \verb:Coq < :
represent user input, terminated by a period.
The following lines usually show \Coq's answer.
When used from a graphical user interface such as
CoqIDE, the prompt is not displayed: user input is given in one window
and \Coq's answers are displayed in a different window.

\chapter{Basic Predicate Calculus}

\section{An overview of the specification language Gallina}

A formal development in Gallina consists in a sequence of {\sl declarations}
and {\sl definitions}.

\subsection{Declarations}

A declaration associates a {\sl name} with a {\sl specification}.
A name corresponds roughly to an identifier in a programming
language, i.e. to a string of letters, digits, and a few ASCII symbols like
underscore (\verb"_") and prime (\verb"'"), starting with a letter. 
We use case distinction, so that the names \verb"A" and \verb"a" are distinct.
Certain strings are reserved as key-words of \Coq, and thus are forbidden 
as user identifiers.

A specification is a formal expression which classifies the notion which is
being declared. There are basically three kinds of specifications: 
{\sl logical propositions}, {\sl mathematical collections}, and
{\sl abstract types}. They are classified by the three basic sorts
of the system, called respectively \verb:Prop:, \verb:Set:, and
\verb:Type:, which are themselves atomic abstract types.

Every valid expression $e$ in Gallina is associated with a specification,
itself a valid expression, called its {\sl type} $\tau(E)$. We write
$e:\tau(E)$ for the judgment that $e$ is of type $E$. 
You may request \Coq{} to return to you the type of a valid expression by using
the command \verb:Check::

\begin{coq_eval}
Set Printing Width 60.
\end{coq_eval}

\begin{coq_example}
Check O.
\end{coq_example}

Thus we know that the identifier \verb:O: (the name `O', not to be
confused with the numeral `0' which is not a proper identifier!) is
known in the current context, and that its type is the specification 
\verb:nat:. This specification is itself classified as a mathematical
collection, as we may readily check:

\begin{coq_example}
Check nat.
\end{coq_example}

The specification \verb:Set: is an abstract type, one of the basic
sorts of the Gallina language, whereas the notions $nat$ and $O$ are
notions which are defined in the arithmetic prelude,
automatically loaded when running the \Coq{} system.

We start by introducing a so-called section name. The role of sections
is to structure the modelisation by limiting the scope of parameters,
hypotheses and definitions. It will also give a convenient way to
reset part of the development.

\begin{coq_example}
Section Declaration.
\end{coq_example}
With what we already know, we may now enter in the system a declaration,
corresponding to the informal mathematics {\sl let n be a natural
  number}. 

\begin{coq_example}
Variable n : nat.
\end{coq_example}

If we want to translate a more precise statement, such as
{\sl let n be a positive natural number},
we have to add another declaration, which will declare explicitly the
hypothesis \verb:Pos_n:, with specification the proper logical
proposition:
\begin{coq_example}
Hypothesis Pos_n : (gt n 0).
\end{coq_example}

Indeed we may check that the relation \verb:gt: is known with the right type
in the current context:

\begin{coq_example}
Check gt.
\end{coq_example}

which tells us that \texttt{gt} is a function expecting two arguments of
type \texttt{nat} in order to build a logical proposition.
What happens here is similar to what we are used to in a functional
programming language: we may compose the (specification) type \texttt{nat}
with the (abstract) type \texttt{Prop} of logical propositions through the
arrow function constructor, in order to get a functional type
\texttt{nat -> Prop}:
\begin{coq_example}
Check (nat -> Prop).
\end{coq_example}
which may be composed once more with \verb:nat: in order to obtain the
type \texttt{nat -> nat -> Prop} of binary relations over natural numbers.
Actually the type \texttt{nat -> nat -> Prop} is an abbreviation for 
\texttt{nat -> (nat -> Prop)}.

Functional notions may be composed in the usual way. An expression $f$
of type $A\ra B$ may be applied to an expression $e$ of type $A$ in order
to form the expression $(f~e)$ of type $B$. Here we get that
the expression \verb:(gt n): is well-formed of type \texttt{nat -> Prop},
and thus that the expression \verb:(gt n O):, which abbreviates
\verb:((gt n) O):, is a well-formed proposition.
\begin{coq_example}
Check gt n O.
\end{coq_example}

\subsection{Definitions}

The initial prelude contains a few arithmetic definitions:
\texttt{nat} is defined as a mathematical collection (type \texttt{Set}),
constants \texttt{O}, \texttt{S}, \texttt{plus}, are defined as objects of
types respectively \texttt{nat}, \texttt{nat -> nat}, and \texttt{nat ->
nat -> nat}.
You may introduce new definitions, which link a name to a well-typed value.
For instance, we may introduce the constant \texttt{one} as being defined
to be equal to the successor of zero:
\begin{coq_example}
Definition one := (S O).
\end{coq_example}
We may optionally indicate the required type:
\begin{coq_example}
Definition two : nat := S one.
\end{coq_example}

Actually \Coq{} allows several possible syntaxes:
\begin{coq_example}
Definition three := S two : nat.
\end{coq_example}

Here is a way to define the doubling function, which expects an
argument \verb:m: of type \verb:nat: in order to build its result as
\verb:(plus m m)::

\begin{coq_example}
Definition double (m : nat) := plus m m.
\end{coq_example}
This introduces the constant \texttt{double} defined as the
expression \texttt{fun m : nat => plus m m}.
The abstraction introduced by \texttt{fun} is explained as follows.
The expression \texttt{fun x : A => e} is well formed of type
\texttt{A -> B} in a context whenever the expression \texttt{e} is
well-formed of type \texttt{B} in the given context to which we add the
declaration that \texttt{x} is of type \texttt{A}. Here \texttt{x} is a
bound, or dummy variable in the expression \texttt{fun x : A => e}.
For instance we could as well have defined \texttt{double} as
\texttt{fun n : nat => (plus n n)}.

Bound (local) variables and free (global) variables may be mixed.
For instance, we may define the function which adds the constant \verb:n:
to its argument as
\begin{coq_example}
Definition add_n (m:nat) := plus m n.
\end{coq_example}
However, note that here we may not rename the formal argument $m$ into $n$
without capturing the free occurrence of $n$, and thus changing the meaning
of the defined notion.

Binding operations are well known for instance in logic, where they
are called quantifiers.  Thus we may universally quantify a
proposition such as $m>0$ in order to get a universal proposition
$\forall m\cdot m>0$. Indeed this operator is available in \Coq, with
the following syntax: \texttt{forall m : nat, gt m O}. Similarly to the
case of the functional abstraction binding, we are obliged to declare
explicitly the type of the quantified variable. We check:
\begin{coq_example}
Check (forall m : nat, gt m 0).
\end{coq_example}

\begin{coq_eval}
Reset Initial.
Set Printing Width 60.
Set Printing Compact Contexts.
\end{coq_eval}

\section{Introduction to the proof engine: Minimal Logic}

In the following, we are going to consider various propositions, built
from atomic propositions $A, B, C$. This may be done easily, by
introducing these atoms as global variables declared of type \verb:Prop:.
It is easy to declare several names with the same specification:
\begin{coq_example}
Section Minimal_Logic.
Variables A B C : Prop.
\end{coq_example}

We shall consider simple implications, such as $A\ra B$, read as 
``$A$ implies $B$''. Remark that we overload the arrow symbol, which
has been used above as the functionality type constructor, and which
may be used as well as propositional connective:
\begin{coq_example}
Check (A -> B).
\end{coq_example}

Let us now embark on a simple proof. We want to prove the easy tautology
$((A\ra (B\ra C))\ra (A\ra B)\ra (A\ra C)$. 
We enter the proof engine by the command
\verb:Goal:, followed by the conjecture we want to verify:
\begin{coq_example}
Goal (A -> B -> C) -> (A -> B) -> A -> C.
\end{coq_example}

The system displays the current goal below a double line, local hypotheses
(there are none initially) being displayed above the line. We call 
the combination of local hypotheses with a goal a {\sl judgment}.
We are now in an inner 
loop of the system, in proof mode. 
New commands are available in this
mode, such as {\sl tactics}, which are proof combining primitives.
A tactic operates on the current goal by attempting to construct a proof
of the corresponding judgment, possibly from proofs of some
hypothetical judgments, which are then added to the current
list of conjectured judgments.
For instance, the \verb:intro: tactic is applicable to any judgment
whose goal is an implication, by moving the proposition to the left
of the application to the list of local hypotheses:
\begin{coq_example}
intro H.
\end{coq_example}

Several introductions may be done in one step:
\begin{coq_example}
intros H' HA.
\end{coq_example}

We notice that $C$, the current goal, may be obtained from hypothesis
\verb:H:, provided the truth of $A$ and $B$ are established.
The tactic \verb:apply: implements this piece of reasoning:
\begin{coq_example}
apply H.
\end{coq_example}

We are now in the situation where we have two judgments as conjectures
that remain to be proved. Only the first is listed in full, for the
others the system displays only the corresponding subgoal, without its
local hypotheses list. Remark that \verb:apply: has kept the local
hypotheses of its father judgment, which are still available for
the judgments it generated.

In order to solve the current goal, we just have to notice that it is
exactly available as hypothesis $HA$:
\begin{coq_example}
exact HA.
\end{coq_example}

Now $H'$ applies:
\begin{coq_example}
apply H'.
\end{coq_example}

And we may now conclude the proof as before, with \verb:exact HA.:
Actually, we may not bother with the name \verb:HA:, and just state that
the current goal is solvable from the current local assumptions:
\begin{coq_example}
assumption.
\end{coq_example}

The proof is now finished. We are now going to ask \Coq{}'s kernel
to check and save the proof.
\begin{coq_example}
Qed.
\end{coq_example}

Let us redo the same proof with a few variations. First of all we may name
the initial goal as a conjectured lemma:
\begin{coq_example}
Lemma distr_impl : (A -> B -> C) -> (A -> B) -> A -> C.
\end{coq_example}

Next, we may omit the names of local assumptions created by the introduction
tactics, they can be automatically created by the proof engine as new
non-clashing names.
\begin{coq_example}
intros.
\end{coq_example}

The \verb:intros: tactic, with no arguments, effects as many individual
applications of \verb:intro: as is legal.

Then, we may compose several tactics together in sequence, or in parallel,
through {\sl tacticals}, that is tactic combinators. The main constructions
are the following:
\begin{itemize}
\item $T_1 ; T_2$ (read $T_1$ then $T_2$) applies tactic $T_1$ to the current
goal, and then tactic $T_2$ to all the subgoals generated by $T_1$.
\item $T; [T_1 | T_2 | ... | T_n]$ applies tactic $T$ to the current
goal, and then tactic $T_1$ to the first newly generated subgoal, 
..., $T_n$ to the nth.
\end{itemize}

We may thus complete the proof of \verb:distr_impl: with one composite tactic:
\begin{coq_example}
apply H; [ assumption | apply H0; assumption ].
\end{coq_example}

You should be aware however that relying on automatically generated names is
not robust to slight updates to this proof script. Consequently, it is
discouraged in finished proof scripts. As for the composition of tactics with
\texttt{:} it may hinder the readability of the proof script and it is also
harder to see what's going on when replaying the proof because composed
tactics are evaluated in one go.

Actually, such an easy combination of tactics \verb:intro:, \verb:apply:
and \verb:assumption: may be found completely automatically by an automatic
tactic, called \verb:auto:, without user guidance:

\begin{coq_eval}
Abort.
\end{coq_eval}
\begin{coq_example}
Lemma distr_impl : (A -> B -> C) -> (A -> B) -> A -> C.
auto.
\end{coq_example}

Let us now save lemma \verb:distr_impl::
\begin{coq_example}
Qed.
\end{coq_example}

\section{Propositional Calculus}

\subsection{Conjunction}

We have seen how \verb:intro: and \verb:apply: tactics could be combined
in order to prove implicational statements. More generally, \Coq{} favors a style
of reasoning, called {\sl Natural Deduction}, which decomposes reasoning into 
so called {\sl introduction rules}, which tell how to prove a goal whose main 
operator is a given propositional connective, and {\sl elimination rules},
which tell how to use an hypothesis whose main operator is the propositional 
connective. Let us show how to use these ideas for the propositional connectives
\verb:/\: and \verb:\/:.

\begin{coq_example}
Lemma and_commutative : A /\ B -> B /\ A.
intro H.
\end{coq_example}

We make use of the conjunctive hypothesis \verb:H: with the \verb:elim: tactic,
which breaks it into its components:
\begin{coq_example}
elim H.
\end{coq_example}

We now use the conjunction introduction tactic \verb:split:, which splits the 
conjunctive goal into the two subgoals:
\begin{coq_example}
split.
\end{coq_example}
and the proof is now trivial. Indeed, the whole proof is obtainable as follows:
\begin{coq_eval}
Abort.
\end{coq_eval}
\begin{coq_example}
Lemma and_commutative : A /\ B -> B /\ A.
intro H; elim H; auto.
Qed.
\end{coq_example}

The tactic \verb:auto: succeeded here because it knows as a hint the 
conjunction introduction operator \verb+conj+
\begin{coq_example}
Check conj.
\end{coq_example}

Actually, the tactic \verb+split+ is just an abbreviation for \verb+apply conj.+

What we have just seen is that the \verb:auto: tactic is more powerful than
just a simple application of local hypotheses; it tries to apply as well 
lemmas which have been specified as hints. A 
\verb:Hint Resolve: command registers a
lemma as a hint to be used from now on by the \verb:auto: tactic, whose power 
may thus be incrementally augmented.

\subsection{Disjunction}

In a similar fashion, let us consider disjunction:

\begin{coq_example}
Lemma or_commutative : A \/ B -> B \/ A.
intro H; elim H.
\end{coq_example}

Let us prove the first subgoal in detail. We use \verb:intro: in order to
be left to prove \verb:B\/A: from \verb:A::

\begin{coq_example}
intro HA.
\end{coq_example}

Here the hypothesis \verb:H: is not needed anymore. We could choose to
actually erase it with the tactic \verb:clear:; in this simple proof it
does not really matter, but in bigger proof developments it is useful to
clear away unnecessary hypotheses which may clutter your screen.
\begin{coq_example}
clear H.
\end{coq_example}

The tactic \verb:destruct: combines the effects of \verb:elim:, \verb:intros:,
and \verb:clear::

\begin{coq_eval}
Abort.
\end{coq_eval}
\begin{coq_example}
Lemma or_commutative : A \/ B -> B \/ A.
intros H; destruct H.
\end{coq_example}

The disjunction connective has two introduction rules, since \verb:P\/Q:
may be obtained from \verb:P: or from \verb:Q:; the two corresponding
proof constructors are called respectively \verb:or_introl: and
\verb:or_intror:; they are applied to the current goal by tactics
\verb:left: and \verb:right: respectively. For instance:
\begin{coq_example}
right.
trivial.
\end{coq_example}
The tactic \verb:trivial: works like \verb:auto: with the hints
database, but it only tries those tactics that can solve the goal in one
step. 

As before, all these tedious elementary steps may be performed automatically,
as shown for the second symmetric case:

\begin{coq_example}
auto.
\end{coq_example}

However, \verb:auto: alone does not succeed in proving the full lemma, because
it does not try any elimination step.
It is a bit disappointing that \verb:auto: is not able to prove automatically 
such a simple tautology. The reason is that we want to keep
\verb:auto: efficient, so that it is always effective to use. 

\subsection{Tauto}

A complete tactic for propositional
tautologies is indeed available in \Coq{} as the \verb:tauto: tactic.
\begin{coq_eval}
Abort.
\end{coq_eval}
\begin{coq_example}
Lemma or_commutative : A \/ B -> B \/ A.
tauto.
Qed.
\end{coq_example}

It is possible to inspect the actual proof tree constructed by \verb:tauto:,
using a standard command of the system, which prints the value of any notion 
currently defined in the context:
\begin{coq_example}
Print or_commutative.
\end{coq_example}

It is not easy to understand the notation for proof terms without some
explanations. The \texttt{fun} prefix, such as \verb+fun H : A\/B =>+, 
corresponds
to \verb:intro H:, whereas a subterm such as 
\verb:(or_intror: \verb:B H0):
corresponds to the sequence of tactics \verb:apply or_intror; exact H0:. 
The generic combinator \verb:or_intror: needs to be instantiated by
the two properties \verb:B: and \verb:A:. Because \verb:A: can be
deduced from the type of \verb:H0:, only  \verb:B: is printed.
The two instantiations are effected automatically by the tactic
\verb:apply: when pattern-matching a goal. The specialist will of course
recognize our proof term as a $\lambda$-term, used as notation for the
natural deduction proof term through the Curry-Howard isomorphism. The
naive user of \Coq{} may safely ignore these formal details.

Let us exercise the \verb:tauto: tactic on a more complex example:
\begin{coq_example}
Lemma distr_and : A -> B /\ C -> (A -> B) /\ (A -> C).
tauto.
Qed.
\end{coq_example}

\subsection{Classical reasoning}

The tactic \verb:tauto: always comes back with an answer. Here is an example where it 
fails:
\begin{coq_example}
Lemma Peirce : ((A -> B) -> A) -> A.
try tauto.
\end{coq_example}

Note the use of the \verb:try: tactical, which does nothing if its tactic
argument fails.

This may come as a surprise to someone familiar with classical reasoning. 
Peirce's lemma is true in Boolean logic, i.e. it evaluates to \verb:true: for 
every truth-assignment to \verb:A: and \verb:B:. Indeed the double negation
of Peirce's law may be proved in \Coq{} using \verb:tauto::
\begin{coq_eval}
Abort.
\end{coq_eval}
\begin{coq_example}
Lemma NNPeirce : ~ ~ (((A -> B) -> A) -> A).
tauto.
Qed.
\end{coq_example}

In classical logic, the double negation of a proposition is equivalent to this 
proposition, but in the constructive logic of \Coq{} this is not so. If you
want to use classical logic in \Coq, you have to import explicitly the
\verb:Classical: module, which will declare the axiom \verb:classic:
of excluded middle, and classical tautologies such as de Morgan's laws.
The \verb:Require: command is used to import a module from \Coq's library:
\begin{coq_example}
Require Import Classical.
Check NNPP.
\end{coq_example}

and it is now easy (although admittedly not the most direct way) to prove
a classical law such as Peirce's:
\begin{coq_example}
Lemma Peirce : ((A -> B) -> A) -> A.
apply NNPP; tauto.
Qed.
\end{coq_example}

Here is one more example of propositional reasoning, in the shape of
a Scottish puzzle. A private club has the following rules:
\begin{enumerate}
\item Every non-scottish member wears red socks
\item Every member wears a kilt or doesn't wear red socks
\item The married members don't go out on Sunday
\item A member goes out on Sunday if and only if he is Scottish
\item Every member who wears a kilt is Scottish and married
\item Every scottish member wears a kilt
\end{enumerate}
Now, we show that these rules are so strict that no one can be accepted.
\begin{coq_example}
Section club.
Variables Scottish RedSocks WearKilt Married GoOutSunday : Prop.
Hypothesis rule1 : ~ Scottish -> RedSocks.
Hypothesis rule2 : WearKilt \/ ~ RedSocks.
Hypothesis rule3 : Married -> ~ GoOutSunday.
Hypothesis rule4 : GoOutSunday <-> Scottish.
Hypothesis rule5 : WearKilt -> Scottish /\ Married.
Hypothesis rule6 : Scottish -> WearKilt.
Lemma NoMember : False.
tauto.
Qed.
\end{coq_example}
At that point \verb:NoMember: is a proof of the absurdity depending on
hypotheses.
We may end the section, in that case, the variables and hypotheses
will be discharged, and the type of \verb:NoMember: will be
generalised.

\begin{coq_example}
End club.
Check NoMember.
\end{coq_example}

\section{Predicate Calculus}

Let us now move into predicate logic, and first of all into first-order
predicate calculus. The essence of predicate calculus is that to try to prove 
theorems in the most abstract possible way, without using the definitions of 
the mathematical notions, but by formal manipulations of uninterpreted 
function and predicate symbols. 

\subsection{Sections and signatures}

Usually one works in some domain of discourse, over which range the individual 
variables and function symbols. In \Coq{}, we speak in a language with a rich
variety of types, so we may mix several domains of discourse, in our 
multi-sorted language. For the moment, we just do a few exercises, over a 
domain of discourse \verb:D: axiomatised as a \verb:Set:, and we consider two 
predicate symbols  \verb:P: and \verb:R: over \verb:D:, of arities 
1 and 2, respectively.

\begin{coq_eval}
Reset Initial.
Set Printing Width 60.
Set Printing Compact Contexts.
\end{coq_eval}

We start by assuming a domain of
discourse \verb:D:, and a binary relation \verb:R:  over \verb:D:: 
\begin{coq_example}
Section Predicate_calculus.
Variable D : Set.
Variable R : D -> D -> Prop.
\end{coq_example}

As a simple example of predicate calculus reasoning, let us assume
that relation \verb:R: is symmetric and transitive, and let us show that
\verb:R: is reflexive in any point \verb:x: which has an \verb:R: successor.
Since we do not want to make the assumptions about \verb:R: global axioms of 
a theory, but rather local hypotheses to a theorem, we open a specific
section to this effect.
\begin{coq_example}
Section R_sym_trans.
Hypothesis R_symmetric : forall x y : D, R x y -> R y x.
Hypothesis R_transitive :
    forall x y z : D, R x y -> R y z -> R x z.
\end{coq_example}

Remark the syntax \verb+forall x : D,+ which stands for universal quantification
$\forall x : D$.

\subsection{Existential quantification}

We now state our lemma, and enter proof mode.
\begin{coq_example}
Lemma refl_if : forall x : D, (exists y, R x y) -> R x x.
\end{coq_example}

Remark that the hypotheses which are local to the currently opened sections
are listed as local hypotheses to the current goals.
The rationale is that these hypotheses are going to be discharged, as we
shall see, when we shall close the corresponding sections.

Note the functional syntax for existential quantification. The existential
quantifier is built from the operator \verb:ex:, which expects a 
predicate as argument:
\begin{coq_example}
Check ex.
\end{coq_example}
and the notation \verb+(exists x : D, P x)+ is just concrete syntax for 
the expression \verb+(ex D (fun x : D => P x))+. 
Existential quantification is handled in \Coq{} in a similar
fashion to the connectives \verb:/\: and \verb:\/:: it is introduced by
the proof combinator \verb:ex_intro:, which is invoked by the specific 
tactic \verb:exists:, and its elimination provides a witness \verb+a : D+ to
\verb:P:, together with an assumption \verb+h : (P a)+ that indeed \verb+a+
verifies \verb:P:. Let us see how this works on this simple example.
\begin{coq_example}
intros x x_Rlinked.
\end{coq_example}

Remark that \verb:intros: treats universal quantification in the same way
as the premises of implications. Renaming of bound variables occurs
when it is needed; for instance, had we started with \verb:intro y:,
we would have obtained the goal:
\begin{coq_eval}
Undo.
\end{coq_eval}
\begin{coq_example}
intro y.
\end{coq_example}
\begin{coq_eval}
Undo.
intros x x_Rlinked.
\end{coq_eval}

Let us now use the existential hypothesis \verb:x_Rlinked: to 
exhibit an R-successor y of x. This is done in two steps, first with
\verb:elim:, then with \verb:intros:

\begin{coq_example}
elim x_Rlinked.
intros y Rxy.
\end{coq_example}

Now we want to use \verb:R_transitive:. The \verb:apply: tactic will know
how to match \verb:x: with \verb:x:, and \verb:z: with \verb:x:, but needs
help on how to instantiate \verb:y:, which appear in the hypotheses of
\verb:R_transitive:, but not in its conclusion. We give the proper hint
to \verb:apply: in a \verb:with: clause, as follows:
\begin{coq_example}
apply R_transitive with y.
\end{coq_example}

The rest of the proof is routine:
\begin{coq_example}
assumption.
apply R_symmetric; assumption.
\end{coq_example}
\begin{coq_example*}
Qed.
\end{coq_example*}

Let us now close the current section.
\begin{coq_example}
End R_sym_trans.
\end{coq_example}

All the local hypotheses have been
discharged in the statement of \verb:refl_if:, which now becomes a general
theorem in the first-order language declared in section 
\verb:Predicate_calculus:. In this particular example, section
\verb:R_sym_trans: has not been really useful, since we could have
instead stated theorem \verb:refl_if: in its general form, and done 
basically the same proof, obtaining \verb:R_symmetric: and
\verb:R_transitive: as local hypotheses by initial \verb:intros: rather
than as global hypotheses in the context. But if we had pursued the
theory by proving more theorems about relation \verb:R:,
we would have obtained all general statements at the closing of the section,
with minimal dependencies on the hypotheses of symmetry and transitivity.

\subsection{Paradoxes of classical predicate calculus}

Let us illustrate this feature by pursuing our \verb:Predicate_calculus:
section with an enrichment of our language: we declare a unary predicate
\verb:P: and a constant \verb:d::
\begin{coq_example}
Variable P :  D -> Prop.
Variable d : D.
\end{coq_example}

We shall now prove a well-known fact from first-order logic: a universal 
predicate is non-empty, or in other terms existential quantification 
follows from universal quantification.
\begin{coq_example}
Lemma weird : (forall x:D, P x) ->  exists a, P a.
 intro UnivP.
\end{coq_example}

First of all, notice the pair of parentheses around
\verb+forall x : D, P x+ in
the statement of lemma \verb:weird:.
If we had omitted them, \Coq's parser would have interpreted the
statement as a truly trivial fact, since we would 
postulate an \verb:x: verifying \verb:(P x):. Here the situation is indeed
more problematic. If we have some element in \verb:Set: \verb:D:, we may
apply \verb:UnivP: to it and conclude, otherwise we are stuck. Indeed
such an element \verb:d: exists, but this is just by virtue of our
new signature. This points out a subtle difference between standard
predicate calculus and \Coq. In standard first-order logic,
the equivalent of lemma \verb:weird: always holds, 
because such a rule is wired in the inference rules for quantifiers, the
semantic justification being that the interpretation domain is assumed to
be non-empty. Whereas in \Coq, where types are not assumed to be 
systematically inhabited, lemma \verb:weird: only holds in signatures
which allow the explicit construction of an element in the domain of
the predicate. 

Let us conclude the proof, in order to show the use of the \verb:exists:
tactic:
\begin{coq_example}
exists d; trivial.
Qed.
\end{coq_example}

Another fact which illustrates the sometimes disconcerting rules of
classical 
predicate calculus is Smullyan's drinkers' paradox: ``In any non-empty
bar, there is a person such that if she drinks, then everyone drinks''.
We modelize the bar by Set \verb:D:, drinking by predicate \verb:P:.
We shall need classical reasoning. Instead of loading the \verb:Classical:
module as we did above, we just state the law of excluded middle as a
local hypothesis schema at this point:
\begin{coq_example}
Hypothesis EM : forall A : Prop, A \/ ~ A.
Lemma drinker :  exists x : D, P x -> forall x : D, P x.
\end{coq_example}
The proof goes by cases on whether or not
there is someone who does not drink. Such reasoning by cases proceeds
by invoking the excluded middle principle, via \verb:elim: of the
proper instance of \verb:EM::
\begin{coq_example}
elim (EM (exists x, ~ P x)).
\end{coq_example}

We first look at the first case. Let Tom be the non-drinker.
The following combines at once the effect of \verb:intros: and
\verb:destruct::
\begin{coq_example}
intros (Tom, Tom_does_not_drink).
\end{coq_example}

We conclude in that case by considering Tom, since his drinking leads to
a contradiction:
\begin{coq_example}
exists Tom; intro Tom_drinks.
\end{coq_example}

There are several ways in which we may eliminate a contradictory case;
in this case, we use \verb:contradiction: to let \Coq{} find out the
two contradictory hypotheses:
\begin{coq_example}
contradiction.
\end{coq_example}

We now proceed with the second case, in which actually any person will do;
such a John Doe is given by the non-emptiness witness \verb:d::
\begin{coq_example}
intro No_nondrinker; exists d; intro d_drinks.
\end{coq_example}

Now we consider any Dick in the bar, and reason by cases according to its
drinking or not:
\begin{coq_example}
intro Dick; elim (EM (P Dick)); trivial.
\end{coq_example}

The only non-trivial case is again treated by contradiction:
\begin{coq_example}
intro Dick_does_not_drink; absurd (exists x, ~ P x); trivial.
exists Dick; trivial.
Qed.
\end{coq_example}

Now, let us close the main section and look at the complete statements
we proved:
\begin{coq_example}
End Predicate_calculus.
Check refl_if.
Check weird.
Check drinker.
\end{coq_example}

Remark how the three theorems are completely generic in the most general 
fashion;
the domain \verb:D: is discharged in all of them, \verb:R: is discharged in
\verb:refl_if: only, \verb:P: is discharged only in \verb:weird: and
\verb:drinker:, along with the hypothesis that \verb:D: is inhabited. 
Finally, the excluded middle hypothesis is discharged only in 
\verb:drinker:.

Note that the name \verb:d: has vanished as well from
the statements of \verb:weird: and \verb:drinker:, 
since \Coq's pretty-printer replaces
systematically a quantification such as \texttt{forall d : D, E},
where \texttt{d} does not occur in \texttt{E},
by the functional notation \texttt{D -> E}.
Similarly the name \texttt{EM} does not appear in \texttt{drinker}.

Actually, universal quantification, implication, 
as well as function formation, are
all special cases of one general construct of type theory called
{\sl dependent product}. This is the mathematical construction 
corresponding to an indexed family of functions. A function 
$f\in \Pi x:D\cdot Cx$ maps an element $x$ of its domain $D$ to its
(indexed) codomain $Cx$. Thus a proof of $\forall x:D\cdot Px$ is
a function mapping an element $x$ of $D$ to a proof of proposition $Px$.


\subsection{Flexible use of local assumptions}

Very often during the course of a proof we want to retrieve a local
assumption and reintroduce it explicitly in the goal, for instance
in order to get a more general induction hypothesis. The tactic
\verb:generalize: is what is needed here:

\begin{coq_example}
Section Predicate_Calculus.
Variables P Q : nat -> Prop.
Variable R :  nat -> nat -> Prop.
Lemma PQR :
 forall x y:nat, (R x x -> P x -> Q x) -> P x -> R x y -> Q x.
intros.
generalize H0.
\end{coq_example}

Sometimes it may be convenient to state an intermediate fact.
The tactic \verb:assert: does this and introduces a new subgoal
for this fact to be proved first. The tactic \verb:enough: does
the same while keeping this goal for later.
\begin{coq_example}
enough (R x x) by auto.
\end{coq_example}
We clean the goal by doing an \verb:Abort: command.
\begin{coq_example*}
Abort.
\end{coq_example*}


\subsection{Equality}

The basic equality provided in \Coq{} is Leibniz equality, noted infix like
\texttt{x = y}, when \texttt{x} and \texttt{y} are two expressions of
type the same Set. The replacement of \texttt{x} by \texttt{y} in any
term is effected by a variety of tactics, such as \texttt{rewrite}
and \texttt{replace}.

Let us give a few examples of equality replacement. Let us assume that
some arithmetic function \verb:f: is null in zero:
\begin{coq_example}
Variable f : nat -> nat.
Hypothesis foo : f 0 = 0.
\end{coq_example}

We want to prove the following conditional equality:
\begin{coq_example*}
Lemma L1 : forall k:nat, k = 0 -> f k = k.
\end{coq_example*}

As usual, we first get rid of local assumptions with \verb:intro::
\begin{coq_example}
intros k E.
\end{coq_example}

Let us now use equation \verb:E: as a left-to-right rewriting:
\begin{coq_example}
rewrite E.
\end{coq_example}
This replaced both occurrences of \verb:k: by \verb:O:. 

Now \verb:apply foo: will finish the proof:

\begin{coq_example}
apply foo.
Qed.
\end{coq_example}

When one wants to rewrite an equality in a right to left fashion, we should
use \verb:rewrite <- E: rather than \verb:rewrite E: or the equivalent
\verb:rewrite -> E:. 
Let us now illustrate the tactic \verb:replace:.
\begin{coq_example}
Hypothesis f10 : f 1 = f 0.
Lemma L2 : f (f 1) = 0.
replace (f 1) with 0.
\end{coq_example}
What happened here is that the replacement left the first subgoal to be
proved, but another proof obligation was generated by the \verb:replace:
tactic, as the second subgoal. The first subgoal is solved immediately
by applying lemma \verb:foo:; the second one transitivity and then 
symmetry of equality, for instance with tactics \verb:transitivity: and 
\verb:symmetry::
\begin{coq_example}
apply foo.
transitivity (f 0); symmetry; trivial.
\end{coq_example}
In case the equality $t=u$ generated by \verb:replace: $u$ \verb:with:
$t$ is an assumption
(possibly modulo symmetry), it will be automatically proved and the
corresponding goal will not appear. For instance:

\begin{coq_eval}
Restart.
\end{coq_eval}
\begin{coq_example}
Lemma L2 : f (f 1) = 0.
replace (f 1) with (f 0).
replace (f 0) with 0; trivial.
Qed.
\end{coq_example}

\section{Using definitions}

The development of mathematics does not simply proceed by logical 
argumentation from first principles: definitions are used in an essential way.
A formal development proceeds by a dual process of abstraction, where one
proves abstract statements in predicate calculus, and use of definitions, 
which in the contrary one instantiates general statements with particular 
notions in order to use the structure of mathematical values for the proof of
more specialised properties.

\subsection{Unfolding definitions}

Assume that we want to develop the theory of sets represented as characteristic
predicates over some universe \verb:U:. For instance:
\begin{coq_example}
Variable U : Type.
Definition set := U -> Prop.
Definition element (x : U) (S : set) := S x.
Definition subset (A B : set) := 
  forall x : U, element x A -> element x B.
\end{coq_example}

Now, assume that we have loaded a module of general properties about
relations over some abstract type \verb:T:, such as transitivity:

\begin{coq_example}
Definition transitive (T : Type) (R : T -> T -> Prop) :=
  forall x y z : T, R x y -> R y z -> R x z.
\end{coq_example}

We want to prove that \verb:subset: is a \verb:transitive:
relation. 
\begin{coq_example}
Lemma subset_transitive : transitive set subset.
\end{coq_example}

In order to make any progress, one needs to use the definition of
\verb:transitive:. The \verb:unfold: tactic, which replaces all
occurrences of a defined notion by its definition in the current goal,
may be used here.
\begin{coq_example}
unfold transitive.
\end{coq_example}

Now, we must unfold \verb:subset::
\begin{coq_example}
unfold subset.
\end{coq_example}
Now, unfolding \verb:element: would be a mistake, because indeed a simple proof
can be found by \verb:auto:, keeping \verb:element: an abstract predicate:
\begin{coq_example}
auto.
\end{coq_example}

Many variations on \verb:unfold: are provided in \Coq. For instance,
instead of unfolding all occurrences of \verb:subset:, we may want to
unfold only one designated occurrence:
\begin{coq_eval}
Undo 2.
\end{coq_eval}
\begin{coq_example}
unfold subset at 2.
\end{coq_example}

One may also unfold a definition in a given local hypothesis, using the
\verb:in: notation:
\begin{coq_example}
intros.
unfold subset in H.
\end{coq_example}

Finally, the tactic \verb:red: does only unfolding of the head occurrence
of the current goal:
\begin{coq_example}
red.
auto.
Qed.
\end{coq_example}


\subsection{Principle of proof irrelevance}

Even though in principle the proof term associated with a verified lemma
corresponds to a defined value of the corresponding specification, such
definitions cannot be unfolded in \Coq: a lemma is considered an {\sl opaque}
definition. This conforms to the mathematical tradition of {\sl proof
irrelevance}: the proof of a logical proposition does not matter, and the
mathematical justification of a logical development relies only on
{\sl provability} of the lemmas used in the formal proof. 

Conversely, ordinary mathematical definitions can be unfolded at will, they
are {\sl transparent}. 

\chapter{Induction}

\begin{coq_eval}
Reset Initial.
Set Printing Width 60.
Set Printing Compact Contexts.
\end{coq_eval}

\section{Data Types as Inductively Defined Mathematical Collections}

All the notions which were studied until now pertain to traditional
mathematical logic. Specifications of objects were abstract properties
used in reasoning more or less constructively; we are now entering
the realm of inductive types, which specify the existence of concrete
mathematical constructions.

\subsection{Booleans}

Let us start with the collection of booleans, as they are specified
in the \Coq's \verb:Prelude: module: 
\begin{coq_example}
Inductive bool :  Set := true | false.
\end{coq_example}

Such a declaration defines several objects at once. First, a new
\verb:Set: is declared, with name \verb:bool:. Then the {\sl constructors}
of this \verb:Set: are declared, called \verb:true: and \verb:false:.
Those are analogous to introduction rules of the new Set \verb:bool:.
Finally, a specific elimination rule for \verb:bool: is now available, which
permits to reason by cases on \verb:bool: values. Three instances are
indeed defined as new combinators in the global context: \verb:bool_ind:,
a proof combinator corresponding to reasoning by cases,
\verb:bool_rec:, an if-then-else programming construct,
and \verb:bool_rect:, a similar combinator at the level of types.
Indeed:
\begin{coq_example}
Check bool_ind.
Check bool_rec.
Check bool_rect.
\end{coq_example}

Let us for instance prove that every Boolean is true or false.
\begin{coq_example}
Lemma duality : forall b:bool, b = true \/ b = false.
intro b.
\end{coq_example}

We use the knowledge that \verb:b: is a \verb:bool: by calling tactic
\verb:elim:, which is this case will appeal to combinator \verb:bool_ind:
in order to split the proof according to the two cases:
\begin{coq_example}
elim b.
\end{coq_example}

It is easy to conclude in each case:
\begin{coq_example}
left; trivial.
right; trivial.
\end{coq_example}

Indeed, the whole proof can be done with the combination of the
 \verb:destruct:, which combines \verb:intro: and \verb:elim:,
with good old \verb:auto::
\begin{coq_eval}
Abort.
\end{coq_eval}
\begin{coq_example}
Lemma duality : forall b:bool, b = true \/ b = false.
destruct b; auto.
Qed.
\end{coq_example}

\subsection{Natural numbers}

Similarly to Booleans, natural numbers are defined in the \verb:Prelude:
module with constructors \verb:S: and \verb:O::
\begin{coq_example}
Inductive nat : Set :=
  | O : nat
  | S : nat -> nat.
\end{coq_example}

The elimination principles which are automatically generated are Peano's
induction principle, and a recursion operator:
\begin{coq_example}
Check nat_ind.
Check nat_rec.
\end{coq_example}

Let us start by showing how to program the standard primitive recursion
operator \verb:prim_rec: from the more general \verb:nat_rec::
\begin{coq_example}
Definition prim_rec := nat_rec (fun i : nat => nat).
\end{coq_example}

That is, instead of computing for natural \verb:i: an element of the indexed
\verb:Set: \verb:(P i):, \verb:prim_rec: computes uniformly an element of 
\verb:nat:. Let us check the type of \verb:prim_rec::
\begin{coq_example}
About prim_rec.
\end{coq_example}

Oops! Instead of the expected type \verb+nat->(nat->nat->nat)->nat->nat+ we
get an apparently more complicated expression.
In fact, the two types are convertible and one way of having the proper
type would be to do some computation before actually defining \verb:prim_rec:
as such:

\begin{coq_eval}
Reset Initial.
Set Printing Width 60.
Set Printing Compact Contexts.
\end{coq_eval}

\begin{coq_example}
Definition prim_rec :=
    Eval compute in nat_rec (fun i : nat => nat).
About prim_rec.
\end{coq_example}

Let us now show how to program addition with primitive recursion:
\begin{coq_example}
Definition addition (n m:nat) :=
    prim_rec m (fun p rec : nat => S rec) n.
\end{coq_example}

That is, we specify that \verb+(addition n m)+ computes by cases on \verb:n:
according to its main constructor; when \verb:n = O:, we get \verb:m:;
 when \verb:n = S p:, we get \verb:(S rec):, where \verb:rec: is the result
of the recursive computation \verb+(addition p m)+. Let us verify it by
asking \Coq{} to compute for us say $2+3$:
\begin{coq_example}
Eval compute in (addition (S (S O)) (S (S (S O)))).
\end{coq_example}

Actually, we do not have to do all explicitly. {\Coq} provides a
special syntax {\tt Fixpoint/match} for generic primitive recursion,
and we could thus have defined directly addition as:

\begin{coq_example}
Fixpoint plus (n m:nat) {struct n} : nat :=
  match n with
  | O => m
  | S p => S (plus p m)
  end.
\end{coq_example}

\begin{coq_eval}
\begin{coq_example}
Reset Initial.
Set Printing Width 60.
Set Printing Compact Contexts.
\end{coq_eval}

\subsection{Simple proofs by induction}

Let us now show how to do proofs by structural induction. We start with easy
properties of the \verb:plus: function we just defined. Let us first
show that $n=n+0$.
\begin{coq_example}
Lemma plus_n_O : forall n : nat, n = n + 0.
intro n; elim n.
\end{coq_example}

What happened was that \texttt{elim n}, in order to construct a \texttt{Prop}
(the initial goal) from a \texttt{nat} (i.e. \texttt{n}), appealed to the
corresponding induction principle \texttt{nat\_ind} which we saw was indeed
exactly Peano's induction scheme. Pattern-matching instantiated the 
corresponding predicate \texttt{P} to \texttt{fun n : nat => n = n + 0},
and we get as subgoals the corresponding instantiations of the base case
\texttt{(P O)}, and of the inductive step
\texttt{forall y : nat, P y -> P (S y)}.
In each case we get an instance of function \texttt{plus} in which its second
argument starts with a constructor, and is thus amenable to simplification
by primitive recursion. The \Coq{} tactic \texttt{simpl} can be used for
this purpose:
\begin{coq_example}
simpl.
auto.
\end{coq_example}

We proceed in the same way for the base step:
\begin{coq_example}
simpl; auto.
Qed.
\end{coq_example}

Here \verb:auto: succeeded, because it used as a hint lemma \verb:eq_S:,
which say that successor preserves equality:
\begin{coq_example}
Check eq_S.
\end{coq_example}

Actually, let us see how to declare our lemma \verb:plus_n_O: as a hint
to be used by \verb:auto::
\begin{coq_example}
Hint Resolve plus_n_O .
\end{coq_example}

We now proceed to the similar property concerning the other constructor
\verb:S::
\begin{coq_example}
Lemma plus_n_S : forall n m:nat, S (n + m) = n + S m.
\end{coq_example}

We now go faster, using the tactic \verb:induction:, which does the
necessary \verb:intros: before applying \verb:elim:. Factoring simplification
and automation in both cases thanks to tactic composition, we prove this
lemma in one line:
\begin{coq_example}
induction n; simpl; auto.
Qed.
Hint Resolve plus_n_S .
\end{coq_example}

Let us end this exercise with the commutativity of \verb:plus::

\begin{coq_example}
Lemma plus_com : forall n m:nat, n + m = m + n.
\end{coq_example}

Here we have a choice on doing an induction on \verb:n: or on \verb:m:, the
situation being symmetric. For instance:
\begin{coq_example}
induction m as [ | m IHm ]; simpl; auto.
\end{coq_example}

Here \verb:auto: succeeded on the base case, thanks to our hint
\verb:plus_n_O:, but the induction step requires rewriting, which
\verb:auto: does not handle:

\begin{coq_example}
rewrite <- IHm; auto.
Qed.
\end{coq_example}

\subsection{Discriminate}

It is also possible to define new propositions by primitive recursion.
Let us for instance define the predicate which discriminates between
the constructors \verb:O: and \verb:S:: it computes to \verb:False: 
when its argument is \verb:O:, and to \verb:True: when its argument is 
of the form \verb:(S n)::
\begin{coq_example}
Definition Is_S (n : nat) := match n with
                             | O => False
                             | S p => True
                             end.
\end{coq_example}

Now we may use the computational power of \verb:Is_S: to prove
trivially that \verb:(Is_S (S n))::
\begin{coq_example}
Lemma S_Is_S : forall n:nat, Is_S (S n).
simpl; trivial.
Qed.
\end{coq_example}

But we may also use it to transform a \verb:False: goal into 
\verb:(Is_S O):. Let us show a particularly important use of this feature;
we want to prove that \verb:O: and \verb:S: construct different values, one
of Peano's axioms:
\begin{coq_example}
Lemma no_confusion : forall n:nat, 0 <> S n.
\end{coq_example}

First of all, we replace negation by its definition, by reducing the
goal with tactic \verb:red:; then we get contradiction by successive
\verb:intros::
\begin{coq_example}
red; intros n H.
\end{coq_example}

Now we use our trick:
\begin{coq_example}
change (Is_S 0).
\end{coq_example}

Now we use equality in order to get a subgoal which computes out to 
\verb:True:, which finishes the proof:
\begin{coq_example}
rewrite H; trivial.
simpl; trivial.
\end{coq_example}

Actually, a specific tactic \verb:discriminate: is provided
to produce mechanically such proofs, without the need for the user to define
explicitly the relevant discrimination predicates:

\begin{coq_eval}
Abort.
\end{coq_eval}
\begin{coq_example}
Lemma no_confusion : forall n:nat, 0 <> S n.
intro n; discriminate.
Qed.
\end{coq_example}


\section{Logic programming}

In the same way as we defined standard data-types above, we
may define inductive families, and for instance inductive predicates.
Here is the definition of predicate $\le$ over type \verb:nat:, as
given in \Coq's \verb:Prelude: module:
\begin{coq_example*}
Inductive le (n : nat) : nat -> Prop :=
  | le_n : le n n
  | le_S : forall m : nat, le n m -> le n (S m).
\end{coq_example*}

This definition introduces a new predicate
\verb+le : nat -> nat -> Prop+,
and the two constructors \verb:le_n: and \verb:le_S:, which are the
defining clauses of \verb:le:. That is, we get not only the ``axioms''
\verb:le_n: and \verb:le_S:, but also the converse property, that 
\verb:(le n m): if and only if this statement can be obtained as a
consequence of these defining clauses; that is, \verb:le: is the
minimal predicate verifying clauses \verb:le_n: and \verb:le_S:. This is
insured, as in the case of inductive data types, by an elimination principle,
which here amounts to an induction principle \verb:le_ind:, stating this 
minimality property:
\begin{coq_example}
Check le.
Check le_ind.
\end{coq_example}

Let us show how proofs may be conducted with this principle.
First we show that $n\le m \Rightarrow n+1\le m+1$:
\begin{coq_example}
Lemma le_n_S : forall n m : nat, le n m -> le (S n) (S m).
intros n m n_le_m.
elim n_le_m.
\end{coq_example}

What happens here is similar to the behaviour of \verb:elim: on natural
numbers: it appeals to the relevant induction principle, here \verb:le_ind:,
which generates the two subgoals, which may then be solved easily
with the help of the defining clauses of \verb:le:.
\begin{coq_example}
apply le_n; trivial.
intros; apply le_S; trivial.
\end{coq_example}

Now we know that it is a good idea to give the defining clauses as hints,
so that the proof may proceed with a simple combination of 
\verb:induction: and \verb:auto:. \verb:Hint Constructors le:
is just an abbreviation for \verb:Hint Resolve le_n le_S:.
\begin{coq_eval}
Abort.
\end{coq_eval}
\begin{coq_example}
Hint Constructors le.
Lemma le_n_S : forall n m : nat, le n m -> le (S n) (S m).
\end{coq_example}

We have a slight problem however. We want to say ``Do an induction on
hypothesis \verb:(le n m):'', but we have no explicit name for it. What we
do in this case is to say ``Do an induction on the first unnamed hypothesis'',
as follows.
\begin{coq_example}
induction 1; auto.
Qed.
\end{coq_example}

Here is a more tricky problem. Assume we want to show that
$n\le 0 \Rightarrow n=0$. This reasoning ought to follow simply from the
fact that only the first defining clause of \verb:le: applies.
\begin{coq_example} 
Lemma tricky : forall n:nat, le n 0 -> n = 0.
\end{coq_example}

However, here trying something like \verb:induction 1: would lead
nowhere (try it and see what happens). 
An induction on \verb:n: would not be convenient either.
What we must do here is analyse the definition of \verb"le" in order
to match hypothesis \verb:(le n O): with the defining clauses, to find
that only \verb:le_n: applies, whence the result. 
This analysis may be performed by the ``inversion'' tactic
\verb:inversion_clear: as follows:
\begin{coq_example} 
intros n H; inversion_clear H.
trivial.
Qed.
\end{coq_example}

\chapter{Modules}

\begin{coq_eval}
Reset Initial.
Set Printing Width 60.
Set Printing Compact Contexts.
\end{coq_eval}

\section{Opening library modules}

When you start \Coq{} without further requirements in the command line,
you get a bare system with few libraries loaded.  As we saw, a standard
prelude module provides the standard logic connectives, and a few
arithmetic notions. If you want to load and open other modules from
the library, you have to use the \verb"Require" command, as we saw for
classical logic above. For instance, if you want more arithmetic
constructions, you should request:
\begin{coq_example*}
Require Import Arith.
\end{coq_example*}

Such a command looks for a (compiled) module file \verb:Arith.vo: in
the libraries registered by \Coq. Libraries inherit the structure of
the file system of the operating system and are registered with the
command \verb:Add LoadPath:. Physical directories are mapped to
logical directories. Especially the standard library of \Coq{} is
pre-registered as a library of name \verb=Coq=.  Modules have absolute
unique names denoting their place in \Coq{} libraries.  An absolute
name is a sequence of single identifiers separated by dots.  E.g. the
module \verb=Arith= has full name \verb=Coq.Arith.Arith= and because
it resides in eponym subdirectory \verb=Arith= of the standard
library, it can be as well required by the command

\begin{coq_example*}
Require Import Coq.Arith.Arith.
\end{coq_example*}

This may be useful to avoid ambiguities if somewhere, in another branch
of the libraries known by Coq, another module is also called
\verb=Arith=. Notice that by default, when a library is registered,
all its contents, and all the contents of its subdirectories recursively are
visible and accessible by a short (relative) name as \verb=Arith=.
Notice also that modules or definitions not explicitly registered in
a library are put in a default library called \verb=Top=.

The loading of a compiled file is quick, because the corresponding
development is not type-checked again. 

\section{Creating your own modules}

You may create your own module files, by writing {\Coq} commands in a file,
say \verb:my_module.v:. Such a module may be simply loaded in the current
context, with command \verb:Load my_module:. It may also be compiled,
in ``batch'' mode, using the UNIX command
\verb:coqc:. Compiling the module \verb:my_module.v: creates a 
file \verb:my_module.vo:{} that can be reloaded with command
\verb:Require: \verb:Import: \verb:my_module:. 

If a required module depends on other modules then the latters are
automatically required beforehand. However their contents is not
automatically visible.  If you want a module \verb=M= required in a
module \verb=N= to be automatically visible when \verb=N= is required,
you should use \verb:Require Export M: in your module \verb:N:.

\section{Managing the context}

It is often difficult to remember the names of all lemmas and
definitions available in the current context, especially if large
libraries have been loaded. A convenient \verb:Search: command
is available to lookup all known facts 
concerning a given predicate. For instance, if you want to know all the
known lemmas about both the successor and the less or equal relation, just ask:
\begin{coq_eval}
Reset Initial.
Set Printing Width 60.
Set Printing Compact Contexts.
\end{coq_eval}
\begin{coq_example}
Search S le.
\end{coq_example}
Another command \verb:SearchHead: displays only lemmas where the searched
predicate appears at the head position in the conclusion.
\begin{coq_example}
SearchHead le.
\end{coq_example}

The \verb:Search: commands also allows finding the theorems
containing a given pattern, where \verb:_: can be used in
place of an arbitrary term. As shown in this example, \Coq{}
provides usual infix notations for arithmetic operators.

\begin{coq_example}
Search (_ + _ = _).
\end{coq_example}

\section{Now you are on your own}

This tutorial is necessarily incomplete. If you wish to pursue serious
proving in \Coq, you should now get your hands on \Coq's Reference Manual,
which contains a complete description of all the tactics we saw, 
plus many more.
You also should look in the library of developed theories which is distributed
with \Coq, in order to acquaint yourself with various proof techniques.


\end{document}

