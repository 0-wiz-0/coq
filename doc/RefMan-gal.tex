\chapter{The {\sf Gallina} specification language}
\label{Gallina}\index{Gallina}

This chapter describes \gallina, the specification language of Coq.
It allows to develop mathematical theories and to prove specifications
of programs.  The theories are built from axioms, hypotheses,
parameters, lemmas, theorems and definitions of constants, functions,
predicates and sets. The syntax of logical objects involved in
theories is described in section \ref{term}. The language of
commands, called {\em The Vernacular} is described in section
\ref{Vernacular}.

In Coq, logical objects are typed to ensure their logical
correctness. The rules implemented by the typing algorithm are described in
chapter \ref{Cic}.

\subsection*{About the grammars in the manual}

Grammars are presented in Backus-Naur form (BNF). Terminal symbols are
set in {\tt typewriter font}.  In addition, there are special
notations for regular expressions.

An expression enclosed in square brackets \zeroone{\ldots} means at
most one occurrence of this expression (this corresponds to an
optional component).

The notation ``\nelist{\gensymbol}{sep}'' stands for a non empty
sequence of expressions parsed by the ``{\gensymbol}'' entry and
separated by the literal ``{\tt sep}''\footnote{This is similar to the
expression ``{\gensymbol} $\{$ {\tt sep} {\gensymbol} $\}$'' in
standard BNF, or ``{\gensymbol} $($ {\tt sep} {\gensymbol} $)$*'' in
the syntax of regular expressions.}.

Similarly, the notation ``\nelist{\gensymbol}{}'' stands for a non
empty sequence of expressions parsed by the ``{\gensymbol}'' entry,
without any separator between.

At the end, the notation ``\sequence{\gensymbol}{\tt sep}'' stands for a
possibly empty sequence of expressions parsed by the ``{\gensymbol}'' entry,
separated by the literal ``{\tt sep}''.

\section{Lexical conventions}
\label{lexical}\index{Lexical conventions}

\paragraph{Blanks}
Space, newline and horizontal tabulation are considered as blanks.
Blanks are ignored but they separate tokens.

\paragraph{Comments}

Comments in {\Coq} are enclosed between {\tt (*} and {\tt
  *)}\index{Comments}, and can be nested. Comments are treated as
blanks.

\paragraph{Identifiers and access identifiers}

Identifiers, written {\ident}, are sequences of letters, digits,
\verb!_! %, \verb!$!
and \verb!'!, that do not start with a digit or \verb!'!.  That is,
they are recognized by the following lexical class:

\index{ident@\ident}
\begin{center}
\begin{tabular}{rcl} 
{\firstletter} & ::= & \ml{a..z}\op\ml{A..Z}\op\ml{\_}%\op\ml{\$}
 \\
{\subsequentletter} & ::= & \ml{a..z}\op\ml{A..Z}\op\ml{0..9}\op\ml{\_}%\op\ml{\$}
\op\ml{'} \\
{\ident} & ::= & {\firstletter} \sequencewithoutblank{\subsequentletter}{}\\
\end{tabular}
\end{center}
Identifiers can contain at most 80 characters, and all characters are
meaningful. In particular, identifiers are case-sensitive.

Access identifiers, written {\accessident}, are identifiers prefixed by \verb!.! (dot).

\paragraph{Natural numbers and integers}
Numerals are sequences of digits. Integers are numerals optionally preceded by a minus sign.

\index{num@{\num}}
\begin{center}
\begin{tabular}{r@{\quad::=\quad}l}
{\digit} & \ml{0..9} \\
{\num} & \nelistwithoutblank{\digit}{} \\
{\integer} & \zeroone{\ml{-}}{\num} \\
\end{tabular}
\end{center}

\paragraph{Strings}
Strings are delimited by \verb!"! (double quote), and enclose a
sequence of any characters different from \verb!"! and \verb!\!, or
one of the following sequences
\begin{center}
\begin{tabular}{|l|l|}
\hline
Sequence        & Character denoted \\
\hline
\verb"\\"       & backslash (\verb"\") \\
\verb'\"'       & double quote (\verb'"') \\
\verb"\n"       & newline (LF) \\
\verb"\r"       & return (CR) \\
\verb"\t"       & horizontal tabulation (TAB) \\
\verb"\b"       & backspace (BS) \\
\verb"\"$ddd$  & the character with ASCII code $ddd$ in decimal \\
\hline
\end{tabular}
\end{center}
Strings can be split on several lines using a backslash (\verb!\!)  at
the end of each line, just before the newline. For instance,

\begin{flushleft}
\begin{small}
\begin{verbatim}
Add LoadPath "/usr/local/coq/\
contrib/Rocq/LAMBDA".
\end{verbatim}
\end{small}
\end{flushleft}
      
is correctly parsed, and equivalent to

\begin{flushleft}
\begin{small}
\begin{verbatim}
Add LoadPath "/usr/local/coq/contrib/Rocq/LAMBDA".
\end{verbatim}
\end{small}
\end{flushleft}

\paragraph{Keywords}
The following identifiers are reserved keywords, and cannot be
employed otherwise:
\begin{center}
\begin{tabular}{lllll}
\verb!as!         &
\verb!end!        &
\verb!in!         &
\verb!of!         &
\verb!using!      \\
\verb!with!       &
\verb!Axiom!      &
\verb!Cases!      &
\verb!CoFixpoint! &
\verb!CoInductive!\\
\verb!Compile!    &
\verb!Definition! &
\verb!Fixpoint!   &
\verb!Grammar!    &
\verb!Hypothesis! \\
\verb!Inductive!  &
\verb!Load!       &
%%\verb!Orelse!     &
\verb!Parameter!  &
\verb!Proof!      &
\verb!Prop!       \\
\verb!Qed!        &
\verb!Quit!       &
\verb!Set!        &
\verb!Syntax!     &
\verb!Theorem!    \\
\verb!Type!       &
\verb!Variable!   & & &
\end{tabular}
\end{center}

Although they are not considered as keywords, it is not advised to use
words of the following list as identifiers:
\begin{center}
\begin{tabular}{lllll}
\verb!Add!        &
\verb!AddPath!    &
\verb!Abort!      &
\verb!Abstraction!&
\verb!All!        \\
\verb!Begin!      &
\verb!Cd!         &
\verb!Chapter!    &
\verb!Check!      &
\verb!Compute!    \\
% \verb!Conjectures! \\ que dans Show Conjectures sans conflit avec ident
\verb!Defined!    &
\verb!DelPath!    &
\verb!Drop!       &
\verb!End!        &
\verb!Eval!       \\
% \verb!Explain!  n'est pas documente 
\verb!Extraction! &
\verb!Fact!       &
\verb!Focus!      &
% \verb!for! n'intervient que pour Scheme ... Induction ... sans conflit
% \verb!Go! n'est pas documente et semble peu robuste aux erreurs
\verb!Goal!       &
\verb!Guarded!    \\
\verb!Hint!       &
\verb!Immediate!  &
\verb!Induction!  &
\verb!Infix!      &
\verb!Inspect!    \\
\verb!Lemma!      &
\verb!Let!        &
\verb!LoadPath!   &
\verb!Local!      &
\verb!Minimality! \\
\verb!ML!         &
\verb!Module!     &
\verb!Modules!    &
\verb!Mutual!     &
% \verb!Node!   que dans Show Node sans conflit avec ident
\verb!Opaque!     \\
\verb!Parameters! &
\verb!Print!      &
\verb!Pwd!        &
\verb!Remark!     &
\verb!Remove!     \\
\verb!Require!    &
\verb!Reset!      &
\verb!Restart!    &
\verb!Restore!    &
\verb!Resume!     \\
\verb!Save!       &
\verb!Scheme!     &
% \verb!Script!  que dans Show Script sans conflit avec ident
\verb!Search!     &
\verb!Section!    &
\verb!Show!       \\
\verb!Silent!     &
\verb!State!      &
\verb!States!     &
\verb!Suspend!    &
\verb!Syntactic!  \\
\verb!Test!       &
\verb!Transparent!&
% \verb!Tree! que dans Show Tree et Explain Proof Tree sans conflit avec id
\verb!Undo!       &
\verb!Unset!      &
\verb!Unfocus!    \\
\verb!Variables!  &
\verb!Write!      & & &
\end{tabular}
\end{center}

\paragraph{Special tokens}
The following sequences of characters are special tokens:
\begin{center}
\begin{tabular}{lllllll}
\verb!|!  &
\verb!:!  &
\verb!:=! &
\verb!=!  &
\verb!>!  &
\verb!>>! &
\verb!<>! \\
\verb!<<! &
\verb!<!  &
\verb!->! &
\verb!;!  &
\verb!#!  &
\verb!*!  &
\verb!,!  \\
\verb!?!  &
\verb!@!  &
\verb!::! &
\verb!/!  &
\verb!<-! &
\verb!=>! &
\end{tabular}
\end{center}

Lexical ambiguities are resolved according to the ``longest match''
rule: when a sequence of non alphanumerical characters can be decomposed
into several different ways, then the first token is the longest
possible one (among all tokens defined at this moment), and so on.

\section{Terms}\label{term}\index{Terms}
\subsection{Syntax of terms}

Figure \ref{term-syntax} describes the basic set of terms which form
the {\em Calculus
of Inductive Constructions} (also called \CIC). The formal
presentation of {\CIC} is given in chapter \ref{Cic}. Extensions of
this syntax are given in chapter
\ref{Gallina-extension}. How to customize the syntax is described in
chapter \ref{Addoc-syntax}.

\begin{figure}[htb]
\begin{tabular}{|lcl|}
\hline
{\term} & ::= & {\qualid} \\
  & $|$ & {\sort} \\
 & $|$ & {\term} {\tt ->} {\term} \\
 & $|$ & {\tt (} {\nelist{\typedidents}{;}} {\tt )} {\term}\\
 & $|$ & {\tt [} {\nelist{\localdecls}{;}} {\tt ]} {\term}\\
 & $|$ & {\tt (} \nelist{\term}{} {\tt )}\\
 & $|$ & {\tt \zeroone{\annotation}} {\tt Cases} {\term} {\tt of}
        \sequence{\eqn}{|} {\tt end}\\
 & $|$ & {\tt Fix} {\ident} \verb.{. \nelist{\fixpointbody}{with} \verb.}.\\
 & $|$ & {\tt CoFix} {\ident} \verb.{. \nelist{\cofixpointbody}{with} \verb.}.\\
 & & \\
{\qualid} & ::= & {\ident} \\
 & $|$ & {\qualid} {\accessident} \\
 & & \\
{\sort} & ::= & {\tt Prop} \\
 & $|$ & {\tt Set} \\
 & $|$ & {\tt Type} \\
 & & \\
{\annotation} & ::= & \verb!<! {\term} \verb!>!\\
 & & \\
{\typedidents} & ::= &   \nelist{\ident}{,} {\tt :} {\term}\\
{\localassums} & ::= &   \nelist{\ident}{,} \zeroone{{\tt :} {\term}}\\
{\localdef} & ::= & {\ident} {\tt :=} {\term} \zeroone{{\tt :} {\term}}\\
{\localdecls} & ::= &   \localassums \\
 & $|$ & {\localdef} \\
 & & \\
{\fixpointbody} & ::= & {\ident} {\tt [} \nelist{\typedidents}{;}  {\tt ]}\verb.:. 
    {\term} \verb.:=. {\term} \\
{\cofixpointbody} & ::= & {\ident} \verb.:. 
    {\term} \verb.:=. {\term} \\
  & &\\
{\simplepattern}  & ::= & {\ident} \\
 & $|$ & \verb!(! \nelist{\ident}{} \verb!)! \\
{\eqn} & ::= &  {\simplepattern} ~\verb!=>! ~\term \\
\hline
\end{tabular}
\caption{Syntax of terms}
\label{term-syntax}
\index{term@{\term}}
\index{sort@{\sort}}
\end{figure}

%%%%%%%
\subsection{Qualified identifiers and simple identifiers}
\label{qualid}\label{ident}

{\em Qualified identifiers} ({\qualid}) denote {\em global constants}
(definitions, lemmas, theorems, remarks or facts), {\em global
variables} (parameters or axioms), {\em inductive
types} or {\em constructors of inductive types}.
{\em Simple identifiers} (or shortly {\em identifiers}) are a
syntactic subset of qualified identifiers.  Identifiers may also
denote {\em variables}, what qualified identifiers do not.

\subsection{Sorts}\index{Sorts}
\index{Type@{\Type}}
\index{Set@{\Set}}
\index{Prop@{\Prop}}
\index{Sorts}
\label{Gallina-sorts}

There are three sorts \Set, \Prop\ and \Type.

\begin{itemize}
\item \Prop\ is the universe of {\em logical propositions}.
The logical propositions themselves are typing the proofs.
We denote propositions by {\form}. This constitutes a semantic
subclass of the syntactic class {\term}.
\index{form@{\form}}
\item \Set\ is is the universe of {\em program
types} or {\em specifications}.
The specifications themselves are typing the programs.
We denote specifications by {\specif}. This constitutes a semantic
subclass of the syntactic class {\term}.
\index{specif@{\specif}}
\item {\Type} is the type of {\Set} and {\Prop}
\end{itemize}

\noindent More on sorts can be found in section \ref{Sorts}.

\subsection{Types}

{\Coq} terms are typed. {\Coq} types are recognized by the same
syntactic class as {\term}. We denote by {\type} the semantic subclass
of types inside the syntactic class {\term}.
\index{type@{\type}}

\subsection{Abstractions}
\index{abstractions}

The expression ``{\tt [} {\ident} {\tt :} \type {\tt ]}
{\term}'' denotes the {\em abstraction} of the variable {\ident}
of type {\type}, over the term {\term}.

One can abstract several variables successively:
the notation {\tt [} {\ident$_{1}$} {\tt ,} {\ldots} {\tt ,} 
{\ident$_{n}$} {\tt :} \type {\tt ]} {\term} stands for 
{\tt [} {\ident$_{1}$} {\tt :} \type {\tt ](} {\ldots} 
{\tt ([} {\ident$_{n}$} {\tt :} \type {\tt ]} {\term} {\tt )} 
\ldots {\tt )} and the notation {\tt [} {\localassums$_{1}$} {\tt ;}
{\ldots} {\tt ;} {\localassums$_{m}$} {\tt ]} {\term} is a shorthand
for {\tt [}~{\localassums$_{1}$} {\tt ](} {\ldots} {\tt ([}
{\localassums$_{m}$} {\tt ]} {\term} {\tt )}.

\medskip
\Rem The types of variables may be omitted in an
abstraction when they can be synthetized by the system.

\Rem Local definitions may appear inside brackets mixed with
assumptions. Obviously, this is expanded into unary abstractions
separated by let-in's.


\subsection{Products}
\index{products}

The expression ``{\tt (}~{\ident} {\tt :} \type {\tt )}
{\term}'' denotes the {\em product} of the variable {\ident}
of type {\type}, over the term {\term}.

Similarly, the expression {\tt (} {\ident$_{1}$} {\tt ,} {\ldots} {\tt
,} {\ident$_{n}$} {\tt :} \type {\tt )} {\term} is equivalent to {\tt
(} {\ident$_{1}$} {\tt :} \type {\tt )(} {\ldots} {\tt ((}
{\ident$_{n}$} {\tt :} \type {\tt )} {\term} {\tt )} \ldots {\tt )}
and the expression {\tt (} {\typedidents$_{1}$} {\tt ;} {\ldots} {\tt
;} {\typedidents$_{m}$} {\tt )} {\term} is a equivalent to {\tt
(}~{\typedidents$_{1}$} {\tt )(} {\ldots} {\tt ((}
{\typedidents$_{m}$} {\tt )} {\term} {\tt )} \ldots {\tt )}

\subsection{Applications}
\index{applications}

{\tt (}\term$_0$ \term$_1${\tt)} denotes the application of
  term \term$_0$ to \term$_1$.

The expression {\tt (}\term$_0$ \term$_1$ ...  \term$_n${\tt)}
denotes the application of the term \term$_0$ to the arguments
\term$_1$ ... then \term$_n$.  It is equivalent to {\tt (} {\ldots}
{\tt (} {\term$_0$} {\term$_1$} {\tt )} {\ldots} {\term$_n$} {\tt )}:
associativity is to the left.

\subsection{Local definitions (let-in)}
\index{Local definitions}
\index{let-in}

{\tt [}{\ident}{\tt :=}{\term$_1$}{\tt ]}{\term$_2$} denotes the local
binding of \term$_1$ to the variable $\ident$ in \term$_2$.

\medskip
\Rem The expression {\tt [}{\ident}{\tt :=}{\term$_1$}{\tt
:}{\type}{\tt ]}{\term$_2$} is an alternative form for the expression {\tt [}{\ident}{\tt
:=(}{\term$_1$}{\tt ::}{\type}{\tt )]}{\term$_2$}.

\Rem An alternative equivalent syntax for let-in is {\tt let} {\ident}
{\tt =} {\term} {\tt in} {\term}.  For historical reasons, the syntax
{\tt [} {\ident} {\tt =} {\term} {\tt ]} {\term} is also available but
is not recommended.

\subsection{Definition by case analysis}
\index{Cases@{\tt Cases\ldots of\ldots end}}

In a simple pattern \verb!(! \nelist{\ident}{} \verb!)!, the first {\ident}
is intended to be a constructor.

The expression {\tt \zeroone{\annotation}} {\tt Cases} {\term$_0$} {\tt of}
{\pattern$_1$} {\tt =>} {\term$_1$} {\tt $|$} {\ldots} {\tt $|$}
{\pattern$_n$} {\tt =>} {\term$_n$} {\tt end}, denotes a
{\em pattern-matching} over the term {\term$_0$} (expected to be of an
inductive type). 

The {\annotation} is the resulting type of the whole {\tt Cases}
expression. Most of the time, when this type is the same as the
types of all the {\term$_i$}, the annotation is not needed\footnote{except
if no equation is given, to match the term in an empty type, e.g. the
type \texttt{False}}. The annotation has to be given when the 
resulting type of the whole {\tt Cases} depends on the actual {\term$_0$}
matched.


\subsection{Recursive functions}
\index{Fix@{Fix \ident$_i$\{\dots\}}}

The expression {\tt Fix} {\ident$_i$} \verb.{. \ident$_1$ {\tt [}
\binders$_1$ {\tt ] :} {\type$_1$} \texttt{:=} \term$_1$ 
{\tt with} {\ldots} 
{\tt with} \ident$_n$ {\tt [} \binders$_n$~{\tt{]} :} {\type$_n$}
\texttt{:=} \term$_n$ \verb.}. denotes
the $i$th component of a block of functions defined by mutual
well-founded recursion.

The expression {\tt CoFix} {\ident$_i$} \verb.{. \ident$_1$
{\tt :} {\type$_1$} {\tt with} {\ldots} {\tt with}
\ident$_n$ {\tt [} \binders$_n$ {\tt ] :} {\type$_n$}~\verb.}. denotes
the $i$th component of a block of terms defined by a mutual guarded recursion. 

\section{\em The Vernacular}
\label{Vernacular}

Figure \ref{sentences-syntax} describes {\em The Vernacular} which is the
language of commands of \gallina.  A sentence of the vernacular
language, like in many natural languages, begins with a capital letter
and ends with a dot.
\begin{figure}
\label{sentences-syntax}
\begin{tabular}{|lcl|}
\hline
{\sentence} & ::= & {\declaration} \\
           & $|$ & {\definition} \\
           & $|$ & {\statement} \\
           & $|$ & {\inductive} \\
           & $|$ & {\fixpoint} \\
           & $|$ & {\statement} ~~ {\proof} \\
       & & \\
{\params} & ::= & \nelist{\typedidents}{;} \\
       & & \\
{\declaration} & ::= & 
  {\tt Axiom} {\ident} \verb.:. {\term} \verb:.: \\
 & $|$ &  {\declarationkeyword} {\params}  \verb:.: \\
       & & \\
{\declarationkeyword} & ::= & 
{\tt Parameter} $|$ {\tt Parameters} \\
  & $|$  & {\tt Variable}  $|$ {\tt Variables}  \\
  & $|$  & {\tt Hypothesis}  $|$ {\tt Hypotheses}\\
           & & \\

{\definition} & ::= & 
  {\tt Definition} {\ident} \zeroone{{\tt :} {\term}} \verb.:=. {\term} \verb:.: \\
 & $|$ &  {\tt Local} {\ident} \zeroone{{\tt :} {\term}} \verb.:=. {\term} \verb:.: \\
           & & \\

{\inductive} & ::= & 
 \zeroone{\texttt{Mutual}} {\tt Inductive} \nelist{\inductivebody}{with} 
    \verb:.: \\
 & $|$ & 
 \zeroone{\texttt{Mutual}} {\tt CoInductive} \nelist{\inductivebody}{with}
    \verb:.: \\
           & & \\
{\inductivebody} & ::= & 
{\ident} \zeroone{{\tt [} {\params} {\tt ]}} \verb.:. {\term}
      \verb.:=. 
      \sequence{\constructor}{|} \\
           & & \\
{\constructor} & ::= & {\ident} \verb.:. {\term} \\
  & &\\

{\fixpoint} & ::= & {\tt Fixpoint} \nelist{\fixpointbody}{with} 
   \verb:.: \\
& $|$ &  {\tt CoFixpoint} \nelist{\cofixpointbody}{with} 
   \verb:.: \\
  & &\\

{\statement} & ::= & {\tt Theorem} {\ident} {\tt :} {\term} \verb:.: \\
 & $|$ &  {\tt Lemma} {\ident} {\tt :} {\term} \verb:.: \\
 & $|$ &  {\tt Definition} {\ident} {\tt :} {\term} \verb:.: \\
           & & \\

{\proof} & ::= & {\tt Proof} {\tt .} {\dots} {\tt Qed} {\tt .}\\
   & $|$ & {\tt Proof} {\tt .} {\dots} {\tt Defined} {\tt .}\\
\hline
\end{tabular}
\caption{Syntax of sentences}
\end{figure}
The different kinds of command are described hereafter. They all suppose
that the terms occurring in the sentences are well-typed.

\subsection{Declarations}\index{Declarations}\label{Declarations}
The declaration mechanism allows the user to specify his own basic
objects. Declared objects play the role of axioms or parameters in
mathematics. A declared object is an {\ident} associated to a \term. A
declaration is accepted by {\Coq} iff this {\term} is a correct type 
in the current context of the declaration and \ident\ was
not previously defined in the same module. This {\term}
is considered to be the type, or specification, of the \ident.

\subsubsection{{\tt Axiom {\ident} : {\term}}.}
\comindex{Axiom}
\label{Axiom}
This command links {\term} to the name {\ident} as its specification in the
global context. The fact asserted by {\term} is thus assumed
as a postulate.

\begin{ErrMsgs}
\item \errindex{Clash with previous constant {\ident}}
\end{ErrMsgs}

\begin{Variants} 
\item {\tt Parameter {\ident} : {\term}.}
  \comindex{Parameter}\\
  Is equivalent to {\tt Axiom {\ident} : {\term}}
\item {\tt Parameter \nelist{\nelist{\ident}{,} : {\term}}{;} {\tt .}} \\
%        Is equivalent to {\tt Axiom {\lident} : {\term}}  
  Links the {\term}'s to the names comprising the lists \nelist{\nelist{\ident}{,} : {\term}}{;}.
\end{Variants}

\noindent {\bf Remark: } It is possible to replace {\tt Parameter} by
{\tt Parameters} when more than one parameter are given.

\subsubsection{{\tt Variable {\ident} : {\term}}.}\comindex{Variable}
\comindex{Variables}
This command links {\term} to the name {\ident} in the context of the
current section (see \ref{Section} for a description of the section
mechanism). The name {\ident} will be unknown when the current
section will be closed. One says that the variable is {\em
  discharged}. Using the {\tt Variable} command out of any section is
equivalent to {\tt Axiom}.

\begin{ErrMsgs}
\item \errindex{Clash with previous constant {\ident}}
\end{ErrMsgs}

\begin{Variants}
\item {\tt Variable \nelist{\nelist{\ident}{,}:{\term}}{;} {\tt .}}\\
  Links {\term} to the
  names comprising the list \nelist{\nelist{\ident}{,}:{\term}}{;}
\item {\tt Hypothesis \nelist{\nelist{\ident}{,} $\;$:$\;$ {\term}}{;} {\tt
 .}} \comindex{Hypothesis}\\  % Ligne trop longue
  \texttt{Hypothsis} is a synonymous of \texttt{Variable}
\end{Variants}

\noindent {\bf Remark: } It is possible to replace {\tt Variable} by
{\tt Variables} and \ml{Hypothesis} by {\tt Hypotheses}
 when more than one variable or one hypothesis are given.

It is advised to use the keywords \verb:Axiom: and \verb:Hypothesis:
for logical postulates (i.e. when the assertion {\term} is of sort
\verb:Prop:), and to use the keywords \verb:Parameter: and
\verb:Variable: in other cases (corresponding to the declaration of an
abstract mathematical entity).

\subsection{Definitions}\index{Definitions}\label{Simpl-definitions}
Definitions differ from declarations since they allow to give a name
to a term whereas declarations were just giving a type to a name. That
is to say that the name of a defined object can be replaced at any
time by its definition.  This replacement is called
$\delta$-conversion\index{delta-reduction@$\delta$-reduction} (see section
\ref{delta}).  A defined object is accepted by the system iff the
defining term is well-typed in the current context of the definition.
Then the type of the name is the type of term. The defined name is
called a {\em constant}\index{Constant} and one says that {\it the
  constant is added to the environment}\index{Environment}.

A formal presentation of constants and environments is given in
section \ref{Typed-terms}.


\subsubsection{\tt Definition {\ident} := {\term}.}
\comindex{Definition}
This command binds the value {\term} to the name {\ident} in the
environment, provided that {\term} is well-typed.

\begin{ErrMsgs}
\item \errindex{Clash with previous constant {\ident}}
\end{ErrMsgs}

\begin{Variants}
\item {\tt Definition {\ident} : {\term$_1$} := {\term$_2$}.} It
  checks that the type of {\term$_2$} is definitionally equal to
  {\term$_1$}, and registers {\ident} as being of type {\term$_1$},
  and bound to value {\term$_2$}.
\end{Variants}

\begin{ErrMsgs}
\item \errindex{In environment \dots the term: {\term$_2$} does not have type
    {\term$_1$}}.\\
    \texttt{Actually, it has type {\term$_3$}}.
\end{ErrMsgs}

\SeeAlso sections \ref{Opaque}, \ref{Transparent}, \ref{Unfold}

\subsubsection{\tt Local {\ident} := {\term}.}\comindex{Local}
This command binds the value {\term} to the name {\ident} in the
environment of the current section. The name {\ident} will be unknown
when the current section will be closed and all occurrences of
{\ident} in persistent objects (such as theorems) defined within the
section will be replaced by \term.  One can say that the {\tt Local}
definition is a kind of {\em macro}.

\begin{ErrMsgs}
\item \errindex{Clash with previous constant {\ident}}
\end{ErrMsgs}

\begin{Variants}
\item {\tt Local {\ident} : {\term$_1$} := {\term$_2$}.}
\end{Variants}

\SeeAlso \ref{Section} (section mechanism), \ref{Opaque},
\ref{Transparent} (opaque/transparent constants), \ref{Unfold}

% Let comme forme de Definition n'existe plus
%
%\subsubsection{\tt Let {\ident} := {\term}.} 
%\comindex{Let}

%This command makes definition that are stronger than \texttt{Local},
%and weaker than \texttt{Definition}. The name {\ident} will be kown in the
%current section and after having closed the current section, but will
%be unknown after closing the section above the current section.

%\begin{Variants}
%\item {\tt Local {\ident} : {\term$_1$} := {\term$_2$}.}
%\end{Variants}

%\SeeAlso \ref{Section}

\subsection{Inductive definitions}
\index{Inductive definitions} \label{gal_Inductive_Definitions}
\comindex{Inductive}\label{Inductive}

We gradually explain simple inductive types, simple
annotated inductive types, simple parametric inductive types, 
mutually inductive types. We explain also co-inductive types.

\subsubsection{Simple inductive types}

The definition of a simple inductive type has the following form:

\medskip
{\tt 
\begin{tabular}{l}
Inductive {\ident} : {\sort} :=  \\
\begin{tabular}{clcl}
   & {\ident$_1$}  &:& {\type$_1$} \\
 | & {\ldots} && \\
 | & {\ident$_n$} &:& {\type$_n$}
\end{tabular}
\end{tabular}
}
\medskip

The name {\ident} is the name of the inductively defined type and
{\sort} is the universes where it lives.
The names {\ident$_1$}, {\ldots}, {\ident$_n$}
are the names of its constructors and {\type$_1$}, {\ldots},
{\type$_n$} their respective types. The types of the constructors have
to satisfy a {\em positivity condition} (see section \ref{Positivity})
for {\ident}.  This condition ensures the soundness of the inductive
definition.  If this is the case, the constants {\ident},
{\ident$_1$}, {\ldots}, {\ident$_n$} are added to the environment with
their respective types.  Accordingly to the universe where
the inductive type lives({\it e.g.} its type {\sort}), {\Coq} provides a
number of destructors for {\ident}.  Destructors are named
{\ident}{\tt\_ind}, {\ident}{\tt \_rec} or {\ident}{\tt \_rect} which
respectively correspond to elimination principles on {\tt Prop}, {\tt
Set} and {\tt Type}.  The type of the destructors expresses structural
induction/recursion principles over objects of {\ident}. We give below
two examples of the use of the {\tt Inductive} definitions.

The set of natural numbers is defined as:
\begin{coq_example}
Inductive nat : Set := O : nat | S : nat -> nat.
\end{coq_example}

The type {\tt nat} is defined as the least \verb:Set: containing {\tt
  O} and closed by the {\tt S} constructor. The constants {\tt nat},
{\tt O} and {\tt S} are added to the environment.

Now let us have a look at the elimination principles. They are three :
{\tt nat\_ind}, {\tt nat\_rec} and {\tt nat\_rect}.  The type of {\tt
  nat\_ind} is:
\begin{coq_example}
Check nat_ind.
\end{coq_example}

This is the well known structural induction principle over natural
numbers, i.e. the second-order form of Peano's induction principle.
It allows to prove some universal property of natural numbers ({\tt
(n:nat)(P n)}) by induction on {\tt n}. Recall that {\tt (n:nat)(P n)}
is \gallina's syntax for the universal quantification $\forall
n:nat\cdot P(n).$\\ The types of {\tt nat\_rec} and {\tt nat\_rect}
are similar, except that they pertain to {\tt (P:nat->Set)} and {\tt
(P:nat->Type)} respectively . They correspond to primitive induction
principles (allowing dependent types) respectively over sorts
\verb:Set: and \verb:Type:. The constant {\ident}{\tt \_ind} is always
provided, whereas {\ident}{\tt \_rec} and {\ident}{\tt \_rect} can be
impossible to derive (for example, when {\ident} is a proposition).

\subsubsection{Simple annotated inductive types}

In an annotated inductive types, the universe where the inductive
type is defined is no longer a simple sort, but what is called an
arity, which is a type whose conclusion is a sort.

As an example of annotated inductive types, let us define the
$even$ predicate:

\begin{coq_example}
Inductive even : nat->Prop :=
 | even_0  : (even O)
 | even_SS : (n:nat)(even n)->(even (S (S n))).
\end{coq_example}

The type {\tt nat->Prop} means that {\tt even} is a unary predicate
(inductively defined) over natural numbers.  The type of its two
constructors are the defining clauses of the predicate {\tt even}. The
type of {\tt even\_ind} is:

\begin{coq_example}
Check even_ind.
\end{coq_example}

From a mathematical point of view it asserts that the natural numbers
satisfying the predicate {\tt even} are exactly the naturals satisfying
the clauses {\tt even\_0} or {\tt even\_SS}. This is why, when we want
to prove any predicate {\tt P} over elements of {\tt even}, it is
enough to prove it for {\tt O} and to prove that if any natural number
{\tt n} satisfies {\tt P} its double successor {\tt (S (S n))}
satisfies also {\tt P}. This is indeed analogous to the structural
induction principle we got for {\tt nat}.

\begin{ErrMsgs}
\item \errindex{Non strictly positive occurrence of {\ident} in {\type}}
\item \errindex{Type of Constructor not well-formed}
\end{ErrMsgs}

\begin{Variants}
\item {\tt Inductive {\ident} [ {\params} ] : {\term} :=
    {\ident$_1$}:{\term$_1$} | {\ldots} | {\ident$_n$}:\term$_n$.}\\ Allows
  to define parameterized inductive types.\\
  For instance, one can define
  parameterized lists as:
\begin{coq_example*}
Inductive list [X:Set] : Set :=
 Nil : (list X) | Cons : X->(list X)->(list X).
\end{coq_example*}
Notice that, in the type of {\tt Nil} and {\tt Cons}, we write {\tt
  (list X)} and not just {\tt list}.\\ The constants {\tt Nil} and
{\tt Cons} will have respectively types:

\begin{coq_example}
Check Nil.
\end{coq_example}
and

\begin{coq_example}
Check Cons.
\end{coq_example}

Types of destructors will be also quantified with {\tt (X:Set)}.
\item  {\tt Inductive {\sort} {\ident} :=
{\ident$_1$}:{\term$_1$} | {\ldots} | {\ident$_n$}:\term$_n$.}\\ 
with {\sort} being one of {\tt Prop, Type, Set} is
equivalent to \\ {\tt Inductive {\ident} : {\sort} :=
  {\ident$_1$}:{\term$_1$} | {\ldots} | {\ident$_n$}:\term$_n$.}
\item {\tt Inductive {\sort} {\ident} [ {\params} ]:=
{\ident$_1$}:{\term$_1$} | {\ldots} | {\ident$_n$}:\term$_n$.}\\ 
      Same as before but with parameters.
\end{Variants}

\SeeAlso sections \ref{Cic-inductive-definitions}, \ref{Elim}

\subsubsection{Mutually inductive types}
\comindex{Mutual Inductive}\label{Mutual-Inductive}

The definition of a block of mutually inductive types has the form:

\medskip
{\tt 
\begin{tabular}{l}
Inductive {\ident$_1$} : {\type$_1$} :=  \\
\begin{tabular}{clcl}
   & {\ident$_1^1$}     &:& {\type$_1^1$} \\
 | & {\ldots} && \\
 | & {\ident$_{n_1}^1$} &:& {\type$_{n_1}^1$}
\end{tabular}  \\
with\\
~{\ldots} \\
with {\ident$_m$} : {\type$_m$} := \\
\begin{tabular}{clcl}
   & {\ident$_1^m$}     &:& {\type$_1^m$} \\
 | & {\ldots} \\
 | & {\ident$_{n_m}^m$} &:& {\type$_{n_m}^m$}.
\end{tabular}
\end{tabular}
}
\medskip

\noindent {\bf Remark: } The word {\tt Mutual} can be optionally
inserted in front of {\tt Inductive}.

It has the same semantics as the above {\tt Inductive} definition for
each \ident$_1$, {\ldots}, \ident$_m$. All names \ident$_1$, {\ldots},
\ident$_m$ and \ident$_1^1$, \dots, \ident$_{n_m}^m$ are
simultaneously added to the environment. Then
well-typing of constructors can be checked. Each one of the
\ident$_1$, {\ldots}, \ident$_m$ can be used on its own.

It is also possible to parameterize these inductive definitions.
However, parameters correspond to a local
context in which the whole set of inductive declarations is done.  For
this reason, the parameters must be strictly the same for each
inductive types The extented syntax is:

\medskip
{\tt 
Inductive {{\ident$_1$} [{\rm\sl params} ] : {\type$_1$} :=  \\
\mbox{}\hspace{0.4cm}  {\ident$_1^1$} : {\type$_1^1$} \\
\mbox{}\hspace{0.1cm}| ..  \\
\mbox{}\hspace{0.1cm}| {\ident$_{n_1}^1$} : {\type$_{n_1}^1$}  \\
with\\
\mbox{}\hspace{0.1cm} .. \\
with {\ident$_m$} [{\rm\sl params} ] : {\type$_m$} := \\
\mbox{}\hspace{0.4cm}{\ident$_1^m$} : {\type$_1^m$} \\
\mbox{}\hspace{0.1cm}| .. \\
\mbox{}\hspace{0.1cm}| {\ident$_{n_m}^m$} : {\type$_{n_m}^m$}.
}}
\medskip

\Example
The typical example of a mutual inductive data type is the one for
trees and forests. We assume given two types $A$ and $B$ as variables.
It can be declared the following way.

\begin{coq_example*}
Variables A,B:Set.
Inductive tree : Set :=  node : A -> forest -> tree
with forest : Set := 
   | leaf : B -> forest 
   | cons : tree -> forest -> forest.
\end{coq_example*}

This declaration generates automatically six induction
principles. They are respectively 
called {\tt tree\_rec}, {\tt tree\_ind}, {\tt
  tree\_rect}, {\tt forest\_rec}, {\tt forest\_ind}, {\tt
  forest\_rect}.  These ones are not the most general ones but are
just the induction principles corresponding to each inductive part
seen as a single inductive definition.

To illustrate this point on our example, we give the types of {\tt
  tree\_rec} and {\tt forest\_rec}.

\begin{coq_example}
Check tree_rec.
Check forest_rec.
\end{coq_example}

Assume we want to parameterize our mutual inductive definitions with
the two type variables $A$ and $B$, the declaration should be done the
following way:

\begin{coq_eval}
Reset tree.
\end{coq_eval}
\begin{coq_example*}
Inductive
     tree   [A,B:Set] : Set := node : A -> (forest A B) -> (tree A B)
with forest [A,B:Set] : Set := leaf : B -> (forest A B)
                   | cons : (tree A B) -> (forest A B) -> (forest A B).
\end{coq_example*}
\begin{coq_eval}
Save State toto.
\end{coq_eval}

Assume we define an inductive definition inside a section.  When the
section is closed, the variables declared in the section and occurring
free in the declaration are added as parameters to the inductive
definition. 

\SeeAlso \ref{Section}

\subsubsection{Co-inductive types}
\comindex{CoInductive}

The objects of an inductive type are well-founded with respect to the
constructors of the type. In other words, such objects contain only a
{\it finite} number constructors. Co-inductive types arise from
relaxing this condition, and admitting types whose objects contain an
infinity of constructors. Infinite objects are introduced by a
non-ending (but effective) process of construction, defined in terms
of the constructors of the type.

An example of a co-inductive type is the type of infinite sequences of
natural numbers, usually called streams. It can be introduced in \Coq\
using the \texttt{CoInductive} command:
\begin{coq_example}
CoInductive Set Stream  := Seq : nat->Stream->Stream.
\end{coq_example}

The syntax of this command is the same as the command \texttt{Inductive}
(cf. section \ref{gal_Inductive_Definitions}). Notice that no
principle of induction is derived from the definition of a
co-inductive type, since such principles only make sense for inductive
ones. For co-inductive ones, the only elimination principle is case
analysis. For example, the usual destructors on streams
\texttt{hd:Stream->nat} and \texttt{tl:Str->Str} can be defined as
follows:
\begin{coq_example}
Definition hd  := [x:Stream]Cases x of (Seq a s) => a  end.
Definition tl  := [x:Stream]Cases x of (Seq a s) => s  end.
\end{coq_example}

Definition of co-inductive predicates and blocks of mutually
co-inductive definitions are also allowed. An example of a
co-inductive predicate is the extensional equality on streams:

\begin{coq_example}
CoInductive EqSt  : Stream->Stream->Prop := 
            eqst : (s1,s2:Stream)
                    (hd s1)=(hd s2)->
                     (EqSt (tl s1) (tl s2))->(EqSt s1 s2).
\end{coq_example}

In order to prove the extensionally equality of two streams $s_1$ and
$s_2$ we have to construct and infinite proof of equality, that is,
an infinite object of type $(\texttt{EqSt}\;s_1\;s_2)$. We will see
how to introduce infinite objects in section \ref{CoFixpoint}.

\subsection{Definition of recursive functions}

\subsubsection{\tt Fixpoint {\ident} [ \ident$_1$ : \type$_1$ ] :
\type$_0$ := \term$_0$}
\comindex{Fixpoint}\label{Fixpoint}

This command allows to define inductive objects using a fixed point
construction. The meaning of this declaration is to define {\it
ident} a recursive function with one argument \ident$_1$ of type
\term$_1$ such that ({\it ident}~\ident$_1$) has type \type$_0$ and is
equivalent to the expression \term$_0$. The type of the {\ident} is
consequently {(\ident$_1$ : \type$_1$)\type$_0$} and the value is
equivalent to [\ident$_1$ : \type$_1$]\term$_0$.  The argument
{\ident$_1$} (of type {\type$_1$}) is called the {\em recursive
variable} of {\ident}.  Its type should be an inductive definition.

To be accepted, a {\tt Fixpoint} definition has to satisfy some
syntactical constraints on this recursive variable. They are needed to
ensure that the {\tt Fixpoint} definition always terminates. For
instance, one can define the addition function as :

\begin{coq_example}
Fixpoint add [n:nat] : nat->nat 
   := [m:nat]Cases n of O => m | (S p) => (S (add p m)) end.
\end{coq_example}

The {\tt Cases} operator matches a value (here \verb:n:) with the
various constructors of its (inductive) type. The remaining arguments
give the respective values to be returned, as functions of the
parameters of the corresponding constructor. Thus here when \verb:n:
equals \verb:O: we return \verb:m:, and when \verb:n: equals 
\verb:(S p): we return \verb:(S (add p m)):.

The {\tt Cases} operator is formally described
in detail in section \ref{Caseexpr}.  The system recognizes that in
the inductive call {\tt (add p m)} the first argument actually
decreases because it is a {\em pattern variable} coming from {\tt Cases
  n of}.

\begin{Variants}
\item {\tt Fixpoint {\ident} [ {\params} ] : \type$_0$ :=
\term$_0$.}\\
        It declares a list of identifiers with their type
        usable in the type \type$_0$ and the definition body \term$_0$
        and the last identifier in {\params} is the recursion variable.
\item {\tt Fixpoint {\ident$_1$} [ {\params$_1$} ] :
        {\type$_1$} := {\term$_1$}\\
        with {\ldots} \\
        with {\ident$_m$}  [ {\params$_m$} ] : {\type$_m$} := 
        {\type$_m$}}\\
        Allows to define simultaneously {\ident$_1$}, {\ldots},
        {\ident$_m$}.
\end{Variants}

\Example The following definition is not correct and generates an
error message:

\begin{coq_example}
Fixpoint wrongplus [n:nat] : nat->nat 
   := [m:nat]Cases m of O => n | (S p) => (S (wrongplus n p)) end.
\end{coq_example}

because the declared decreasing argument {\tt n} actually does not
decrease in the recursive call.  The function computing the addition
over the second argument should rather be written:

\begin{coq_example*}
Fixpoint plus [n,m:nat] : nat 
   := Cases m of O => n | (S p) => (S (plus n p)) end.
\end{coq_example*}

The ordinary match operation on natural numbers can be mimicked in the
following way.
\begin{coq_example*}
Fixpoint nat_match [C:Set;f0:C;fS:nat->C->C;n:nat] : C 
   := Cases n of O => f0 | (S p) => (fS p (nat_match C f0 fS p)) end.
\end{coq_example*}
The recursive call may not only be on direct subterms of the recursive
variable {\tt n} but also on a deeper subterm and we can directly
write the function {\tt mod2} which gives the remainder modulo 2 of a
natural number.
\begin{coq_example*}
Fixpoint mod2 [n:nat] : nat
   := Cases n of
         O    => O
      | (S p) => Cases p of O => (S O) | (S q) => (mod2 q) end
      end.
\end{coq_example*}
In order to keep the strong normalisation property, the fixed point
reduction will only be performed when the argument in position of the
recursive variable (whose type should be in an inductive definition)
starts with a constructor.

The {\tt Fixpoint} construction enjoys also the {\tt with} extension
to define functions over mutually defined inductive types or more
generally any mutually recursive definitions.

\Example 
The size of trees and forests can be defined the following way: 
\begin{coq_eval}
Restore State Initial.
Variables A,B:Set.
Inductive tree : Set :=  node : A -> forest -> tree
with forest : Set := leaf : B -> forest 
                   | cons : tree -> forest -> forest.
\end{coq_eval}
\begin{coq_example*}
Fixpoint tree_size [t:tree] : nat :=
    Cases t of (node a f) => (S (forest_size f)) end
with forest_size [f:forest] : nat := 
    Cases f of (leaf b) => (S O)
          | (cons t f') => (plus (tree_size t) (forest_size f')) 
         end.
\end{coq_example*}
A generic command {\tt Scheme} is useful to build automatically various
mutual induction principles. It is described in section \ref{Scheme}.

\subsubsection{\tt CoFixpoint {\ident} :
\type$_0$ := \term$_0$.}\comindex{CoFixpoint}\label{CoFixpoint}

The {\tt CoFixpoint} command introduces a method for constructing an
infinite object of a coinduc\-tive type. For example, the stream
containing all natural numbers can be introduced applying the
following method to the number \texttt{O}:

\begin{coq_example*}
CoInductive Set Stream  := Seq : nat->Stream->Stream.
Definition hd  := [x:Stream]Cases x of (Seq a s) => a  end.
Definition tl  := [x:Stream]Cases x of (Seq a s) => s  end.
\end{coq_example*}
\begin{coq_example}
CoFixpoint from  : nat->Stream := [n:nat](Seq n (from (S n))).
\end{coq_example}

Oppositely to recursive ones, there is no decreasing argument in a
co-recursive definition. To be admissible, a method of construction
must provide at least one extra constructor of the infinite object for
each iteration. A syntactical guard condition is imposed on
co-recursive definitions in order to ensure this: each recursive call
in the definition must be protected by at least one constructor, and
only by constructors. That is the case in the former definition, where
the single recursive call of \texttt{from} is guarded by an
application of \texttt{Seq}. On the contrary, the following recursive
function does not satisfy the guard condition:

\begin{coq_example*}
CoFixpoint filter  : (nat->bool)->Stream->Stream := 
   [p:nat->bool]
     [s:Stream]
       if (p (hd s)) then (Seq (hd s) (filter p (tl s)))
       else (filter p (tl s)).
\end{coq_example*}

\noindent Notice that the definition contains an unguarded recursive
call of \texttt{filter} on the \texttt{else} branch of the test.

The elimination of co-recursive definition is done lazily, i.e. the
definition is expanded only when it occurs at the head of an
application which is the argument of a case expression.  Isolate, it
is considered as a canonical expression which is completely
evaluated. We can test this using the command \texttt{Eval},
which computes the normal forms of a term:

\begin{coq_example}
Eval Compute in (from O).
Eval Compute in (hd (from O)).
Eval Compute in (tl (from O)).  
\end{coq_example}

As in the \texttt{Fixpoint} command (cf. section~\ref{Fixpoint}), it
is possible to introduce a block of mutually dependent methods. The
general syntax for this case is:

{\tt CoFixpoint {\ident$_1$}  :{\type$_1$} := {\term$_1$}\\
     with\\
        \mbox{}\hspace{0.1cm} $\ldots$  \\
        with {\ident$_m$}   : {\type$_m$} := {\term$_m$}}

\subsection{Statement and proofs}

A statement claims a goal of which the proof is then interactively done
using tactics. More on the proof editing mode, statements and proofs can be
found in chapter \ref{Proof-handling}.

\subsubsection{\tt Theorem {\ident} : {\type}.}
\comindex{Theorem}
This command binds {\type} to the name {\ident} in the
environment, provided that a proof of {\type} is next given.

After a statement, Coq needs a proof.

\begin{Variants} 

\item {\tt Lemma {\ident} : {\type}.}\comindex{Lemma}\\ 
It is a synonymous of \texttt{Theorem}
\item {\tt Remark {\ident} : {\type}.}\comindex{Remark}\\ 
Same as {\tt Theorem} except
that if this statement is in a section then the name {\ident} will be unknown
when the current section (see \ref{Section}) will be closed. All
proofs of  persistent objects (such as theorems) referring to {\ident}
within the section will be replaced by the proof of  {\ident}.
\item {\tt Definition {\ident} : {\type}.} \\
Allow to define a term of type {\type} using the proof editing mode. It
behaves as {\tt Theorem} except the defined term will be transparent (see
\ref{Transparent}, \ref{Unfold}). 
\end{Variants}

\subsubsection{{\tt Proof} {\tt .} \dots {\tt Qed} {\tt .}}

A proof starts by the keyword {\tt Proof}.  Then {\Coq} enters the
proof editing mode until the proof is completed. The proof editing
mode essentially contains tactics that are described in chapter
\ref{Tactics}. Besides tactics, there are commands to manage the proof
editing mode. They are described in chapter \ref{Proof-handling}. When
the proof is completed it should be validated and put in the
environment using the keyword {\tt Qed}.
\medskip

\ErrMsg
\begin{enumerate}
\item \errindex{Clash with previous constant {\ident}}
\end{enumerate}

\begin{Remarks}
\item Several statements can be simultaneously opened.
\item Not only other statements but any vernacular command can be given
within the proof editing mode. In this case, the command is
understood as if it would have been given before the statements still to be
proved. 
\item {\tt Proof} is recommended but can currently be omitted. On the
opposite, {\tt Qed} is mandatory to validate a proof.
\end{Remarks}

\begin{Variants}
\item {\tt Proof} {\tt .} \dots {\tt Defined} {\tt .}\\
  \comindex{Proof}
  Same as {\tt Proof} {\tt .} \dots {\tt Qed} {\tt .} but it is
  intended to surround a definition built using the proof-editing mode.
\item {\tt Proof} {\tt .} \dots {\tt Save.}\\
  Same as {\tt Proof} {\tt .} \dots {\tt Qed} {\tt .}
\item {\tt Goal} \type \dots {\tt Save} \ident \\
  Same as {\tt Lemma} \ident {\tt :} \type \dots {\tt Save.}
  This is intended to be used in the interactive mode. Conversely to named
  lemmas, anonymous goals cannot be nested.
\end{Variants}
\comindex{Proof}
\comindex{Qed}
\comindex{Save}
\comindex{Goal}

% Local Variables: 
% mode: LaTeX
% TeX-master: "Reference-Manual"
% End: 

% $Id$ 
