
%\documentstyle[11pt]{article}
%%%%%%%%%%%%%%%%%%%%%%%%%%%%%%%%%
% File title.tex
% Page formatting commands
% Macro \coverpage
%%%%%%%%%%%%%%%%%%%%%%%%%%%%%%%%

%\setlength{\marginparwidth}{0pt}
%\setlength{\oddsidemargin}{0pt}
%\setlength{\evensidemargin}{0pt}
%\setlength{\marginparsep}{0pt}
%\setlength{\topmargin}{0pt}
%\setlength{\textwidth}{16.9cm}
%\setlength{\textheight}{22cm}
%\usepackage{fullpage}

%\newcommand{\printingdate}{\today}
%\newcommand{\isdraft}{\Large\bf\today\\[20pt]}
%\newcommand{\isdraft}{\vspace{20pt}}

\newcommand{\coverpage}[3]{
\thispagestyle{empty}
\begin{center}
\bfseries % for the rest of this page, until \end{center}
\Huge
The Coq Proof Assistant\\[12pt]
#1\\[20pt]
\Large\today\\[20pt]
Version \coqversion%\footnote[1]{This research was partly supported by IST working group ``Types''}

\vspace{0pt plus .5fill}
#2
\par\vfill
The Coq Development Team

\vspace*{15pt}
\end{center}
\newpage

\thispagestyle{empty}
\hbox{}\vfill % without \hbox \vfill does not work at the top of the page
\begin{flushleft}
%BEGIN LATEX
V\coqversion, \today
\par\vspace{20pt}
%END LATEX
\copyright INRIA 1999-2004 ({\Coq} versions 7.x)

\copyright INRIA 2004-2010 ({\Coq} versions 8.x)

#3
\end{flushleft}
} % end of \coverpage definition


% \newcommand{\shorttitle}[1]{
% \begin{center}
% \begin{huge}
% \begin{bf}
% The Coq Proof Assistant\\
% \vspace{10pt}
%     #1\\
% \end{bf}
% \end{huge}
% \end{center}
% \vspace{5pt}
% }

% Local Variables: 
% mode: LaTeX
% TeX-master: ""
% End: 

% $Id$ 


%%%%%%%%%%%%%%%%%%%%%%%%%%%%%%%%%%%%%%%%%%%
% MACROS FOR THE REFERENCE MANUAL OF COQ %
%%%%%%%%%%%%%%%%%%%%%%%%%%%%%%%%%%%%%%%%%%

\newcommand{\coqversion}{7.3}

% For commentaries (define \com as {} for the release manual)
%\newcommand{\com}[1]{{\it(* #1 *)}}
\newcommand{\com}[1]{}

%%%%%%%%%%%%%%%%%%%%%%%
% Formatting commands %
%%%%%%%%%%%%%%%%%%%%%%%

\newcommand{\ErrMsg}{\medskip \noindent {\bf Error message: }}
\newcommand{\ErrMsgx}{\medskip \noindent {\bf Error messages: }}
\newcommand{\variant}{\medskip \noindent {\bf Variant: }}
\newcommand{\variants}{\medskip \noindent {\bf Variants: }}
\newcommand{\SeeAlso}{\medskip \noindent {\bf See also: }}
\newcommand{\Rem}{\medskip \noindent {\bf Remark: }}
\newcommand{\Rems}{\medskip \noindent {\bf Remarks: }}
\newcommand{\Example}{\medskip \noindent {\bf Example: }}
\newcommand{\Warning}{\medskip \noindent {\bf Warning: }}
\newcommand{\Warns}{\medskip \noindent {\bf Warnings: }}
\newcounter{ex}
\newcommand{\firstexample}{\setcounter{ex}{1}}
\newcommand{\example}[1]{
\medskip \noindent \textbf{Example \arabic{ex}: }\textit{#1}
\addtocounter{ex}{1}}

\newenvironment{Variant}{\variant\begin{enumerate}}{\end{enumerate}}
\newenvironment{Variants}{\variants\begin{enumerate}}{\end{enumerate}}
\newenvironment{ErrMsgs}{\ErrMsgx\begin{enumerate}}{\end{enumerate}}
\newenvironment{Remarks}{\Rems\begin{enumerate}}{\end{enumerate}}
\newenvironment{Warnings}{\Warns\begin{enumerate}}{\end{enumerate}}
\newenvironment{Examples}{\medskip\noindent{\bf Examples:}
\begin{enumerate}}{\end{enumerate}}

\newcommand{\rr}{\raggedright}

\newcommand{\tinyskip}{\rule{0mm}{4mm}}

\newcommand{\bd}{\noindent\bf}
\newcommand{\sbd}{\vspace{8pt}\noindent\bf}
\newcommand{\sdoll}[1]{\begin{small}$ #1~ $\end{small}}
\newcommand{\sdollnb}[1]{\begin{small}$ #1 $\end{small}}
\newcommand{\kw}[1]{\textsf{#1}}
\newcommand{\spec}[1]{\{\,#1\,\}}

% Building regular expressions
\newcommand{\zeroone}[1]{{\sl [}#1{\sl ]}}
%\newcommand{\zeroonemany}[1]{$\{$#1$\}$*}
%\newcommand{\onemany}[1]{$\{$#1$\}$+}
\newcommand{\nelist}[2]{{#1} {\tt #2} {\ldots} {\tt #2} {#1}}
\newcommand{\sequence}[2]{{\sl [}{#1} {\tt #2} {\ldots} {\tt #2} {#1}{\sl ]}}
\newcommand{\nelistwithoutblank}[2]{#1{\tt #2}\ldots{\tt #2}#1}
\newcommand{\sequencewithoutblank}[2]{$[$#1{\tt #2}\ldots{\tt #2}#1$]$}

% Used for RefMan-gal
\newcommand{\ml}[1]{\hbox{\tt{#1}}}
\newcommand{\op}{\,|\,}

%%%%%%%%%%%%%%%%%%%%%%%%
% Trademarks and so on %
%%%%%%%%%%%%%%%%%%%%%%%%

\newcommand{\Coq}{{\sf Coq}}
\newcommand{\ocaml}{{\sf Objective Caml}}
\newcommand{\camlpppp}{{\sf Camlp4}}
\newcommand{\emacs}{{\sf GNU Emacs}}
\newcommand{\gallina}{\textsf{Gallina}}
\newcommand{\CIC}{\mbox{\sc Cic}}
\newcommand{\FW}{\mbox{$F_{\omega}$}}
\newcommand{\bn}{{\sf BNF}}

%%%%%%%%%%%%%%%%%%%
% Name of tactics %
%%%%%%%%%%%%%%%%%%%

\newcommand{\Natural}{\mbox{\tt Natural}}

%%%%%%%%%%%%%%%%%
% \rm\sl series %
%%%%%%%%%%%%%%%%%

\newcommand{\Fwterm}{\textrm{\textsl{Fwterm}}}
\newcommand{\Index}{\textrm{\textsl{index}}}
\newcommand{\abbrev}{\textrm{\textsl{abbreviation}}}
\newcommand{\annotation}{\textrm{\textsl{annotation}}}
\newcommand{\atomictac}{\textrm{\textsl{atomic\_tactic}}}
\newcommand{\binders}{\textrm{\textsl{bindings}}}
\newcommand{\binder}{\textrm{\textsl{binding}}}
\newcommand{\bindinglist}{\textrm{\textsl{bindings\_list}}}
\newcommand{\cast}{\textrm{\textsl{cast}}}
\newcommand{\cofixpointbody}{\textrm{\textsl{cofix\_body}}}
\newcommand{\coinductivebody}{\textrm{\textsl{coind\_body}}}
\newcommand{\commandtac}{\textrm{\textsl{tactic\_invocation}}}
\newcommand{\constructor}{\textrm{\textsl{constructor}}}
\newcommand{\convtactic}{\textrm{\textsl{conv\_tactic}}}
\newcommand{\declarationkeyword}{\textrm{\textsl{declaration\_keyword}}}
\newcommand{\declaration}{\textrm{\textsl{declaration}}}
\newcommand{\definition}{\textrm{\textsl{definition}}}
\newcommand{\digit}{\textrm{\textsl{digit}}}
\newcommand{\eqn}{\textrm{\textsl{equation}}}
\newcommand{\exteqn}{\textrm{\textsl{ext\_eqn}}}
\newcommand{\field}{\textrm{\textsl{field}}}
\newcommand{\firstletter}{\textrm{\textsl{first\_letter}}}
\newcommand{\fixpg}{\textrm{\textsl{fix\_pgm}}}
\newcommand{\fixpointbody}{\textrm{\textsl{fix\_body}}}
\newcommand{\fixpoint}{\textrm{\textsl{fixpoint}}}
\newcommand{\flag}{\textrm{\textsl{flag}}}
\newcommand{\form}{\textrm{\textsl{form}}}
\newcommand{\gensymbol}{\textrm{\textsl{symbol}}}
\newcommand{\localassums}{\textrm{\textsl{local\_assums}}}
\newcommand{\localdef}{\textrm{\textsl{local\_def}}}
\newcommand{\localdecls}{\textrm{\textsl{local\_decls}}}
\newcommand{\ident}{\textrm{\textsl{ident}}}
\newcommand{\accessident}{\textrm{\textsl{access\_ident}}}
\newcommand{\inductivebody}{\textrm{\textsl{ind\_body}}}
\newcommand{\inductive}{\textrm{\textsl{inductive}}}
\newcommand{\naturalnumber}{\textrm{\textsl{natural}}}
\newcommand{\integer}{\textrm{\textsl{integer}}}
\newcommand{\multpattern}{\textrm{\textsl{mult\_pattern}}}
\newcommand{\mutualcoinductive}{\textrm{\textsl{mutual\_coinductive}}}
\newcommand{\mutualinductive}{\textrm{\textsl{mutual\_inductive}}}
\newcommand{\nestedpattern}{\textrm{\textsl{nested\_pattern}}}
\newcommand{\num}{\textrm{\textsl{num}}}
\newcommand{\params}{\textrm{\textsl{params}}}
\newcommand{\pattern}{\textrm{\textsl{pattern}}}
\newcommand{\pat}{\textrm{\textsl{pat}}}
\newcommand{\pgs}{\textrm{\textsl{pgms}}}
\newcommand{\pg}{\textrm{\textsl{pgm}}}
\newcommand{\proof}{\textrm{\textsl{proof}}}
\newcommand{\record}{\textrm{\textsl{record}}}
\newcommand{\rewrule}{\textrm{\textsl{rewriting\_rule}}}
\newcommand{\sentence}{\textrm{\textsl{sentence}}}
\newcommand{\simplepattern}{\textrm{\textsl{simple\_pattern}}}
\newcommand{\sort}{\textrm{\textsl{sort}}}
\newcommand{\specif}{\textrm{\textsl{specif}}}
\newcommand{\statement}{\textrm{\textsl{statement}}}
\newcommand{\str}{\textrm{\textsl{string}}}
\newcommand{\subsequentletter}{\textrm{\textsl{subsequent\_letter}}}
\newcommand{\switch}{\textrm{\textsl{switch}}}
\newcommand{\tac}{\textrm{\textsl{tactic}}}
\newcommand{\terms}{\textrm{\textsl{terms}}}
\newcommand{\term}{\textrm{\textsl{term}}}
\newcommand{\module}{\textrm{\textsl{module}}}
\newcommand{\modexpr}{\textrm{\textsl{module\_expression}}}
\newcommand{\modtype}{\textrm{\textsl{module\_type}}}
\newcommand{\onemodbinding}{\textrm{\textsl{module\_binding}}}
\newcommand{\modbindings}{\textrm{\textsl{module\_bindings}}}
\newcommand{\qualid}{\textrm{\textsl{qualid}}}
\newcommand{\class}{\textrm{\textsl{class}}}
\newcommand{\dirpath}{\textrm{\textsl{dirpath}}}
\newcommand{\typedidents}{\textrm{\textsl{typed\_idents}}}
\newcommand{\type}{\textrm{\textsl{type}}}
\newcommand{\vref}{\textrm{\textsl{ref}}}
\newcommand{\zarithformula}{\textrm{\textsl{zarith\_formula}}}
\newcommand{\zarith}{\textrm{\textsl{zarith}}}

%%%%%%%%%%%%%%%%%%%%%%%%%%%%%%%%%%%%%%%%%%%%%%%%%%%%%%%
% \mbox{\sf } series for roman text in maths formulas %
%%%%%%%%%%%%%%%%%%%%%%%%%%%%%%%%%%%%%%%%%%%%%%%%%%%%%%%

\newcommand{\alors}{\mbox{\textsf{then}}}
\newcommand{\alter}{\mbox{\textsf{alter}}}
\newcommand{\bool}{\mbox{\textsf{bool}}}
\newcommand{\conc}{\mbox{\textsf{conc}}}
\newcommand{\cons}{\mbox{\textsf{cons}}}
\newcommand{\consf}{\mbox{\textsf{consf}}}
\newcommand{\emptyf}{\mbox{\textsf{emptyf}}}
\newcommand{\EqSt}{\mbox{\textsf{EqSt}}}
\newcommand{\false}{\mbox{\textsf{false}}}
\newcommand{\filter}{\mbox{\textsf{filter}}}
\newcommand{\forest}{\mbox{\textsf{forest}}}
\newcommand{\from}{\mbox{\textsf{from}}}
\newcommand{\hd}{\mbox{\textsf{hd}}}
\newcommand{\Length}{\mbox{\textsf{Length}}}
\newcommand{\length}{\mbox{\textsf{length}}}
\newcommand{\LengthA}{\mbox {\textsf{Length\_A}}}
\newcommand{\List}{\mbox{\textsf{List}}}
\newcommand{\ListA}{\mbox{\textsf{List\_A}}}
\newcommand{\LNil}{\mbox{\textsf{Lnil}}}
\newcommand{\LCons}{\mbox{\textsf{Lcons}}}
\newcommand{\nat}{\mbox{\textsf{nat}}}
\newcommand{\nO}{\mbox{\textsf{O}}}
\newcommand{\nS}{\mbox{\textsf{S}}}
\newcommand{\node}{\mbox{\textsf{node}}}
\newcommand{\Nil}{\mbox{\textsf{nil}}}
\newcommand{\Prop}{\mbox{\textsf{Prop}}}
\newcommand{\Set}{\mbox{\textsf{Set}}}
\newcommand{\si}{\mbox{\textsf{if}}}
\newcommand{\sinon}{\mbox{\textsf{else}}}
\newcommand{\Str}{\mbox{\textsf{Stream}}}
\newcommand{\tl}{\mbox{\textsf{tl}}}
\newcommand{\tree}{\mbox{\textsf{tree}}}
\newcommand{\true}{\mbox{\textsf{true}}}
\newcommand{\Type}{\mbox{\textsf{Type}}}
\newcommand{\unfold}{\mbox{\textsf{unfold}}}
\newcommand{\zeros}{\mbox{\textsf{zeros}}}

%%%%%%%%%
% Misc. %
%%%%%%%%%
\newcommand{\T}{\texttt{T}}
\newcommand{\U}{\texttt{U}}
\newcommand{\real}{\textsf{Real}}
\newcommand{\Spec}{\textit{Spec}}
\newcommand{\Data}{\textit{Data}}
\newcommand{\In} {{\textbf{in }}}
\newcommand{\AND} {{\textbf{and}}}
\newcommand{\If}{{\textbf{if }}}
\newcommand{\Else}{{\textbf{else }}}
\newcommand{\Then} {{\textbf{then }}}
\newcommand{\Let}{{\textbf{let }}}
\newcommand{\Where}{{\textbf{where rec }}}
\newcommand{\Function}{{\textbf{function }}}
\newcommand{\Rec}{{\textbf{rec }}}
\newcommand{\cn}{\centering}

%%%%%%%%%%%%%%%%%%%%%%%%%%%%%
% Math commands and symbols %
%%%%%%%%%%%%%%%%%%%%%%%%%%%%%

\newcommand{\la}{\leftarrow}
\newcommand{\ra}{\rightarrow}
\newcommand{\Ra}{\Rightarrow}
\newcommand{\rt}{\Rightarrow}
\newcommand{\lla}{\longleftarrow}
\newcommand{\lra}{\longrightarrow}
\newcommand{\Llra}{\Longleftrightarrow}
\newcommand{\mt}{\mapsto}
\newcommand{\ov}{\overrightarrow}
\newcommand{\wh}{\widehat}
\newcommand{\up}{\uparrow}
\newcommand{\dw}{\downarrow}
\newcommand{\nr}{\nearrow}
\newcommand{\se}{\searrow}
\newcommand{\sw}{\swarrow}
\newcommand{\nw}{\nwarrow}

\newcommand{\vm}[1]{\vspace{#1em}}
\newcommand{\vx}[1]{\vspace{#1ex}}
\newcommand{\hm}[1]{\hspace{#1em}}
\newcommand{\hx}[1]{\hspace{#1ex}}
\newcommand{\sm}{\mbox{ }}
\newcommand{\mx}{\mbox}

\newcommand{\nq}{\neq}
\newcommand{\eq}{\equiv}
\newcommand{\fa}{\forall}
\newcommand{\ex}{\exists}
\newcommand{\impl}{\rightarrow}
\newcommand{\Or}{\vee}
\newcommand{\And}{\wedge}
\newcommand{\ms}{\models}
\newcommand{\bw}{\bigwedge}
\newcommand{\ts}{\times}
\newcommand{\cc}{\circ}
\newcommand{\es}{\emptyset}
\newcommand{\bs}{\backslash}
\newcommand{\vd}{\vdash}
\newcommand{\lan}{{\langle }}
\newcommand{\ran}{{\rangle }}

\newcommand{\al}{\alpha}
\newcommand{\bt}{\beta}
\newcommand{\io}{\iota}
\newcommand{\lb}{\lambda}
\newcommand{\sg}{\sigma}
\newcommand{\sa}{\Sigma}
\newcommand{\om}{\Omega}
\newcommand{\tu}{\tau}

%%%%%%%%%%%%%%%%%%%%%%%%%
% Custom maths commands %
%%%%%%%%%%%%%%%%%%%%%%%%%

\newcommand{\sumbool}[2]{\{#1\}+\{#2\}}
\newcommand{\myifthenelse}[3]{\kw{if} ~ #1 ~\kw{then} ~ #2 ~ \kw{else} ~ #3}
\newcommand{\fun}[2]{\item[]{\tt {#1}}. \quad\\ #2}
\newcommand{\WF}[2]{\ensuremath{{\cal W\!F}(#1)[#2]}}
\newcommand{\WFE}[1]{\WF{E}{#1}}
\newcommand{\WT}[4]{\ensuremath{#1[#2] \vdash #3 : #4}}
\newcommand{\WTE}[3]{\WT{E}{#1}{#2}{#3}}
\newcommand{\WTEG}[2]{\WTE{\Gamma}{#1}{#2}}

\newcommand{\WTM}[3]{\WT{#1}{}{#2}{#3}}
\newcommand{\WFT}[2]{\ensuremath{#1[] \vdash {\cal W\!F}(#2)}}
\newcommand{\WS}[3]{\ensuremath{#1[] \vdash #2 <: #3}}
\newcommand{\WSE}[2]{\WS{E}{#1}{#2}}

\newcommand{\WTRED}[5]{\mbox{$#1[#2] \vdash #3 #4 #5$}}
\newcommand{\WTERED}[4]{\mbox{$E[#1] \vdash #2 #3 #4$}}
\newcommand{\WTELECONV}[3]{\WTERED{#1}{#2}{\leconvert}{#3}}
\newcommand{\WTEGRED}[3]{\WTERED{\Gamma}{#1}{#2}{#3}}
\newcommand{\WTECONV}[3]{\WTERED{#1}{#2}{\convert}{#3}}
\newcommand{\WTEGCONV}[2]{\WTERED{\Gamma}{#1}{\convert}{#2}}
\newcommand{\WTEGLECONV}[2]{\WTERED{\Gamma}{#1}{\leconvert}{#2}}

\newcommand{\lab}[1]{\mathit{labels}(#1)}
\newcommand{\dom}[1]{\mathit{dom}(#1)}

\newcommand{\CI}[2]{\mbox{$\{#1\}^{#2}$}}
\newcommand{\CIP}[3]{\mbox{$\{#1\}_{#2}^{#3}$}}
\newcommand{\CIPV}[1]{\CIP{#1}{I_1.. I_k}{P_1.. P_k}}
\newcommand{\CIPI}[1]{\CIP{#1}{I}{P}}
\newcommand{\CIF}[1]{\mbox{$\{#1\}_{f_1.. f_n}$}}
\newcommand{\NInd}[3]{\mbox{{\sf Ind}$(#1)(\begin{array}[t]{l}#2:=#3
                                              \,)\end{array}$}}
\newcommand{\Ind}[4]{\mbox{{\sf Ind}$(#1)[#2](\begin{array}[t]{l}#3:=#4
                                                 \,)\end{array}$}}
\newcommand{\Indp}[5]{\mbox{{\sf Ind}$_{#5}(#1)[#2](\begin{array}[t]{l}#3:=#4
                                                 \,)\end{array}$}}
\newcommand{\Def}[4]{\mbox{{\sf Def}$(#1)(#2:=#3:#4)$}}
\newcommand{\Assum}[3]{\mbox{{\sf Assum}$(#1)(#2:#3)$}}
\newcommand{\Match}[3]{\mbox{$<\!#1\!>\!{\mbox{\tt Match}}~#2~{\mbox{\tt with}}~#3~{\mbox{\tt end}}$}}
\newcommand{\Case}[3]{\mbox{$<\!#1\!>\!{\mbox{\tt Cases}}~#2~{\mbox{\tt of}}~#3~{\mbox{\tt end}}$}}
\newcommand{\Fix}[2]{\mbox{\tt Fix}~#1\{#2\}}
\newcommand{\CoFix}[2]{\mbox{\tt CoFix}~#1\{#2\}}
\newcommand{\With}[2]{\mbox{\tt ~with~}}
\newcommand{\subst}[3]{#1\{#2/#3\}}
\newcommand{\substs}[4]{#1\{(#2/#3)_{#4}\}}
\newcommand{\Sort}{\mbox{$\cal S$}}
\newcommand{\convert}{=_{\beta\delta\iota\zeta}}
\newcommand{\leconvert}{\leq_{\beta\delta\iota\zeta}}
\newcommand{\NN}{\mbox{I\hspace{-7pt}N}}
\newcommand{\inference}[1]{$${#1}$$}
\newcommand{\compat}[2]{\mbox{$[#1|#2]$}}
\newcommand{\tristackrel}[3]{\mathrel{\mathop{#2}\limits_{#3}^{#1}}}

\newcommand{\Impl}{{\it Impl}}
\newcommand{\Mod}[3]{{\sf Mod}({#1}:{#2}:={#3})}
\newcommand{\ModType}[2]{{\sf ModType}({#1}:={#2})}
\newcommand{\ModS}[2]{{\sf ModS}({#1}:{#2})}
\newcommand{\ModSEq}[3]{{\sf ModSEq}({#1}:{#2}=={#3})}
\newcommand{\functor}[3]{\ensuremath{{\sf Functor}[#1:#2]\;#3}}
\newcommand{\funsig}[3]{\ensuremath{{\sf Funsig}(#1:#2)\;#3}}
\newcommand{\sig}[1]{\ensuremath{{\sf Sig}~#1~{\sf End}}}
\newcommand{\struct}[1]{\ensuremath{{\sf Struct}~#1~{\sf End}}}


\newbox\tempa
\newbox\tempb
\newdimen\tempc
\newcommand{\mud}[1]{\hfil $\displaystyle{\mathstrut #1}$\hfil}
\newcommand{\rig}[1]{\hfil $\displaystyle{#1}$}
\newcommand{\irulehelp}[3]{\setbox\tempa=\hbox{$\displaystyle{\mathstrut #2}$}%
                        \setbox\tempb=\vbox{\halign{##\cr
        \mud{#1}\cr
        \noalign{\vskip\the\lineskip}
        \noalign{\hrule height 0pt}
        \rig{\vbox to 0pt{\vss\hbox to 0pt{${\; #3}$\hss}\vss}}\cr
        \noalign{\hrule}
        \noalign{\vskip\the\lineskip}
        \mud{\copy\tempa}\cr}}
                      \tempc=\wd\tempb
                      \advance\tempc by \wd\tempa
                      \divide\tempc by 2 }
\newcommand{\irule}[3]{{\irulehelp{#1}{#2}{#3}
                     \hbox to \wd\tempa{\hss \box\tempb \hss}}}

\newcommand{\sverb}[1]{\tt #1}
\newcommand{\mover}[2]{{#1\over #2}}
\newcommand{\jd}[2]{#1 \vdash #2}
\newcommand{\mathline}[1]{\[#1\]}
\newcommand{\zrule}[2]{#2: #1}
\newcommand{\orule}[3]{#3: {\mover{#1}{#2}}}
\newcommand{\trule}[4]{#4: \mover{#1  \qquad #2} {#3}}
\newcommand{\thrule}[5]{#5: {\mover{#1  \qquad #2 \qquad #3}{#4}}}


% $Id$ 


%%% Local Variables: 
%%% mode: latex
%%% TeX-master: "Reference-Manual"
%%% End: 

%\makeindex

%\begin{document}
%\coverpage{The module {\tt Equality}}{Cristina CORNES}

%\tableofcontents

\chapter{Tactics for inductive types and families}
\label{Addoc-equality}

This chapter details a few special tactics useful for inferring facts
from inductive hypotheses. They can be considered as tools that
macro-generate complicated uses of the basic elimination tactics for
inductive types. 

Sections \ref{inversion_introduction} to \ref{inversion_using}  present
inversion tactics and section \ref{scheme} describes
a command {\tt Scheme} for automatic generation of induction schemes
for mutual inductive types.

%\end{document}
%\documentstyle[11pt]{article}
%%%%%%%%%%%%%%%%%%%%%%%%%%%%%%%%%
% File title.tex
% Page formatting commands
% Macro \coverpage
%%%%%%%%%%%%%%%%%%%%%%%%%%%%%%%%

%\setlength{\marginparwidth}{0pt}
%\setlength{\oddsidemargin}{0pt}
%\setlength{\evensidemargin}{0pt}
%\setlength{\marginparsep}{0pt}
%\setlength{\topmargin}{0pt}
%\setlength{\textwidth}{16.9cm}
%\setlength{\textheight}{22cm}
%\usepackage{fullpage}

%\newcommand{\printingdate}{\today}
%\newcommand{\isdraft}{\Large\bf\today\\[20pt]}
%\newcommand{\isdraft}{\vspace{20pt}}

\newcommand{\coverpage}[3]{
\thispagestyle{empty}
\begin{center}
\bfseries % for the rest of this page, until \end{center}
\Huge
The Coq Proof Assistant\\[12pt]
#1\\[20pt]
\Large\today\\[20pt]
Version \coqversion%\footnote[1]{This research was partly supported by IST working group ``Types''}

\vspace{0pt plus .5fill}
#2
\par\vfill
The Coq Development Team

\vspace*{15pt}
\end{center}
\newpage

\thispagestyle{empty}
\hbox{}\vfill % without \hbox \vfill does not work at the top of the page
\begin{flushleft}
%BEGIN LATEX
V\coqversion, \today
\par\vspace{20pt}
%END LATEX
\copyright INRIA 1999-2004 ({\Coq} versions 7.x)

\copyright INRIA 2004-2010 ({\Coq} versions 8.x)

#3
\end{flushleft}
} % end of \coverpage definition


% \newcommand{\shorttitle}[1]{
% \begin{center}
% \begin{huge}
% \begin{bf}
% The Coq Proof Assistant\\
% \vspace{10pt}
%     #1\\
% \end{bf}
% \end{huge}
% \end{center}
% \vspace{5pt}
% }

% Local Variables: 
% mode: LaTeX
% TeX-master: ""
% End: 

% $Id$ 


%\begin{document}
%\coverpage{Module Inv: Inversion Tactics}{Cristina CORNES}

\section{Generalities about inversion}
\label{inversion_introduction}
When working with (co)inductive predicates, we are very often faced to
some of these situations:
\begin{itemize}
\item we have an inconsistent instance of an inductive predicate in the
  local context of hypotheses. Thus, the current goal can be trivially
  proved by absurdity. 

\item we have a hypothesis that is an instance of an inductive
  predicate, and the instance has some variables whose constraints we
  would like to derive.
\end{itemize}

The inversion tactics are very useful to simplify the work in these
cases. Inversion tools can be classified in three groups:
\begin{enumerate}
\item tactics for inverting an instance without stocking the inversion
  lemma in the context: 
  (\texttt{Dependent})  \texttt{Inversion} and
 (\texttt{Dependent}) \texttt{Inversion\_clear}.
\item commands for generating and stocking in the context the inversion
  lemma corresponding to an instance: \texttt{Derive}
  (\texttt{Dependent}) \texttt{Inversion}, \texttt{Derive}
  (\texttt{Dependent}) \texttt{Inversion\_clear}.
\item tactics for inverting an instance using an already defined
  inversion lemma: \texttt{Inversion \ldots using}.
\end{enumerate}

These tactics work for inductive types of arity $(\vec{x}:\vec{T})s$ 
where $s \in \{Prop,Set,Type\}$. Sections \ref{inversion_primitive},
\ref{inversion_derivation} and \ref{inversion_using} 
describe respectively each group of tools.

As inversion proofs may be large in size, we recommend the user to
stock the lemmas whenever the same instance needs to be inverted
several times.\\

Let's consider the relation \texttt{Le} over natural numbers and the
following variables:

\begin{coq_eval}
Restore State Initial.
\end{coq_eval}

\begin{coq_example*}
Inductive Le : nat->nat->Set :=
  LeO : (n:nat)(Le O n)  |  LeS : (n,m:nat) (Le n m)-> (Le (S n) (S m)).
Variable P:nat->nat->Prop.
Variable Q:(n,m:nat)(Le n m)->Prop.
\end{coq_example*}

For example purposes we defined  \verb+Le: nat->nat->Set+
 but we may have defined
it \texttt{Le} of type \verb+nat->nat->Prop+ or \verb+nat->nat->Type+.


\section{Inverting an instance}
\label{inversion_primitive}
\subsection{The non dependent case}
\begin{itemize}

\item \texttt{Inversion\_clear} \ident~\\
\index{Inversion-clear@{\tt Inversion\_clear}}
  Let the type of \ident~ in the local context be $(I~\vec{t})$,
  where $I$ is a (co)inductive predicate. Then,
  \texttt{Inversion} applied to \ident~ derives for each possible
  constructor $c_i$ of $(I~\vec{t})$, {\bf all} the necessary
  conditions that should hold for the instance $(I~\vec{t})$ to be
  proved by $c_i$. Finally it erases \ident~ from the context.



For example, consider the goal:
\begin{coq_eval}
Lemma ex : (n,m:nat)(Le (S n) m)->(P n m).
Intros.
\end{coq_eval}

\begin{coq_example}
Show.
\end{coq_example}

To prove the goal we may need to reason by cases on \texttt{H} and to 
 derive that \texttt{m}  is necessarily of
the form $(S~m_0)$ for certain $m_0$ and that $(Le~n~m_0)$.  
Deriving these conditions corresponds to prove that the
only possible constructor of \texttt{(Le (S n) m)}  is
\texttt{LeS} and that we can invert the 
\texttt{->} in the type  of \texttt{LeS}.  
This inversion is possible because \texttt{Le} is the smallest set closed by
the constructors \texttt{LeO} and \texttt{LeS}.


\begin{coq_example}
Inversion_clear  H.
\end{coq_example}

Note that \texttt{m} has been substituted in the goal for \texttt{(S m0)}
and that the hypothesis \texttt{(Le n m0)} has been added to the
context.

\item \texttt{Inversion} \ident~\\
\index{Inversion@{\tt Inversion}}
  This tactic differs from   {\tt Inversion\_clear} in the fact that
  it adds the equality constraints in the context and
  it does not erase  the hypothesis \ident.


In the previous example, {\tt Inversion\_clear} 
has substituted \texttt{m} by \texttt{(S m0)}. Sometimes it is
interesting to have the equality \texttt{m=(S m0)} in the
context to use it after. In that case we can use  \texttt{Inversion} that
does not clear the equalities:

\begin{coq_example*}
Undo.
\end{coq_example*}
\begin{coq_example}
Inversion H.
\end{coq_example}

\begin{coq_eval}
Undo.
\end{coq_eval}

Note that the hypothesis \texttt{(S m0)=m} has been deduced and 
\texttt{H} has not been cleared from the context.

\end{itemize}

\begin{Variants}

\item \texttt{Inversion\_clear } \ident~ \texttt{in} \ident$_1$ \ldots
  \ident$_n$\\ 
\index{Inversion_clear...in@{\tt Inversion\_clear...in}}
  Let \ident$_1$ \ldots \ident$_n$, be identifiers in the local context. This
  tactic behaves as generalizing \ident$_1$ \ldots \ident$_n$, and then performing
  {\tt Inversion\_clear}.

\item \texttt{Inversion } \ident~ \texttt{in} \ident$_1$ \ldots \ident$_n$\\
\index{Inversion ... in@{\tt Inversion ... in}}
  Let \ident$_1$ \ldots \ident$_n$, be identifiers in the local context. This
  tactic behaves as generalizing \ident$_1$ \ldots \ident$_n$, and then performing
  \texttt{Inversion}.


\item \texttt{Simple Inversion} \ident~ \\
\index{Simple Inversion@{\tt Simple Inversion}}
  It is a very primitive inversion tactic that derives all the necessary
  equalities  but it does not simplify
  the  constraints as \texttt{Inversion} and
  {\tt Inversion\_clear} do.

\end{Variants}


\subsection{The dependent case}
\begin{itemize}
\item \texttt{Dependent Inversion\_clear} \ident~\\
\index{Dependent Inversion-clear@{\tt Dependent Inversion\_clear}}
  Let the type of \ident~ in the local context be $(I~\vec{t})$,
  where $I$ is a (co)inductive predicate, and let the goal  depend both on
  $\vec{t}$ and \ident. Then, 
  \texttt{Dependent Inversion\_clear} applied to \ident~ derives 
   for each possible constructor $c_i$ of $(I~\vec{t})$, {\bf all} the
  necessary conditions that should hold for the instance $(I~\vec{t})$ to be
  proved by $c_i$. It also substitutes  \ident~ for the corresponding
  term  in the goal and  it erases \ident~ from the context.


For example, consider the  goal:
\begin{coq_eval}
Lemma ex_dep : (n,m:nat)(H:(Le (S n) m))(Q (S n) m H).
Intros.
\end{coq_eval}

\begin{coq_example}
Show.
\end{coq_example}

As \texttt{H} occurs in the goal, we may want to reason by cases on its
structure and so, we would like  inversion tactics to
substitute \texttt{H} by the corresponding term in constructor form. 
Neither \texttt{Inversion} nor  {\tt Inversion\_clear} make such a
substitution. To have such a behavior we use the dependent inversion tactics:

\begin{coq_example}
Dependent Inversion_clear H.
\end{coq_example}

Note that \texttt{H} has been substituted by \texttt{(LeS n m0 l)} and
\texttt{m} by \texttt{(S m0)}.


\end{itemize}

\begin{Variants}

\item \texttt{Dependent Inversion\_clear } \ident~ \texttt{ with } \term\\
\index{Dependent Inversion_clear...with@{\tt Dependent Inversion\_clear...with}}
 \noindent Behaves as \texttt{Dependent Inversion\_clear} but allows to give
  explicitly the good generalization of the goal. It is useful when
  the system fails to generalize the goal automatically. If
  \ident~ has type $(I~\vec{t})$ and $I$ has type
  $(\vec{x}:\vec{T})s$,   then \term~  must be of type
  $I:(\vec{x}:\vec{T})(I~\vec{x})\rightarrow s'$ where $s'$ is the
  type of the goal.



\item \texttt{Dependent Inversion} \ident~\\
\index{Dependent Inversion@{\tt Dependent Inversion}}
 This tactic differs from   \texttt{Dependent Inversion\_clear} in the fact that
  it also  adds the equality constraints in the context and
  it does not erase  the hypothesis \ident~.

\item \texttt{Dependent Inversion } \ident~ \texttt{ with } \term \\
\index{Dependent Inversion...with@{\tt Dependent Inversion...with}}
  Analogous to \texttt{Dependent Inversion\_clear .. with..} above. 
\end{Variants}



\section{Deriving the inversion lemmas}
\label{inversion_derivation}
\subsection{The non dependent case}

The tactics (\texttt{Dependent}) \texttt{Inversion} and (\texttt{Dependent})
{\tt Inversion\_clear} work on a
certain instance $(I~\vec{t})$ of an inductive predicate. At each
application, they inspect the given instance and derive the
corresponding inversion lemma.  If we have to invert the same
instance several times it is recommended to stock the lemma in the
context and to reuse it whenever we need it.

The families of commands \texttt{Derive Inversion}, \texttt{Derive
Dependent Inversion}, \texttt{Derive} \\ {\tt Inversion\_clear} and \texttt{Derive Dependent Inversion\_clear}
allow to generate inversion lemmas for given instances and sorts.  Next
section describes the tactic \texttt{Inversion}$\ldots$\texttt{using} that refines the
goal with a specified inversion lemma.

\begin{itemize}

\item \texttt{Derive Inversion\_clear} \ident~ \texttt{with}
  $(\vec{x}:\vec{T})(I~\vec{t})$ \texttt{Sort} \sort~ \\ 
\index{Derive Inversion_clear...with@{\tt Derive Inversion\_clear...with}}
  Let $I$ be an inductive predicate and $\vec{x}$ the variables
  occurring in $\vec{t}$. This command generates and stocks
  the inversion lemma for the sort  \sort~  corresponding to the instance
  $(\vec{x}:\vec{T})(I~\vec{t})$ with the name \ident~ in the {\bf
    global} environment. When applied it is equivalent to have
  inverted the instance with the tactic {\tt Inversion\_clear}.


  For example, to generate the inversion lemma for the instance
 \texttt{(Le (S n) m)} and the sort \texttt{Prop} we do:
\begin{coq_example}
Derive Inversion_clear leminv with (n,m:nat)(Le (S n) m) Sort Prop.
\end{coq_example}

Let us  inspect the type of the generated lemma:
\begin{coq_example}
Check leminv.
\end{coq_example}



\end{itemize}

%\variants
%\begin{enumerate} 
%\item \verb+Derive Inversion_clear+  \ident$_1$ \ident$_2$ \\ 
%\index{Derive Inversion_clear@{\tt Derive Inversion\_clear}}
%  Let \ident$_1$ have type $(I~\vec{t})$ in the local context ($I$
%  an inductive predicate).  Then, this command has the same semantics
%  as \verb+Derive Inversion_clear+ \ident$_2$~ \verb+with+
%  $(\vec{x}:\vec{T})(I~\vec{t})$ \verb+Sort Prop+ where $\vec{x}$ are the free
%  variables of $(I~\vec{t})$ declared in the local context (variables
%  of the global context are considered as constants). 
%\item \verb+Derive Inversion+ \ident$_1$~ \ident$_2$~\\
%\index{Derive Inversion@{\tt Derive Inversion}}
% Analogous to the previous command. 
%\item \verb+Derive Inversion+ $num$ \ident~ \ident~ \\ 
%\index{Derive Inversion@{\tt Derive Inversion}}
%  This command behaves as \verb+Derive Inversion+ \ident~ {\it
%    namehyp} performed on the goal number $num$.
%
%\item \verb+Derive Inversion_clear+ $num$ \ident~ \ident~ \\ 
%\index{Derive Inversion_clear@{\tt Derive Inversion\_clear}}
%  This command behaves as \verb+Derive Inversion_clear+ \ident~
%  \ident~ performed on the goal number $num$.
%\end{enumerate}



A derived inversion lemma is  adequate for inverting the instance
with which it was generated,  \texttt{Derive} applied to
different instances yields different lemmas. In general, if we generate
the inversion lemma with 
an instance $(\vec{x}:\vec{T})(I~\vec{t})$ and a sort $s$,  the inversion lemma will
expect a predicate of type $(\vec{x}:\vec{T})s$ as first argument. \\

\begin{Variant}
\item \texttt{Derive Inversion} \ident~ \texttt{with} 
  $(\vec{x}:\vec{T})(I~\vec{t})$ \texttt{Sort} \sort\\ 
\index{Derive Inversion...with@{\tt Derive Inversion...with}}
     Analogous of \texttt{Derive Inversion\_clear .. with ..} but 
 when applied it is equivalent to having
  inverted the instance with the tactic \texttt{Inversion}.
\end{Variant}

\subsection{The dependent case}
\begin{itemize}
\item \texttt{Derive Dependent Inversion\_clear} \ident~ \texttt{with}
  $(\vec{x}:\vec{T})(I~\vec{t})$ \texttt{Sort} \sort~ \\ 
\index{Derive Dependent Inversion\_clear...with@{\tt Derive Dependent Inversion\_clear...with}}
  Let $I$ be an inductive predicate. This command generates and stocks
  the dependent inversion lemma for the sort  \sort~  corresponding to the instance
  $(\vec{x}:\vec{T})(I~\vec{t})$ with the name \ident~ in the {\bf
    global} environment. When applied it is equivalent to having
  inverted the instance with the tactic \texttt{Dependent Inversion\_clear}.
\end{itemize}

\begin{coq_example}
Derive Dependent Inversion_clear leminv_dep 
  with (n,m:nat)(Le (S n) m) Sort Prop.
\end{coq_example}

\begin{coq_example}
Check leminv_dep.
\end{coq_example}

\begin{Variants}
\item \texttt{Derive Dependent Inversion} \ident~ \texttt{with}
  $(\vec{x}:\vec{T})(I~\vec{t})$ \texttt{Sort} \sort~ \\ 
\index{Derive Dependent Inversion...with@{\tt Derive Dependent Inversion...with}}
  Analogous to \texttt{Derive Dependent Inversion\_clear}, but when
  applied  it is equivalent to having
  inverted the instance with the tactic \texttt{Dependent Inversion}.

\end{Variants}

\section{Using already defined inversion  lemmas}
\label{inversion_using}
\begin{itemize}
\item \texttt{Inversion} \ident \texttt{ using} \ident$'$ \\
\index{Inversion...using@{\tt Inversion...using}}
  Let \ident~ have type $(I~\vec{t})$ ($I$ an inductive
  predicate) in the local context, and \ident$'$ be a (dependent) inversion
  lemma. Then, this tactic refines the current goal with the specified
  lemma. 


\begin{coq_eval}
Abort.
\end{coq_eval}

\begin{coq_example}
Show.
\end{coq_example}
\begin{coq_example}
Inversion H using leminv.
\end{coq_example}


\end{itemize}
\variant
\begin{enumerate}
\item \texttt{Inversion} \ident~ \texttt{using} \ident$'$ \texttt{in} \ident$_1$\ldots \ident$_n$\\
\index{Inversion...using...in@{\tt Inversion...using...in}}
This tactic behaves as generalizing  \ident$_1$\ldots \ident$_n$,
then doing \texttt{Use Inversion} \ident~\ident$'$.
\end{enumerate}

\section{\tt Scheme ...}\index{Scheme@{\tt Scheme}}\label{Scheme}
\label{scheme}
The {\tt Scheme} command is a high-level tool for generating
automatically (possibly mutual) induction principles for given types
and sorts.  Its syntax follows the schema :

\noindent
{\tt Scheme {\ident$_1$} := Induction for \term$_1$ Sort {\sort$_1$} \\
  with\\
  \mbox{}\hspace{0.1cm} .. \\
        with {\ident$_m$} := Induction for {\term$_m$} Sort
        {\sort$_m$}}\\
\term$_1$ \ldots \term$_m$ are different inductive types belonging to
the same package of mutual inductive definitions. This command
generates {\ident$_1$}\ldots{\ident$_m$} to be mutually recursive
definitions. Each term {\ident$_i$} proves a general principle 
of mutual induction for objects in type {\term$_i$}. 

\Example
The definition of principle of mutual induction for {\tt tree} and
{\tt forest} over the sort {\tt Set} is defined by the command:
\begin{coq_eval}
Restore State Initial.
Variables A,B:Set.
Mutual Inductive tree : Set :=  node : A -> forest -> tree
with forest : Set := leaf : B -> forest 
                   | cons : tree -> forest -> forest.
\end{coq_eval}
\begin{coq_example*}
Scheme tree_forest_rec := Induction for tree Sort Set 
with forest_tree_rec := Induction for forest Sort Set.
\end{coq_example*}
You may now look at the type of {\tt tree\_forest\_rec} :
\begin{coq_example}
Check tree_forest_rec.
\end{coq_example}
This principle involves two different predicates for {\tt trees} and
{\tt forests}; it also has three premises each one corresponding to a
constructor of one of the inductive definitions.

The principle {\tt tree\_forest\_rec} shares exactly the same
premises, only the conclusion now refers to the property of forests.
\begin{coq_example}
Check forest_tree_rec.
\end{coq_example}

\begin{Variant}
\item {\tt Scheme {\ident$_1$} := Minimality for \term$_1$ Sort {\sort$_1$} \\
  with\\
  \mbox{}\hspace{0.1cm} .. \\
        with {\ident$_m$} := Minimality for {\term$_m$} Sort
        {\sort$_m$}}\\
Same as before but defines a non-dependent elimination principle more
natural in case of inductively defined relations. 
\end{Variant}

\Example
With the predicates {\tt odd} and {\tt even} inductively defined as:
\begin{coq_eval}
Restore State Initial.
\end{coq_eval}
\begin{coq_example*}
Mutual Inductive odd : nat->Prop := 
    oddS : (n:nat)(even n)->(odd (S n))
with even : nat -> Prop := 
    evenO : (even O) 
  | evenS : (n:nat)(odd n)->(even (S n)).  
\end{coq_example*}
The following command generates a powerful elimination
principle:
\begin{coq_example*}
Scheme odd_even := Minimality for odd Sort Prop
with   even_odd := Minimality for even Sort Prop.
\end{coq_example*}
The type of {\tt odd\_even} for instance will be:
\begin{coq_example}
Check odd_even.
\end{coq_example}
The type of {\tt even\_odd} shares the same premises but the
conclusion is {\tt (n:nat)(even n)->(Q n)}.



%\end{document}

% $Id$ 
