\achapter{The \texttt{Ring} tactic}
\aauthor{Patrick Loiseleur and Samuel Boutin}
\label{Ring}
\tacindex{Ring}

This chapter presents  the \texttt{Ring} tactic.

\asection{What does this tactic?}

\texttt{Ring} does associative-commutative rewriting in ring and semi-ring
structures. Assume you have two binary functions $\oplus$ and $\otimes$
that are associative and commutative, with $\oplus$ distributive on
$\otimes$, and two constants 0 and 1 that are unities for $\oplus$ and
$\otimes$. A \textit{polynomial} is an expression built on variables $V_0, V_1,
\dots$ and constants by application of $\oplus$ and $\otimes$.

Let an {\it ordered product} be a product of variables $V_{i_1} \otimes
\ldots \otimes V_{i_n}$ verifying $i_1 \le i_2 \le \dots \le i_n$. Let a
\textit{monomial} be the product of a constant (possibly equal to 1, in
which case we omit it) and an ordered product.  We can order the
monomials by the lexicographic order on products of variables. Let a
\textit{canonical sum} be an ordered sum of monomials that are all
different, i.e. each monomial in the sum is strictly less than the
following monomial according to the lexicographic order. It is an easy
theorem to show that every polynomial is equivalent (modulo the ring
properties) to exactly one canonical sum. This canonical sum is called
the \textit{normal form} of the polynomial. So what does \texttt{Ring}? It
normalizes polynomials over any ring or semi-ring structure. The basic
utility of \texttt{Ring} is to simplify ring expressions, so that the user
does not have to deal manually with the theorems of associativity and
commutativity.

\begin{Examples}
\item In the ring of integers, the normal form of 
$x (3 + yx + 25(1 - z)) + zx$ is $28x + (-24)xz + xxy$.
\item For the classical propositional calculus (or the boolean rings)
  the normal form is what logicians call \textit{disjunctive normal
    form}: every formula is equivalent to a disjunction of
  conjunctions of atoms. (Here $\oplus$ is $\vee$, $\otimes$ is
  $\wedge$, variables are atoms and the only constants are T and F)
\end{Examples}

\asection{The variables map}

It is frequent to have an expression built with + and
  $\times$, but rarely on variables only.
Let us associate a number to each subterm of a ring
expression in the \gallina\ language. For example in the ring
\texttt{nat}, consider the expression:

\begin{quotation}
\begin{verbatim}
(plus (mult (plus (f (5)) x) x)
      (mult (if b then (4) else (f (3))) (2)))
\end{verbatim}
\end{quotation}

\noindent As a ring expression, is has 3 subterms. Give each subterm a
number in an arbitrary order:

\begin{tabular}{ccl}
0 & $\mapsto$ & \verb|if b then (4) else (f (3))| \\
1 & $\mapsto$ & \verb|(f (5))| \\
2 & $\mapsto$ & \verb|x| \\
\end{tabular}

\noindent Then normalize the ``abstract'' polynomial 

$$((V_1 \otimes V_2) \oplus V_2) \oplus (V_0 \otimes 2) $$

\noindent In our example the normal form is:

$$(2 \otimes V_0) \oplus (V_1 \otimes V_2) \oplus (V_2 \otimes V_2)$$

\noindent Then substitute the variables by their values in the variables map to
get the concrete normal polynomial:

\begin{quotation}
\begin{verbatim}
(plus (mult (2) (if b then (4) else (f (3)))) 
      (plus (mult (f (5)) x) (mult x x))) 
\end{verbatim}
\end{quotation}

\asection{Is it automatic?}

Yes, building the variables map and doing the substitution after
normalizing is automatically done by the tactic. So you can just forget
this paragraph and use the tactic according to your intuition.

\asection{Concrete usage in \Coq}

Under a session launched by \texttt{coqtop} or \texttt{coqtop -full},
load the \texttt{Ring} files with the command:

\begin{quotation}
\begin{verbatim}
Require Ring.
\end{verbatim}
\end{quotation}

It does not work under \texttt{coqtop -opt} because the compiled ML
objects used by the tactic are not linked in this binary image, and
dynamic loading of native code is not possible in \ocaml.

In order to use \texttt{Ring} on naturals, load \texttt{ArithRing}
instead; for binary integers, load the module \texttt{ZArithRing}.

Then, to normalize the terms $term_1$, \dots, $term_n$ in
the current subgoal, use the tactic:

\begin{quotation}
\texttt{Ring} $term_1$ \dots{} $term_n$
\end{quotation}
\tacindex{Ring}

Then the tactic guesses the type of given terms, the ring theory to
use, the variables map, and replace each term with its normal form.
The variables map is common to all terms

\Warning \texttt{Ring $term_1$; Ring $term_2$} is not equivalent to 
\texttt{Ring $term_1$ $term_2$}. In the latter case the variables map
is shared between the two terms, and common subterm $t$ of $term_1$
and $term_2$ will have the same associated variable number.

\begin{ErrMsgs}
\item \errindex{All terms must have the same type}
\item \errindex{Don't know what to do with this goal}
\item \errindex{No Declared Ring Theory for \term.}\\
  \texttt{Use Add [Semi] Ring to declare it}\\
  That happens when all terms have the same type \term, but there is
  no declared ring theory for this set. See below.
\end{ErrMsgs}

\begin{Variants}
\item \texttt{Ring}\\
  That works if the current goal is an equality between two
  polynomials. It will normalize both sides of the
  equality, solve it if the normal forms are equal and in other cases
  try to simplify the equality using \texttt{congr\_eqT} and \texttt{refl\_equal}
  to reduce $x + y = x + z$ to $y = z$ and $x * z = x * y$ to $y = z$.

  \ErrMsg\errindex{This goal is not an equality}

\end{Variants}

\asection{Add a ring structure}

It can be done in the \Coq toplevel (No ML file to edit and to link
with \Coq). First, \texttt{Ring} can handle two kinds of structure:
rings and semi-rings. Semi-rings are like rings without an opposite to
addition. Their precise specification (in \gallina) can be found in
the file

\begin{quotation}
\begin{verbatim}
tactics/contrib/polynom/Ring_theory.v
\end{verbatim}
\end{quotation}

The typical example of ring is \texttt{Z}, the typical
example of semi-ring is \texttt{nat}.

The specification of a
ring is divided in two parts: first the record of constants
($\oplus$, $\otimes$, 1, 0, $\ominus$) and then the theorems
(associativity, commutativity, etc.).

\begin{small}
\begin{flushleft}
\begin{verbatim}
Section Theory_of_semi_rings.

Variable A : Type.
Variable Aplus : A -> A -> A.
Variable Amult : A -> A -> A.
Variable Aone : A.
Variable Azero : A.
(* There is also a "weakly decidable" equality on A. That means 
  that if (A_eq x y)=true then x=y but x=y can arise when 
  (A_eq x y)=false. On an abstract ring the function [x,y:A]false
  is a good choice. The proof of A_eq_prop is in this case easy. *)
Variable Aeq : A -> A -> bool.

Record Semi_Ring_Theory : Prop :=
{ SR_plus_sym  : (n,m:A)[| n + m == m + n |];
  SR_plus_assoc : (n,m,p:A)[| n + (m + p) == (n + m) + p |];

  SR_mult_sym : (n,m:A)[| n*m == m*n |];
  SR_mult_assoc : (n,m,p:A)[| n*(m*p) == (n*m)*p |];
  SR_plus_zero_left :(n:A)[| 0 + n == n|];
  SR_mult_one_left : (n:A)[| 1*n == n |];
  SR_mult_zero_left : (n:A)[| 0*n == 0 |];
  SR_distr_left   : (n,m,p:A) [| (n + m)*p == n*p + m*p |];
  SR_plus_reg_left : (n,m,p:A)[| n + m == n + p |] -> m==p;
  SR_eq_prop : (x,y:A) (Is_true (Aeq x y)) -> x==y
}.
\end{verbatim}
\end{flushleft}
\end{small}

\begin{small}
\begin{flushleft}
\begin{verbatim}
Section Theory_of_rings.

Variable A : Type.

Variable Aplus : A -> A -> A.
Variable Amult : A -> A -> A.
Variable Aone : A.
Variable Azero : A.
Variable Aopp : A -> A.
Variable Aeq : A -> A -> bool.


Record Ring_Theory : Prop :=
{ Th_plus_sym  : (n,m:A)[| n + m == m + n |];
  Th_plus_assoc : (n,m,p:A)[| n + (m + p) == (n + m) + p |];
  Th_mult_sym : (n,m:A)[| n*m == m*n |];
  Th_mult_assoc : (n,m,p:A)[| n*(m*p) == (n*m)*p |];
  Th_plus_zero_left :(n:A)[| 0 + n == n|];
  Th_mult_one_left : (n:A)[| 1*n == n |];
  Th_opp_def : (n:A) [| n + (-n) == 0 |];
  Th_distr_left   : (n,m,p:A) [| (n + m)*p == n*p + m*p |];
  Th_eq_prop : (x,y:A) (Is_true (Aeq x y)) -> x==y
}.
\end{verbatim}
\end{flushleft}
\end{small}

To define a ring structure on A, you must provide an addition, a
multiplication, an opposite function and two unities 0 and 1.

You must then prove all theorems that make
(A,Aplus,Amult,Aone,Azero,Aeq) 
a ring structure, and pack them with the \verb|Build_Ring_Theory| 
constructor.

Finally to register a ring the syntax is:

\comindex{Add Ring}
\begin{quotation}
  \texttt{Add Ring} \textit{A Aplus Amult Aone Azero Ainv Aeq T}
  \texttt{[} \textit{c1 \dots cn} \texttt{].}
\end{quotation}

\noindent where \textit{A} is a term of type \texttt{Set}, 
\textit{Aplus} is a term of type \texttt{A->A->A},
\textit{Amult} is a term of type \texttt{A->A->A},
\textit{Aone} is a term of type \texttt{A},
\textit{Azero} is a term of type \texttt{A},
\textit{Ainv} is a term of type \texttt{A->A},
\textit{Aeq} is a term of type \texttt{A->bool},
\textit{T} is a term of type 
\texttt{(Ring\_Theory }\textit{A Aplus Amult Aone Azero Ainv
  Aeq}\texttt{)}.
The arguments \textit{c1 \dots cn}, 
are the names of constructors which define closed terms: a
subterm will be considered as a constant if it is either one of the
terms \textit{c1 \dots cn} or the application of one of these terms to
closed terms. For \texttt{nat}, the given constructors are \texttt{S}
and \texttt{O}, and the closed terms are \texttt{O}, \texttt{(S O)},
\texttt{(S (S O))}, \ldots

\begin{Variants}
\item \texttt{Add Semi Ring} \textit{A Aplus Amult Aone Azero Aeq T} 
  \texttt{[} \textit{c1 \dots\ cn} \texttt{].}\comindex{Add Semi
    Ring}\\
  There are two differences with the \texttt{Add Ring} command: 
  there is no inverse function and the term $T$ must be of type 
  \texttt{(Semi\_Ring\_Theory }\textit{A Aplus Amult Aone Azero
  Aeq}\texttt{)}.  

\item \texttt{Add Abstract Ring} \textit{A Aplus Amult Aone Azero Ainv 
    Aeq T}\texttt{.}\comindex{Add Abstract Ring}\\
  This command should be used for when the operations of rings are not 
  computable; for example the real numbers of
  \texttt{theories/REALS/}. Here $0+1$ is not beta-reduced to $1$ but
  you still may want to \textit{rewrite} it to $1$ using the ring
  axioms. The argument \texttt{Aeq} is not used; a good choice for
  that function is \verb+[x:A]false+.

\item \texttt{Add Abstract Semi Ring} \textit{A Aplus Amult Aone Azero
    Aeq T}\texttt{.}\comindex{Add Abstract Semi Ring}\\

\end{Variants}

\begin{ErrMsgs}
\item \errindex{Not a valid (semi)ring theory}.\\ 
  That happens when the typing condition does not hold.
\end{ErrMsgs}

Currently, the hypothesis is made than no more than one ring structure
may be declared for a given type in \texttt{Set} or \texttt{Type}.
This allows automatic detection of the theory used to achieve the
normalization. On popular demand, we can change that and allow several
ring structures on the same set.

The table of theories of \texttt{Ring} is compatible with the \Coq\ 
sectioning mechanism. If you declare a Ring inside a section, the
declaration will be thrown away when closing the section.
And when you load a compiled file, all the \texttt{Add Ring}
commands of this file that are not inside a section will be loaded.

The typical example of ring is \texttt{Z}, and the typical example of
semi-ring is \texttt{nat}. Another ring structure is defined on the
booleans. 

\Warning Only the ring of booleans is loaded by default with the
\texttt{Ring} module. To load the ring structure for \texttt{nat},
load the module \texttt{ArithRing}, and for \texttt{Z},
load the module \texttt{ZArithRing}.


\asection{How does it work?}

The code of \texttt{Ring} a good  example of tactic written using
\textit{reflection} (or \textit{internalization}, it is 
synonymous). What is reflection? Basically,  it is writing \Coq\ tactics
in \Coq, rather than in \ocaml. From the philosophical point of view,
it is using the ability of the Calculus of Constructions to speak and
reason about itself.
For the \texttt{Ring} tactic we used 
\Coq\ as a programming language and also as a proof environment to build a
tactic and to prove it correctness.

The interested reader is strongly advised to have a look at the file
\texttt{Ring\_normalize.v}. Here a type for polynomials is defined: 

\begin{small}
\begin{flushleft}
\begin{verbatim}
Inductive Type polynomial := 
  Pvar : idx -> polynomial
| Pconst : A -> polynomial
| Pplus : polynomial -> polynomial -> polynomial
| Pmult : polynomial -> polynomial -> polynomial
| Popp : polynomial -> polynomial.
\end{verbatim}
\end{flushleft}
\end{small}

There is also a type to represent variables maps, and an
interpretation function, that maps a variables map and a polynomial to an
element of the concrete ring:

\begin{small}
\begin{flushleft}
\begin{verbatim}
Definition polynomial_simplify := [...]
Definition interp : (varmap A) -> (polynomial A) -> A := [...]
\end{verbatim}
\end{flushleft}
\end{small}

A function to normalize polynomials is defined, and the big theorem is
its correctness w.r.t interpretation, that is:

\begin{small}
\begin{flushleft}
\begin{verbatim}
Theorem polynomial_simplify_correct : (v:(varmap A))(p:polynomial)
  (interp v (polynomial_simplify p))
  ==(interp v p).
\end{verbatim}
\end{flushleft}
\end{small}

(The actual code is slightly more complex: for efficiency,
there is a special datatype to represent normalized polynomials,
i.e. ``canonical sums''. But the idea is still the same).

So now, what is the scheme for a normalization proof? Let \texttt{p}
be the polynomial expression that the user wants to normalize. First a
little piece of ML code guesses the type of \texttt{p}, the ring
theory \texttt{T} to use, an abstract polynomial \texttt{ap} and a
variables map \texttt{v} such that \texttt{p} is
$\beta\delta\iota$-equivalent to \verb|(interp v ap)|. Then we
replace it by \verb|(interp v (polynomial_simplify ap))|, using the
main correctness theorem and we reduce it to a concrete expression
\texttt{p'}, which is the concrete normal form of
\texttt{p}. This is summarized in this diagram:

\medskip
\begin{tabular}{rcl}
\texttt{p} & $\rightarrow_{\beta\delta\iota}$  
   & \texttt{(interp v ap)} \\
 & & $=_{\mathrm{(by\ the\ main\ correctness\ theorem)}}$ \\
\texttt{p'} 
   & $\leftarrow_{\beta\delta\iota}$ 
   & \texttt{(interp v (polynomial\_simplify ap))}
\end{tabular}
\medskip

The user do not see the right part of the diagram. 
From outside, the tactic behaves like a
$\beta\delta\iota$ simplification extended with AC rewriting rules.
Basically, the proof is only the application of the main
correctness theorem to well-chosen arguments.

\asection{History of \texttt{Ring}}

First Samuel Boutin designed the tactic \texttt{ACDSimpl}. 
This tactic did lot of rewriting. But the proofs
terms generated by rewriting were too big for \Coq's type-checker.
Let us see why:

\begin{coq_eval}
Require ZArith.
\end{coq_eval}
\begin{coq_example}
Goal (x,y,z:Z)`x + 3 + y + y*z = x + 3 + y + z*y`.
\end{coq_example}
\begin{coq_example*}
Intros; Rewrite (Zmult_sym y z); Reflexivity. 
Save toto.
\end{coq_example*}
\begin{coq_example}
Print toto.
\end{coq_example}

At each step of rewriting, the whole context is duplicated in the proof
term. Then, a tactic that does hundreds of rewriting generates huge proof
terms. Since \texttt{ACDSimpl} was too slow, Samuel Boutin rewrote it
using reflection (see his article in TACS'97 \cite{Bou97}). Later, the
stuff was rewritten by Patrick
Loiseleur: the new tactic does not any more require \texttt{ACDSimpl}
to compile and it makes use of $\beta\delta\iota$-reduction 
not only to replace the rewriting steps, but also to achieve the
interleaving of computation and 
reasoning (see \ref{DiscussReflection}). He also wrote a
few ML code for the \texttt{Add Ring} command, that allow to register
new rings dynamically.

Proofs terms generated by \texttt{Ring} are quite small, they are
linear in the number of $\oplus$ and $\otimes$ operations in the
normalized terms. Type-checking those terms requires some time because it
makes a large use of the conversion rule, but
memory requirements are much smaller. 

\asection{Discussion}
\label{DiscussReflection}

Efficiency is not the only motivation to use reflection
here. \texttt{Ring} also deals with constants, it rewrites for example the
expression $34 + 2*x -x + 12$ to the expected result $x + 46$. For the
tactic \texttt{ACDSimpl}, the only constants were 0 and 1. So the
expression $34 + 2*(x - 1) + 12$ is interpreted as 
$V_0 \oplus V_1 \otimes (V_2 \ominus 1) \oplus V_3$, 
with the variables mapping 
$\{V_0 \mt 34; V_1 \mt 2; V_2 \mt x; V_3 \mt 12 \}$. Then it is
rewritten to $34 - x + 2*x + 12$, very far from the expected
result. Here rewriting is not sufficient: you have to do some kind of
reduction (some kind of \textit{computation}) to achieve the
normalization.

The tactic \texttt{Ring} is not only faster than a classical one:
using reflection, we get for free integration of computation and
reasoning that would be very complex to implement in the classic fashion.

Is it the ultimate way to write tactics? 
The answer is: yes and no. The \texttt{Ring} tactic
uses intensively the conversion
rule of \CIC, that is replaces proof by computation the most as it is
possible. It can be useful in all situations where a classical tactic
generates huge proof terms. Symbolic Processing and Tautologies are
in that case. But there are also tactics like \texttt{Auto} or
\texttt{Linear}: that do many complex computations, using side-effects
and backtracking, and generate
 a small proof term. Clearly, it would be a non-sense to
replace them by tactics using reflection.

Another argument against the reflection is that \Coq, as a
programming language, has many nice features, like dependent types, 
but is very far from the
speed and the expressive power of \ocaml. Wait a minute! With \Coq\
it is possible to extract ML code from \CIC\ terms, right? So, why not
to link the extracted code with \Coq\ to inherit the benefits of the
reflection and the speed of ML tactics? That is called \textit{total
  reflection}, and is still an active research subject. With these
technologies it will become possible to bootstrap the type-checker of
\CIC, but there is still some work to achieve that goal.

Another brilliant idea from Benjamin Werner: reflection could be used
to couple a external tool (a rewriting program or a model checker)
with \Coq. We define (in \Coq) a type of terms, a type of
\emph{traces}, and prove a correction theorem that states that
\emph{replaying traces} is safe w.r.t some interpretation. Then we let 
the external tool do every computation (using side-effects,
backtracking, exception, or others features that are not available in
pure lambda calculus) to produce the trace: now we replay the trace in
Coq{}, and apply the correction lemma. So internalization seems to be
the best way to import \dots{} external proofs!

 
%%% Local Variables: 
%%% mode: latex
%%% TeX-master: "Reference-Manual"
%%% End: 
