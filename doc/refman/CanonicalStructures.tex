\achapter{Canonical Structures}
\aauthor{Assia Mahboubi and Enrico Tassi}

\label{CS-full}
\index{Canonical Structures!presentation}

This chapter explains the basics of Canonical Structure and how thy can be used
to overload notations and build a hierarchy of algebraic structures.
The examples are taken from~\cite{CSwcu}.  We invite the interested reader
to refer to this paper for all the details that are omitted here for brevity.
The interested reader shall also find in~\cite{CSlessadhoc} a detailed
description of another, complementary, use of Canonical Structures:
advanced proof search.  This latter papers also presents many techniques one
can employ to tune the inference of Canonical Structures.

\section{Notation overloading}

We build an infix notation $==$ for a comparison predicate.  Such notation
will be overloaded, and its meaning will depend on the types of the terms
that are compared.

\begin{coq_eval}
Require Import Arith.
\end{coq_eval}

\begin{coq_example}
Module EQ.
  Record class (T : Type) := Class { cmp : T -> T -> Prop }.
  Structure type := Pack { obj : Type; class_of : class obj }.
  Definition op (e : type) : obj e -> obj e -> Prop :=
    let 'Pack _ (Class _ the_cmp) := e in the_cmp.
  Check op.
  Arguments op {e} x y : simpl never.
  Arguments Class {T} cmp.
  Module theory.
    Notation "x == y" := (op x y) (at level 70).
  End theory.
End EQ.
\end{coq_example}

We use Coq modules as name spaces.  This allows to follow the same pattern
and naming convention for the rest of the chapter.  The base name space
contains the definitions of the algebraic structure.  To keep the example
small, the algebraic structure $EQ.type$ we are defining is very simplistic,
and characterizes terms on which a binary relation is defined, without
requiring such relation to validate any property.
The inner $theory$ module contains the overloaded notation $==$ and
will eventually contain lemmas holding on all the instances of the
algebraic structure (in this case there are no lemmas).

Note that in practice the user may want to declare $EQ.obj$ as a coercion,
but we will not do that here.

The following line tests that, when we assume a type $e$ that is in the
$EQ$ class, then we can relates two of its objects with $==$.

\begin{coq_example}
Import EQ.theory.
Check forall (e : EQ.type) (a b : EQ.obj e), a == b.
\end{coq_example}

Still, no concrete type is in the $EQ$ class.  We amend that by equipping $nat$
with a comparison relation.

\begin{coq_example}
Fail Check 3 == 3.
Definition nat_eq (x y : nat) := nat_compare x y = Eq.
Definition nat_EQcl : EQ.class nat := EQ.Class nat_eq.
Canonical Structure nat_EQty : EQ.type := EQ.Pack nat nat_EQcl.
Check 3 == 3.
Eval compute in 3 == 4.
\end{coq_example}

This last test shows that Coq is now not only able to typecheck $3==3$, but
also that the infix relation was bound to the $nat\_eq$ relation.  This
relation is selected whenever $==$ is used on terms of type $nat$.  This
can be read in the line declaring the canonical structure $nat\_EQty$,
where the first argument to $Pack$ is the key and its second argument
a group of canonical values associated to the key.  In this case we associate
to $nat$ only one canonical value (since its class, $nat\_EQcl$ has just one
member).  The use of the projection $op$ requires its argument to be in
the class $EQ$, and uses such a member (function) to actually compare
its arguments.

Similarly, we could equip any other type with a comparison relation, and
use the $==$ notation on terms of this type.

\subsection{Derived Canonical Structures}

We know how to use $==$ on base types, like $nat$, $bool$, $Z$.
Here we show how to deal with type constructors, i.e. how to make the
following example work:

\begin{coq_example}
Fail Check forall (e : EQ.type) (a b : EQ.obj e), (a,b) == (a,b). 
\end{coq_example}

The error message is telling that Coq has no idea on how to compare
pairs of objects.  The following construction is telling Coq exactly how to do
that.

\begin{coq_example}
Definition pair_eq (e1 e2 : EQ.type) (x y : EQ.obj e1 * EQ.obj e2) :=
  fst x == fst y /\ snd x == snd y.
Definition pair_EQcl e1 e2 := EQ.Class (pair_eq e1 e2).
Canonical Structure pair_EQty (e1 e2 : EQ.type) : EQ.type :=
  EQ.Pack (EQ.obj e1 * EQ.obj e2) (pair_EQcl e1 e2).
Check forall (e : EQ.type) (a b : EQ.obj e), (a,b) == (a,b).
Check forall n m : nat, (3,4) == (n,m).
\end{coq_example}

Thanks to the $pair\_EQty$ declaration, Coq is able to build a comparison
relation for pairs whenever it is able to build a comparison relation
for each component of the pair.  The declaration associates to the key
$*$ (the type constructor of pairs) the canonical comparison relation
$pair\_eq$ whenever the type constructor $*$ is applied to two types
being themselves in the $EQ$ class.

\section{Hierarchy of structures}

To get to an interesting example we need another base class to be available.
We choose the class of types that are equipped with an order relation,
to which we associate the infix $<=$ notation.

\begin{coq_example}
Module LE.
  Record class T := Class { cmp : T -> T -> Prop }.
  Structure type := Pack { obj : Type; class_of : class obj }.
  Definition op (e : type) : obj e -> obj e -> Prop :=
    let 'Pack _ (Class _ f) := e in f.
  Arguments op {_} x y : simpl never.
  Arguments Class {T} cmp.
  Module theory.
  Notation "x <= y" := (op x y) (at level 70).
  End theory.
End LE.
\end{coq_example}

As before we register a canonical $LE$ class for $nat$.

\begin{coq_example}
Import LE.theory.
Definition nat_le x y := nat_compare x y <> Gt.
Definition nat_LEcl : LE.class nat := LE.Class nat_le.
Canonical Structure nat_LEty : LE.type := LE.Pack nat nat_LEcl.
\end{coq_example}

And we enable Coq to relate pair of terms with $<=$.

\begin{coq_example}
Definition pair_le e1 e2 (x y : LE.obj e1 * LE.obj e2) :=
  fst x <= fst y /\ snd x <= snd y.
Definition pair_LEcl e1 e2 := LE.Class (pair_le e1 e2).
Canonical Structure pair_LEty (e1 e2 : LE.type) : LE.type :=
  LE.Pack (LE.obj e1 * LE.obj e2) (pair_LEcl e1 e2).
Check (3,4,5) <= (3,4,5).
\end{coq_example}

At the current stage we can use $==$ and $<=$ on concrete types,
like tuples of natural numbers, but we can't develop an algebraic
theory over the types that are equipped with both relations.

\begin{coq_example}
Check 2 <= 3 /\ 2 == 2.
Fail Check forall (e : EQ.type) (x y : EQ.obj e), x <= y -> y <= x -> x == y.
Fail Check forall (e : LE.type) (x y : LE.obj e), x <= y -> y <= x -> x == y.
\end{coq_example}

We need to define a new class that inherits from both $EQ$ and $LE$.

\begin{coq_example}
Module LEQ.
  Record mixin (e : EQ.type) (le : EQ.obj e -> EQ.obj e -> Prop) :=
    Mixin { compat : forall x y : EQ.obj e, le x y /\ le y x <-> x == y }.
  Record class T := Class {
    EQ_class    : EQ.class T;
    LE_class    : LE.class T;
    extra : mixin (EQ.Pack T EQ_class) (LE.cmp T LE_class) }.
  Structure type := _Pack { obj : Type; class_of : class obj }.
  Arguments Mixin {e le} _.
  Arguments Class {T} _ _ _.
\end{coq_example}

The $mixin$ component of the $LEQ$ class contains all the extra content
we are adding to $EQ$ and $LE$.  In particular it contains the requirement
that the two relations we are combining are compatible.

Unfortunately there is still an obstacle to developing the algebraic theory
of this new class.

\begin{coq_example}
  Module theory.
  Fail Check forall (le : type) (n m : obj le), n <= m -> n <= m -> n == m.
\end{coq_example}

The problem is that the two classes $LE$ and $LEQ$ are not yet related by
a subclass relation.  In other words Coq does not see that an object
of the $LEQ$ class is also an object of the $LE$ class.

The following two constructions tell Coq how to canonically build
the $LE.type$ and $EQ.type$ structure given an $LEQ.type$ structure
on the same type.

\begin{coq_example}
  Definition to_EQ (e : type) : EQ.type :=
    EQ.Pack (obj e) (EQ_class _ (class_of e)).
  Canonical Structure to_EQ.
  Definition to_LE (e : type) : LE.type :=
    LE.Pack (obj e) (LE_class _ (class_of e)).
  Canonical Structure to_LE.
\end{coq_example}
We can now formulate out first theorem on the objects of the $LEQ$ structure.
\begin{coq_example}
  Lemma lele_eq (e : type) (x y : obj e) : x <= y -> y <= x -> x == y.
   now intros; apply (compat _ _ (extra _ (class_of e)) x y); split. Qed.
  Arguments lele_eq {e} x y _ _.
  End theory.
End LEQ.
Import LEQ.theory.
Check lele_eq.
\end{coq_example}

Of course one would like to apply results proved in the algebraic
setting to any concrete instate of the algebraic structure.

\begin{coq_example}
Example test_algebraic (n m : nat) :  n <= m -> m <= n -> n == m.
 Fail apply (lele_eq n m). Abort.
Example test_algebraic2 (l1 l2 : LEQ.type) (n m : LEQ.obj l1 * LEQ.obj l2) :
  n <= m -> m <= n -> n == m.
 Fail apply (lele_eq n m). Abort.
\end{coq_example}

Again one has to tell Coq that the type $nat$ is in the $LEQ$ class, and how
the type constructor $*$ interacts with the $LEQ$ class.  In the following
proofs are omitted for brevity.

\begin{coq_example}
Lemma nat_LEQ_compat (n m : nat) : n <= m /\ m <= n <-> n == m.
\end{coq_example}
\begin{coq_eval}

split.
  unfold EQ.op; unfold LE.op; simpl; unfold nat_le; unfold nat_eq.
  case (nat_compare_spec n m); [ reflexivity | | now intros _ [H _]; case H ].
  now intro H; apply nat_compare_gt in H; rewrite -> H; intros [_ K]; case K.
unfold EQ.op; unfold LE.op; simpl; unfold nat_le; unfold nat_eq.
case (nat_compare_spec n m); [ | intros H1 H2; discriminate H2 .. ].
intro H; rewrite H; intros _; split; [ intro H1; discriminate H1 | ].
case (nat_compare_eq_iff m m); intros _ H1.
now rewrite H1; auto; intro H2; discriminate H2.
Qed.
\end{coq_eval}
\begin{coq_example}
Definition nat_LEQmx := LEQ.Mixin nat_LEQ_compat.
Lemma pair_LEQ_compat (l1 l2 : LEQ.type) (n m : LEQ.obj l1 * LEQ.obj l2) :
n <= m /\ m <= n <-> n == m.
\end{coq_example}
\begin{coq_eval}

case n; case m; unfold EQ.op; unfold LE.op; simpl.
intros n1 n2 m1 m2; split; [ intros [[Le1 Le2] [Ge1 Ge2]] | intros [H1 H2] ].
  now split; apply lele_eq.
case (LEQ.compat _ _ (LEQ.extra _ (LEQ.class_of l1)) m1 n1).
case (LEQ.compat _ _ (LEQ.extra _ (LEQ.class_of l2)) m2 n2).
intros _ H3 _ H4; apply H3 in H2; apply H4 in H1; clear H3 H4.
now case H1; case H2; split; split.
Qed.
\end{coq_eval}
\begin{coq_example}
Definition pair_LEQmx l1 l2 := LEQ.Mixin (pair_LEQ_compat l1 l2).
\end{coq_example}

The following script registers an $LEQ$ class for $nat$ and for the
type constructor $*$.  It also tests that they work as expected.

Unfortunately, these declarations are very verbose.  In the following
subsection we show how to make these declaration more compact.

\begin{coq_example}
Module Add_instance_attempt.
  Canonical Structure nat_LEQty : LEQ.type :=
    LEQ._Pack nat (LEQ.Class nat_EQcl nat_LEcl nat_LEQmx).
  Canonical Structure pair_LEQty (l1 l2 : LEQ.type) : LEQ.type :=
    LEQ._Pack (LEQ.obj l1 * LEQ.obj l2)
      (LEQ.Class
        (EQ.class_of (pair_EQty (to_EQ l1) (to_EQ l2)))
        (LE.class_of (pair_LEty (to_LE l1) (to_LE l2)))
        (pair_LEQmx l1 l2)).
  Example test_algebraic (n m : nat) : n <= m -> m <= n -> n == m.
   now apply (lele_eq n m). Qed.
  Example test_algebraic2 (n m : nat * nat) : n <= m -> m <= n -> n == m.
   now apply (lele_eq n m). Qed.
End Add_instance_attempt.
\end{coq_example}

Note that no direct proof of $n <= m -> m <= n -> n == m$ is provided by the
user for $n$ and $m$ of type $nat * nat$.  What the user provides is a proof of
this statement for $n$ and $m$ of type $nat$ and a proof that the pair
constructor preserves this property.  The combination of these two facts is a
simple form of proof search that Coq performs automatically while inferring
canonical structures.

\subsection{Compact declaration of Canonical Structures}

We need some infrastructure for that.

\begin{coq_example}
Module infrastructure.
  Require Import Strings.String.
  Inductive phantom {T : Type} (t : T) : Type := Phantom.
  Definition unify {T1 T2} (t1 : T1) (t2 : T2) (s : option string) := phantom t1 -> phantom t2.
  Definition id {T} {t : T} (x : phantom t) := x.
  Notation "[find v | t1 ~ t2 ] p" := (fun v (_ : unify t1 t2 None) => p) (at level 50, v ident, only parsing).
  Notation "[find v | t1 ~ t2 | s ] p" := (fun v (_ : unify t1 t2 (Some s)) => p) (at level 50, v ident, only parsing).
  Notation "'Error : t : s" := (unify _ t (Some s)) (at level 50, format "''Error'  :  t  :  s").
  Open Scope string_scope.
End infrastructure.
\end{coq_example}

To explain the notation $[find v | t1 \textasciitilde t2]$ lets pick one
of its instances: $[find e  | EQ.obj e \textasciitilde T | "is not an EQ.type" ]$.
It should be read as: ``find a class e such that its objects have type T
or fail with message "T is not an EQ.type"''.

The other utilities are used to ask Coq to solve a specific unification
problem, that will in turn require the inference of some canonical
structures.  They are explained in mode details in~\cite{CSwcu}.

We now have all we need to create a compact ``packager'' to declare
instances of the $LEQ$ class.

\begin{coq_example}
Import infrastructure.
Definition packager T e0 le0 (m0 : LEQ.mixin e0 le0) :=
    [find e  | EQ.obj e ~ T       | "is not an EQ.type" ]
    [find o  | LE.obj o ~ T       | "is not an LE.type" ]
    [find ce | EQ.class_of e ~ ce ]
    [find co | LE.class_of o ~ co ]
    [find m  | m ~ m0             | "is not the right mixin" ]
  LEQ._Pack T (LEQ.Class ce co m).
Notation Pack T m := (packager T _ _ m _ id _ id _ id _ id _ id).
\end{coq_example}

The object $Pack$ takes a type $T$ (the key) and a mixin $m$.  It infers all
the other pieces of the class $LEQ$ and declares them as canonical values
associated to the $T$ key.  All in all, the only new piece of information
we add in the $LEQ$ class is the mixin, all the rest is already canonical
for $T$ and hence can be inferred by Coq.

$Pack$ is a notation, hence it is not type checked at the time of its
declaration.  It will be type checked when it is used, an in that case
$T$ is going to be a concrete type.  The odd arguments $\_$ and $id$ we
pass to the 
packager represent respectively the classes to be inferred (like $e$, $o$, etc) and a token ($id$) to force their inference.  Again, for all the details the
reader can refer to~\cite{CSwcu}. 

The declaration of canonical instances can now be way more compact:

\begin{coq_example}
Canonical Structure nat_LEQty := Eval hnf in Pack nat nat_LEQmx.
Canonical Structure pair_LEQty (l1 l2 : LEQ.type) :=
  Eval hnf in Pack (LEQ.obj l1 * LEQ.obj l2) (pair_LEQmx l1 l2).
\end{coq_example}

Error messages are also quite intelligible (if one skips to the end of
the message).

\begin{coq_example}
Fail Canonical Structure err := Eval hnf in Pack bool nat_LEQmx.
\end{coq_example}

%%% Local Variables: 
%%% mode: latex
%%% TeX-master: "Reference-Manual"
%%% End: 
