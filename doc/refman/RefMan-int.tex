%BEGIN LATEX
\setheaders{Introduction}
%END LATEX
\chapter*{Introduction}

This document is the Reference Manual of version \coqversion{} of the \Coq\ 
proof assistant. A companion volume, the \Coq\ Tutorial, is provided
for the beginners. It is advised to read the Tutorial first.
A book~\cite{CoqArt} on practical uses of the \Coq{} system was published in 2004 and is a good support for both the beginner and
the advanced user.

%The system \Coq\ is designed to develop mathematical proofs. It can be
%used by mathematicians to develop mathematical theories and by
%computer scientists to write formal specifications,
The \Coq{} system is designed to develop mathematical proofs, and
especially to write formal specifications, programs and to verify that
programs are correct with respect to their specification. It provides
a specification language named \gallina. Terms of \gallina\ can
represent programs as well as properties of these programs and proofs
of these properties. Using the so-called \textit{Curry-Howard
  isomorphism}, programs, properties and proofs are formalized in the
same language called \textit{Calculus of Inductive Constructions},
that is a $\lambda$-calculus with a rich type system.  All logical
judgments in \Coq\ are typing judgments. The very heart of the Coq
system is the type-checking algorithm that checks the correctness of
proofs, in other words that checks that a program complies to its
specification. \Coq\ also provides an interactive proof assistant to
build proofs using specific programs called \textit{tactics}.

All services of the \Coq\ proof assistant are accessible by
interpretation of a command language called \textit{the vernacular}.

\Coq\ has an interactive mode in which commands are interpreted as the
user types them in from the keyboard and a compiled mode where
commands are processed from a file.  

\begin{itemize}
\item The interactive mode may be used as a debugging mode in which
  the user can develop his theories and proofs step by step,
  backtracking if needed and so on. The interactive mode is run with
  the {\tt coqtop} command from the operating system (which we shall
  assume to be some variety of UNIX in the rest of this document).
\item The compiled mode acts as a proof checker taking a file
  containing a whole development in order to ensure its correctness.
  Moreover, \Coq's compiler provides an output file containing a
  compact representation of its input. The compiled mode is run with
  the {\tt coqc} command from the operating system. 

\end{itemize}
These two modes are documented in Chapter~\ref{Addoc-coqc}.

Other modes of interaction with \Coq{} are possible: through an emacs
shell window, an emacs generic user-interface for proof assistant
({\ProofGeneral}~\cite{ProofGeneral}) or through a customized interface
(PCoq~\cite{Pcoq}).  These facilities are not documented here.  There
is also a \Coq{} Integrated Development Environment described in
Chapter~\ref{Addoc-coqide}.

\section*{How to read this book}

This is a Reference Manual, not a User Manual, then it is not made for a
continuous reading. However, it has some structure that is explained
below.

\begin{itemize}
\item The first part describes the specification language,
  Gallina. Chapters~\ref{Gallina} and~\ref{Gallina-extension}
  describe the concrete syntax as well as the meaning of programs,
  theorems and proofs in the Calculus of Inductive
  Constructions. Chapter~\ref{Theories} describes the standard library
  of \Coq. Chapter~\ref{Cic} is a mathematical description of the
  formalism. Chapter~\ref{chapter:Modules} describes the module system.

\item The second part describes the proof engine. It is divided in
  five chapters. Chapter~\ref{Vernacular-commands} presents all
  commands (we call them \emph{vernacular commands}) that are not
  directly related to interactive proving: requests to the
  environment, complete or partial evaluation, loading and compiling
  files. How to start and stop proofs, do multiple proofs in parallel
  is explained in Chapter~\ref{Proof-handling}. In
  Chapter~\ref{Tactics}, all commands that realize one or more steps
  of the proof are presented: we call them \emph{tactics}. The
  language to combine these tactics into complex proof strategies is
  given in Chapter~\ref{TacticLanguage}. Examples of tactics are
  described in Chapter~\ref{Tactics-examples}.

%\item The third part describes how to extend the system in two ways:
%  adding parsing and pretty-printing rules
%  (Chapter~\ref{Addoc-syntax}) and writing new tactics
%  (Chapter~\ref{TacticLanguage}). 

\item The third part describes how to extend the syntax of \Coq. It
corresponds to the Chapter~\ref{Addoc-syntax}.

\item In the fourth part more practical tools are documented. First in
  Chapter~\ref{Addoc-coqc}, the usage of \texttt{coqc} (batch mode)
  and \texttt{coqtop} (interactive mode) with their options is
  described. Then, in Chapter~\ref{Utilities},
  various utilities that come with the \Coq\ distribution are
  presented.
  Finally, Chapter~\ref{Addoc-coqide} describes the \Coq{} integrated
  development environment. 

\item The fifth part documents a number of advanced features, including
  coercions, canonical structures, typeclasses, program extraction, and
  specialized solvers and tactics.  See the table of contents for a complete
  list.
\end{itemize}

At the end of the document, after the global index, the user can find
specific indexes for tactics, vernacular commands, and error
messages. 

\section*{List of additional documentation}

This manual does not contain all the documentation the user may need
about \Coq{}. Various informations can be found in the following
documents:  
\begin{description}

\item[Tutorial] 
  A companion volume to this reference manual, the \Coq{} Tutorial, is
  aimed at gently introducing new users to developing proofs in \Coq{}
  without assuming prior knowledge of type theory. In a second step, the
  user can read also the tutorial on recursive types (document {\tt
    RecTutorial.ps}).

\item[Installation] A text file INSTALL that comes with the sources
  explains how to install \Coq{}.

\item[The \Coq{} standard library]
A commented version of sources of the \Coq{} standard library
(including only the specifications, the proofs are removed) 
is given in the additional document {\tt Library.ps}.

\end{description}


%%% Local Variables: 
%%% mode: latex
%%% TeX-master: "Reference-Manual"
%%% End:
