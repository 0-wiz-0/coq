\chapter[Extensions of \Gallina{}]{Extensions of \Gallina{}\label{Gallina-extension}\index{Gallina}}

{\gallina} is the kernel language of {\Coq}. We describe here extensions of
the Gallina's syntax.

\section{Record types
\comindex{Record}
\comindex{Inductive}
\comindex{CoInductive}
\label{Record}}

The \verb+Record+ construction is a macro allowing the definition of
records as is done in many programming languages.  Its syntax is
described on Figure~\ref{record-syntax}.  In fact, the \verb+Record+
macro is more general than the usual record types, since it allows
also for ``manifest'' expressions. In this sense, the \verb+Record+
construction allows defining ``signatures''.

\begin{figure}[h]
\begin{centerframe}
\begin{tabular}{lcl}
{\sentence} & ++= & {\record}\\
  & & \\
{\record} & ::= &
   {\recordkw} {\ident} \zeroone{\binders} \zeroone{{\tt :} {\sort}} \verb.:=. \\
&& ~~~~\zeroone{\ident}
       \verb!{! \zeroone{\nelist{\field}{;}} \verb!}! \verb:.:\\
  & & \\
{\recordkw} & ::= &
   {\tt Record} $|$ {\tt Inductive} $|$ {\tt CoInductive}\\
  & & \\
{\field} & ::= & {\name} \zeroone{\binders} : {\type} [ {\tt where} {\it notation} ] \\
 & $|$ & {\name} \zeroone{\binders} {\typecstr} := {\term}
\end{tabular}
\end{centerframe}
\caption{Syntax for the definition of {\tt Record}}
\label{record-syntax}
\end{figure}

\noindent In the expression

\smallskip
{\tt Record} {\ident} {\params} \texttt{:} 
   {\sort} := {\ident$_0$} \verb+{+
 {\ident$_1$} \binders$_1$ \texttt{:} {\term$_1$}; 
              \dots
  {\ident$_n$} \binders$_n$ \texttt{:} {\term$_n$} \verb+}+.
\smallskip
 
\noindent the identifier {\ident} is the name of the defined record
and {\sort} is its type. The identifier {\ident$_0$} is the name of
its constructor. If {\ident$_0$} is omitted, the default name {\tt
Build\_{\ident}} is used. If {\sort} is omitted, the default sort is ``{\Type}''.
The identifiers {\ident$_1$}, ..,
{\ident$_n$} are the names of fields and {\tt forall} \binders$_1${\tt ,} {\term$_1$}, ..., {\tt forall} \binders$_n${\tt ,} {\term$_n$}
their respective types. Remark that the type of {\ident$_i$} may
depend on the previous {\ident$_j$} (for $j<i$). Thus the order of the
fields is important. Finally, {\params} are the parameters of the
record.

More generally, a record may have explicitly defined (a.k.a.
manifest) fields. For instance, {\tt Record} {\ident} {\tt [}
{\params} {\tt ]} \texttt{:} {\sort} := \verb+{+ {\ident$_1$}
\texttt{:} {\type$_1$} \verb+;+ {\ident$_2$} \texttt{:=} {\term$_2$}
\verb+;+ {\ident$_3$} \texttt{:} {\type$_3$} \verb+}+ in which case
the correctness of {\type$_3$} may rely on the instance {\term$_2$} of
{\ident$_2$} and {\term$_2$} in turn may depend on {\ident$_1$}.


\Example
The set of rational numbers may be defined as:
\begin{coq_eval}
Reset Initial.
\end{coq_eval}
\begin{coq_example}
Record Rat : Set := mkRat
  {sign : bool;
   top : nat;
   bottom : nat;
   Rat_bottom_cond : 0 <> bottom;
   Rat_irred_cond :
    forall x y z:nat, (x * y) = top /\ (x * z) = bottom -> x = 1}.
\end{coq_example}

Remark here that the field
\verb+Rat_cond+ depends on the field \verb+bottom+. 

%Let us now see the work done by the {\tt Record} macro.
%First the macro generates an inductive definition
%with just one constructor:
%
%\medskip
%\noindent
%{\tt Inductive {\ident} \zeroone{\binders} : {\sort} := \\
%\mbox{}\hspace{0.4cm} {\ident$_0$} : forall ({\ident$_1$}:{\term$_1$}) .. 
%({\ident$_n$}:{\term$_n$}), {\ident} {\rm\sl params}.}
%\medskip

Let us now see the work done by the {\tt Record} macro.  First the
macro generates a variant type definition with just one constructor:
\begin{quote}
{\tt Variant {\ident} {\params} :{\sort} :=} \\
\qquad {\tt
  {\ident$_0$} ({\ident$_1$}:{\term$_1$}) .. ({\ident$_n$}:{\term$_n$}).}
\end{quote}
To build an object of type {\ident}, one should provide the
constructor {\ident$_0$} with $n$ terms filling the fields of
the record.

As an example, let us define the rational $1/2$:
\begin{coq_example*}
Require Import Arith.
Theorem one_two_irred :
 forall x y z:nat, x * y = 1 /\ x * z = 2 -> x = 1.
\end{coq_example*}
\begin{coq_eval}
Lemma mult_m_n_eq_m_1 : forall m n:nat, m * n = 1 -> m = 1.
destruct m; trivial.
intros; apply f_equal with (f := S).
destruct m; trivial.
destruct n; simpl in H.
 rewrite <- mult_n_O in H.
   discriminate.
 rewrite <- plus_n_Sm in H.
   discriminate.
Qed.

intros x y z [H1 H2].
 apply mult_m_n_eq_m_1 with (n := y); trivial.
\end{coq_eval}
\ldots
\begin{coq_example*}
Qed.
\end{coq_example*}
\begin{coq_example}
Definition half := mkRat true 1 2 (O_S 1) one_two_irred.
\end{coq_example}
\begin{coq_example}
Check half.
\end{coq_example}

The macro generates also, when it is possible, the projection
functions for destructuring an object of type {\ident}.  These
projection functions have the same name that the corresponding
fields. If a field is named ``\verb=_='' then no projection is built
for it.  In our example:

\begin{coq_example}
Eval compute in half.(top).
Eval compute in half.(bottom).
Eval compute in half.(Rat_bottom_cond).
\end{coq_example}
\begin{coq_eval}
Reset Initial.
\end{coq_eval}

Records defined  with the {\tt Record}  keyword are not  allowed to be
recursive (references  to the record's name  in the type  of its field
raises an  error). To define recursive  records, one can  use the {\tt
  Inductive} and {\tt CoInductive} keywords, resulting in an inductive
or  co-inductive record.  A  \emph{caveat}, however,  is that  records
cannot appear in mutually inductive (or co-inductive) definitions.
Induction schemes are automatically generated for inductive records.
Automatic generation of induction schemes for non-recursive records
defined with the {\tt Record} keyword can be activated with the
{\tt Nonrecursive Elimination Schemes} option
(see~\ref{set-nonrecursive-elimination-schemes}).

\begin{Warnings}
\item {\tt {\ident$_i$} cannot be defined.}

  It can happen that the definition of a projection is impossible.
  This message is followed by an explanation of this impossibility.
  There may be three reasons:
   \begin{enumerate}
   \item The name {\ident$_i$} already exists in the environment (see
     Section~\ref{Axiom}).
   \item The body of {\ident$_i$} uses an incorrect elimination for
     {\ident} (see Sections~\ref{Fixpoint} and~\ref{Caseexpr}).
   \item The type of the projections {\ident$_i$} depends on previous
   projections which themselves could not be defined.
   \end{enumerate}  
\end{Warnings}     

\begin{ErrMsgs}

\item \errindex{Records declared with the keyword Record or Structure cannot be recursive.}

  The record name {\ident} appears in the type of its fields, but uses
  the keyword  {\tt Record}. Use  the keyword {\tt Inductive}  or {\tt
    CoInductive} instead.
\item \errindex{Cannot handle mutually (co)inductive records.}

  Records  cannot  be  defined  as  part  of  mutually  inductive  (or
  co-inductive) definitions,  whether with records only  or mixed with
  standard definitions.
\item During the definition of the one-constructor inductive
  definition, all the errors of inductive definitions, as described in
  Section~\ref{gal-Inductive-Definitions}, may also occur.

\end{ErrMsgs}

\SeeAlso Coercions and records in Section~\ref{Coercions-and-records}
of the chapter devoted to coercions.

\Rem {\tt Structure} is a synonym of the keyword {\tt Record}.

\Rem Creation of an object of record type can be done by calling {\ident$_0$}
and passing arguments in the correct order.

\begin{coq_example}
Record point := { x : nat; y : nat }.
Definition a := Build_point 5 3.
\end{coq_example}

The following syntax allows creating objects by using named fields. The
fields do not have to be in any particular order, nor do they have to be all
present if the missing ones can be inferred or prompted for (see
Section~\ref{Program}).

\begin{coq_example}
Definition b := {| x := 5; y := 3 |}.
Definition c := {| y := 3; x := 5 |}.
\end{coq_example}

This syntax can be disabled globally for printing by
\begin{quote}
{\tt Unset Printing Records.}
\optindex{Printing Records}
\end{quote}
For a given type, one can override this using either
\begin{quote}
{\tt Add Printing Record {\ident}.}
\end{quote}
to get record syntax or
\begin{quote}
{\tt Add Printing Constructor {\ident}.}
\end{quote}
to get constructor syntax.

This syntax can also be used for pattern matching.

\begin{coq_example}
Eval compute in (
  match b with
  | {| y := S n |} => n
  | _ => 0
  end).
\end{coq_example}

\begin{coq_eval}
Reset Initial.
\end{coq_eval}

\Rem An experimental syntax for projections based on a dot notation is
available. The command to activate it is
\optindex{Printing Projections}
\begin{quote}
{\tt Set Printing Projections.}
\end{quote}

\begin{figure}[t]
\begin{centerframe}
\begin{tabular}{lcl}
{\term} & ++= & {\term} {\tt .(} {\qualid} {\tt )}\\
 & $|$ & {\term} {\tt .(} {\qualid} \nelist{\termarg}{} {\tt )}\\
 & $|$ & {\term} {\tt .(} {@}{\qualid} \nelist{\term}{} {\tt )}
\end{tabular}
\end{centerframe}
\caption{Syntax of \texttt{Record} projections}
\label{fig:projsyntax}
\end{figure}

The corresponding grammar rules are given Figure~\ref{fig:projsyntax}.
When {\qualid} denotes a projection, the syntax {\tt
  {\term}.({\qualid})} is equivalent to {\qualid~\term}, the syntax
{\term}{\tt .(}{\qualid}~{\termarg}$_1$ {\ldots} {\termarg}$_n${\tt )} to
{\qualid~{\termarg}$_1$ {\ldots} {\termarg}$_n$~\term}, and the syntax
{\term}{\tt .(@}{\qualid}~{\term}$_1$~\ldots~{\term}$_n${\tt )} to
{@\qualid~{\term}$_1$ {\ldots} {\term}$_n$~\term}. In each case, {\term}
is the object projected and the other arguments are the parameters of
the inductive type.

To deactivate the printing of projections, use 
{\tt Unset Printing Projections}.

\subsection{Primitive Projections}
\optindex{Primitive Projections}
\optindex{Printing Primitive Projection Parameters}
\optindex{Printing Primitive Projection Compatibility}
\index{Primitive projections}
\label{prim-proj}

The option {\tt Set Primitive Projections} turns on the use of primitive
projections when defining subsequent records. Primitive projections
extended the Calculus of Inductive Constructions with a new binary term
constructor {\tt r.(p)} representing a primitive projection p applied to
a record object {\tt r} (i.e., primitive projections are always
applied). Even if the record type has parameters, these do not appear at
applications of the projection, considerably reducing the sizes of terms
when manipulating parameterized records and typechecking time. On the
user level, primitive projections are a transparent replacement
for the usual defined ones.

  % - r.(p) and (p r) elaborate to native projection application, and
  %   the parameters cannot be mentioned. The following arguments are
  %   parsed according to the remaining implicit arguments declared for the
  %   projection (i.e. the implicit arguments after the record type
  %   argument). In dot notation, the record type argument is considered
  %   explicit no matter what its implicit status is.
  % - r.(@p params) and @p args are parsed as regular applications of the
  %   projection with explicit parameters.
  % - [simpl p] is forbidden, but [simpl @p] will simplify both the projection
  %   and its explicit [@p] version.
  % - [unfold p] has no effect on projection applications unless it is applied
  %   to a constructor. If the explicit version appears it reduces to the
  %   projection application.
  % - [pattern x at n], [rewrite x at n] and in general abstraction and selection
  %   of occurrences may fail due to the disappearance of parameters.

For compatibility, the parameters still appear to the user when printing terms
even though they are absent in the actual AST manipulated by the kernel. This
can be changed by unsetting the {\tt Printing Primitive Projection Parameters}
flag. Further compatibility printing can be deactivated thanks to the
{\tt Printing Primitive Projection Compatibility} option which governs the
printing of pattern-matching over primitive records.

\section{Variants and extensions of {\mbox{\tt match}}
\label{Extensions-of-match}
\index{match@{\tt match\ldots with\ldots end}}}

\subsection{Multiple and nested pattern-matching
\index{ML-like patterns}
\label{Mult-match}}

The basic version of \verb+match+ allows pattern-matching on simple
patterns. As an extension, multiple nested patterns or disjunction of
patterns are allowed, as in ML-like languages.

The extension just acts as a macro that is expanded during parsing
into a sequence of {\tt match} on simple patterns. Especially, a
construction defined using the extended {\tt match} is generally
printed under its expanded form (see~\texttt{Set Printing Matching} in
section~\ref{SetPrintingMatching}).

\SeeAlso Chapter~\ref{Mult-match-full}.

\subsection{Pattern-matching on boolean values: the {\tt if} expression
\label{if-then-else}
\index{if@{\tt if ... then ... else}}}

For inductive types with exactly two constructors and for
pattern-matchings expressions which do not depend on the arguments of
the constructors, it is possible to use a {\tt if ... then ... else}
notation. For instance, the definition

\begin{coq_example}
Definition not (b:bool) :=
  match b with
  | true => false
  | false => true
  end.
\end{coq_example}

\noindent can be alternatively written

\begin{coq_eval}
Reset not.
\end{coq_eval}
\begin{coq_example}
Definition not (b:bool) := if b then false else true.
\end{coq_example}

More generally, for an inductive type with constructors {\tt C$_1$}
and {\tt C$_2$}, we have the following equivalence

\smallskip

{\tt if {\term} \zeroone{\ifitem} then {\term}$_1$ else {\term}$_2$} $\equiv$
\begin{tabular}[c]{l}
{\tt match {\term} \zeroone{\ifitem} with}\\
{\tt \verb!|! C$_1$ \_ {\ldots} \_ \verb!=>! {\term}$_1$} \\
{\tt \verb!|! C$_2$ \_ {\ldots} \_ \verb!=>! {\term}$_2$} \\
{\tt end}
\end{tabular}

Here is an example.

\begin{coq_example}
Check (fun x (H:{x=0}+{x<>0}) =>
  match H with
  | left _ => true
  | right _ => false
  end).
\end{coq_example}

Notice that the printing uses the {\tt if} syntax because {\tt sumbool} is
declared as such (see Section~\ref{printing-options}).

\subsection{Irrefutable patterns: the destructuring {\tt let} variants 
\index{let in@{\tt let ... in}}
\label{Letin}}

Pattern-matching on terms inhabiting inductive type having only one
constructor can be alternatively written using {\tt let ... in ...}
constructions. There are two variants of them.

\subsubsection{First destructuring {\tt let} syntax}
The expression {\tt let
(}~{\ident$_1$},\ldots,{\ident$_n$}~{\tt ) :=}~{\term$_0$}~{\tt
in}~{\term$_1$} performs case analysis on a {\term$_0$} which must be in
an inductive type with one constructor having itself $n$ arguments. Variables
{\ident$_1$}\ldots{\ident$_n$} are bound to the $n$ arguments of the
constructor in expression {\term$_1$}. For instance, the definition

\begin{coq_example}
Definition fst (A B:Set) (H:A * B) := match H with
                                      | pair x y => x
                                      end.
\end{coq_example}

can be alternatively written 

\begin{coq_eval}
Reset fst.
\end{coq_eval}
\begin{coq_example}
Definition fst (A B:Set) (p:A * B) := let (x, _) := p in x.
\end{coq_example}
Notice that reduction is different from regular {\tt let ... in ...}
construction since it happens only if {\term$_0$} is in constructor
form. Otherwise, the reduction is blocked.

The pretty-printing of a definition by matching on a
irrefutable pattern can either be done using {\tt match} or the {\tt
let} construction (see Section~\ref{printing-options}).

If {\term} inhabits an inductive type with one constructor {\tt C},
we have an equivalence between

{\tt let ({\ident}$_1$,\ldots,{\ident}$_n$) \zeroone{\ifitem} := {\term} in {\term}'}

\noindent and

{\tt match {\term} \zeroone{\ifitem} with C {\ident}$_1$ {\ldots} {\ident}$_n$ \verb!=>! {\term}' end}


\subsubsection{Second destructuring {\tt let} syntax\index{let '... in@\texttt{let '... in}}}

Another destructuring {\tt let} syntax is available for inductive types with
one constructor by giving an arbitrary pattern instead of just a tuple
for all the arguments. For example, the preceding example can be written:
\begin{coq_eval}
Reset fst.
\end{coq_eval}
\begin{coq_example}
Definition fst (A B:Set) (p:A*B) := let 'pair x _ := p in x.
\end{coq_example}

This is useful to match deeper inside tuples and also to use notations
for the pattern, as the syntax {\tt let 'p := t in b} allows arbitrary
patterns to do the deconstruction. For example:

\begin{coq_example}
Definition deep_tuple (A:Set) (x:(A*A)*(A*A)) : A*A*A*A :=
  let '((a,b), (c, d)) := x in (a,b,c,d).
Notation " x 'With' p " := (exist _ x p) (at level 20).
Definition proj1_sig' (A:Set) (P:A->Prop) (t:{ x:A | P x }) : A :=
  let 'x With p := t in x.
\end{coq_example}

When printing definitions which are written using this construct it
takes precedence over {\tt let} printing directives for the datatype
under consideration (see Section~\ref{printing-options}).

\subsection{Controlling pretty-printing of {\tt match} expressions
\label{printing-options}}

The following commands give some control over the pretty-printing of
{\tt match} expressions.

\subsubsection{Printing nested patterns
\label{SetPrintingMatching}
\optindex{Printing Matching}}

The Calculus of Inductive Constructions knows pattern-matching only
over simple patterns. It is however convenient to re-factorize nested
pattern-matching into a single pattern-matching over a nested pattern.
{\Coq}'s printer try to do such limited re-factorization.

\begin{quote}
{\tt Set Printing Matching.}
\end{quote}
This tells {\Coq} to try to use nested patterns. This is the default
behavior.

\begin{quote}
{\tt Unset Printing Matching.}
\end{quote}
This tells {\Coq} to print only simple pattern-matching problems in
the same way as the {\Coq} kernel handles them.

\begin{quote}
{\tt Test Printing Matching.}
\end{quote}
This tells if the printing matching mode is on or off. The default is
on.

\subsubsection{Printing of wildcard pattern
\optindex{Printing Wildcard}}

Some variables in a pattern may not occur in the right-hand side of
the pattern-matching clause.  There are options to control the
display of these variables.

\begin{quote}
{\tt Set Printing Wildcard.}
\end{quote}
The variables having no occurrences in the right-hand side of the
pattern-matching clause are just printed using the wildcard symbol
``{\tt \_}''.

\begin{quote}
{\tt Unset Printing Wildcard.}
\end{quote}
The variables, even useless, are printed using their usual name. But some
non dependent variables have no name. These ones are still printed
using a ``{\tt \_}''.

\begin{quote}
{\tt Test Printing Wildcard.}
\end{quote}
This tells if the wildcard printing mode is on or off. The default is
to print wildcard for useless variables.

\subsubsection{Printing of the elimination predicate
\optindex{Printing Synth}}

In most of the cases, the type of the result of a matched term is
mechanically synthesizable. Especially, if the result type does not
depend of the matched term.

\begin{quote}
{\tt Set Printing Synth.}
\end{quote}
The result type is not printed when {\Coq} knows that it can
re-synthesize it.

\begin{quote}
{\tt Unset Printing Synth.}
\end{quote}
This forces the result type to be always printed.

\begin{quote}
{\tt Test Printing Synth.}
\end{quote}
This tells if the non-printing of synthesizable types is on or off.
The default is to not print synthesizable types.

\subsubsection{Printing matching on irrefutable pattern
\label{AddPrintingLet}
\comindex{Add Printing Let {\ident}}
\comindex{Remove Printing Let {\ident}}
\comindex{Test Printing Let for {\ident}}
\comindex{Print Table Printing Let}}

If an inductive type has just one constructor,
pattern-matching can be written using the first destructuring let syntax.

\begin{quote}
{\tt Add Printing Let {\ident}.}
\end{quote}
This adds {\ident} to the list of inductive types for which
pattern-matching is written using a {\tt let} expression.

\begin{quote}
{\tt Remove Printing Let {\ident}.}
\end{quote}
This removes {\ident} from this list. Note that removing an inductive
type from this list has an impact only for pattern-matching written using
\texttt{match}. Pattern-matching explicitly written using a destructuring
let are not impacted.

\begin{quote}
{\tt Test Printing Let for {\ident}.}
\end{quote}
This tells if {\ident} belongs to the list.

\begin{quote}
{\tt Print Table Printing Let.}
\end{quote}
This prints the list of inductive types for which pattern-matching is
written using a {\tt let} expression.

The list of inductive types for which pattern-matching is written
using a {\tt let} expression is managed synchronously. This means that
it is sensible to the command {\tt Reset}.

\subsubsection{Printing matching on booleans
\comindex{Add Printing If {\ident}}
\comindex{Remove Printing If {\ident}}
\comindex{Test Printing If for {\ident}}
\comindex{Print Table Printing If}}

If an inductive type is isomorphic to the boolean type,
pattern-matching can be written using {\tt if} ... {\tt then} ... {\tt
  else} ...

\begin{quote}
{\tt Add Printing If {\ident}.}
\end{quote}
This adds {\ident} to the list of inductive types for which
pattern-matching is written using an {\tt if} expression.

\begin{quote}
{\tt Remove Printing If {\ident}.}
\end{quote}
This removes {\ident} from this list.

\begin{quote}
{\tt Test Printing If for {\ident}.}
\end{quote}
This tells if {\ident} belongs to the list.

\begin{quote}
{\tt Print Table Printing If.}
\end{quote}
This prints the list of inductive types for which pattern-matching is
written using an {\tt if} expression.

The list of inductive types for which pattern-matching is written
using an {\tt if} expression is managed synchronously. This means that
it is sensible to the command {\tt Reset}.

\subsubsection{Example}

This example emphasizes what the printing options offer.

\begin{coq_example}
Definition snd (A B:Set) (H:A * B) := match H with
                                      | pair x y => y
                                      end.
Test Printing Let for prod.
Print snd.
Remove Printing Let prod.
Unset Printing Synth.
Unset Printing Wildcard.
Print snd.
\end{coq_example}
\begin{coq_eval}
Reset Initial.
\end{coq_eval}

% \subsection{Still not dead old notations}

% The following variant of {\tt match} is inherited from older version
% of {\Coq}. 

% \medskip
% \begin{tabular}{lcl}
% {\term} & ::= & {\annotation} {\tt Match} {\term} {\tt with} {\terms} {\tt end}\\
% \end{tabular}
% \medskip

% This syntax is a macro generating a combination of {\tt match} with {\tt
% Fix} implementing a combinator for primitive recursion equivalent to
% the {\tt Match} construction of \Coq\ V5.8. It is provided only for
% sake of compatibility with \Coq\ V5.8. It is recommended to avoid it.
% (see Section~\ref{Matchexpr}).

% There is also a notation \texttt{Case} that is the
% ancestor of \texttt{match}. Again, it is still in the code for
% compatibility with old versions but the user should not use it.

% Explained in RefMan-gal.tex
%% \section{Forced type}

%% In some cases, one may wish to assign a particular type to a term. The
%% syntax to force the type of a term is the following:

%% \medskip
%% \begin{tabular}{lcl}
%% {\term} & ++= & {\term} {\tt :} {\term}\\
%% \end{tabular}
%% \medskip

%% It forces the first term to be of type the second term. The
%% type must be compatible with
%% the term. More precisely it must be either a type convertible to
%% the automatically inferred type (see Chapter~\ref{Cic}) or a type
%% coercible to it, (see \ref{Coercions}). When the type of a
%% whole expression is forced, it is usually not necessary to give the types of
%% the variables involved in the term.

%% Example:

%% \begin{coq_example}
%% Definition ID := forall X:Set, X -> X.
%% Definition id := (fun X x => x):ID.
%% Check id.
%% \end{coq_example}

\section{Advanced recursive functions}

The \emph{experimental} command 
\begin{center}
   \texttt{Function {\ident} {\binder$_1$}\ldots{\binder$_n$}
     \{decrease\_annot\} : type$_0$ := \term$_0$}
   \comindex{Function}
   \label{Function}
\end{center}
can be seen as a generalization of {\tt Fixpoint}.  It is actually a
wrapper for several ways of defining a function \emph{and other useful
  related objects}, namely: an induction principle that reflects the
recursive structure of the function (see \ref{FunInduction}), and its
fixpoint equality.  The meaning of this
declaration is to define a function {\it ident}, similarly to {\tt
  Fixpoint}. Like in {\tt Fixpoint}, the decreasing argument must be
given (unless the function is not recursive), but it must not
necessary be \emph{structurally} decreasing. The point of the {\tt
  \{\}} annotation is to name the decreasing argument \emph{and} to
describe which kind of decreasing criteria must be used to ensure
termination of recursive calls.

The {\tt Function} construction enjoys also the {\tt with} extension
to define mutually recursive definitions. However, this feature does
not work for non structural recursive functions. % VRAI??

See the documentation of {\tt functional induction}
(see Section~\ref{FunInduction}) and {\tt Functional Scheme}
(see Section~\ref{FunScheme} and \ref{FunScheme-examples}) for how to use the
induction principle to easily reason about the function.

\noindent {\bf Remark: } To obtain the right principle, it is better
to put rigid parameters of the function as first arguments. For
example it is better to define plus like this:

\begin{coq_example*}
Function plus (m n : nat) {struct n} : nat :=
  match n with
  | 0 => m
  | S p => S (plus m p)
  end.
\end{coq_example*}
\noindent than like this:
\begin{coq_eval}
Reset plus.
\end{coq_eval}
\begin{coq_example*}
Function plus (n m : nat) {struct n} : nat :=
  match n with
  | 0 => m
  | S p => S (plus p m)
  end.
\end{coq_example*}

\paragraph[Limitations]{Limitations\label{sec:Function-limitations}}
\term$_0$ must be build as a \emph{pure pattern-matching tree}
(\texttt{match...with}) with applications only \emph{at the end} of
each branch.  

Function does not support partial application of the function being defined. Thus, the following example cannot be accepted due to the presence of partial application of \ident{wrong} into the body of \ident{wrong}~:
\begin{coq_eval}
Require List.
\end{coq_eval}
\begin{coq_example*}
Fail Function wrong (C:nat) : nat :=
  List.hd 0 (List.map wrong (C::nil)).
\end{coq_example*}

For now dependent cases are not treated for non structurally terminating functions.



\begin{ErrMsgs}
\item \errindex{The recursive argument must be specified}
\item \errindex{No argument name \ident}
\item \errindex{Cannot use mutual definition with well-founded
    recursion or measure}

\item \errindex{Cannot define graph for \ident\dots} (warning)

  The generation of the graph relation \texttt{(R\_\ident)} used to
  compute the induction scheme of \ident\ raised a typing error. Only
  the ident is defined, the induction scheme will not be generated.

  This error happens generally when:

  \begin{itemize}
  \item the definition uses pattern matching on dependent types, which
    \texttt{Function} cannot deal with yet.
  \item the definition is not a \emph{pattern-matching tree} as
    explained above.
  \end{itemize}

\item \errindex{Cannot define principle(s) for \ident\dots} (warning)

  The generation of the graph relation \texttt{(R\_\ident)} succeeded
  but the induction principle could not be built. Only the ident is
  defined. Please report.

\item \errindex{Cannot build functional inversion principle} (warning)

  \texttt{functional inversion} will not be available for the
  function.
\end{ErrMsgs}


\SeeAlso{\ref{FunScheme}, \ref{FunScheme-examples}, \ref{FunInduction}}

Depending on the {\tt \{$\ldots$\}} annotation, different definition
mechanisms are used by {\tt Function}. More precise description
given below.

\begin{Variants}
\item \texttt{ Function {\ident} {\binder$_1$}\ldots{\binder$_n$}
    : type$_0$ := \term$_0$}

  Defines the not recursive function \ident\ as if declared with
  \texttt{Definition}.  Moreover the following are defined:

  \begin{itemize}
  \item {\tt\ident\_rect}, {\tt\ident\_rec} and {\tt\ident\_ind},
    which reflect the pattern matching structure of \term$_0$ (see the
    documentation of {\tt Inductive} \ref{Inductive});
  \item The inductive \texttt{R\_\ident} corresponding to the graph of
    \ident\ (silently);
  \item \texttt{\ident\_complete} and \texttt{\ident\_correct} which are
    inversion information linking the function and its graph.
  \end{itemize}
\item \texttt{Function {\ident} {\binder$_1$}\ldots{\binder$_n$}
    {\tt \{}{\tt struct} \ident$_0${\tt\}} : type$_0$ := \term$_0$}
  
  Defines the structural recursive function \ident\ as if declared
  with \texttt{Fixpoint}.  Moreover the following are defined:

  \begin{itemize}
  \item The same objects as above;
  \item The fixpoint equation of \ident: \texttt{\ident\_equation}.
  \end{itemize}
  
\item \texttt{Function {\ident} {\binder$_1$}\ldots{\binder$_n$} {\tt
      \{}{\tt measure \term$_1$} \ident$_0${\tt\}} : type$_0$ :=
    \term$_0$}
\item \texttt{Function {\ident} {\binder$_1$}\ldots{\binder$_n$}
 {\tt \{}{\tt wf \term$_1$} \ident$_0${\tt\}} : type$_0$ := \term$_0$}

Defines a recursive function by well founded recursion. \textbf{The
module \texttt{Recdef} of the standard library must be loaded for this
feature}. The {\tt \{\}} annotation is mandatory and must be one of
the following:
\begin{itemize}
\item {\tt \{measure} \term$_1$ \ident$_0${\tt\}} with \ident$_0$
      being the decreasing argument and \term$_1$ being a function
      from type of \ident$_0$ to \texttt{nat} for which value on the
      decreasing argument decreases (for the {\tt lt} order on {\tt
      nat}) at each recursive call of \term$_0$, parameters of the
      function are bound in  \term$_0$;
\item {\tt \{wf} \term$_1$ \ident$_0${\tt\}} with \ident$_0$ being
      the decreasing argument and \term$_1$ an ordering relation on
      the type of \ident$_0$ (i.e. of type T$_{\ident_0}$
      $\to$ T$_{\ident_0}$ $\to$ {\tt Prop}) for which
      the decreasing argument decreases at each recursive call of
      \term$_0$. The order must be well founded. parameters of the
      function are bound in  \term$_0$.
\end{itemize} 

Depending on the annotation, the user is left with some proof
obligations that will be used to define the function. These proofs
are: proofs that each recursive call is actually decreasing with
respect to the given criteria, and (if the criteria is \texttt{wf}) a
proof that the ordering relation is well founded.

%Completer sur measure et wf

Once proof obligations are discharged, the following objects are
defined:

\begin{itemize}
\item The same objects as with the \texttt{struct};
\item The lemma \texttt{\ident\_tcc} which collects all proof
  obligations in one property;
\item The lemmas \texttt{\ident\_terminate} and \texttt{\ident\_F}
  which is needed to be inlined during extraction of \ident.
\end{itemize}



%Complete!!
The way this recursive function is defined is the subject of several
papers by Yves Bertot and Antonia Balaa on the one hand, and Gilles Barthe,
Julien Forest, David Pichardie, and Vlad Rusu on the other hand.

%Exemples ok ici

\bigskip

\noindent {\bf Remark: } Proof obligations are presented as several
subgoals belonging to a Lemma {\ident}{\tt\_tcc}. % These subgoals are independent which means that in order to
% abort them you will have to abort each separately.



%The decreasing argument cannot be dependent of another??

%Exemples faux ici
\end{Variants}


\section{Section mechanism
\index{Sections}
\label{Section}}

The sectioning mechanism can be used to to organize a proof in
structured sections. Then local declarations become available (see
Section~\ref{Basic-definitions}).

\subsection{\tt Section {\ident}\comindex{Section}}

This command is used to open a section named {\ident}.

%% Discontinued ?
%% \begin{Variants}
%% \comindex{Chapter}
%% \item{\tt Chapter {\ident}}\\
%%         Same as {\tt Section {\ident}}
%% \end{Variants}

\subsection{\tt End {\ident}
\comindex{End}}

This command closes the section named {\ident}. After closing of the
section, the local declarations (variables and local definitions) get
{\em discharged}, meaning that they stop being visible and that all
global objects defined in the section are generalized with respect to
the variables and local definitions they each depended on in the
section.


Here is an example :
\begin{coq_example}
Section s1.
Variables x y : nat.
Let y' := y.
Definition x' := S x.
Definition x'' := x' + y'.
Print x'.
End s1.
Print x'.
Print x''.
\end{coq_example}
Notice the difference between the value of {\tt x'} and {\tt x''}
inside section {\tt s1} and outside.

\begin{ErrMsgs}
\item \errindex{This is not the last opened section}
\end{ErrMsgs}

\begin{Remarks}
\item Most commands, like {\tt Hint}, {\tt Notation}, option management, ...
which appear inside a section are canceled when the
section is closed.
% see Section~\ref{LongNames}
%\item Usually all identifiers must be distinct. 
%However, a name already used in a closed section (see \ref{Section})
%can be reused. In this case, the old name is no longer accessible.

% Obsolète
%\item A module implicitly open a section. Be careful not to name a
%module with an identifier already used in the module (see \ref{compiled}).
\end{Remarks}

\section{Module system
\index{Modules}
\label{section:Modules}}

The module system provides a way of packaging related elements
together, as well as a mean of massive abstraction.

\begin{figure}[t]
\begin{centerframe}
\begin{tabular}{rcl}
{\modtype}  & ::= & {\ident} \\
 & $|$ & {\modtype} \texttt{ with Definition }{\ident} := {\term} \\
 & $|$ & {\modtype} \texttt{ with Module }{\ident} := {\qualid} \\
 &&\\

{\onemodbinding}  & ::= & {\tt ( \nelist{\ident}{} : {\modtype} )}\\
 &&\\

{\modbindings} & ::= & \nelist{\onemodbinding}{}\\
 &&\\

{\modexpr} & ::= & \nelist{\qualid}{} 
\end{tabular}
\end{centerframe}
\caption{Syntax of modules}
\end{figure}

\subsection{\tt Module {\ident}
\comindex{Module}}

This command is used to start an interactive module named {\ident}.

\begin{Variants}

\item{\tt Module {\ident} {\modbindings}}

  Starts an interactive functor with parameters given by {\modbindings}.

\item{\tt Module {\ident} \verb.:. {\modtype}}

  Starts an interactive module specifying its module type. 

\item{\tt Module {\ident} {\modbindings} \verb.:. {\modtype}}

  Starts an interactive functor with parameters given by
  {\modbindings}, and output module type {\modtype}.

\item{\tt Module {\ident} \verb.<:. {\modtype}}

  Starts an interactive module satisfying {\modtype}. 

\item{\tt Module {\ident} {\modbindings} \verb.<:. {\modtype}}

  Starts an interactive functor with parameters given by
  {\modbindings}. The output module type is verified against the
  module type {\modtype}.

\end{Variants}

\subsection{\tt End {\ident}
\comindex{End}}

This command closes the interactive module {\ident}. If the module type
was given the content of the module is matched against it and an error
is signaled if the matching fails. If the module is basic (is not a
functor) its components (constants, inductive types, submodules etc) are
now available through the dot notation.

\begin{ErrMsgs}
\item \errindex{No such label {\ident}}
\item \errindex{Signature components for label {\ident} do not match}
\item \errindex{This is not the last opened module}
\end{ErrMsgs}


\subsection{\tt Module {\ident} := {\modexpr}
\comindex{Module}}

This command defines the module identifier {\ident} to be equal to
{\modexpr}. 

\begin{Variants}
\item{\tt Module {\ident} {\modbindings} := {\modexpr}}

 Defines a functor with parameters given by {\modbindings} and body {\modexpr}.

% Particular cases of the next 2 items
%\item{\tt Module {\ident} \verb.:. {\modtype} := {\modexpr}}
%
%  Defines a module with body {\modexpr} and interface {\modtype}.
%\item{\tt Module {\ident} \verb.<:. {\modtype} := {\modexpr}}
%
%  Defines a module with body {\modexpr}, satisfying {\modtype}.

\item{\tt Module {\ident} {\modbindings} \verb.:. {\modtype} :=
    {\modexpr}}

  Defines a functor with parameters given by {\modbindings} (possibly none),
  and output module type {\modtype}, with body {\modexpr}. 

\item{\tt Module {\ident} {\modbindings} \verb.<:. {\modtype} :=
    {\modexpr}}

  Defines a functor with parameters given by {\modbindings} (possibly none) 
  with body {\modexpr}. The body is checked against {\modtype}.

\end{Variants}

\subsection{\tt Module Type {\ident}
\comindex{Module Type}}

This command is used to start an interactive module type {\ident}.

\begin{Variants}

\item{\tt Module Type {\ident} {\modbindings}}

  Starts an interactive functor type with parameters given by {\modbindings}.

\end{Variants}

\subsection{\tt End {\ident}
\comindex{End}}

This command closes the interactive module type {\ident}.

\begin{ErrMsgs}
\item \errindex{This is not the last opened module type}
\end{ErrMsgs}

\subsection{\tt Module Type {\ident} := {\modtype}}

Defines a module type {\ident} equal to {\modtype}.

\begin{Variants}
\item {\tt Module Type {\ident} {\modbindings} := {\modtype}}

  Defines a functor type {\ident} specifying functors taking arguments
  {\modbindings} and returning {\modtype}.
\end{Variants}

\subsection{\tt Declare Module {\ident}}

Starts an interactive module declaration. This command is available
only in module types. 

\begin{Variants}

\item{\tt Declare Module {\ident} {\modbindings}}

  Starts an interactive declaration of a functor with parameters given
  by {\modbindings}.

% Particular case of the next item
%\item{\tt Declare Module {\ident} \verb.<:. {\modtype}}
%
%  Starts an interactive declaration of a module satisfying {\modtype}.

\item{\tt Declare Module {\ident} {\modbindings} \verb.<:. {\modtype}}

  Starts an interactive declaration of a functor with parameters given
  by {\modbindings} (possibly none). The declared output module type is
  verified against the module type {\modtype}.

\end{Variants}

\subsection{\tt End {\ident}}

This command closes the interactive declaration of module {\ident}.

\subsection{\tt Declare Module {\ident} : {\modtype}}

Declares a module of {\ident} of type {\modtype}. This command is available
only in module types. 

\begin{Variants}

\item{\tt Declare Module {\ident} {\modbindings} \verb.:. {\modtype}}

  Declares a functor with parameters {\modbindings} and output module
  type {\modtype}.

\item{\tt Declare Module {\ident} := {\qualid}}

  Declares a module equal to the module {\qualid}.

\item{\tt Declare Module {\ident} \verb.<:. {\modtype} := {\qualid}}

  Declares a module equal to the module {\qualid}, verifying that the
  module type of the latter is a subtype of {\modtype}.

\end{Variants}


\subsubsection{Example}

Let us define a simple module.
\begin{coq_example}
Module M.
  Definition T := nat.
  Definition x := 0.
  Definition y : bool.
    exact true.
  Defined.
End M.
\end{coq_example}
Inside a module one can define constants, prove theorems and do any
other things that can be done in the toplevel. Components of a closed
module can be accessed using the dot notation:
\begin{coq_example}
Print M.x.
\end{coq_example}
A simple module type:
\begin{coq_example}
Module Type SIG.
  Parameter T : Set.
  Parameter x : T.
End SIG.
\end{coq_example}
Inside a module type the proof editing mode is not available.
Consequently commands like \texttt{Definition}\ without body,
\texttt{Lemma}, \texttt{Theorem} are not allowed.  In order to declare
constants, use \texttt{Axiom} and \texttt{Parameter}.

Now we can create a new module from \texttt{M}, giving it a less
precise specification: the \texttt{y} component is dropped as well
as the body of \texttt{x}.

\begin{coq_example}
Module N  :  SIG with Definition T := nat  :=  M.
Print N.T.
Print N.x.
Print N.y.
\end{coq_example}
\begin{coq_eval}
Reset N.
\end{coq_eval}

\noindent
The definition of \texttt{N} using the module type expression
\texttt{SIG with Definition T:=nat} is equivalent to the following
one:

\begin{coq_example*}
Module Type SIG'.
  Definition T : Set := nat.
  Parameter x : T.
End SIG'.
Module N : SIG' := M.
\end{coq_example*}
If we just want to be sure that the our implementation satisfies a
given module type without restricting the interface, we can use a
transparent constraint
\begin{coq_example}
Module P <: SIG := M.
Print P.y.
\end{coq_example}
Now let us create a functor, i.e. a parametric module
\begin{coq_example}
Module Two (X Y: SIG).
\end{coq_example}
\begin{coq_example*}
  Definition T := X.T * Y.T.
  Definition x := (X.x, Y.x).
\end{coq_example*}
\begin{coq_example}
End Two.
\end{coq_example}
and apply it to our modules and do some computations
\begin{coq_example}
Module Q := Two M N.
Eval compute in (fst Q.x + snd Q.x).
\end{coq_example}
In the end, let us define a module type with two sub-modules, sharing
some of the fields and give one of its possible implementations:
\begin{coq_example}
Module Type SIG2.
  Declare Module M1 : SIG.
  Declare Module M2 <: SIG.
    Definition T := M1.T.
    Parameter x : T.
  End M2.
End SIG2.
\end{coq_example}
\begin{coq_example*}
Module Mod <: SIG2.
  Module M1.
    Definition T := nat.
    Definition x := 1.
  End M1.
  Module M2 := M.
\end{coq_example*}
\begin{coq_example}
End Mod.
\end{coq_example}
Notice that \texttt{M} is a correct body for the component \texttt{M2}
since its \texttt{T} component is equal \texttt{nat} and hence
\texttt{M1.T} as specified.
\begin{coq_eval}
Reset Initial.
\end{coq_eval}

\begin{Remarks}
\item Modules and module types can be nested components of each other.
\item When a module declaration is started inside a module type,
  the proof editing mode is still unavailable.
\item One can have sections inside a module or a module type, but
  not a module or a module type inside a section.
\item Commands like \texttt{Hint} or \texttt{Notation} can
  also appear inside modules and module types. Note that in case of a
  module definition like:

    \medskip
    \noindent
    {\tt Module N : SIG := M.} 
    \medskip

    or

    \medskip
    {\tt Module N : SIG.\\
      \ \ \dots\\
      End N.}
    \medskip 
    
    hints and the like valid for \texttt{N} are not those defined in
    \texttt{M} (or the module body) but the ones defined in
    \texttt{SIG}.

\end{Remarks}

\subsection{Import {\qualid}
\comindex{Import}
\label{Import}}

If {\qualid} denotes a valid basic module (i.e. its module type is a
signature), makes its components available by their short names.

Example:

\begin{coq_example}
Module Mod.
\end{coq_example}
\begin{coq_example}
  Definition T:=nat.
  Check T.
\end{coq_example}
\begin{coq_example}
End Mod.
Check Mod.T.
Check T. (* Incorrect ! *)
Import Mod.
Check T. (* Now correct *)
\end{coq_example}
\begin{coq_eval}
Reset Mod.
\end{coq_eval}


\begin{Variants}
\item{\tt Export {\qualid}}\comindex{Export}

  When the module containing the command {\tt Export {\qualid}} is
  imported, {\qualid} is imported as well.
\end{Variants}

\begin{ErrMsgs}
  \item \errindexbis{{\qualid} is not a module}{is not a module}
% this error is impossible in the import command
%  \item \errindex{Cannot mask the absolute name {\qualid} !}
\end{ErrMsgs}

\begin{Warnings}
  \item Warning: Trying to mask the absolute name {\qualid} !
\end{Warnings}

\subsection{\tt Print Module {\ident}
\comindex{Print Module}}

Prints the module type and (optionally) the body of the module {\ident}.

\subsection{\tt Print Module Type {\ident}
\comindex{Print Module Type}}

Prints the module type corresponding to {\ident}.


%%% Local Variables: 
%%% mode: latex
%%% TeX-master: "Reference-Manual"
%%% End: 


\section{Libraries and qualified names}

\subsection{Names of libraries
\label{Libraries}
\index{Libraries}}

The theories developed in {\Coq} are stored in {\em library files}
which are hierarchically classified into {\em libraries} and {\em
  sublibraries}. To express this hierarchy, library names are
represented by qualified identifiers {\qualid}, i.e. as list of
identifiers separated by dots (see Section~\ref{qualid}). For
instance, the library file {\tt Mult} of the standard {\Coq} library
{\tt Arith} is named {\tt Coq.Arith.Mult}. The identifier that starts
the name of a library is called a {\em library root}.  All library
files of the standard library of {\Coq} have the reserved root {\tt Coq}
but library file names based on other roots can be obtained by using
{\Coq} commands ({\tt coqc}, {\tt coqtop}, {\tt coqdep}, \dots) options
{\tt -Q} or {\tt -R} (see Section~\ref{coqoptions}). Also, when an
interactive {\Coq} session starts, a library of root {\tt Top} is
started, unless option {\tt -top} is set (see
Section~\ref{coqoptions}).

\subsection{Qualified names
\label{LongNames}
\index{Qualified identifiers}
\index{Absolute names}}

Library files are modules which possibly contain submodules which
eventually contain constructions (axioms, parameters, definitions,
lemmas, theorems, remarks or facts). The {\em absolute name}, or {\em
full name}, of a construction in some library file is a qualified
identifier starting with the logical name of the library file,
followed by the sequence of submodules names encapsulating the
construction and ended by the proper name of the construction.
Typically, the absolute name {\tt Coq.Init.Logic.eq} denotes Leibniz'
equality defined in the module {\tt Logic} in the sublibrary {\tt
Init} of the standard library of \Coq.

The proper name that ends the name of a construction is the {\it short
name} (or sometimes {\it base name}) of the construction (for
instance, the short name of {\tt Coq.Init.Logic.eq} is {\tt eq}). Any
partial suffix of the absolute name is a {\em partially qualified name}
(e.g. {\tt Logic.eq} is a partially qualified name for {\tt
Coq.Init.Logic.eq}).  Especially, the short name of a construction is
its shortest partially qualified name.

{\Coq} does not accept two constructions (definition, theorem, ...)
with the same absolute name but different constructions can have the
same short name (or even same partially qualified names as soon as the
full names are different).

Notice that the notion of absolute, partially qualified and
short names also applies to library file names.

\paragraph{Visibility}

{\Coq} maintains a table called {\it name table} which maps partially
qualified names of constructions to absolute names. This table is
updated by the commands {\tt Require} (see \ref{Require}), {\tt
Import} and {\tt Export} (see \ref{Import}) and also each time a new
declaration is added to the context. An absolute name is called {\it
visible} from a given short or partially qualified name when this
latter name is enough to denote it. This means that the short or
partially qualified name is mapped to the absolute name in {\Coq} name
table. Definitions flagged as {\tt Local} are only accessible with their
fully qualified name (see \ref{Definition}).

It may happen that a visible name is hidden by the short name or a
qualified name of another construction. In this case, the name that
has been hidden must be referred to using one more level of
qualification. To ensure that a construction always remains
accessible, absolute names can never be hidden.

Examples:
\begin{coq_eval}
Reset Initial.
\end{coq_eval}
\begin{coq_example}
Check 0.
Definition nat := bool.
Check 0.
Check Datatypes.nat.
Locate nat.
\end{coq_example}

\SeeAlso Command {\tt Locate} in Section~\ref{Locate} and {\tt Locate
Library} in Section~\ref{Locate Library}.

\subsection{Libraries and filesystem\label{loadpath}\index{Loadpath}
\index{Physical paths} \index{Logical paths}}

Please note that the questions described here have been subject to
redesign in Coq v8.5. Former versions of Coq use the same terminology
to describe slightly different things.

Compiled files (\texttt{.vo} and \texttt{.vio}) store sub-libraries. In
order to refer to them inside {\Coq}, a translation from file-system
names to {\Coq} names is needed. In this translation, names in the
file system are called {\em physical} paths while {\Coq} names are
contrastingly called {\em logical} names.

A logical prefix {\tt Lib} can be associated to a physical path
\textrm{\textsl{path}} using the command line option {\tt -Q}
\textrm{\textsl{path}} {\tt Lib}. All subfolders of {\textsl{path}} are
recursively associated to the logical path {\tt Lib} extended with the
corresponding suffix coming from the physical path. For instance, the
folder {\tt path/fOO/Bar} maps to {\tt Lib.fOO.Bar}. Subdirectories
corresponding to invalid {\Coq} identifiers are skipped, and, by
convention, subdirectories named {\tt CVS} or {\tt \_darcs} are
skipped too.

Thanks to this mechanism, {\texttt{.vo}} files are made available through the
logical name of the folder they are in, extended with their own basename. For
example, the name associated to the file {\tt path/fOO/Bar/File.vo} is
{\tt Lib.fOO.Bar.File}. The same caveat applies for invalid identifiers.
When compiling a source file, the {\texttt{.vo}} file stores its logical name,
so that an error is issued if it is loaded with the wrong loadpath afterwards.

Some folders have a special status and are automatically put in the path.
{\Coq} commands associate automatically a logical path to files
in the repository trees rooted at the directory from where the command
is launched, \textit{coqlib}\texttt{/user-contrib/}, the directories
listed in the \verb:$COQPATH:, \verb:${XDG_DATA_HOME}/coq/: and
\verb:${XDG_DATA_DIRS}/coq/: environment variables (see
\url{http://standards.freedesktop.org/basedir-spec/basedir-spec-latest.html})
with the same physical-to-logical translation and with an empty logical prefix.

The command line option \texttt{-R} is a variant of \texttt{-Q} which has the
strictly same behavior regarding loadpaths, but which also makes the
corresponding \texttt{.vo} files available through their short names in a
way not unlike the {\tt Import} command (see~{\ref{Import}}). For instance,
\texttt{-R} \textrm{\textsl{path}} \texttt{Lib} associates to the file
\texttt{path/fOO/Bar/File.vo} the logical name \texttt{Lib.fOO.Bar.File}, but
allows this file to be accessed through the short names \texttt{fOO.Bar.File},
\texttt{Bar.File} and \texttt{File}. If several files with identical base name
are present in different subdirectories of a recursive loadpath, which of
these files is found first may be system-dependent and explicit
qualification is recommended. The {\tt From} argument of the {\tt Require}
command can be used to bypass the implicit shortening by providing an absolute
root to the required file (see~\ref{Require}).

There also exists another independent loadpath mechanism attached to {\ocaml}
object files (\texttt{.cmo} or \texttt{.cmxs}) rather than {\Coq} object files
as described above. The {\ocaml} loadpath is managed using the option
\texttt{-I path} (in the {\ocaml} world, there is neither a notion of logical
name prefix nor a way to access files in subdirectories of \texttt{path}).
See the command \texttt{Declare ML Module} in Section~\ref{compiled} to
understand the need of the {\ocaml} loadpath.

See Section~\ref{coqoptions} for a more general view over the {\Coq}
command line options.

%% \paragraph{The special case of remarks and facts}
%% 
%% In contrast with definitions, lemmas, theorems, axioms and parameters,
%% the absolute name of remarks includes the segment of sections in which
%% it is defined. Concretely, if a remark {\tt R} is defined in
%% subsection {\tt S2} of section {\tt S1} in module {\tt M}, then its
%% absolute name is {\tt M.S1.S2.R}. The same for facts, except that the
%% name of the innermost section is dropped from the full name. Then, if
%% a fact {\tt F} is defined in subsection {\tt S2} of section {\tt S1}
%% in module {\tt M}, then its absolute name is {\tt M.S1.F}.

\section{Implicit arguments
\index{Implicit arguments}
\label{Implicit Arguments}}

An implicit argument of a function is an argument which can be
inferred from contextual knowledge. There are different kinds of
implicit arguments that can be considered implicit in different
ways. There are also various commands to control the setting or the
inference of implicit arguments.

\subsection{The different kinds of implicit arguments}

\subsubsection{Implicit arguments inferable from the knowledge of other 
arguments of a function}

The first kind of implicit arguments covers the arguments that are
inferable from the knowledge of the type of other arguments of the
function, or of the type of the surrounding context of the
application.  Especially, such implicit arguments correspond to 
parameters dependent in the type of the function. Typical implicit
arguments are the type arguments in polymorphic functions.  
There are several kinds of such implicit arguments.

\paragraph{Strict Implicit Arguments.} 
An implicit argument can be either strict or non strict. An implicit
argument is said {\em strict} if, whatever the other arguments of the
function are, it is still inferable from the type of some other
argument. Technically, an implicit argument is strict if it
corresponds to a parameter which is not applied to a variable which
itself is another parameter of the function (since this parameter
may erase its arguments), not in the body of a {\tt match}, and not
itself applied or matched against patterns (since the original
form of the argument can be lost by reduction).

For instance, the first argument of
\begin{quote}
\verb|cons: forall A:Set, A -> list A -> list A|
\end{quote}
in module {\tt List.v} is strict because {\tt list} is an inductive
type and {\tt A} will always be inferable from the type {\tt
list A} of the third argument of {\tt cons}.
On the contrary, the second argument of a term of type 
\begin{quote}
\verb|forall P:nat->Prop, forall n:nat, P n -> ex nat P|
\end{quote}
is implicit but not strict, since it can only be inferred from the
type {\tt P n} of the third argument and if {\tt P} is, e.g., {\tt
fun \_ => True}, it reduces to an expression where {\tt n} does not
occur any longer. The first argument {\tt P} is implicit but not
strict either because it can only be inferred from {\tt P n} and {\tt
P} is not canonically inferable from an arbitrary {\tt n} and the
normal form of {\tt P n} (consider e.g. that {\tt n} is {\tt 0} and
the third argument has type {\tt True}, then any {\tt P} of the form
{\tt fun n => match n with 0 => True | \_ => \mbox{\em anything} end} would
be a solution of the inference problem).

\paragraph{Contextual Implicit Arguments.} 
An implicit argument can be {\em contextual} or not. An implicit
argument is said {\em contextual} if it can be inferred only from the
knowledge of the type of the context of the current expression. For
instance, the only argument of
\begin{quote}
\verb|nil : forall A:Set, list A|
\end{quote}
is contextual. Similarly, both arguments of a term of type
\begin{quote}
\verb|forall P:nat->Prop, forall n:nat, P n \/ n = 0|
\end{quote}
are contextual (moreover, {\tt n} is strict and {\tt P} is not).

\paragraph{Reversible-Pattern Implicit Arguments.}
There is another class of implicit arguments that can be reinferred
unambiguously if all the types of the remaining arguments are
known. This is the class of implicit arguments occurring in the type
of another argument in position of reversible pattern, which means it
is at the head of an application but applied only to uninstantiated
distinct variables. Such an implicit argument is called {\em
reversible-pattern implicit argument}. A typical example is the
argument {\tt P} of {\tt nat\_rec} in
\begin{quote}
{\tt nat\_rec : forall P : nat -> Set,
       P 0 -> (forall n : nat, P n -> P (S n)) -> forall x : nat, P x}.
\end{quote}
({\tt P} is reinferable by abstracting over {\tt n} in the type {\tt P n}).

See Section~\ref{SetReversiblePatternImplicit} for the automatic declaration
of reversible-pattern implicit arguments.

\subsubsection{Implicit arguments inferable by resolution}

This corresponds to a class of non dependent implicit arguments that
are solved based on the structure of their type only.

\subsection{Maximal or non maximal insertion of implicit arguments}

In case a function is partially applied, and the next argument to be
applied is an implicit argument, two disciplines are applicable. In the
first case, the function is considered to have no arguments furtherly:
one says that the implicit argument is not maximally inserted. In
the second case, the function is considered to be implicitly applied
to the implicit arguments it is waiting for: one says that the
implicit argument is maximally inserted.

Each implicit argument can be declared to have to be inserted
maximally or non maximally. This can be governed argument per argument
by the command {\tt Implicit Arguments} (see~\ref{ImplicitArguments})
or globally by the command {\tt Set Maximal Implicit Insertion}
(see~\ref{SetMaximalImplicitInsertion}). See also
Section~\ref{PrintImplicit}.

\subsection{Casual use of implicit arguments}

In a given expression, if it is clear that some argument of a function
can be inferred from the type of the other arguments, the user can
force the given argument to be guessed by replacing it by ``{\tt \_}''. If
possible, the correct argument will be automatically generated.

\begin{ErrMsgs}

\item \errindex{Cannot infer a term for this placeholder}

  {\Coq} was not able to deduce an instantiation of a ``{\tt \_}''.

\end{ErrMsgs}

\subsection{Declaration of implicit arguments
\comindex{Arguments}}
\label{ImplicitArguments}

In case one wants that some arguments of a given object (constant,
inductive types, constructors, assumptions, local or not) are always
inferred by Coq, one may declare once and for all which are the expected
implicit arguments of this object. There are two ways to do this,
a priori and a posteriori.

\subsubsection{Implicit Argument Binders}

In the first setting, one wants to explicitly give the implicit
arguments of a declared object as part of its definition. To do this, one has
to surround the bindings of implicit arguments by curly braces:
\begin{coq_eval}
Reset Initial.
\end{coq_eval}
\begin{coq_example}
Definition id {A : Type} (x : A) : A := x.
\end{coq_example}

This automatically declares the argument {\tt A} of {\tt id} as a
maximally inserted implicit argument. One can then do as-if the argument
was absent in every situation but still be able to specify it if needed:
\begin{coq_example}
Definition compose {A B C} (g : B -> C) (f : A -> B) := 
  fun x => g (f x).
Goal forall A, compose id id = id (A:=A).
\end{coq_example}

The syntax is supported in all top-level definitions: {\tt Definition},
{\tt Fixpoint}, {\tt Lemma} and so on. For (co-)inductive datatype
declarations, the semantics are the following: an inductive parameter
declared as an implicit argument need not be repeated in the inductive
definition but will become implicit for the constructors of the
inductive only, not the inductive type itself. For example:

\begin{coq_example}
Inductive list {A : Type} : Type :=
| nil : list
| cons : A -> list -> list.
Print list.
\end{coq_example}

One can always specify the parameter if it is not uniform using the
usual implicit arguments disambiguation syntax.

\subsubsection{Declaring Implicit Arguments}

To set implicit arguments a posteriori, one can use the
command:
\begin{quote}
\tt Arguments {\qualid} \nelist{\possiblybracketedident}{}
\end{quote}
where the list of {\possiblybracketedident} is a prefix of the list of arguments
of {\qualid} where the ones to be declared implicit are surrounded by square
brackets and the ones to be declared as maximally inserted implicits are
surrounded by curly braces.

After the above declaration is issued, implicit arguments can just (and
have to) be skipped in any expression involving an application of
{\qualid}.

Implicit arguments can be cleared with the following syntax:

\begin{quote}
{\tt Arguments {\qualid} : clear implicits
\comindex{Arguments}}
\end{quote}

\begin{Variants}
\item {\tt Global Arguments {\qualid} \nelist{\possiblybracketedident}{}
\comindex{Global Arguments}}

Tell to recompute the implicit arguments of {\qualid} after ending of
the current section if any, enforcing the implicit arguments known
from inside the section to be the ones declared by the command.

\item {\tt Local Arguments {\qualid} \nelist{\possiblybracketedident}{}
\comindex{Local Arguments}}

When in a module, tell not to activate the implicit arguments of
{\qualid} declared by this command to contexts that require the
module.

\item {\tt \zeroone{Global {\sl |} Local} Arguments {\qualid} \sequence{\nelist{\possiblybracketedident}{}}{,}}

For names of constants, inductive types, constructors, lemmas which
can only be applied to a fixed number of arguments (this excludes for
instance constants whose type is polymorphic), multiple 
implicit arguments decflarations can be given. 
Depending on the number of arguments {\qualid} is applied
to in practice, the longest applicable list of implicit arguments is
used to select which implicit arguments are inserted.

For printing, the omitted arguments are the ones of the longest list
of implicit arguments of the sequence.

\end{Variants}

\Example
\begin{coq_eval}
Reset Initial.
\end{coq_eval}
\begin{coq_example*}
Inductive list (A:Type) : Type :=
 | nil : list A 
 | cons : A -> list A -> list A.
\end{coq_example*}
\begin{coq_example}
Check (cons nat 3 (nil nat)).
Arguments cons [A] _ _.
Arguments nil [A].
Check (cons 3 nil).
Fixpoint map (A B:Type) (f:A->B) (l:list A) : list B :=
  match l with nil => nil | cons a t => cons (f a) (map A B f t) end.
Fixpoint length (A:Type) (l:list A) : nat :=
  match l with nil => 0 | cons _ m => S (length A m) end.
Arguments map [A B] f l.
Arguments length {A} l. (* A has to be maximally inserted *)
Check (fun l:list (list nat) => map length l).
Arguments map [A B] f l, [A] B f l, A B f l.
Check (fun l => map length l = map (list nat) nat length l).
\end{coq_example}

\Rem To know which are the implicit arguments of an object, use the command
{\tt Print Implicit} (see \ref{PrintImplicit}).

\subsection{Automatic declaration of implicit arguments}

{\Coq} can also automatically detect what are the implicit arguments
of a defined object. The command is just
\begin{quote}
{\tt Arguments {\qualid} : default implicits
\comindex{Arguments}}
\end{quote}
The auto-detection is governed by options telling if strict,
contextual, or reversible-pattern implicit arguments must be
considered or not (see
Sections~\ref{SetStrictImplicit},~\ref{SetContextualImplicit},~\ref{SetReversiblePatternImplicit}
and also~\ref{SetMaximalImplicitInsertion}).

\begin{Variants}
\item {\tt Global Arguments {\qualid} : default implicits
\comindex{Global Arguments}}

Tell to recompute the implicit arguments of {\qualid} after ending of
the current section if any.

\item {\tt Local Arguments {\qualid} : default implicits
\comindex{Local Arguments}}

When in a module, tell not to activate the implicit arguments of
{\qualid} computed by this declaration to contexts that requires the
module.

\end{Variants}

\Example
\begin{coq_eval}
Reset Initial.
\end{coq_eval}
\begin{coq_example*}
Inductive list (A:Set) : Set := 
  | nil : list A 
  | cons : A -> list A -> list A.
\end{coq_example*}
\begin{coq_example}
Arguments cons : default implicits.
Print Implicit cons.
Arguments nil : default implicits.
Print Implicit nil.
Set Contextual Implicit.
Arguments nil : default implicits.
Print Implicit nil.
\end{coq_example}

The computation of implicit arguments takes account of the
unfolding of constants.  For instance, the variable {\tt p} below has
type {\tt (Transitivity R)} which is reducible to {\tt forall x,y:U, R x
y -> forall z:U, R y z -> R x z}. As the variables {\tt x}, {\tt y} and
{\tt z} appear strictly in body of the type, they are implicit.

\begin{coq_example*}
Variable X : Type.
Definition Relation := X -> X -> Prop.
Definition Transitivity (R:Relation) :=
  forall x y:X, R x y -> forall z:X, R y z -> R x z.
Variables (R : Relation) (p : Transitivity R).
Arguments p : default implicits.
\end{coq_example*}
\begin{coq_example}
Print p.
Print Implicit p.
\end{coq_example}
\begin{coq_example*}
Variables (a b c : X) (r1 : R a b) (r2 : R b c).
\end{coq_example*}
\begin{coq_example}
Check (p r1 r2).
\end{coq_example}

\subsection{Mode for automatic declaration of implicit arguments
\label{Auto-implicit}
\optindex{Implicit Arguments}}

In case one wants to systematically declare implicit the arguments
detectable as such, one may switch to the automatic declaration of
implicit arguments mode by using the command
\begin{quote}
\tt Set Implicit Arguments.
\end{quote}
Conversely, one may unset the mode by using {\tt Unset Implicit
Arguments}.  The mode is off by default. Auto-detection of implicit
arguments is governed by options controlling whether strict and
contextual implicit arguments have to be considered or not.

\subsection{Controlling strict implicit arguments
\optindex{Strict Implicit}
\label{SetStrictImplicit}}

When the mode for automatic declaration of implicit arguments is on,
the default is to automatically set implicit only the strict implicit
arguments plus, for historical reasons, a small subset of the non
strict implicit arguments. To relax this constraint and to
set implicit all non strict implicit arguments by default, use the command
\begin{quote}
\tt Unset Strict Implicit.
\end{quote}
Conversely, use the command {\tt Set Strict Implicit} to
restore the original mode that declares implicit only the strict implicit arguments plus a small subset of the non strict implicit arguments.

In the other way round, to capture exactly the strict implicit arguments and no more than the strict implicit arguments, use the command:
\optindex{Strongly Strict Implicit}
\begin{quote}
\tt Set Strongly Strict Implicit.
\end{quote}
Conversely, use the command {\tt Unset Strongly Strict Implicit} to
let the option ``{\tt Strict Implicit}'' decide what to do.

\Rem In versions of {\Coq} prior to version 8.0, the default was to
declare the strict implicit arguments as implicit.

\subsection{Controlling contextual implicit arguments
\optindex{Contextual Implicit}
\label{SetContextualImplicit}}

By default, {\Coq} does not automatically set implicit the contextual
implicit arguments. To tell {\Coq} to infer also contextual implicit
argument, use command  
\begin{quote}
\tt Set Contextual Implicit. 
\end{quote}
Conversely, use command {\tt Unset Contextual Implicit} to
unset the contextual implicit mode.

\subsection{Controlling reversible-pattern implicit arguments
\optindex{Reversible Pattern Implicit}
\label{SetReversiblePatternImplicit}}

By default, {\Coq} does not automatically set implicit the reversible-pattern
implicit arguments. To tell {\Coq} to infer also reversible-pattern implicit
argument, use command  
\begin{quote}
\tt Set Reversible Pattern Implicit. 
\end{quote}
Conversely, use command {\tt Unset Reversible Pattern Implicit} to
unset the reversible-pattern implicit mode.

\subsection{Controlling the insertion of implicit arguments not followed by explicit arguments
\optindex{Maximal Implicit Insertion}
\label{SetMaximalImplicitInsertion}}

Implicit arguments can be declared to be automatically inserted when a
function is partially applied and the next argument of the function is
an implicit one. In case the implicit arguments are automatically
declared (with the command {\tt Set Implicit Arguments}), the command
\begin{quote}
\tt Set Maximal Implicit Insertion. 
\end{quote}
is used to tell to declare the implicit arguments with a maximal
insertion status. By default, automatically declared implicit
arguments are not declared to be insertable maximally.  To restore the
default mode for maximal insertion, use command {\tt Unset Maximal
Implicit Insertion}.

\subsection{Explicit applications
\index{Explicitly given implicit arguments}
\label{Implicits-explicitation}
\index{qualid@{\qualid}} \index{\symbol{64}}}

In presence of non strict or contextual argument, or in presence of
partial applications, the synthesis of implicit arguments may fail, so
one may have to give explicitly certain implicit arguments of an
application. The syntax for this is {\tt (\ident:=\term)} where {\ident}
is the name of the implicit argument and {\term} is its corresponding
explicit term. Alternatively, one can locally deactivate the hiding of
implicit arguments of a function by using the notation
{\tt @{\qualid}~{\term}$_1$..{\term}$_n$}. This syntax extension is
given Figure~\ref{fig:explicitations}.
\begin{figure}
\begin{centerframe}
\begin{tabular}{lcl}
{\term} & ++= & @ {\qualid} \nelist{\term}{}\\
& $|$ & @ {\qualid}\\
& $|$ & {\qualid} \nelist{\textrm{\textsl{argument}}}{}\\
\\
{\textrm{\textsl{argument}}} & ::= & {\term} \\
& $|$ & {\tt ({\ident}:={\term})}\\
\end{tabular}
\end{centerframe}
\caption{Syntax for explicitly giving implicit arguments}
\label{fig:explicitations}
\end{figure}

\noindent {\bf Example (continued): }
\begin{coq_example}
Check (p r1 (z:=c)).
Check (p (x:=a) (y:=b) r1 (z:=c) r2).
\end{coq_example}

\subsection{Renaming implicit arguments
\comindex{Arguments}
}

Implicit arguments names can be redefined using the following syntax:
\begin{quote}
{\tt Arguments {\qualid} \nelist{\name}{}  : rename}
\end{quote}

With the {\tt assert} flag, {\tt Arguments} can be used to assert
that a given object has the expected number of arguments and that
these arguments are named as expected.

\noindent {\bf Example (continued): }
\begin{coq_example}
Arguments p [s t] _ [u] _: rename.
Check (p r1 (u:=c)).
Check (p (s:=a) (t:=b) r1 (u:=c) r2).
Fail Arguments p [s t] _ [w] _ : assert.
\end{coq_example}


\subsection{Displaying what the implicit arguments are
\comindex{Print Implicit}
\label{PrintImplicit}}

To display the implicit arguments associated to an object, and to know
if each of them is to be used maximally or not, use the command
\begin{quote}
\tt Print Implicit {\qualid}.
\end{quote}

\subsection{Explicit displaying of implicit arguments for pretty-printing
\optindex{Printing Implicit}
\optindex{Printing Implicit Defensive}}

By default the basic pretty-printing rules hide the inferable implicit
arguments of an application. To force printing all implicit arguments,
use command
\begin{quote}
{\tt Set Printing Implicit.}
\end{quote}
Conversely, to restore the hiding of implicit arguments, use command
\begin{quote}
{\tt Unset Printing Implicit.}
\end{quote}

By default the basic pretty-printing rules display the implicit arguments that are not detected as strict implicit arguments. This ``defensive'' mode can quickly make the display cumbersome so this can be deactivated by using the command
\begin{quote}
{\tt Unset Printing Implicit Defensive.}
\end{quote}
Conversely, to force the display of non strict arguments, use command
\begin{quote}
{\tt Set Printing Implicit Defensive.}
\end{quote}

\SeeAlso {\tt Set Printing All} in Section~\ref{SetPrintingAll}.

\subsection{Interaction with subtyping}

When an implicit argument can be inferred from the type of more than
one of the other arguments, then only the type of the first of these
arguments is taken into account, and not an upper type of all of
them.  As a consequence, the inference of the implicit argument of
``='' fails in
\begin{coq_example*}
Fail Check nat = Prop.
\end{coq_example*}

but succeeds in
\begin{coq_example*}
Check Prop = nat.
\end{coq_example*}

\subsection{Deactivation of implicit arguments for parsing}
\optindex{Parsing Explicit}

Use of implicit arguments can be deactivated by issuing the command:
\begin{quote}
{\tt Set Parsing Explicit.}
\end{quote}

In this case, all arguments of constants, inductive types,
constructors, etc, including the arguments declared as implicit, have
to be given as if none arguments were implicit. By symmetry, this also
affects printing. To restore parsing and normal printing of implicit
arguments, use:
\begin{quote}
{\tt Set Parsing Explicit.}
\end{quote}

\subsection{Canonical structures
\comindex{Canonical Structure}}

A canonical structure is an instance of a record/structure type that
can be used to solve unification problems involving a projection
applied to an unknown structure instance (an implicit argument) and
a value.  The complete documentation of canonical structures can be found
in Chapter~\ref{CS-full}, here only a simple example is given.

Assume that {\qualid} denotes an object $(Build\_struc~ c_1~ \ldots~ c_n)$ in
the
structure {\em struct} of which the fields are $x_1$, ...,
$x_n$. Assume that {\qualid} is declared as a canonical structure
using the command
\begin{quote}
{\tt Canonical Structure {\qualid}.}
\end{quote}
Then, each time an equation of the form $(x_i~
\_)=_{\beta\delta\iota\zeta}c_i$ has to be solved during the
type-checking process, {\qualid} is used as a solution. Otherwise
said, {\qualid} is canonically used to extend the field $c_i$ into a
complete structure built on $c_i$.

Canonical structures are particularly useful when mixed with
coercions and strict implicit arguments. Here is an example.
\begin{coq_example*}
Require Import Relations.
Require Import EqNat.
Set Implicit Arguments.
Unset Strict Implicit.
Structure Setoid : Type := 
  {Carrier :> Set;
   Equal : relation Carrier;
   Prf_equiv : equivalence Carrier Equal}.
Definition is_law (A B:Setoid) (f:A -> B) :=
  forall x y:A, Equal x y -> Equal (f x) (f y).
Axiom eq_nat_equiv : equivalence nat eq_nat.
Definition nat_setoid : Setoid := Build_Setoid eq_nat_equiv.
Canonical Structure nat_setoid.
\end{coq_example*}

Thanks to \texttt{nat\_setoid} declared as canonical, the implicit
arguments {\tt A} and {\tt B} can be synthesized in the next statement.
\begin{coq_example}
Lemma is_law_S : is_law S.
\end{coq_example}

\Rem If a same field occurs in several canonical structure, then
only the structure declared first as canonical is considered.

\begin{Variants}
\item {\tt Canonical Structure {\ident} := {\term} : {\type}.}\\
 {\tt Canonical Structure {\ident} := {\term}.}\\
 {\tt Canonical Structure {\ident} : {\type} := {\term}.}

These are equivalent to a regular definition of {\ident} followed by
the declaration 

{\tt Canonical Structure {\ident}}.
\end{Variants}

\SeeAlso more examples in user contribution \texttt{category}
(\texttt{Rocq/ALGEBRA}).

\subsubsection{Print Canonical Projections.
\comindex{Print Canonical Projections}}

This displays the list of global names that are components of some
canonical structure. For each of them, the canonical structure of
which it is a projection is indicated. For instance, the above example 
gives the following output:

\begin{coq_example}
Print Canonical Projections.
\end{coq_example}

\subsection{Implicit types of variables}
\comindex{Implicit Types}

It is possible to bind variable names to a given type (e.g. in a
development using arithmetic, it may be convenient to bind the names
{\tt n} or {\tt m} to the type {\tt nat} of natural numbers). The
command for that is
\begin{quote}
\tt Implicit Types \nelist{\ident}{} : {\type}
\end{quote}
The effect of the command is to automatically set the type of bound
variables starting with {\ident} (either {\ident} itself or
{\ident} followed by one or more single quotes, underscore or digits)
to be {\type} (unless the bound variable is already declared with an
explicit type in which case, this latter type is considered).

\Example
\begin{coq_example}
Require Import List.
Implicit Types m n : nat.
Lemma cons_inj_nat : forall m n l, n :: l = m :: l -> n = m.
intros m n.
Lemma cons_inj_bool : forall (m n:bool) l, n :: l = m :: l -> n = m.
\end{coq_example}

\begin{Variants}
\item {\tt Implicit Type {\ident} : {\type}}\\
This is useful for declaring the implicit type of a single variable.
\item
 {\tt Implicit Types\,%
(\,{\ident$_{1,1}$}\ldots{\ident$_{1,k_1}$}\,{\tt :}\,{\term$_1$} {\tt )}\,%
\ldots\,{\tt (}\,{\ident$_{n,1}$}\ldots{\ident$_{n,k_n}$}\,{\tt :}\,%
{\term$_n$} {\tt )}.}\\ 
  Adds $n$ blocks of implicit types with different specifications.
\end{Variants}


\subsection{Implicit generalization
\label{implicit-generalization}
\comindex{Generalizable Variables}}

Implicit generalization is an automatic elaboration of a statement with
free variables into a closed statement where these variables are
quantified explicitly. Implicit generalization is done inside binders
starting with a \texttt{\`{}} and terms delimited by \texttt{\`{}\{ \}} and
\texttt{\`{}( )}, always introducing maximally inserted implicit arguments for
the generalized variables. Inside implicit generalization
delimiters, free variables in the current context are automatically
quantified using a product or a lambda abstraction to generate a closed
term. In the following statement for example, the variables \texttt{n}
and \texttt{m} are automatically generalized and become explicit
arguments of the lemma as we are using \texttt{\`{}( )}:

\begin{coq_example}
Generalizable All Variables.
Lemma nat_comm : `(n = n + 0).
\end{coq_example}
\begin{coq_eval}
Abort.
\end{coq_eval}
One can control the set of generalizable identifiers with the
\texttt{Generalizable} vernacular command to avoid unexpected
generalizations when mistyping identifiers. There are three variants of
the command:

\begin{quote}
{\tt Generalizable (All|No) Variable(s)? ({\ident$_1$ \ident$_n$})?.}
\end{quote}

\begin{Variants}
\item {\tt Generalizable All Variables.} All variables are candidate for 
  generalization if they appear free in the context under a
  generalization delimiter. This may result in confusing errors in
  case of typos. In such cases, the context will probably contain some
  unexpected generalized variable.

\item {\tt Generalizable No Variables.} Disable implicit generalization 
  entirely. This is the default behavior.

\item {\tt Generalizable Variable(s)? {\ident$_1$ \ident$_n$}.} 
  Allow generalization of the given identifiers only. Calling this
  command multiple times adds to the allowed identifiers.

\item {\tt Global Generalizable} Allows to export the choice of
  generalizable variables.
\end{Variants}

One can also use implicit generalization for binders, in which case the
generalized variables are added as binders and set maximally implicit.
\begin{coq_example*}
Definition id `(x : A) : A := x.
\end{coq_example*}
\begin{coq_example}
Print id.
\end{coq_example}

The generalizing binders \texttt{\`{}\{ \}} and \texttt{\`{}( )} work similarly to
their explicit counterparts, only binding the generalized variables
implicitly, as maximally-inserted arguments. In these binders, the
binding name for the bound object is optional, whereas the type is
mandatory, dually to regular binders.

\section{Coercions
\label{Coercions}
\index{Coercions}}

Coercions can be used to implicitly inject terms from one {\em class} in
which they reside into another one. A {\em class} is either a sort
(denoted by the keyword {\tt Sortclass}), a product type (denoted by the
keyword {\tt Funclass}), or a type constructor (denoted by its name),
e.g. an inductive type or any constant with a type of the form
\texttt{forall} $(x_1:A_1) .. (x_n:A_n),~s$ where $s$ is a sort.

Then the user is able to apply an
object that is not a function, but can be coerced to a function, and
more generally to consider that a term of type A is of type B provided
that there is a declared coercion between A and B. The main command is
\comindex{Coercion}
\begin{quote}
\tt Coercion {\qualid} : {\class$_1$} >-> {\class$_2$}.
\end{quote}
which declares the construction denoted by {\qualid} as a
coercion between {\class$_1$} and {\class$_2$}.

More details and examples, and a description of the commands related
to coercions are provided in Chapter~\ref{Coercions-full}.

\section[Printing constructions in full]{Printing constructions in full\label{SetPrintingAll}
\optindex{Printing All}}

Coercions, implicit arguments, the type of pattern-matching, but also
notations (see Chapter~\ref{Addoc-syntax}) can obfuscate the behavior
of some tactics (typically the tactics applying to occurrences of
subterms are sensitive to the implicit arguments). The command
\begin{quote}
{\tt Set Printing All.}
\end{quote}
deactivates all high-level printing features such as coercions,
implicit arguments, returned type of pattern-matching, notations and
various syntactic sugar for pattern-matching or record projections.
Otherwise said, {\tt Set Printing All} includes the effects
of the commands {\tt Set Printing Implicit}, {\tt Set Printing
Coercions}, {\tt Set Printing Synth}, {\tt Unset Printing Projections}
and {\tt Unset Printing Notations}.  To reactivate the high-level
printing features, use the command
\begin{quote}
{\tt Unset Printing All.}
\end{quote}

\section[Printing universes]{Printing universes\label{PrintingUniverses}
\optindex{Printing Universes}}

The following command:
\begin{quote}
{\tt Set Printing Universes}
\end{quote}
activates the display of the actual level of each occurrence of
{\Type}. See Section~\ref{Sorts} for details.  This wizard option, in
combination with \texttt{Set Printing All} (see
section~\ref{SetPrintingAll}) can help to diagnose failures to unify
terms apparently identical but internally different in the Calculus of
Inductive Constructions. To reactivate the display of the actual level
of the occurrences of {\Type}, use
\begin{quote}
{\tt Unset Printing Universes.}
\end{quote}

\comindex{Print Universes}
\comindex{Print Sorted Universes}

The constraints on the internal level of the occurrences of {\Type}
(see Section~\ref{Sorts}) can be printed using the command
\begin{quote}
{\tt Print \zeroone{Sorted} Universes.}
\end{quote}
If the optional {\tt Sorted} option is given, each universe will be
made equivalent to a numbered label reflecting its level (with a
linear ordering) in the universe hierarchy.

This command also accepts an optional output filename:
\begin{quote}
\tt Print \zeroone{Sorted} Universes {\str}.
\end{quote}
If {\str} ends in \texttt{.dot} or \texttt{.gv}, the constraints are
printed in the DOT language, and can be processed by Graphviz
tools. The format is unspecified if {\str} doesn't end in
\texttt{.dot} or \texttt{.gv}.

\section[Existential variables]{Existential variables\label{ExistentialVariables}}
\label{evars}

Coq terms can include existential variables which
represents unknown subterms to eventually be replaced by actual
subterms.

Existential variables are generated in place of unsolvable implicit
arguments or ``{\tt \_}'' placeholders when using commands such as
\texttt{Check} (see Section~\ref{Check}) or when using tactics such as
\texttt{refine}~(see Section~\ref{refine}), as well as in place of unsolvable
instances when using tactics such that \texttt{eapply} (see
Section~\ref{eapply}). An existential variable is defined in a
context, which is the context of variables of the placeholder which
generated the existential variable, and a type, which is the expected
type of the placeholder. 

As a consequence of typing constraints, existential variables can be
duplicated in such a way that they possibly appear in different
contexts than their defining context. Thus, any occurrence of a given
existential variable comes with an instance of its original context. In the
simple case, when an existential variable denotes the placeholder
which generated it, or is used in the same context as the one in which
it was generated, the context is not displayed and the existential
variable is represented by ``?'' followed by an identifier.

\begin{coq_example}
Parameter identity : forall (X:Set), X -> X.
Check identity _ _.
Check identity _ (fun x => _).
\end{coq_example}

In the general case, when an existential variable ?{\ident}
appears outside of its context of definition, its instance, written under
the form \verb!@{id1:=term1; ...; idn:=termn}!, is appending to its
name, indicating how the variables of its defining context are
instantiated.  The variables of the context of the existential
variables which are instantiated by themselves are not written, unless
the flag {\tt Printing Existential Instances} is on (see
Section~\ref{SetPrintingExistentialInstances}), and this is why an
existential variable used in the same context as its context of
definition is written with no instance.

\begin{coq_example}
Check (fun x y => _) 0 1.
Set Printing Existential Instances.
Check (fun x y => _) 0 1.
\end{coq_example}

\begin{coq_eval}
Unset Printing Existential Instances.
\end{coq_eval}

\subsection{Explicit displaying of existential instances for pretty-printing
\label{SetPrintingExistentialInstances}
\optindex{Printing Existential Instances}}

The command:
\begin{quote}
{\tt Set Printing Existential Instances}
\end{quote}
activates the full display of how the context of an existential variable is
instantiated at each of the occurrences of the existential variable.

To deactivate the full display of the instances of existential
variables, use
\begin{quote}
{\tt Unset Printing Existential Instances.}
\end{quote}

\subsection{Solving existential variables using tactics}
\ttindex{ltac:( \ldots )}

\def\expr{\textrm{\textsl{tacexpr}}}

Instead of letting the unification engine try to solve an existential variable
by itself, one can also provide an explicit hole together with a tactic to solve
it. Using the syntax {\tt ltac:(\expr)}, the user can put a
tactic anywhere a term is expected. The order of resolution is not specified and
is implementation-dependent. The inner tactic may use any variable defined in
its scope, including repeated alternations between variables introduced by term
binding as well as those introduced by tactic binding. The expression {\expr}
can be any tactic expression as described at section~\ref{TacticLanguage}.

\begin{coq_example*}
Definition foo (x : nat) : nat := ltac:(exact x).
\end{coq_example*}

This construction is useful when one wants to define complicated terms using
highly automated tactics without resorting to writing the proof-term by means of
the interactive proof engine.

This mechanism is comparable to the {\tt Declare Implicit Tactic} command
defined at~\ref{DeclareImplicit}, except that the used tactic is local to each
hole instead of being declared globally.

%%% Local Variables: 
%%% mode: latex
%%% TeX-master: "Reference-Manual"
%%% End: 
