% TODO: unify the use of \form and \type to mean a type
% or use \form specifically for a type of type Prop
\chapter{Tactics
\index{Tactics}
\label{Tactics}}

A deduction rule is a link between some (unique) formula, that we call
the {\em conclusion} and (several) formulas that we call the {\em
premises}. A deduction rule can be read in two ways. The first
one says: {\it ``if I know this and this then I can deduce
this''}. For instance, if I have a proof of $A$ and a proof of $B$
then I have a proof of $A \land B$. This is forward reasoning from
premises to conclusion. The other way says: {\it ``to prove this I
have to prove this and this''}. For instance, to prove $A \land B$, I
have to prove $A$ and I have to prove $B$. This is backward reasoning
from conclusion to premises. We say that the conclusion
is the {\em goal}\index{goal} to prove and premises are the {\em
subgoals}\index{subgoal}.  The tactics implement {\em backward
reasoning}. When applied to a goal, a tactic replaces this goal with
the subgoals it generates. We say that a tactic reduces a goal to its
subgoal(s).

Each (sub)goal is denoted with a number. The current goal is numbered
1. By default, a tactic is applied to the current goal, but one can
address a particular goal in the list by writing {\sl n:\tac} which
means {\it ``apply tactic {\tac} to goal number {\sl n}''}.
We can show the list of subgoals by typing {\tt Show} (see
Section~\ref{Show}).

Since not every rule applies to a given statement, every tactic cannot be
used to reduce any goal. In other words, before applying a tactic to a
given goal, the system checks that some {\em preconditions} are
satisfied. If it is not the case, the tactic raises an error message.

Tactics are built from atomic tactics and tactic expressions (which
extends the folklore notion of tactical) to combine those atomic
tactics. This chapter is devoted to atomic tactics. The tactic
language will be described in Chapter~\ref{TacticLanguage}.

\section{Invocation of tactics
\label{tactic-syntax}
\index{tactic@{\tac}}}

A tactic is applied as an ordinary command. It may be preceded by a
goal selector: {\tt all} if the tactic is to be applied to every
focused goal simultaneously, or a natural number $n$ if it is to be
applied to the $n$-th goal. If no selector is specified, the default
selector (see Section \ref{default-selector}) is used.

\newcommand{\selector}{\nterm{selector}}
\begin{tabular}{lcl}
{\selector} & := & {\tt all} | {\num}\\
{\commandtac} & ::= & {\selector} {\tt :} {\tac} {\tt .}\\
 & $|$ & {\tac} {\tt .}
\end{tabular}
\subsection[\tt Set Default Goal Selector ``\selector''.]
           {\tt Set Default Goal Selector ``\selector''.
            \optindex{Default Goal Selector}
            \label{default-selector}}
After using this command, the default selector -- used when no selector
is specified when applying a tactic -- is set to the chosen value. The
initial value is $1$, hence the tactics are, by default, applied to
the first goal. Using {\tt Set Default Goal Selector ``all''} will
make is so that tactics are, by default, applied to every goal
simultaneously. Then, to apply a tactic {\tt tac} to the first goal
only, you can write {\tt 1:tac}.

\subsection[\tt Test Default Goal Selector.]
           {\tt Test Default Goal Selector.}
This command displays the current default selector.

\subsection{Bindings list
\index{Binding list}
\label{Binding-list}}

Tactics that take a term as argument may also support a bindings list, so
as to instantiate some parameters of the term by name or position.
The general form of a term equipped with a bindings list is {\tt
{\term} with {\bindinglist}} where {\bindinglist} may be of two
different forms:

\begin{itemize}
\item In a bindings list of the form {\tt (\vref$_1$ := \term$_1$)
  \dots\ (\vref$_n$ := \term$_n$)}, {\vref} is either an {\ident} or a
  {\num}. The references are determined according to the type of
  {\term}. If \vref$_i$ is an identifier, this identifier has to be
  bound in the type of {\term} and the binding provides the tactic
  with an instance for the parameter of this name.  If \vref$_i$ is
  some number $n$, this number denotes the $n$-th non dependent
  premise of the {\term}, as determined by the type of {\term}.

  \ErrMsg \errindex{No such binder}

\item A bindings list can also be a simple list of terms {\tt
  \term$_1$ \dots\ \term$_n$}. In that case the references to
  which these terms correspond are determined by the tactic. In case
  of {\tt induction}, {\tt destruct}, {\tt elim} and {\tt case} (see
  Section~\ref{elim}) the terms have to provide instances for all the
  dependent products in the type of \term\ while in the case of {\tt
  apply}, or of {\tt constructor} and its variants, only instances for
  the dependent products that are not bound in the conclusion of the
  type are required.

  \ErrMsg \errindex{Not the right number of missing arguments}
\end{itemize}

\subsection{Occurrences sets and occurrences clauses}
\label{Occurrences_clauses}
\index{Occurrences clauses}

An occurrences clause is a modifier to some tactics that obeys the
following syntax:

\begin{tabular}{lcl}
{\occclause} & ::= & {\tt in} {\occgoalset} \\
{\occgoalset} & ::= &
    \zeroone{{\ident$_1$} \zeroone{\atoccurrences} {\tt ,} \\
&   & {\dots} {\tt ,}\\
&   & {\ident$_m$} \zeroone{\atoccurrences}}\\
&   & \zeroone{{\tt |-} \zeroone{{\tt *} \zeroone{\atoccurrences}}}\\
& | &
    {\tt *} {\tt |-} \zeroone{{\tt *} \zeroone{\atoccurrences}}\\
& | &
    {\tt *}\\
{\atoccurrences} & ::= & {\tt at} {\occlist}\\
{\occlist} & ::= & \zeroone{{\tt -}} {\num$_1$} \dots\ {\num$_n$}
\end{tabular}

The role of an occurrence clause is to select a set of occurrences of
a {\term} in a goal. In the first case, the {{\ident$_i$}
\zeroone{{\tt at} {\num$_1^i$} \dots\ {\num$_{n_i}^i$}}} parts
indicate that occurrences have to be selected in the hypotheses named
{\ident$_i$}.  If no numbers are given for hypothesis {\ident$_i$},
then all the occurrences of {\term} in the hypothesis are selected. If
numbers are given, they refer to occurrences of {\term} when the term
is printed using option {\tt Set Printing All} (see
Section~\ref{SetPrintingAll}), counting from left to right. In
particular, occurrences of {\term} in implicit arguments (see
Section~\ref{Implicit Arguments}) or coercions (see
Section~\ref{Coercions}) are counted.

If a minus sign is given between {\tt at} and the list of occurrences,
it negates the condition so that the clause denotes all the occurrences except
the ones explicitly mentioned after the minus sign.

As an exception to the left-to-right order, the occurrences in the
{\tt return} subexpression of a {\tt match} are considered {\em
before} the occurrences in the matched term.

In the second case, the {\tt *} on the left of {\tt |-} means that
all occurrences of {\term} are selected in every hypothesis.

In the first and second case, if {\tt *} is mentioned on the right of
{\tt |-}, the occurrences of the conclusion of the goal have to be
selected. If some numbers are given, then only the occurrences denoted
by these numbers are selected. In no numbers are given, all
occurrences of {\term} in the goal are selected.

Finally, the last notation is an abbreviation for {\tt * |- *}. Note
also that {\tt |-} is optional in the first case when no {\tt *} is
given.

Here are some tactics that understand occurrences clauses:
{\tt set}, {\tt remember}, {\tt induction}, {\tt destruct}.

\SeeAlso~Sections~\ref{tactic:set}, \ref{Tac-induction}, \ref{SetPrintingAll}.

\section{Applying theorems}

\subsection{\tt exact \term}
\tacindex{exact}
\label{exact}

This tactic applies to any goal. It gives directly the exact proof
term of the goal. Let {\T} be our goal, let {\tt p} be a term of type
{\tt U} then {\tt exact p} succeeds iff {\tt T} and {\tt U} are
convertible (see Section~\ref{conv-rules}).

\begin{ErrMsgs}
\item \errindex{Not an exact proof}
\end{ErrMsgs}

\begin{Variants}
  \item \texttt{eexact \term}\tacindex{eexact}

    This tactic behaves like \texttt{exact} but is able to handle terms
    and goals with meta-variables.

\end{Variants}

\subsection{\tt assumption}
\tacindex{assumption}

This tactic looks in the local context for an
hypothesis which type is equal to the goal.  If it is the case, the
subgoal is proved. Otherwise, it fails.

\begin{ErrMsgs}
\item  \errindex{No such assumption}
\end{ErrMsgs}

\begin{Variants}
\tacindex{eassumption}
  \item \texttt{eassumption}

    This tactic behaves like \texttt{assumption} but is able to handle
    goals with meta-variables.

\end{Variants}

\subsection{\tt refine \term}
\tacindex{refine}
\label{refine}
\label{refine-example}
\index{?@{\texttt{?}}}

This tactic applies to any goal. It behaves like {\tt exact} with a big
difference: the user can leave some holes (denoted by \texttt{\_} or
{\tt (\_:\type)}) in the term. {\tt refine} will generate as
many subgoals as there are holes in the term. The type of holes must be
either synthesized by the system or declared by an
explicit cast like \verb|(_:nat->Prop)|. Any subgoal that occurs in other
subgoals is automatically shelved, as if calling {\tt shelve\_unifiable}
(see Section~\ref{shelve}).
This low-level tactic can be useful to advanced users.

\Example

\begin{coq_example*}
Inductive Option : Set :=
  | Fail : Option
  | Ok : bool -> Option.
\end{coq_example}
\begin{coq_example}
Definition get : forall x:Option, x <> Fail -> bool.
refine
 (fun x:Option =>
    match x return x <> Fail -> bool with
    | Fail => _
    | Ok b => fun _ => b
    end).
intros; absurd (Fail = Fail); trivial.
\end{coq_example}
\begin{coq_example*}
Defined.
\end{coq_example*}

\begin{ErrMsgs}
\item \errindex{invalid argument}:
  the tactic \texttt{refine} does not know what to do
  with the term you gave.
\item \texttt{Refine passed ill-formed term}: the term you gave is not
  a valid proof (not easy to debug in general).
  This message may also occur in higher-level tactics that call
  \texttt{refine} internally.
\item \errindex{Cannot infer a term for this placeholder}:
  there is a hole in the term you gave
  which type cannot be inferred. Put a cast around it.
\end{ErrMsgs}

\begin{Variants}
\item {\tt simple refine \term}\tacindex{simple refine}

  This tactic behaves like {\tt refine}, but it does not shelve any
  subgoal. It does not perform any beta-reduction either.
\end{Variants}

\subsection{\tt apply \term}
\tacindex{apply}
\label{apply}
\label{eapply}

This tactic applies to any goal.  The argument {\term} is a term
well-formed in the local context.  The tactic {\tt apply} tries to
match the current goal against the conclusion of the type of {\term}.
If it succeeds, then the tactic returns as many subgoals as the number
of non-dependent premises of the type of {\term}. If the conclusion of
the type of {\term} does not match the goal {\em and} the conclusion
is an inductive type isomorphic to a tuple type, then each component
of the tuple is recursively matched to the goal in the left-to-right
order.

The tactic {\tt apply} relies on first-order unification with
dependent types unless the conclusion of the type of {\term} is of the
form {\tt ($P$ $t_1$ \dots\ $t_n$)} with $P$ to be instantiated.  In
the latter case, the behavior depends on the form of the goal. If the
goal is of the form {\tt (fun $x$ => $Q$)~$u_1$~\ldots~$u_n$} and the
$t_i$ and $u_i$ unifies, then $P$ is taken to be {\tt (fun $x$ => $Q$)}.
Otherwise, {\tt apply} tries to define $P$ by abstracting over
$t_1$~\ldots ~$t_n$ in the goal. See {\tt pattern} in
Section~\ref{pattern} to transform the goal so that it gets the form
{\tt (fun $x$ => $Q$)~$u_1$~\ldots~$u_n$}.

\begin{ErrMsgs}
\item \errindex{Impossible to unify \dots\ with \dots}

  The {\tt apply}
  tactic failed to match the conclusion of {\term} and the current goal.
  You can help the {\tt apply} tactic by transforming your
  goal with the {\tt change} or {\tt pattern} tactics (see
  sections~\ref{pattern},~\ref{change}).

\item \errindex{Unable to find an instance for the variables
{\ident} \dots\ {\ident}}

  This occurs when some instantiations of the premises of {\term} are not
  deducible from the unification. This is the case, for instance, when
  you want to apply a transitivity property. In this case, you have to
  use one of the variants below:

\end{ErrMsgs}

\begin{Variants}

\item{\tt apply {\term} with {\term$_1$} \dots\ {\term$_n$}}
  \tacindex{apply \dots\ with}

  Provides {\tt apply} with explicit instantiations for all dependent
  premises of the type of {\term} that do not occur in the conclusion
  and consequently cannot be found by unification. Notice that
  {\term$_1$} \mbox{\dots} {\term$_n$} must be given according to the order
  of these dependent premises of the type of {\term}.

  \ErrMsg \errindex{Not the right number of missing arguments}

\item{\tt apply {\term} with ({\vref$_1$} := {\term$_1$}) \dots\ ({\vref$_n$}
    := {\term$_n$})}

  This also provides {\tt apply} with values for instantiating
  premises. Here, variables are referred by names and non-dependent
  products by increasing numbers (see syntax in Section~\ref{Binding-list}).

\item {\tt apply \term$_1$ , \mbox{\dots} , \term$_n$}

  This is a shortcut for {\tt apply} {\term$_1$} {\tt ; [ ..~|}
   \ldots~{\tt ; [ ..~| {\tt apply} {\term$_n$} ]} \ldots~{\tt ]}, i.e. for the
   successive applications of {\term$_{i+1}$} on the last subgoal
   generated by {\tt apply} {\term$_i$}, starting from the application
   of {\term$_1$}.

\item {\tt eapply \term}\tacindex{eapply}

  The tactic {\tt eapply} behaves like {\tt apply} but it does not fail
  when no instantiations are deducible for some variables in the
  premises. Rather, it turns these variables into 
  existential variables which are variables still to instantiate (see
  Section~\ref{evars}). The instantiation is intended to be found
  later in the proof.

\item {\tt simple apply {\term}} \tacindex{simple apply}

  This behaves like {\tt apply} but it reasons modulo conversion only
  on subterms that contain no variables to instantiate. For instance,
  the following example does not succeed because it would require the
  conversion of {\tt id ?foo} and {\tt O}.

\begin{coq_eval}
Reset Initial.
\end{coq_eval}
\begin{coq_example*}
Definition id (x : nat) := x.
Hypothesis H : forall y, id y = y.
Goal O = O.
\end{coq_example*}
\begin{coq_example}
Fail simple apply H.
\end{coq_example}

  Because it reasons modulo a limited amount of conversion, {\tt
  simple apply} fails quicker than {\tt apply} and it is then
  well-suited for uses in used-defined tactics that backtrack often.
  Moreover, it does not traverse tuples as {\tt apply} does.

\item \zeroone{{\tt simple}} {\tt apply} {\term$_1$} \zeroone{{\tt with}
  {\bindinglist$_1$}} {\tt ,} \ldots {\tt ,} {\term$_n$} \zeroone{{\tt with}
  {\bindinglist$_n$}}\\
  \zeroone{{\tt simple}} {\tt eapply} {\term$_1$} \zeroone{{\tt with}
  {\bindinglist$_1$}} {\tt ,} \ldots {\tt ,} {\term$_n$} \zeroone{{\tt with}
  {\bindinglist$_n$}}

  This summarizes the different syntaxes for {\tt apply} and {\tt eapply}.

\item {\tt lapply {\term}} \tacindex{lapply}

  This tactic applies to any goal, say {\tt G}.  The argument {\term}
  has to be well-formed in the current context, its type being
  reducible to a non-dependent product {\tt A -> B} with {\tt B}
  possibly containing products. Then it generates two subgoals {\tt
  B->G} and {\tt A}. Applying {\tt lapply H} (where {\tt H} has type
  {\tt A->B} and {\tt B} does not start with a product) does the same
  as giving the sequence {\tt cut B. 2:apply H.} where {\tt cut} is
  described below.

  \Warning When {\term} contains more than one non
  dependent product the tactic {\tt lapply} only takes into account the
  first product.

\end{Variants}

\Example
Assume we have a transitive relation {\tt R} on {\tt nat}:
\label{eapply-example}

\begin{coq_eval}
Reset Initial.
\end{coq_eval}
\begin{coq_example*}
Variable R : nat -> nat -> Prop.
Hypothesis Rtrans : forall x y z:nat, R x y -> R y z -> R x z.
Variables n m p : nat.
Hypothesis Rnm : R n m.
Hypothesis Rmp : R m p.
\end{coq_example*}

Consider the goal {\tt (R n p)} provable using the transitivity of
{\tt R}:

\begin{coq_example*}
Goal R n p.
\end{coq_example*}

The direct application of {\tt Rtrans} with {\tt apply} fails because
no value for {\tt y} in {\tt Rtrans} is found by {\tt apply}:

%\begin{coq_eval}
%Set Printing Depth 50.
%(********** The following is not correct and should produce **********)
%(**** Error: generated subgoal (R n ?17) has metavariables in it *****)
%\end{coq_eval}
\begin{coq_example}
Fail apply Rtrans.
\end{coq_example}

A solution is to apply {\tt (Rtrans n m p)} or {\tt (Rtrans n m)}.

\begin{coq_example}
apply (Rtrans n m p).
\end{coq_example}

\begin{coq_eval}
Undo.
\end{coq_eval}

Note that {\tt n} can be inferred from the goal, so the following would
work too.

\begin{coq_example*}
apply (Rtrans _ m).
\end{coq_example*}
\begin{coq_eval}
Undo.
\end{coq_eval}

More elegantly, {\tt apply Rtrans with (y:=m)} allows only mentioning
the unknown {\tt m}:

\begin{coq_example*}
apply Rtrans with (y := m).
\end{coq_example*}
\begin{coq_eval}
Undo.
\end{coq_eval}

Another solution is to mention the proof of {\tt (R x y)} in {\tt
Rtrans} \ldots

\begin{coq_example}
apply Rtrans with (1 := Rnm).
\end{coq_example}
\begin{coq_eval}
Undo.
\end{coq_eval}

\ldots or the proof of {\tt (R y z)}.

\begin{coq_example}
apply Rtrans with (2 := Rmp).
\end{coq_example}
\begin{coq_eval}
Undo.
\end{coq_eval}

On the opposite, one can use {\tt eapply} which postpones the problem
of finding {\tt m}. Then one can apply the hypotheses {\tt Rnm} and {\tt
Rmp}. This instantiates the existential variable and completes the proof.

\begin{coq_example}
eapply Rtrans.
apply Rnm.
apply Rmp.
\end{coq_example}

\begin{coq_eval}
Reset R.
\end{coq_eval}

\noindent {\bf Remark: } When the conclusion of the type of the term
to apply is an inductive type isomorphic to a tuple type and {\em apply}
looks recursively whether a component of the tuple matches the goal,
it excludes components whose statement would result in applying an
universal lemma of the form {\tt forall A, ... -> A}. Excluding this
kind of lemma can be avoided by setting the following option:

\begin{quote}
\optindex{Universal Lemma Under Conjunction}
{\tt Set Universal Lemma Under Conjunction}
\end{quote}

This option, which preserves compatibility with versions of {\Coq}
prior to 8.4 is also available for {\tt apply {\term} in {\ident}}
(see Section~\ref{apply-in}).

\subsection{\tt apply {\term} in {\ident}}
\label{apply-in}
\tacindex{apply \dots\ in}

This tactic applies to any goal.  The argument {\term} is a term
well-formed in the local context and the argument {\ident} is an
hypothesis of the context.  The tactic {\tt apply {\term} in {\ident}}
tries to match the conclusion of the type of {\ident} against a
non-dependent premise of the type of {\term}, trying them from right to
left.  If it succeeds, the statement of hypothesis {\ident} is
replaced by the conclusion of the type of {\term}. The tactic also
returns as many subgoals as the number of other non-dependent premises
in the type of {\term} and of the non-dependent premises of the type
of {\ident}.  If the conclusion of the type of {\term} does not match
the goal {\em and} the conclusion is an inductive type isomorphic to a
tuple type, then the tuple is (recursively) decomposed and the first
component of the tuple of which a non-dependent premise matches the
conclusion of the type of {\ident}. Tuples are decomposed in a
width-first left-to-right order (for instance if the type of {\tt H1}
is a \verb=A <-> B= statement, and the type of {\tt H2} is \verb=A=
then {\tt apply H1 in H2} transforms the type of {\tt H2} into {\tt
  B}). The tactic {\tt apply} relies on first-order pattern-matching
with dependent types.

\begin{ErrMsgs}
\item \errindex{Statement without assumptions}

This happens if the type of {\term} has no non dependent premise.

\item \errindex{Unable to apply}

This happens if the conclusion of {\ident} does not match any of the
non dependent premises of the type of {\term}.
\end{ErrMsgs}

\begin{Variants}
\item {\tt apply \nelist{\term}{,} in {\ident}}

This applies each of {\term} in sequence in {\ident}.

\item {\tt apply \nelist{{\term} with {\bindinglist}}{,} in {\ident}}

This does the same but uses the bindings in each {\bindinglist} to
instantiate the parameters of the corresponding type of {\term}
(see syntax of bindings in Section~\ref{Binding-list}).

\item {\tt eapply \nelist{{\term} with {\bindinglist}}{,} in {\ident}}
\tacindex{eapply \dots\ in}

This works as {\tt apply \nelist{{\term} with {\bindinglist}}{,} in
{\ident}} but turns unresolved bindings into existential variables, if
any, instead of failing.

\item {\tt apply \nelist{{\term} with {\bindinglist}}{,} in {\ident} as {\intropattern}}

This works as {\tt apply \nelist{{\term} with {\bindinglist}}{,} in
{\ident}} then applies the {\intropattern} to the hypothesis {\ident}.

\item {\tt eapply \nelist{{\term} with {\bindinglist}}{,} in {\ident} as {\intropattern}}

This works as {\tt apply \nelist{{\term} with {\bindinglist}}{,} in {\ident} as {\intropattern}} but using {\tt eapply}.

\item {\tt simple apply {\term} in {\ident}}
\tacindex{simple apply \dots\ in}
\tacindex{simple eapply \dots\ in}

This behaves like {\tt apply {\term} in {\ident}} but it reasons
modulo conversion only on subterms that contain no variables to
instantiate. For instance, if {\tt id := fun x:nat => x} and {\tt H :
  forall y, id y = y -> True} and {\tt H0 :\ O = O} then {\tt simple
  apply H in H0} does not succeed because it would require the
conversion of {\tt id ?1234} and {\tt O} where {\tt ?1234} is a variable to
instantiate.  Tactic {\tt simple apply {\term} in {\ident}} does not
either traverse tuples as {\tt apply {\term} in {\ident}} does.

\item {\tt \zeroone{simple} apply \nelist{{\term} \zeroone{with {\bindinglist}}}{,} in {\ident} \zeroone{as {\intropattern}}}\\
{\tt \zeroone{simple} eapply \nelist{{\term} \zeroone{with {\bindinglist}}}{,} in {\ident} \zeroone{as {\intropattern}}}

This summarizes the different syntactic variants of {\tt apply {\term}
  in {\ident}} and {\tt eapply {\term} in {\ident}}.
\end{Variants}

\subsection{\tt constructor \num}
\label{constructor}
\tacindex{constructor}

This tactic applies to a goal such that its conclusion is
an inductive type (say {\tt I}).  The argument {\num} must be less
or equal to the numbers of constructor(s) of {\tt I}. Let {\tt ci} be
the {\tt i}-th constructor of {\tt I}, then {\tt constructor i} is
equivalent to {\tt intros; apply ci}.

\begin{ErrMsgs}
\item \errindex{Not an inductive product}
\item \errindex{Not enough constructors}
\end{ErrMsgs}

\begin{Variants}
\item \texttt{constructor}

  This tries \texttt{constructor 1} then \texttt{constructor 2},
  \dots\ , then \texttt{constructor} \textit{n} where \textit{n} is
  the number of constructors of the head of the goal.

\item {\tt constructor \num~with} {\bindinglist}

  Let {\tt ci} be the {\tt i}-th constructor of {\tt I}, then {\tt
    constructor i with \bindinglist} is equivalent to {\tt intros;
    apply ci with \bindinglist}.

  \Warning the terms in the \bindinglist\ are checked
  in the context where {\tt constructor} is executed and not in the
  context where {\tt apply} is executed (the introductions are not
  taken into account).

% To document?
% \item {\tt constructor {\tactic}}

\item {\tt split}\tacindex{split}

  This applies only if {\tt I} has a single constructor. It is then
  equivalent to {\tt constructor 1}. It is typically used in the case
  of a conjunction $A\land B$.

  \ErrMsg \errindex{Not an inductive goal with 1 constructor}

\item {\tt exists {\bindinglist}}\tacindex{exists}

  This applies only if {\tt I} has a single constructor. It is then
  equivalent to {\tt intros; constructor 1 with \bindinglist}. It is
  typically used in the case of an existential quantification $\exists
  x, P(x)$.

  \ErrMsg \errindex{Not an inductive goal with 1 constructor}

\item {\tt exists \nelist{\bindinglist}{,}}

  This iteratively applies {\tt exists {\bindinglist}}.

\item {\tt left}\tacindex{left}\\
      {\tt right}\tacindex{right}

  These tactics apply only if {\tt I} has two constructors, for instance
  in the case of a
  disjunction $A\lor B$. Then, they are respectively equivalent to {\tt
    constructor 1} and {\tt constructor 2}.

  \ErrMsg \errindex{Not an inductive goal with 2 constructors}

\item {\tt left with \bindinglist}\\
      {\tt right with \bindinglist}\\
      {\tt split with \bindinglist}

  As soon as the inductive type has the right number of constructors,
    these expressions are equivalent to calling {\tt
    constructor $i$ with \bindinglist} for the appropriate $i$.

\item \texttt{econstructor}\tacindex{econstructor}\\
      \texttt{eexists}\tacindex{eexists}\\
      \texttt{esplit}\tacindex{esplit}\\
      \texttt{eleft}\tacindex{eleft}\\
      \texttt{eright}\tacindex{eright}

  These tactics and their variants behave like \texttt{constructor},
  \texttt{exists}, \texttt{split}, \texttt{left}, \texttt{right} and
  their variants but they introduce existential variables instead of
  failing when the instantiation of a variable cannot be found (cf
  \texttt{eapply} and Section~\ref{eapply-example}).

\end{Variants}

\section{Managing the local context}

\subsection{\tt intro}
\tacindex{intro}
\label{intro}

This tactic applies to a goal that is either a product or starts with
a let binder. If the goal is a product, the tactic implements the
``Lam''\index{Typing rules!Lam} rule given in
Section~\ref{Typed-terms}\footnote{Actually, only the second subgoal will be
generated since the other one can be automatically checked.}.  If the
goal starts with a let binder, then the tactic implements a mix of the
``Let''\index{Typing rules!Let} and ``Conv''\index{Typing rules!Conv}.

If the current goal is a dependent product $\forall x:T,~U$ (resp {\tt
let $x$:=$t$ in $U$}) then {\tt intro} puts {\tt $x$:$T$} (resp {\tt $x$:=$t$})
 in the local context.
% Obsolete (quantified names already avoid hypotheses names):
% Otherwise, it puts
% {\tt x}{\it n}{\tt :T} where {\it n} is such that {\tt x}{\it n} is a
%fresh name.
The new subgoal is $U$.
% If the {\tt x} has been renamed {\tt x}{\it n} then it is replaced
% by {\tt x}{\it n} in {\tt U}.

If the goal is a non-dependent product $T \to U$, then it puts
in the local context either {\tt H}{\it n}{\tt :$T$} (if $T$ is of
type {\tt Set} or {\tt Prop}) or {\tt X}{\it n}{\tt :$T$} (if the type
of $T$ is {\tt Type}). The optional index {\it n} is such that {\tt
H}{\it n} or {\tt X}{\it n} is a fresh identifier.
In both cases, the new subgoal is $U$.

If the goal is neither a product nor starting with a let definition,
the tactic {\tt intro} applies the tactic {\tt hnf} until the tactic
{\tt intro} can be applied or the goal is not head-reducible.

\begin{ErrMsgs}
\item \errindex{No product even after head-reduction}
\item \errindexbis{{\ident} is already used}{is already used}
\end{ErrMsgs}

\begin{Variants}

\item {\tt intros}\tacindex{intros}

  This repeats {\tt intro} until it meets the head-constant. It never reduces
  head-constants and it never fails.

\item {\tt intro {\ident}}

  This applies {\tt intro} but forces {\ident} to be the name of the
  introduced hypothesis.

  \ErrMsg \errindex{name {\ident} is already used}

  \Rem If a name used by {\tt intro} hides the base name of a global
  constant then the latter can still be referred to by a qualified name
  (see \ref{LongNames}).

\item {\tt intros \ident$_1$ \dots\ \ident$_n$}

  This is equivalent to the composed tactic {\tt intro \ident$_1$; \dots\ ;
    intro \ident$_n$}.

  More generally, the \texttt{intros} tactic takes a pattern as
  argument in order to introduce names for components of an inductive
  definition or to clear introduced hypotheses. This is explained
  in~\ref{intros-pattern}.

\item {\tt intros until {\ident}} \tacindex{intros until}

  This repeats {\tt intro} until it meets a premise of the goal having form
  {\tt (} {\ident}~{\tt :}~{\term} {\tt )} and discharges the variable
  named {\ident} of the current goal.

  \ErrMsg \errindex{No such hypothesis in current goal}

\item {\tt intros until {\num}}  \tacindex{intros until}

  This repeats {\tt intro} until the {\num}-th non-dependent product. For
  instance, on the subgoal %
  \verb+forall x y:nat, x=y -> y=x+ the tactic \texttt{intros until 1}
  is equivalent to \texttt{intros x y H}, as \verb+x=y -> y=x+ is the
  first non-dependent product. And on the subgoal %
  \verb+forall x y z:nat, x=y -> y=x+ the tactic \texttt{intros until 1}
  is equivalent to \texttt{intros x y z} as the product on \texttt{z}
  can be rewritten as a non-dependent product: %
  \verb+forall x y:nat, nat -> x=y -> y=x+


  \ErrMsg \errindex{No such hypothesis in current goal}

  This happens when {\num} is 0 or is greater than the number of non-dependent
  products of the goal.

\item {\tt intro after \ident} \tacindex{intro after}\\
      {\tt intro before \ident} \tacindex{intro before}\\
      {\tt intro at top} \tacindex{intro at top}\\
      {\tt intro at bottom} \tacindex{intro at bottom}

  These tactics apply {\tt intro} and move the freshly introduced hypothesis
  respectively after the hypothesis \ident{}, before the hypothesis
  \ident{}, at the top of the local context, or at the bottom of the
  local context. All hypotheses on which the new hypothesis depends
  are moved too so as to respect the order of dependencies between
  hypotheses. Note that {\tt intro at bottom} is a synonym for {\tt
  intro} with no argument.

  \ErrMsg \errindex{No such hypothesis} : {\ident}

\item {\tt intro \ident$_1$ after \ident$_2$}\\
      {\tt intro \ident$_1$ before \ident$_2$}\\
      {\tt intro \ident$_1$ at top}\\
      {\tt intro \ident$_1$ at bottom}

  These tactics behave as previously but naming the introduced hypothesis
  \ident$_1$.  It is equivalent to {\tt intro \ident$_1$} followed by
  the appropriate call to {\tt move}~(see Section~\ref{move}).

\end{Variants}

\subsection{\tt intros {\intropattern} \mbox{\dots} \intropattern}
\label{intros-pattern}
\tacindex{intros \intropattern}
\index{Introduction patterns}
\index{Naming introduction patterns}
\index{Disjunctive/conjunctive introduction patterns}
\index{Disjunctive/conjunctive introduction patterns}
\index{Equality introduction patterns}


This extension of the tactic {\tt intros} combines introduction of
variables or hypotheses and case analysis. An {\em introduction pattern} is
either:
\begin{itemize}
\item a {\em naming introduction pattern}, i.e. either one of:
  \begin{itemize}
  \item the pattern \texttt{?}
  \item the pattern \texttt{?\ident}
  \item an identifier
  \end{itemize}
\item an {\em action introduction pattern} which itself classifies into:
  \begin{itemize}
  \item a {\em disjunctive/conjunctive introduction pattern}, i.e. either one of:
    \begin{itemize}
    \item a disjunction of lists of patterns:
      {\tt [$p_{11}$ \dots\ $p_{1m_1}$ | \dots\ | $p_{11}$ \dots\ $p_{nm_n}$]}
    \item a conjunction of patterns: {\tt ($p_1$ , \dots\ , $p_n$)}
    \item a list of patterns {\tt ($p_1$ \&\ \dots\ \&\ $p_n$)}
      for sequence of right-associative binary constructs
    \end{itemize}
  \item an {\em equality introduction pattern}, i.e. either one of:
    \begin{itemize}
    \item a pattern for decomposing an equality: {\tt [= $p_1$ \dots\ $p_n$]}
    \item the rewriting orientations: {\tt ->} or {\tt <-}
    \end{itemize}
  \item the on-the-fly application of lemmas: $p${\tt \%{\term$_1$}}
    \ldots {\tt \%{\term$_n$}} where $p$ itself is not a pattern for
    on-the-fly application of lemmas (note: syntax is in experimental stage)
  \end{itemize}
\item the wildcard: {\tt \_}
\end{itemize}

Introduction patterns can be combined into lists. An {\em introduction
  pattern list} is a list of introduction patterns possibly containing
the filling introduction patterns {\tt *} and {\tt **}.

Assuming a goal of type $Q \to P$ (non-dependent product), or
of type $\forall x:T,~P$ (dependent product), the behavior of
{\tt intros $p$} is defined inductively over the structure of the
introduction pattern~$p$:
\begin{itemize}
\item introduction on \texttt{?} performs the introduction, and lets {\Coq}
  choose a fresh name for the variable;
\item introduction on \texttt{?\ident} performs the introduction, and
  lets {\Coq} choose a fresh name for the variable based on {\ident};
\item introduction on \texttt{\ident} behaves as described in
  Section~\ref{intro};
\item introduction over a disjunction of list of patterns {\tt
  [$p_{11}$ \dots\ $p_{1m_1}$ | \dots\ | $p_{11}$ \dots\ $p_{nm_n}$]}
  expects the product to be over an inductive type
  whose number of constructors is $n$ (or more generally over a type
  of conclusion an inductive type built from $n$ constructors,
  e.g. {\tt C -> A\textbackslash/B if $n=2$}): it destructs the introduced
  hypothesis as {\tt destruct} (see Section~\ref{destruct}) would and
  applies on each generated subgoal the corresponding tactic;
  \texttt{intros}~$p_{i1}$ {\ldots} $p_{im_i}$; if the disjunctive
  pattern is part of a sequence of patterns and is not the last
  pattern of the sequence, then {\Coq} completes the pattern so that all
  the argument of the constructors of the inductive type are
  introduced (for instance, the list of patterns {\tt [$\;$|$\;$] H}
  applied on goal {\tt forall x:nat, x=0 -> 0=x} behaves the same as
  the list of patterns {\tt [$\,$|$\,$?$\,$] H});
\item introduction over a conjunction of patterns {\tt ($p_1$, \ldots,
  $p_n$)} expects the goal to be a product over an inductive type $I$ with a
  single constructor that itself has at least $n$ arguments: it
  performs a case analysis over the hypothesis, as {\tt destruct}
  would, and applies the patterns $p_1$~\ldots~$p_n$ to the arguments
  of the constructor of $I$ (observe that {\tt ($p_1$, {\ldots},
  $p_n$)} is an alternative notation for {\tt [$p_1$ {\ldots}
  $p_n$]});
\item introduction via {\tt ($p_1$ \& \dots\ \& $p_n$)}
  is a shortcut for introduction via
  {\tt ($p_1$,(\ldots,(\dots,$p_n$)\ldots))}; it expects the
  hypothesis to be a sequence of right-associative binary inductive
  constructors such as {\tt conj} or {\tt ex\_intro}; for instance, an
  hypothesis with type {\tt A\verb|/\|exists x, B\verb|/\|C\verb|/\|D} can be
  introduced via pattern {\tt (a \& x \& b \& c \& d)};
\item if the product is over an equality type, then a pattern of the
  form {\tt [=$p_{1}$ \dots\ $p_n$]} applies either {\tt injection}
  (see Section~\ref{injection}) or {\tt discriminate} (see
  Section~\ref{discriminate}) instead of {\tt destruct}; if {\tt
    injection} is applicable, the patterns $p_1$, \ldots, $p_n$ are
  used on the hypotheses generated by {\tt injection}; if the number
  of patterns is smaller than the number of hypotheses generated, the
  pattern \texttt{?} is used to complete the list;
  %TODO!
  %if {\tt discriminate} is applicable, the list of patterns $p_{1}$
  %\dots\ $p_n$ is supposed to be empty;
\item introduction over {\tt ->} (respectively {\tt <-}) expects the
  hypothesis to be an equality and the right-hand-side (respectively
  the left-hand-side) is replaced by the left-hand-side (respectively
  the right-hand-side) in the conclusion of the goal; the hypothesis
  itself is erased; if the term to substitute is a variable, it is
  substituted also in the context of goal and the variable is removed
  too;
\item introduction over a pattern $p${\tt \%{\term$_1$}} \ldots {\tt
  \%{\term$_n$}} first applies {\term$_1$},\ldots, {\term$_n$} on the
  hypothesis to be introduced (as in {\tt apply }{\term}$_1$, \ldots,
  {\term}$_n$ {\tt in}) prior to the application of the introduction
  pattern $p$;
\item introduction on the wildcard depends on whether the product is
  dependent or not: in the non-dependent case, it erases the
  corresponding hypothesis (i.e. it behaves as an {\tt intro} followed
  by a {\tt clear}, cf Section~\ref{clear}) while in the dependent
  case, it succeeds and erases the variable only if the wildcard is
  part of a more complex list of introduction patterns that also
  erases the hypotheses depending on this variable;
\item introduction over {\tt *} introduces all forthcoming quantified
  variables appearing in a row; introduction over {\tt **} introduces
  all forthcoming quantified variables or hypotheses until the goal is
  not any more a quantification or an implication.
\end{itemize}

Then, if $p_1$ ... $p_n$ is a list of introduction patterns possibly
containing {\tt *} or {\tt **}, {\tt intros $p_1$ ... $p_n$}
\begin{itemize}
\item introduction over {\tt *} introduces all forthcoming quantified
  variables appearing in a row;
\item introduction over {\tt **} introduces all forthcoming quantified
  variables or hypotheses until the goal is not any more a
  quantification or an implication;
\item introduction over an introduction pattern $p$ introduces the
  forthcoming quantified variables or premise of the goal and applies
  the introduction pattern $p$ to it.
\end{itemize}

\Example

\begin{coq_example}
Goal forall A B C:Prop, A \/ B /\ C -> (A -> C) -> C.
intros * [a | (_,c)] f.
\end{coq_example}
\begin{coq_eval}
Abort.
\end{coq_eval}

\Rem {\tt intros $p_1~\ldots~p_n$} is not fully equivalent to
\texttt{intros $p_1$;\ldots; intros $p_n$} for the following reasons:
\begin{itemize}
\item A wildcard pattern never succeeds when applied isolated on a
  dependent product, while it succeeds as part of a list of
  introduction patterns if the hypotheses that depends on it are
  erased too.
\item A disjunctive or conjunctive pattern followed by an introduction
  pattern forces the introduction in the context of all arguments of
  the constructors before applying the next pattern while a terminal
  disjunctive or conjunctive pattern does not. Here is an example

\begin{coq_example}
Goal forall n:nat, n = 0 -> n = 0.
intros [ | ] H.
Show 2.
Undo.
intros [ | ]; intros H.
Show 2.
\end{coq_example}

\end{itemize}

\subsection{\tt clear \ident}
\tacindex{clear}
\label{clear}

This tactic erases the hypothesis named {\ident} in the local context
of the current goal. As a consequence, {\ident} is no more displayed and no more
usable in the proof development.

\begin{ErrMsgs}
\item \errindex{No such hypothesis}
\item \errindexbis{{\ident} is used in the conclusion}{is used in the
    conclusion}
\item \errindexbis{{\ident} is used in the hypothesis {\ident'}}{is
    used in the hypothesis}
\end{ErrMsgs}

\begin{Variants}

\item {\tt clear {\ident$_1$} \dots\ {\ident$_n$}}

  This is equivalent to {\tt clear {\ident$_1$}. {\ldots} clear
    {\ident$_n$}.}

\item {\tt clearbody {\ident}}\tacindex{clearbody}

  This tactic expects {\ident} to be a local definition then clears
  its body. Otherwise said, this tactic turns a definition into an
  assumption.

  \ErrMsg \errindexbis{{\ident} is not a local definition}{is not a local definition}

\item \texttt{clear - {\ident$_1$} \dots\ {\ident$_n$}}

  This tactic clears all the hypotheses except the ones depending in
  the hypotheses named {\ident$_1$} {\ldots} {\ident$_n$} and in the
  goal.

\item \texttt{clear}

  This tactic clears all the hypotheses except the ones the goal depends on.

\item {\tt clear dependent \ident \tacindex{clear dependent}}

 This clears the hypothesis \ident\ and all the hypotheses
 that depend on it.

\end{Variants}

\subsection{\tt revert \ident$_1$ \dots\ \ident$_n$}
\tacindex{revert}
\label{revert}

This applies to any goal with variables \ident$_1$ \dots\ \ident$_n$.
It moves the hypotheses (possibly defined) to the goal, if this respects
dependencies. This tactic is the inverse of {\tt intro}.

\begin{ErrMsgs}
\item \errindex{No such hypothesis}
\item \errindexbis{{\ident} is used in the hypothesis {\ident'}}{is
    used in the hypothesis}
\end{ErrMsgs}

\begin{Variants}
\item {\tt revert dependent \ident \tacindex{revert dependent}}

 This moves to the goal the hypothesis {\ident} and all the hypotheses
 that depend on it.

\end{Variants}

\subsection{\tt move {\ident$_1$} after {\ident$_2$}}
\tacindex{move}
\label{move}

This moves the hypothesis named {\ident$_1$} in the local context
after the hypothesis named {\ident$_2$}. The proof term is not changed.

If {\ident$_1$} comes before {\ident$_2$} in the order of dependencies,
then all the hypotheses between {\ident$_1$} and {\ident$_2$} that
(possibly indirectly) depend on {\ident$_1$} are moved too.

If {\ident$_1$} comes after {\ident$_2$} in the order of dependencies,
then all the hypotheses between {\ident$_1$} and {\ident$_2$} that
(possibly indirectly) occur in {\ident$_1$} are moved too.

\begin{Variants}

\item {\tt move {\ident$_1$} before {\ident$_2$}}

This moves {\ident$_1$} towards and just before the hypothesis named {\ident$_2$}.

\item {\tt move {\ident} at top}

This moves {\ident} at the top of the local context (at the beginning of the context).

\item {\tt move {\ident} at bottom}

This moves {\ident} at the bottom of the local context (at the end of the context).

\end{Variants}

\begin{ErrMsgs}

\item \errindex{No such hypothesis}

\item \errindex{Cannot move {\ident$_1$} after {\ident$_2$}:
                   it occurs in {\ident$_2$}}

\item \errindex{Cannot move {\ident$_1$} after {\ident$_2$}:
                   it depends on {\ident$_2$}}

\end{ErrMsgs}

\subsection{\tt rename {\ident$_1$} into {\ident$_2$}}
\tacindex{rename}

This renames hypothesis {\ident$_1$} into {\ident$_2$} in the current
context. The name of the hypothesis in the proof-term, however, is left
unchanged.

\begin{Variants}

\item {\tt rename {\ident$_1$} into {\ident$_2$}, \ldots,
    {\ident$_{2k-1}$} into {\ident$_{2k}$}}

This renames the variables {\ident$_1$} \ldots {\ident$_2k-1$} into respectively
{\ident$_2$} \ldots {\ident$_2k$} in parallel. In particular, the target
identifiers may contain identifiers that exist in the source context, as long
as the latter are also renamed by the same tactic.

\end{Variants}

\begin{ErrMsgs}
\item \errindex{No such hypothesis}
\item \errindexbis{{\ident$_2$} is already used}{is already used}
\end{ErrMsgs}

\subsection{\tt set ( {\ident} := {\term} )}
\label{tactic:set}
\tacindex{set}

This replaces {\term} by {\ident} in the conclusion of the current goal
and adds the new definition {\tt {\ident} := \term} to the local context.

If {\term} has holes (i.e. subexpressions of the form ``\_''), the
tactic first checks that all subterms matching the pattern are
compatible before doing the replacement using the leftmost subterm
matching the pattern.

\begin{ErrMsgs}
\item \errindex{The variable {\ident} is already defined}
\end{ErrMsgs}

\begin{Variants}

\item {\tt set ( {\ident} := {\term} ) in {\occgoalset}}

This notation allows specifying which occurrences of {\term} have to
be substituted in the context. The {\tt in {\occgoalset}} clause is an
occurrence clause whose syntax and behavior are described in
Section~\ref{Occurrences_clauses}.

\item {\tt set ( {\ident} \nelistnosep{\binder} := {\term} )}

  This is equivalent to {\tt set ( {\ident} := fun
  \nelistnosep{\binder} => {\term} )}.

\item {\tt set \term}

  This behaves as {\tt set (} {\ident} := {\term} {\tt )} but {\ident}
  is generated by {\Coq}. This variant also supports an occurrence clause.

\item {\tt set ( {\ident$_0$} \nelistnosep{\binder} := {\term} ) in {\occgoalset}}\\
      {\tt set {\term} in {\occgoalset}}

  These are the general forms that combine the previous possibilities.

\item {\tt remember {\term} as {\ident}}\tacindex{remember}

  This behaves as {\tt set ( {\ident} := {\term} ) in *} and using a
  logical (Leibniz's) equality instead of a local definition.

\item {\tt remember {\term} as {\ident} eqn:{\ident}}

  This behaves as {\tt remember {\term} as {\ident}}, except
  that the name of the generated equality is also given.

\item {\tt remember {\term} as {\ident} in {\occgoalset}}

  This is a more general form of {\tt remember} that remembers the
  occurrences of {\term} specified by an occurrences set.

\item {\tt pose ( {\ident} := {\term} )}\tacindex{pose}

  This adds the local definition {\ident} := {\term} to the current
  context without performing any replacement in the goal or in the
  hypotheses. It is equivalent to {\tt set ( {\ident} {\tt :=}
  {\term} {\tt ) in |-}}.

\item {\tt pose ( {\ident} \nelistnosep{\binder} := {\term} )}

  This is equivalent to {\tt pose (} {\ident} {\tt :=} {\tt fun}
  \nelistnosep{\binder} {\tt =>} {\term} {\tt )}.

\item{\tt pose {\term}}

  This behaves as {\tt pose ( {\ident} := {\term} )} but
  {\ident} is generated by {\Coq}.

\end{Variants}

\subsection{\tt decompose [ {\qualid$_1$} \dots\ {\qualid$_n$} ] \term}
\label{decompose}
\tacindex{decompose}

This tactic recursively decomposes a
complex proposition in order to obtain atomic ones.

\Example

\begin{coq_eval}
Reset Initial.
\end{coq_eval}
\begin{coq_example}
Goal forall A B C:Prop, A /\ B /\ C \/ B /\ C \/ C /\ A -> C.
intros A B C H; decompose [and or] H; assumption.
\end{coq_example}
\begin{coq_example*}
Qed.
\end{coq_example*}

{\tt decompose} does not work on right-hand sides of implications or products.

\begin{Variants}

\item {\tt decompose sum \term}\tacindex{decompose sum}

  This decomposes sum types (like \texttt{or}).

\item {\tt decompose record \term}\tacindex{decompose record}

  This decomposes record types (inductive types with one constructor,
  like \texttt{and} and \texttt{exists} and those defined with the
  \texttt{Record} macro, see Section~\ref{Record}).

\end{Variants}

\section{Controlling the proof flow}

\subsection{\tt assert ( {\ident} :\ {\form} )}
\tacindex{assert}

This tactic applies to any goal. {\tt assert (H : U)} adds a new
hypothesis of name \texttt{H} asserting \texttt{U} to the current goal
and opens a new subgoal \texttt{U}\footnote{This corresponds to the
  cut rule of sequent calculus.}. The subgoal {\texttt U} comes first
in the list of subgoals remaining to prove.

\begin{ErrMsgs}
\item \errindex{Not a proposition or a type}

  Arises when the argument {\form} is neither of type {\tt Prop}, {\tt
    Set} nor {\tt Type}.

\end{ErrMsgs}

\begin{Variants}

\item{\tt assert {\form}}

  This behaves as {\tt assert ( {\ident} :\ {\form} )} but
  {\ident} is generated by {\Coq}.

\item \texttt{assert {\form} by {\tac}}\tacindex{assert by}

  This tactic behaves like \texttt{assert} but applies {\tac}
  to solve the subgoals generated by \texttt{assert}.

  \ErrMsg \errindex{Proof is not complete}

\item \texttt{assert {\form} as {\intropattern}\tacindex{assert as}}

  If {\intropattern} is a naming introduction pattern (see
  Section~\ref{intros-pattern}), the hypothesis is named after this
  introduction pattern (in particular, if {\intropattern} is {\ident},
  the tactic behaves like \texttt{assert ({\ident} :\ {\form})}).

  If {\intropattern} is a disjunctive/conjunctive 
  introduction pattern, the tactic behaves like \texttt{assert
    {\form}} followed by a {\tt destruct} using this introduction pattern.

  If {\intropattern} is a rewriting intropattern pattern, the tactic
  behaves like \texttt{assert {\form}} followed by a call to {\tt
    subst} on the resulting hypothesis, if applicable, or to {\tt
    rewrite} otherwise.

  If {\intropattern} is an injection intropattern pattern, the tactic
  behaves like \texttt{assert {\form}} followed by {\tt injection}
  using this introduction pattern.

\item \texttt{assert {\form} as {\intropattern} by {\tac}}

  This combines the two previous variants of {\tt assert}.

\item{\tt assert ( {\ident} := {\term} )}

  This behaves as {\tt assert ({\ident} :\ {\type});[exact
      {\term}|idtac]} where {\type} is the type of {\term}. This is
  deprecated in favor of {\tt pose proof}.

  \ErrMsg \errindex{Variable {\ident} is already declared}

\item \texttt{pose proof {\term} as {\intropattern}\tacindex{pose proof}}

  This tactic behaves like \texttt{assert T as {\intropattern} by
  exact {\term}} where \texttt{T} is the type of {\term}.

  In particular, \texttt{pose proof {\term} as {\ident}} behaves as
  \texttt{assert ({\ident} := {\term})} and \texttt{pose proof {\term}
    as {\intropattern}\tacindex{pose proof}} is the same as applying
  the {\intropattern} to {\term}.

\item \texttt{enough ({\ident} :\ {\form})}\tacindex{enough}

  This adds a new hypothesis of name {\ident} asserting {\form} to the
  goal the tactic \texttt{enough} is applied to. A new subgoal stating
  \texttt{\form} is inserted after the initial goal rather than before
  it as \texttt{assert} would do.

\item \texttt{enough {\form}}\tacindex{enough}

  This behaves like \texttt{enough ({\ident} :\ {\form})} with the name
  {\ident} of the hypothesis generated by {\Coq}.

\item \texttt{enough {\form} as {\intropattern}\tacindex{enough as}}

  This behaves like \texttt{enough} {\form} using {\intropattern} to
  name or destruct the new hypothesis.

\item \texttt{enough ({\ident} :\ {\form}) by {\tac}}\tacindex{enough by}\\
 \texttt{enough {\form} by {\tac}}\tacindex{enough by}\\
 \texttt{enough {\form} as {\intropattern} by {\tac}}

  This behaves as above but with {\tac} expected to solve the initial
  goal after the extra assumption {\form} is added and possibly
  destructed. If the \texttt{as} {\intropattern} clause generates more
  than one subgoal, {\tac} is applied to all of them.

\item {\tt cut {\form}}\tacindex{cut}

  This tactic applies to any goal. It implements the non-dependent
  case of the ``App''\index{Typing rules!App} rule given in
  Section~\ref{Typed-terms}. (This is Modus Ponens inference rule.)
  {\tt cut U} transforms the current goal \texttt{T} into the two
  following subgoals: {\tt U -> T} and \texttt{U}.  The subgoal {\tt U
    -> T} comes first in the list of remaining subgoal to prove.

\item {\tt specialize ({\ident} \term$_1$ \dots\ \term$_n$)\tacindex{specialize}} \\
      {\tt specialize {\ident} with \bindinglist}

      The tactic {\tt specialize} works on local hypothesis \ident.
      The premises of this hypothesis (either universal
      quantifications or non-dependent implications) are instantiated
      by concrete terms coming either from arguments \term$_1$
      $\ldots$ \term$_n$ or from a bindings list (see
      Section~\ref{Binding-list} for more about bindings lists). In the
      second form, all instantiation elements must be given, whereas
      in the first form the application to \term$_1$ {\ldots}
      \term$_n$ can be partial. The first form is equivalent to
      {\tt assert (\ident' := {\ident} {\term$_1$} \dots\ \term$_n$);
           clear \ident; rename \ident' into \ident}.

      The name {\ident} can also refer to a global lemma or
      hypothesis. In this case, for compatibility reasons, the
      behavior of {\tt specialize} is close to that of {\tt
        generalize}: the instantiated statement becomes an additional
      premise of the goal.

  \begin{ErrMsgs}
  \item \errindexbis{{\ident} is used in hypothesis \ident'}{is used in hypothesis}
  \item \errindexbis{{\ident} is used in conclusion}{is used in conclusion}
  \end{ErrMsgs}

%% Moreover, the old syntax allows the use of a number after {\tt specialize}
%% for controlling the number of premises to instantiate. Giving this
%% number should not be mandatory anymore (automatic detection of how
%% many premises can be eaten without leaving meta-variables). Hence
%% no documentation for this integer optional argument of specialize

\end{Variants}

\subsection{\tt generalize \term}
\tacindex{generalize}
\label{generalize}

This tactic applies to any goal. It generalizes the conclusion with
respect to some term.

\Example

\begin{coq_eval}
Goal forall x y:nat, (0 <= x + y + y).
intros.
\end{coq_eval}
\begin{coq_example}
Show.
generalize (x + y + y).
\end{coq_example}

\begin{coq_eval}
Abort.
\end{coq_eval}

If the goal is $G$ and $t$ is a subterm of type $T$ in the goal, then
{\tt generalize} \textit{t} replaces the goal by {\tt forall (x:$T$), $G'$}
where $G'$ is obtained from $G$ by replacing all occurrences of $t$ by
{\tt x}. The name of the variable (here {\tt n}) is chosen based on $T$.

\begin{Variants}
\item {\tt generalize {\term$_1$ , \dots\ , \term$_n$}}

  This is equivalent to {\tt generalize \term$_n$; \dots\ ; generalize
    \term$_1$}. Note that the sequence of \term$_i$'s are processed
  from $n$ to $1$.

\item {\tt generalize {\term} at {\num$_1$ \dots\ \num$_i$}}

  This is equivalent to {\tt generalize \term} but it generalizes only over
  the specified occurrences of {\term} (counting from left to right on the
  expression printed using option {\tt Set Printing All}).

\item {\tt generalize {\term} as {\ident}}

  This is equivalent to {\tt generalize \term} but it uses {\ident} to name the
  generalized hypothesis.

\item {\tt generalize {\term$_1$} at {\num$_{11}$ \dots\ \num$_{1i_1}$}
                      as {\ident$_1$}
                      , {\ldots} ,
                      {\term$_n$} at {\num$_{n1}$ \mbox{\dots} \num$_{ni_n}$}
                      as {\ident$_2$}}

  This is the most general form of {\tt generalize} that combines the
  previous behaviors.

\item {\tt generalize dependent \term} \tacindex{generalize dependent}

  This generalizes {\term} but also {\em all} hypotheses that depend
  on {\term}. It clears the generalized hypotheses.

\end{Variants}

\subsection{\tt evar ( {\ident} :\ {\term} )}
\tacindex{evar}
\label{evar}

The {\tt evar} tactic creates a new local definition named \ident\ with
type \term\ in the context. The body of this binding is a fresh
existential variable.

\subsection{\tt instantiate ( {\num} := {\term} )}
\tacindex{instantiate}
\label{instantiate}

The {\tt instantiate} tactic refines (see Section~\ref{refine})
an existential variable
with the term \term. The \num\  argument is the position of the
existential variable from right to left in the conclusion. This cannot be
the number of the existential variable since this number is different
in every session.

When you are referring to hypotheses which you did not name
explicitly, be aware that Coq may make a different decision on how to
name the variable in the current goal and in the context of the
existential variable. This can lead to surprising behaviors.

\begin{Variants}
  \item {\tt instantiate ( {\num} := {\term} ) in \ident}

  \item {\tt instantiate ( {\num} := {\term} ) in ( Value of {\ident} )}

  \item {\tt instantiate ( {\num} := {\term} ) in ( Type of {\ident} )}

These allow to refer respectively to existential variables occurring in
a hypothesis or in the body or the type of a local definition.

  \item {\tt instantiate}

    Without argument, the {\tt instantiate} tactic tries to solve as
    many existential variables as possible, using information gathered
    from other tactics in the same tactical. This is automatically
    done after each complete tactic (i.e. after a dot in proof mode),
    but not, for example, between each tactic when they are sequenced
    by semicolons.

\end{Variants}

\subsection{\tt absurd \term}
\tacindex{absurd}
\label{absurd}

This tactic applies to any goal. The argument {\term} is any
proposition {\tt P} of type {\tt Prop}. This tactic applies {\tt
  False} elimination, that is it deduces the current goal from {\tt
  False}, and generates as subgoals {\tt $\sim$P} and {\tt P}. It is
very useful in proofs by cases, where some cases are impossible. In
most cases, \texttt{P} or $\sim$\texttt{P} is one of the hypotheses of
the local context.

\subsection{\tt contradiction}
\label{contradiction}
\tacindex{contradiction}

This tactic applies to any goal. The {\tt contradiction} tactic
attempts to find in the current context (after all {\tt intros}) one
hypothesis that is equivalent to {\tt False}. It permits to prune
irrelevant cases. This tactic is a macro for the tactics sequence
{\tt intros; elimtype False; assumption}.

\begin{ErrMsgs}
\item \errindex{No such assumption}
\end{ErrMsgs}

\begin{Variants}
\item {\tt contradiction \ident}

The proof of {\tt False} is searched in the hypothesis named \ident.
\end{Variants}

\subsection{\tt contradict \ident}
\label{contradict}
\tacindex{contradict}

This tactic allows manipulating negated hypothesis and goals. The
name \ident\ should correspond to a hypothesis. With
{\tt contradict H}, the current goal and context is transformed in
the following way:
\begin{itemize}
\item  {\tt H:$\neg$A $\vd$  B} \ becomes \ {\tt $\vd$ A}
\item  {\tt H:$\neg$A $\vd$ $\neg$B} \  becomes \ {\tt H: B $\vd$  A }
\item  {\tt H: A $\vd$  B} \ becomes \ {\tt $\vd$ $\neg$A}
\item  {\tt H: A $\vd$ $\neg$B} \ becomes \ {\tt H: B $\vd$ $\neg$A}
\end{itemize}

\subsection{\tt exfalso}
\label{exfalso}
\tacindex{exfalso}

This tactic implements the ``ex falso quodlibet'' logical principle:
an elimination of {\tt False} is performed on the current goal, and the
user is then required to prove that {\tt False} is indeed provable in
the current context. This tactic is a macro for {\tt elimtype False}.

\section{Case analysis and induction}

The tactics presented in this section implement induction or case
analysis on inductive or co-inductive objects (see
Section~\ref{Cic-inductive-definitions}).

\subsection{\tt destruct \term}
\tacindex{destruct}
\label{destruct}

This tactic applies to any goal. The argument {\term} must be of
inductive or co-inductive type and the tactic generates subgoals, one
for each possible form of {\term}, i.e. one for each constructor of
the inductive or co-inductive type. Unlike {\tt induction}, no
induction hypothesis is generated by {\tt destruct}.

There are special cases:

\begin{itemize}

\item If {\term} is an identifier {\ident} denoting a quantified
  variable of the conclusion of the goal, then {\tt destruct {\ident}}
  behaves as {\tt intros until {\ident}; destruct {\ident}}. If
  {\ident} is not anymore dependent in the goal after application of
  {\tt destruct}, it is erased (to avoid erasure, use
  parentheses, as in {\tt destruct ({\ident})}).

\item If {\term} is a {\num}, then {\tt destruct {\num}} behaves as
{\tt intros until {\num}} followed by {\tt destruct} applied to the
last introduced hypothesis. Remark: For destruction of a numeral, use
syntax {\tt destruct ({\num})} (not very interesting anyway).

\item In case {\term} is an hypothesis {\ident} of the context,
  and {\ident} is not anymore dependent in the goal after
  application of {\tt destruct}, it is erased (to avoid erasure, use
  parentheses, as in {\tt destruct ({\ident})}).

\item The argument {\term} can also be a pattern of which holes are
  denoted by ``\_''. In this case, the tactic checks that all subterms
  matching the pattern in the conclusion and the hypotheses are
  compatible and performs case analysis using this subterm.

\end{itemize}

\begin{Variants}
\item{\tt destruct \term$_1$, \ldots, \term$_n$}

  This is a shortcut for {\tt destruct \term$_1$; \ldots; destruct \term$_n$}.

\item{\tt destruct {\term} as {\disjconjintropattern}}

  This behaves as {\tt destruct {\term}} but uses the names in
  {\intropattern} to name the variables introduced in the context.
  The {\intropattern} must have the form {\tt [} $p_{11}$ \ldots
    $p_{1n_1}$ {\tt |} {\ldots} {\tt |} $p_{m1}$ \ldots $p_{mn_m}$
    {\tt ]} with $m$ being the number of constructors of the type of
  {\term}. Each variable introduced by {\tt destruct} in the context
  of the $i^{th}$ goal gets its name from the list $p_{i1}$ \ldots
  $p_{in_i}$ in order. If there are not enough names, {\tt destruct}
  invents names for the remaining variables to introduce. More
  generally, the $p_{ij}$ can be any introduction pattern (see
  Section~\ref{intros-pattern}). This provides a concise notation for
  chaining destruction of an hypothesis.

%  It is recommended to use this variant of {\tt destruct} for
%  robust proof scripts.

\item{\tt destruct {\term} eqn:{\namingintropattern}}

  This behaves as {\tt destruct {\term}} but adds an equation between
  {\term} and the value that {\term} takes in each of the possible
  cases.  The name of the equation is specified by {\namingintropattern}
  (see Section~\ref{intros-pattern}), in particular {\tt ?} can be
  used to let Coq generate a fresh name.

\item{\tt destruct {\term} with \bindinglist}

  This behaves like \texttt{destruct {\term}} providing explicit
  instances for the dependent premises of the type of {\term} (see
  syntax of bindings in Section~\ref{Binding-list}).

\item{\tt edestruct {\term}\tacindex{edestruct}}

  This tactic behaves like \texttt{destruct {\term}} except that it
  does not fail if the instance of a dependent premises of the type of
  {\term} is not inferable. Instead, the unresolved instances are left
  as existential variables to be inferred later, in the same way as
  {\tt eapply} does (see Section~\ref{eapply-example}).

\item{\tt destruct {\term$_1$} using {\term$_2$}}\\
     {\tt destruct {\term$_1$} using {\term$_2$} with {\bindinglist}}

  These are synonyms of {\tt induction {\term$_1$} using {\term$_2$}} and
  {\tt induction {\term$_1$} using {\term$_2$} with {\bindinglist}}.

\item \texttt{destruct {\term} in {\occgoalset}}

  This syntax is used for selecting which occurrences of {\term} the
  case analysis has to be done on. The {\tt in {\occgoalset}} clause is an
  occurrence clause whose syntax and behavior is described in
  Section~\ref{Occurrences_clauses}.

\item{\tt destruct {\term$_1$} with {\bindinglist$_1$}
        as {\disjconjintropattern} eqn:{\namingintropattern}
        using {\term$_2$} with {\bindinglist$_2$} in {\occgoalset}}\\
     {\tt edestruct {\term$_1$} with {\bindinglist$_1$}
        as {\disjconjintropattern} eqn:{\namingintropattern}
        using {\term$_2$} with {\bindinglist$_2$} in {\occgoalset}}

  These are the general forms of {\tt destruct} and {\tt edestruct}.
  They combine the effects of the {\tt with}, {\tt as}, {\tt eqn:}, {\tt using},
  and {\tt in} clauses.

\item{\tt case \term}\label{case}\tacindex{case}

  The tactic {\tt case} is a more basic tactic to perform case
  analysis without recursion. It behaves as {\tt elim \term} but using
  a case-analysis elimination principle and not a recursive one.

\item {\tt case {\term} with {\bindinglist}}

  Analogous to {\tt elim {\term} with {\bindinglist}} above.

\item{\tt ecase {\term}\tacindex{ecase}}\\
  {\tt ecase {\term} with {\bindinglist}}

  In case the type of {\term} has dependent premises, or dependent
  premises whose values are not inferable from the {\tt with
  {\bindinglist}} clause, {\tt ecase} turns them into existential
  variables to be resolved later on.

\item {\tt simple destruct \ident}\tacindex{simple destruct}

  This tactic behaves as {\tt intros until
    {\ident}; case {\tt {\ident}}} when {\ident} is a quantified
  variable of the goal.

\item {\tt simple destruct {\num}}

  This tactic behaves as {\tt intros until
    {\num}; case {\tt {\ident}}} where {\ident} is the name given by
  {\tt intros until {\num}} to the {\num}-th non-dependent premise of
  the goal.

\item{\tt case\_eq \term}\label{case_eq}\tacindex{case\_eq}

 The tactic {\tt case\_eq} is a variant of the {\tt case} tactic that
 allow to perform case analysis on a term without completely
 forgetting its original form. This is done by generating equalities
 between the original form of the term and the outcomes of the case
 analysis.

% The effect of this tactic is similar to the effect of {\tt
% destruct {\term} in |- *} with the exception that no new hypotheses
% are introduced in the context.

\end{Variants}

\subsection{\tt induction \term}
\tacindex{induction}
\label{Tac-induction}

This tactic applies to any goal. The argument {\term} must be of
inductive type and the tactic {\tt induction} generates subgoals,
one for each possible form of {\term}, i.e. one for each constructor
of the inductive type.

If the argument is dependent in either the conclusion or some
hypotheses of the goal, the argument is replaced by the appropriate
constructor form in each of the resulting subgoals and induction
hypotheses are added to the local context using names whose prefix is
{\tt IH}.

There are particular cases:

\begin{itemize}

\item If {\term} is an identifier {\ident} denoting a quantified
  variable of the conclusion of the goal, then {\tt induction
    {\ident}} behaves as {\tt intros until {\ident}; induction
    {\ident}}.  If {\ident} is not anymore dependent in the goal
  after application of {\tt induction}, it is erased (to avoid
  erasure, use parentheses, as in {\tt induction ({\ident})}).

\item If {\term} is a {\num}, then {\tt induction {\num}} behaves as
{\tt intros until {\num}} followed by {\tt induction} applied to the
last introduced hypothesis. Remark: For simple induction on a numeral,
use syntax {\tt induction ({\num})} (not very interesting anyway).

\item In case {\term} is an hypothesis {\ident} of the context,
  and {\ident} is not anymore dependent in the goal after
  application of {\tt induction}, it is erased (to avoid erasure, use
  parentheses, as in {\tt induction ({\ident})}).

\item The argument {\term} can also be a pattern of which holes are
  denoted by ``\_''. In this case, the tactic checks that all subterms
  matching the pattern in the conclusion and the hypotheses are
  compatible and performs induction using this subterm.

\end{itemize}

\Example

\begin{coq_example}
Lemma induction_test : forall n:nat, n = n -> n <= n.
intros n H.
induction n.
\end{coq_example}

\begin{ErrMsgs}
\item \errindex{Not an inductive product}
\item \errindex{Unable to find an instance for the variables
{\ident} \ldots {\ident}}

  Use in this case
  the variant {\tt elim \dots\ with \dots} below.
\end{ErrMsgs}

\begin{Variants}
\item{\tt induction {\term} as {\disjconjintropattern}}

  This behaves as {\tt induction {\term}} but uses the names in
  {\disjconjintropattern} to name the variables introduced in the context.
  The {\disjconjintropattern} must typically be of the form
  {\tt [} $p_{11}$ {\ldots}
  $p_{1n_1}$ {\tt |} {\ldots} {\tt |} $p_{m1}$ {\ldots} $p_{mn_m}$ {\tt
    ]} with $m$ being the number of constructors of the type of
  {\term}. Each variable introduced by {\tt induction} in the context
  of the $i^{th}$ goal gets its name from the list $p_{i1}$ {\ldots}
  $p_{in_i}$ in order. If there are not enough names, {\tt induction}
  invents names for the remaining variables to introduce. More
  generally, the $p_{ij}$ can be any disjunctive/conjunctive
  introduction pattern (see Section~\ref{intros-pattern}). For instance,
  for an inductive type with one constructor, the pattern notation
  {\tt (}$p_{1}$ {\tt ,} {\ldots} {\tt ,} $p_{n}${\tt )} can be used instead of
  {\tt [} $p_{1}$ {\ldots} $p_{n}$ {\tt ]}.

%% \item{\tt induction {\term} eqn:{\namingintropattern}}

%%   This behaves as {\tt induction {\term}} but adds an equation between
%%   {\term} and the value that {\term} takes in each of the induction
%%   case.  The name of the equation is built according to
%%   {\namingintropattern} which can be an identifier, a ``?'', etc, as
%%   indicated in Section~\ref{intros-pattern}.

%% \item{\tt induction {\term} as {\disjconjintropattern} eqn:{\namingintropattern}}

%%   This combines the two previous forms.

\item{\tt induction {\term} with \bindinglist}

  This behaves like \texttt{induction {\term}} providing explicit
  instances for the premises of the type of {\term} (see the syntax of
  bindings in Section~\ref{Binding-list}).

\item{\tt einduction {\term}\tacindex{einduction}}

  This tactic behaves like \texttt{induction {\term}} excepts that it
  does not fail if some dependent premise of the type of {\term} is
  not inferable. Instead, the unresolved premises are posed as
  existential variables to be inferred later, in the same way as {\tt
  eapply} does (see Section~\ref{eapply-example}).

\item {\tt induction {\term$_1$} using {\term$_2$}}

  This behaves as {\tt induction {\term$_1$}} but using {\term$_2$} as
  induction scheme. It does not expect the conclusion of the type of
  {\term$_1$} to be inductive.

\item {\tt induction {\term$_1$} using {\term$_2$} with {\bindinglist}}

  This behaves as {\tt induction {\term$_1$} using {\term$_2$}} but
  also providing instances for the premises of the type of {\term$_2$}.

\item \texttt{induction {\term}$_1$, {\ldots}, {\term}$_n$ using {\qualid}}

  This syntax is used for the case {\qualid} denotes an induction principle
  with complex predicates as the induction principles generated by
  {\tt Function} or {\tt Functional Scheme} may be.

\item \texttt{induction {\term} in {\occgoalset}}

  This syntax is used for selecting which occurrences of {\term} the
  induction has to be carried on. The {\tt in \occgoalset} clause is
  an occurrence clause whose syntax and behavior is described in
  Section~\ref{Occurrences_clauses}. If variables or hypotheses not
  mentioning {\term} in their type are listed in {\occgoalset}, those
  are generalized as well in the statement to prove.

\Example

\begin{coq_example}
Lemma comm x y : x + y = y + x.
induction y in x |- *.
Show 2.
\end{coq_example}

\item {\tt induction {\term$_1$} with {\bindinglist$_1$}
       as {\disjconjintropattern} %% eqn:{\namingintropattern}
       using {\term$_2$} with {\bindinglist$_2$} in {\occgoalset}}\\
     {\tt einduction {\term$_1$} with {\bindinglist$_1$}
       as {\disjconjintropattern} %% eqn:{\namingintropattern}
       using {\term$_2$} with {\bindinglist$_2$} in {\occgoalset}}

  These are the most general forms of {\tt induction} and {\tt
  einduction}.  It combines the effects of the {\tt with}, {\tt as}, %%{\tt eqn:},
  {\tt using}, and {\tt in} clauses.

\item {\tt elim \term}\label{elim}

  This is a more basic induction tactic.  Again, the type of the
  argument {\term} must be an inductive type. Then, according to
  the type of the goal, the tactic {\tt elim} chooses the appropriate
  destructor and applies it as the tactic {\tt apply}
  would do. For instance, if the proof context contains {\tt
  n:nat} and the current goal is {\tt T} of type {\tt
  Prop}, then {\tt elim n} is equivalent to {\tt apply nat\_ind with
  (n:=n)}.  The tactic {\tt elim} does not modify the context of
  the goal, neither introduces the induction loading into the context
  of hypotheses.

  More generally, {\tt elim \term} also works when the type of {\term}
  is a statement with premises and whose conclusion is inductive.  In
  that case the tactic performs induction on the conclusion of the
  type of {\term} and leaves the non-dependent premises of the type as
  subgoals.  In the case of dependent products, the tactic tries to
  find an instance for which the elimination lemma applies and fails
  otherwise.

\item {\tt elim {\term} with {\bindinglist}}

  Allows to give explicit instances to the premises of the type
  of {\term} (see Section~\ref{Binding-list}).

\item{\tt eelim {\term}\tacindex{eelim}}

  In case the type of {\term} has dependent premises, this turns them into
  existential variables to be resolved later on.

\item{\tt elim {\term$_1$} using {\term$_2$}}\\
     {\tt elim {\term$_1$} using {\term$_2$} with {\bindinglist}\tacindex{elim \dots\ using}}

Allows the user to give explicitly an elimination predicate
{\term$_2$} that is not the standard one for the underlying inductive
type of {\term$_1$}. The {\bindinglist} clause allows
instantiating premises of the type of {\term$_2$}.

\item{\tt elim {\term$_1$} with {\bindinglist$_1$} using {\term$_2$} with {\bindinglist$_2$}}\\
     {\tt eelim {\term$_1$} with {\bindinglist$_1$} using {\term$_2$} with {\bindinglist$_2$}}

  These are the most general forms of {\tt elim} and {\tt eelim}.  It
  combines the effects of the {\tt using} clause and of the two uses
  of the {\tt with} clause.

\item {\tt elimtype \form}\tacindex{elimtype}

  The argument {\form} must be inductively defined. {\tt elimtype I}
  is equivalent to {\tt cut I. intro H{\rm\sl n}; elim H{\rm\sl n};
    clear H{\rm\sl n}}. Therefore the hypothesis {\tt H{\rm\sl n}} will
  not appear in the context(s) of the subgoal(s).  Conversely, if {\tt
    t} is a term of (inductive) type {\tt I} that does not occur
  in the goal, then {\tt elim t} is equivalent to {\tt elimtype I; 2:
    exact t.}

\item {\tt simple induction \ident}\tacindex{simple induction}

  This tactic behaves as {\tt intros until
    {\ident}; elim {\tt {\ident}}} when {\ident} is a quantified
  variable of the goal.

\item {\tt simple induction {\num}}

  This tactic behaves as {\tt intros until
    {\num}; elim {\tt {\ident}}} where {\ident} is the name given by
  {\tt intros until {\num}} to the {\num}-th non-dependent premise of
  the goal.

%% \item {\tt simple induction {\term}}\tacindex{simple induction}

%%   If {\term} is an {\ident} corresponding to a quantified variable of
%%   the goal then the tactic behaves as {\tt intros until {\ident}; elim
%%   {\tt {\ident}}}.  If {\term} is a {\num} then the tactic behaves as
%%   {\tt intros until {\ident}; elim {\tt {\ident}}}.  Otherwise, it is
%%   a synonym for {\tt elim {\term}}.

%%   \Rem For simple induction on a numeral, use syntax {\tt simple
%%   induction ({\num})}.

\end{Variants}

%\subsection[\tt FixPoint \dots]{\tt FixPoint \dots\tacindex{Fixpoint}}
%Not yet documented.

\subsection{\tt double induction \ident$_1$ \ident$_2$}
\tacindex{double induction}

This tactic is deprecated and should be replaced by {\tt induction \ident$_1$; induction \ident$_2$} (or {\tt induction \ident$_1$; destruct \ident$_2$} depending on the exact needs).

%% This tactic applies to any goal. If the variables {\ident$_1$} and
%% {\ident$_2$} of the goal have an inductive type, then this tactic
%% performs double induction on these variables.  For instance, if the
%% current goal is \verb+forall n m:nat, P n m+ then, {\tt double induction n
%%   m} yields the four cases with their respective inductive hypotheses.

%% In particular, for proving \verb+(P (S n) (S m))+, the generated induction
%% hypotheses are \verb+(P (S n) m)+ and \verb+(m:nat)(P n m)+ (of the latter,
%% \verb+(P n m)+ and \verb+(P n (S m))+ are derivable).

%% \Rem When the induction hypothesis \verb+(P (S n) m)+ is not
%% needed, {\tt induction \ident$_1$; destruct \ident$_2$} produces
%% more concise subgoals.

\begin{Variant}

\item {\tt double induction \num$_1$ \num$_2$}

This tactic is deprecated and should be replaced by {\tt induction
  \num$_1$; induction \num$_3$} where \num$_3$ is the result of
\num$_2$-\num$_1$.

%% This applies double induction on the \num$_1^{th}$ and \num$_2^{th}$ {\it
%% non dependent} premises of the goal. More generally, any combination of an
%% {\ident} and a {\num} is valid.

\end{Variant}

\subsection{\tt dependent induction \ident}
\tacindex{dependent induction}
\label{DepInduction}

The \emph{experimental} tactic \texttt{dependent induction} performs
induction-inversion on an instantiated inductive predicate.
One needs to first require the {\tt Coq.Program.Equality} module to use
this tactic. The tactic is based on the BasicElim tactic by Conor
McBride \cite{DBLP:conf/types/McBride00} and the work of Cristina Cornes
around inversion \cite{DBLP:conf/types/CornesT95}. From an instantiated
inductive predicate and a goal, it generates an equivalent goal where the
hypothesis has been generalized over its indexes which are then
constrained by equalities to be the right instances. This permits to
state lemmas without resorting to manually adding these equalities and
still get enough information in the proofs.

\Example

\begin{coq_eval}
Reset Initial.
\end{coq_eval}
\begin{coq_example}
Lemma le_minus : forall n:nat, n < 1 -> n = 0.
intros n H ; induction H.
\end{coq_example}

Here we did not get any information on the indexes to help fulfill this
proof. The problem is that, when we use the \texttt{induction} tactic,
we lose information on the hypothesis instance, notably that the second
argument is \texttt{1} here. Dependent induction solves this problem by
adding the corresponding equality to the context.

\begin{coq_eval}
Reset Initial.
\end{coq_eval}
\begin{coq_example}
Require Import Coq.Program.Equality.
Lemma le_minus : forall n:nat, n < 1 -> n = 0.
intros n H ; dependent induction H.
\end{coq_example}

The subgoal is cleaned up as the tactic tries to automatically
simplify the subgoals with respect to the generated equalities.
In this enriched context, it becomes possible to solve this subgoal.
\begin{coq_example}
reflexivity.
\end{coq_example}

Now we are in a contradictory context and the proof can be solved.
\begin{coq_example}
inversion H.
\end{coq_example}

This technique works with any inductive predicate.
In fact, the \texttt{dependent induction} tactic is just a wrapper around
the \texttt{induction} tactic. One can make its own variant by just
writing a new tactic based on the definition found in
\texttt{Coq.Program.Equality}.

\begin{Variants}
\item {\tt dependent induction {\ident} generalizing {\ident$_1$} \dots
    {\ident$_n$}}\tacindex{dependent induction \dots\ generalizing}

  This performs dependent induction on the hypothesis {\ident} but first
  generalizes the goal by the given variables so that they are
  universally quantified in the goal. This is generally what one wants
  to do with the variables that are inside some constructors in the
  induction hypothesis. The other ones need not be further generalized.

\item {\tt dependent destruction {\ident}}\tacindex{dependent destruction}

  This performs the generalization of the instance {\ident} but uses {\tt destruct}
  instead of {\tt induction} on the generalized hypothesis. This gives
  results equivalent to {\tt inversion} or {\tt dependent inversion} if
  the hypothesis is dependent.
\end{Variants}

\SeeAlso \ref{dependent-induction-example} for a larger example of
dependent induction and an explanation of the underlying technique.

\subsection{\tt functional induction (\qualid\ \term$_1$ \dots\ \term$_n$)}
\tacindex{functional induction}
\label{FunInduction}

The tactic \texttt{functional induction} performs
case analysis and induction following the definition of a function. It
makes use of a principle generated by \texttt{Function}
(see Section~\ref{Function}) or \texttt{Functional Scheme}
(see Section~\ref{FunScheme}).

\begin{coq_eval}
Reset Initial.
Import Nat.
\end{coq_eval}
\begin{coq_example}
Functional Scheme minus_ind := Induction for minus Sort Prop.
Check minus_ind.
Lemma le_minus (n m:nat) : n - m <= n.
functional induction (minus n m) using minus_ind; simpl; auto.
\end{coq_example}
\begin{coq_example*}
Qed.
\end{coq_example*}

\Rem \texttt{(\qualid\ \term$_1$ \dots\ \term$_n$)} must be a correct
full application of \qualid. In particular, the rules for implicit
arguments are the same as usual. For example use \texttt{@\qualid} if
you want to write implicit arguments explicitly.

\Rem Parentheses over \qualid \dots \term$_n$ are mandatory.

\Rem \texttt{functional induction (f x1 x2 x3)} is actually a wrapper
for \texttt{induction x1, x2, x3, (f x1 x2 x3) using \qualid} followed by
a cleaning phase, where {\qualid} is the induction principle
registered for $f$ (by the \texttt{Function} (see Section~\ref{Function})
or \texttt{Functional Scheme} (see Section~\ref{FunScheme}) command)
corresponding to the sort of the goal.  Therefore \texttt{functional
  induction} may fail if the induction scheme {\qualid} is
not defined. See also Section~\ref{Function} for the function terms
accepted by \texttt{Function}.

\Rem There is a difference between obtaining an induction scheme for a
function by using \texttt{Function} (see Section~\ref{Function}) and by
using \texttt{Functional Scheme} after a normal definition using
\texttt{Fixpoint} or \texttt{Definition}. See \ref{Function} for
details.

\SeeAlso{\ref{Function},\ref{FunScheme},\ref{FunScheme-examples},
  \ref{sec:functional-inversion}}

\begin{ErrMsgs}
\item \errindex{Cannot find induction information on \qualid}
\item \errindex{Not the right number of induction arguments}
\end{ErrMsgs}

\begin{Variants}
\item {\tt functional induction (\qualid\ \term$_1$ \dots\ \term$_n$)
  as {\disjconjintropattern} using \term$_{m+1}$ with \bindinglist}

 Similarly to \texttt{Induction} and \texttt{elim}
 (see Section~\ref{Tac-induction}), this allows giving explicitly the
 name of the introduced variables, the
 induction principle, and the values of dependent premises of the
 elimination scheme, including \emph{predicates} for mutual induction
 when {\qualid} is part of a mutually recursive definition.

\end{Variants}

\subsection{\tt discriminate \term}
\label{discriminate}
\tacindex{discriminate}


This tactic proves any goal from an assumption stating that two
structurally different terms of an inductive set are equal. For
example, from {\tt (S (S O))=(S O)} we can derive by absurdity any
proposition.

The argument {\term} is assumed to be a proof of a statement
of conclusion {\tt{\term$_1$} = {\term$_2$}} with {\term$_1$} and
{\term$_2$} being elements of an inductive set.  To build the proof,
the tactic traverses the normal forms\footnote{Reminder: opaque
  constants will not be expanded by $\delta$ reductions.} of
{\term$_1$} and {\term$_2$} looking for a couple of subterms {\tt u}
and {\tt w} ({\tt u} subterm of the normal form of {\term$_1$} and
{\tt w} subterm of the normal form of {\term$_2$}), placed at the same
positions and whose head symbols are two different constructors. If
such a couple of subterms exists, then the proof of the current goal
is completed, otherwise the tactic fails.

\Rem The syntax {\tt discriminate {\ident}} can be used to refer to a
hypothesis quantified in the goal. In this case, the quantified
hypothesis whose name is {\ident} is first introduced in the local
context using \texttt{intros until \ident}.

\begin{ErrMsgs}
\item \errindex{No primitive equality found}
\item \errindex{Not a discriminable equality}
\end{ErrMsgs}

\begin{Variants}
\item \texttt{discriminate \num}

  This does the same thing as \texttt{intros until \num} followed by
  \texttt{discriminate \ident} where {\ident} is the identifier for
  the last introduced hypothesis.

\item \texttt{discriminate {\term} with \bindinglist}

  This does the same thing as \texttt{discriminate {\term}} but using
the given bindings to instantiate parameters or hypotheses of {\term}.

\item \texttt{ediscriminate \num}\tacindex{ediscriminate}\\
      \texttt{ediscriminate {\term} \zeroone{with \bindinglist}}

  This works the same as {\tt discriminate} but if the type of {\term},
  or the type of the hypothesis referred to by {\num}, has uninstantiated
  parameters, these parameters are left as existential variables.

\item \texttt{discriminate}

  This behaves like {\tt discriminate {\ident}} if {\ident} is the
  name of an hypothesis to which {\tt discriminate} is applicable; if
  the current goal is of the form {\term$_1$} {\tt <>} {\term$_2$},
  this behaves as {\tt intro {\ident}; discriminate {\ident}}.

  \ErrMsg \errindex{No discriminable equalities}
\end{Variants}

\subsection{\tt injection \term}
\label{injection}
\tacindex{injection}

The {\tt injection} tactic is based on the fact that constructors of
inductive sets are injections. That means that if $c$ is a constructor
of an inductive set, and if $(c~\vec{t_1})$ and $(c~\vec{t_2})$ are two
terms that are equal then $~\vec{t_1}$ and $~\vec{t_2}$ are equal
too.

If {\term} is a proof of a statement of conclusion
 {\tt {\term$_1$} = {\term$_2$}},
then {\tt injection} applies injectivity as deep as possible to
derive the equality of all the subterms of {\term$_1$} and {\term$_2$}
placed in the same positions. For example, from {\tt (S
  (S n))=(S (S (S m)))} we may derive {\tt n=(S m)}.  To use this
tactic {\term$_1$} and {\term$_2$} should be elements of an inductive
set and they should be neither explicitly equal, nor structurally
different. We mean by this that, if {\tt n$_1$} and {\tt n$_2$} are
their respective normal forms, then:
\begin{itemize}
\item {\tt n$_1$} and {\tt n$_2$} should not be syntactically equal,
\item there must not exist any pair of subterms {\tt u} and {\tt w},
  {\tt u} subterm of {\tt n$_1$} and {\tt w} subterm of {\tt n$_2$} ,
  placed in the same positions and having different constructors as
  head symbols.
\end{itemize}
If these conditions are satisfied, then, the tactic derives the
equality of all the subterms of {\term$_1$} and {\term$_2$} placed in
the same positions and puts them as antecedents of the current goal.

\Example Consider the following goal:

\begin{coq_example*}
Inductive list : Set :=
  | nil : list
  | cons : nat -> list -> list.
Variable P : list -> Prop.
\end{coq_example*}
\begin{coq_eval}
Lemma ex :
 forall (l:list) (n:nat), P nil -> cons n l = cons 0 nil -> P l.
intros l n H H0.
\end{coq_eval}
\begin{coq_example}
Show.
injection H0.
\end{coq_example}
\begin{coq_eval}
Abort.
\end{coq_eval}

Beware that \texttt{injection} yields an equality in a sigma type
whenever the injected object has a dependent type $P$ with its two
instances in different types $(P~t_1~...~t_n)$ and
$(P~u_1~...~u_n)$. If $t_1$ and $u_1$ are the same and have for type
an inductive type for which a decidable equality has been declared
using the command {\tt Scheme Equality} (see \ref{Scheme}), the use of
a sigma type is avoided.

\Rem If some quantified hypothesis of the goal is named {\ident}, then
{\tt injection {\ident}} first introduces the hypothesis in the local
context using \texttt{intros until \ident}.

\begin{ErrMsgs}
\item \errindex{Not a projectable equality but a discriminable one}
\item \errindex{Nothing to do, it is an equality between convertible terms}
\item \errindex{Not a primitive equality}
\end{ErrMsgs}

\begin{Variants}
\item \texttt{injection \num}

  This does the same thing as \texttt{intros until \num} followed by
\texttt{injection \ident} where {\ident} is the identifier for the last
introduced hypothesis.

\item \texttt{injection {\term} with \bindinglist}

  This does the same as \texttt{injection {\term}} but using
  the given bindings to instantiate parameters or hypotheses of {\term}.

\item \texttt{einjection \num}\tacindex{einjection}\\
      \texttt{einjection {\term} \zeroone{with \bindinglist}}

  This works the same as {\tt injection} but if the type of {\term},
  or the type of the hypothesis referred to by {\num}, has uninstantiated
  parameters, these parameters are left as existential variables.

\item{\tt injection}

  If the current goal is of the form {\term$_1$} {\tt <>} {\term$_2$},
  this behaves as {\tt intro {\ident}; injection {\ident}}.

  \ErrMsg \errindex{goal does not satisfy the expected preconditions}

\item \texttt{injection {\term} \zeroone{with \bindinglist} as \nelistnosep{\intropattern}}\\
\texttt{injection {\num} as {\intropattern} \dots\ \intropattern}\\
\texttt{injection as {\intropattern} \dots\ \intropattern}\\
\texttt{einjection {\term} \zeroone{with \bindinglist} as \nelistnosep{\intropattern}}\\
\texttt{einjection {\num} as {\intropattern} \dots\ \intropattern}\\
\texttt{einjection as {\intropattern} \dots\ \intropattern}
\tacindex{injection \dots\ as}

These variants apply \texttt{intros} \nelistnosep{\intropattern} after
the call to \texttt{injection} or \texttt{einjection} so that all
equalities generated are moved in the context of hypotheses. The
number of {\intropattern} must not exceed the number of equalities
newly generated. If it is smaller, fresh names are automatically
generated to adjust the list of {\intropattern} to the number of new
equalities. The original equality is erased if it corresponds to an
hypothesis.

\end{Variants}

\subsection{\tt inversion \ident}
\tacindex{inversion}

Let the type of {\ident} in the local context be $(I~\vec{t})$,
where $I$ is a (co)inductive predicate. Then,
\texttt{inversion} applied to \ident~ derives for each possible
constructor $c_i$ of $(I~\vec{t})$, {\bf all} the necessary
conditions that should hold for the instance $(I~\vec{t})$ to be
proved by $c_i$.

\Rem If {\ident} does not denote a hypothesis in the local context
but refers to a hypothesis quantified in the goal, then the
latter is first introduced in the local context using
\texttt{intros until \ident}.

\Rem As inversion proofs may be large in size, we recommend the user to
stock the lemmas whenever the same instance needs to be inverted
several times. See Section~\ref{Derive-Inversion}.

\begin{Variants}
\item \texttt{inversion \num}

  This does the same thing as \texttt{intros until \num} then
  \texttt{inversion \ident} where {\ident} is the identifier for the
  last introduced hypothesis.

\item \tacindex{inversion\_clear} \texttt{inversion\_clear \ident}

  This behaves as \texttt{inversion} and then erases \ident~ from the
  context.

\item \tacindex{inversion \dots\ as} \texttt{inversion {\ident} as \intropattern}

  This generally behaves as \texttt{inversion} but using names in
  {\intropattern} for naming hypotheses. The {\intropattern} must have
  the form {\tt [} $p_{11} \ldots p_{1n_1}$ {\tt |} {\ldots} {\tt |}
  $p_{m1} \ldots p_{mn_m}$ {\tt ]} with $m$ being the number of
  constructors of the type of {\ident}. Be careful that the list must
  be of length $m$ even if {\tt inversion} discards some cases (which
  is precisely one of its roles): for the discarded cases, just use an
  empty list (i.e. $n_i=0$).

  The arguments of the $i^{th}$ constructor and the
  equalities that {\tt inversion} introduces in the context of the
  goal corresponding to the $i^{th}$ constructor, if it exists, get
  their names from the list $p_{i1}$ \ldots $p_{in_i}$ in order. If
  there are not enough names, {\tt inversion} invents names for the
  remaining variables to introduce. In case an equation splits into
  several equations (because {\tt inversion} applies {\tt injection}
  on the equalities it generates), the corresponding name $p_{ij}$ in
  the list must be replaced by a sublist of the form {\tt [$p_{ij1}$
  \mbox{\dots} $p_{ijq}$]} (or, equivalently, {\tt ($p_{ij1}$,
  \dots, $p_{ijq}$)}) where $q$ is the number of subequalities
  obtained from splitting the original equation. Here is an example.

  The \texttt{inversion \dots\ as} variant of \texttt{inversion}
  generally behaves in a slightly more expectable way than
  \texttt{inversion} (no artificial duplication of some hypotheses
  referring to other hypotheses) To take benefit of these
  improvements, it is enough to use \texttt{inversion \dots\ as []},
  letting the names being finally chosen by {\Coq}.

\begin{coq_eval}
Require Import List.
\end{coq_eval}

\begin{coq_example}
Inductive contains0 : list nat -> Prop :=
  | in_hd : forall l, contains0 (0 :: l)
  | in_tl : forall l b, contains0 l -> contains0 (b :: l).
Goal forall l:list nat, contains0 (1 :: l) -> contains0 l.
intros l H; inversion H as [ | l' p Hl' [Heqp Heql'] ].
\end{coq_example}

\begin{coq_eval}
Abort.
\end{coq_eval}

\item \texttt{inversion {\num} as \intropattern}

  This allows naming the hypotheses introduced by
  \texttt{inversion \num} in the context.

\item \tacindex{inversion\_clear \dots\ as} \texttt{inversion\_clear
  {\ident} as \intropattern}

  This allows naming the hypotheses introduced by
  \texttt{inversion\_clear} in the context. Notice that hypothesis
  names can be provided as if \texttt{inversion} were called, even
  though the \texttt{inversion\_clear} will eventually erase the
  hypotheses.

\item \tacindex{inversion \dots\ in} \texttt{inversion {\ident}
  in \ident$_1$ \dots\ \ident$_n$}

  Let \ident$_1$ \dots\ \ident$_n$, be identifiers in the local context. This
  tactic behaves as generalizing \ident$_1$ \dots\ \ident$_n$, and
  then performing \texttt{inversion}.

\item \tacindex{inversion \dots\ as \dots\ in} \texttt{inversion
  {\ident} as {\intropattern} in \ident$_1$ \dots\
  \ident$_n$}

  This allows naming the hypotheses introduced in the context by
  \texttt{inversion {\ident} in \ident$_1$ \dots\ \ident$_n$}.

\item \tacindex{inversion\_clear \dots\ in} \texttt{inversion\_clear
  {\ident} in \ident$_1$ \dots\ \ident$_n$}

  Let \ident$_1$ \dots\ \ident$_n$, be identifiers in the local context. This
  tactic behaves as generalizing \ident$_1$ \dots\ \ident$_n$, and
  then performing {\tt inversion\_clear}.

\item \tacindex{inversion\_clear \dots\ as \dots\ in}
  \texttt{inversion\_clear {\ident} as {\intropattern}
  in \ident$_1$ \dots\ \ident$_n$}

  This allows naming the hypotheses introduced in the context by
  \texttt{inversion\_clear {\ident} in \ident$_1$ \dots\ \ident$_n$}.

\item \tacindex{dependent inversion} \texttt{dependent inversion \ident}

  That must be used when \ident\ appears in the current goal.  It acts
  like \texttt{inversion} and then substitutes \ident\ for the
  corresponding term in the goal.

\item \tacindex{dependent inversion \dots\ as } \texttt{dependent
    inversion {\ident} as \intropattern}

  This allows naming the hypotheses introduced in the context by
  \texttt{dependent inversion} {\ident}.

\item \tacindex{dependent inversion\_clear} \texttt{dependent
    inversion\_clear \ident}

  Like \texttt{dependent inversion}, except that {\ident} is cleared
  from the local context.

\item \tacindex{dependent inversion\_clear \dots\ as}
  \texttt{dependent inversion\_clear {\ident} as \intropattern}

  This allows naming the hypotheses introduced in the context by
  \texttt{dependent inversion\_clear} {\ident}.

\item \tacindex{dependent inversion \dots\ with} \texttt{dependent
    inversion {\ident} with \term}

  This variant allows you to specify the generalization of the goal. It
  is useful when the system fails to generalize the goal automatically. If
  {\ident} has type $(I~\vec{t})$ and $I$ has type
  $\forall (\vec{x}:\vec{T}), s$,   then \term~  must be of type
  $I:\forall (\vec{x}:\vec{T}), I~\vec{x}\to s'$ where $s'$ is the
  type of the goal.

\item \tacindex{dependent inversion \dots\ as \dots\ with}
  \texttt{dependent inversion {\ident} as {\intropattern}
  with \term}

  This allows naming the hypotheses introduced in the context by
  \texttt{dependent inversion {\ident} with \term}.

\item \tacindex{dependent inversion\_clear \dots\ with}
  \texttt{dependent inversion\_clear {\ident} with \term}

  Like \texttt{dependent inversion \dots\ with} but clears {\ident} from
  the local context.

\item \tacindex{dependent inversion\_clear \dots\ as \dots\ with}
  \texttt{dependent inversion\_clear {\ident} as
  {\intropattern} with \term}

  This allows naming the hypotheses introduced in the context by
  \texttt{dependent inversion\_clear {\ident} with \term}.

\item \tacindex{simple inversion} \texttt{simple inversion \ident}

  It is a very primitive inversion tactic that derives all the necessary
  equalities  but it does not simplify the  constraints as
  \texttt{inversion} does.

\item \tacindex{simple inversion \dots\ as} \texttt{simple inversion
  {\ident} as \intropattern}

  This allows naming the hypotheses introduced in the context by
  \texttt{simple inversion}.

\item \tacindex{inversion \dots\ using} \texttt{inversion {\ident}
  using \ident$'$}

  Let {\ident} have type $(I~\vec{t})$ ($I$ an inductive
  predicate) in the local context, and \ident$'$ be a (dependent) inversion
  lemma. Then, this tactic refines the current goal with the specified
  lemma.

\item \tacindex{inversion \dots\ using \dots\ in} \texttt{inversion
  {\ident} using \ident$'$ in \ident$_1$\dots\ \ident$_n$}

  This tactic behaves as generalizing \ident$_1$\dots\ \ident$_n$,
  then doing \texttt{inversion {\ident} using \ident$'$}.

\end{Variants}

\firstexample
\example{Non-dependent inversion}
\label{inversion-examples}

Let us consider the relation \texttt{Le} over natural numbers and the
following variables:

\begin{coq_eval}
Reset Initial.
\end{coq_eval}

\begin{coq_example*}
Inductive Le : nat -> nat -> Set :=
  | LeO : forall n:nat, Le 0 n
  | LeS : forall n m:nat, Le n m -> Le (S n) (S m).
Variable P : nat -> nat -> Prop.
Variable Q : forall n m:nat, Le n m -> Prop.
\end{coq_example*}

Let us consider the following goal:

\begin{coq_eval}
Lemma ex : forall n m:nat, Le (S n) m -> P n m.
intros.
\end{coq_eval}

\begin{coq_example}
Show.
\end{coq_example}

To prove the goal, we may need to reason by cases on \texttt{H} and to
derive that \texttt{m} is necessarily of
the form $(S~m_0)$ for certain $m_0$ and that $(Le~n~m_0)$.
Deriving these conditions corresponds to prove that the
only possible constructor of \texttt{(Le (S n) m)} is
\texttt{LeS} and that we can invert the
\texttt{->} in the type  of \texttt{LeS}.
This inversion is possible because \texttt{Le} is the smallest set closed by
the constructors \texttt{LeO} and \texttt{LeS}.

\begin{coq_example}
inversion_clear H.
\end{coq_example}

Note that \texttt{m} has been substituted in the goal for \texttt{(S m0)}
and that the hypothesis \texttt{(Le n m0)} has been added to the
context.

Sometimes it is
interesting to have the equality \texttt{m=(S m0)} in the
context to use it after. In that case we can use \texttt{inversion} that
does not clear the equalities:

\begin{coq_eval}
Undo.
\end{coq_eval}

\begin{coq_example}
inversion H.
\end{coq_example}

\begin{coq_eval}
Abort.
\end{coq_eval}

\example{Dependent inversion}

Let us consider the following goal:

\begin{coq_eval}
Lemma ex_dep : forall (n m:nat) (H:Le (S n) m), Q (S n) m H.
intros.
\end{coq_eval}

\begin{coq_example}
Show.
\end{coq_example}

As \texttt{H} occurs in the goal, we may want to reason by cases on its
structure and so, we would like  inversion tactics to
substitute \texttt{H} by the corresponding term in constructor form.
Neither \texttt{Inversion} nor  {\tt Inversion\_clear} make such a
substitution.
To have such a behavior we use the dependent inversion tactics:

\begin{coq_example}
dependent inversion_clear H.
\end{coq_example}

Note that \texttt{H} has been substituted by \texttt{(LeS n m0 l)} and
\texttt{m} by \texttt{(S m0)}.

\subsection{\tt fix {\ident} {\num}}
\tacindex{fix}
\label{tactic:fix}

This tactic is a primitive tactic to start a proof by induction. In
general, it is easier to rely on higher-level induction tactics such
as the ones described in Section~\ref{Tac-induction}.

In the syntax of the tactic, the identifier {\ident} is the name given
to the induction hypothesis. The natural number {\num} tells on which
premise of the current goal the induction acts, starting
from 1 and counting both dependent and non dependent
products. Especially, the current lemma must be composed of at least
{\num} products.

Like in a {\tt fix} expression, the induction
hypotheses have to be used on structurally smaller arguments.
The verification that inductive proof arguments are correct is done
only at the time of registering the lemma in the environment. To know
if the use of induction hypotheses is correct at some
time of the interactive development of a proof, use the command {\tt
  Guarded} (see Section~\ref{Guarded}).

\begin{Variants}
  \item {\tt fix \ident$_1$ {\num} with ( \ident$_2$
    \nelistnosep{\binder$_2$} \zeroone{\{ struct \ident$'_2$
      \}} :~\type$_2$ ) \dots\ ( \ident$_n$
    \nelistnosep{\binder$_n$} \zeroone{\{ struct \ident$'_n$ \}} :~\type$_n$ )}

This starts a proof by mutual induction. The statements to be
simultaneously proved are respectively {\tt forall}
  \nelistnosep{{\binder}$_2$}{\tt ,} {\type}$_2$, {\ldots}, {\tt forall}
  \nelistnosep{{\binder}$_n$}{\tt ,} {\type}$_n$.  The identifiers
{\ident}$_1$ {\ldots} {\ident}$_n$ are the names of the induction
hypotheses. The identifiers {\ident}$'_2$ {\ldots} {\ident}$'_n$ are the
respective names of the premises on which the induction is performed
in the statements to be simultaneously proved (if not given, the
system tries to guess itself what they are).

\end{Variants}

\subsection{\tt cofix \ident}
\tacindex{cofix}
\label{tactic:cofix}

This tactic starts a proof by coinduction. The identifier {\ident} is
the name given to the coinduction hypothesis.  Like in a {\tt cofix}
expression, the use of induction hypotheses have to guarded by a
constructor.  The verification that the use of co-inductive hypotheses
is correct is done only at the time of registering the lemma in the
environment. To know if the use of coinduction hypotheses is correct
at some time of the interactive development of a proof, use the
command {\tt Guarded} (see Section~\ref{Guarded}).


\begin{Variants}
  \item {\tt cofix \ident$_1$ with ( \ident$_2$
    \nelistnosep{\binder$_2$} :~\type$_2$ ) \dots\ (
    \ident$_n$ \nelistnosep{\binder$_n$} :~\type$_n$ )}

This starts a proof by mutual coinduction. The statements to be
simultaneously proved are respectively {\tt forall}
\nelistnosep{{\binder}$_2$}{\tt ,} {\type}$_2$, {\ldots}, {\tt forall}
  \nelistnosep{{\binder}$_n$}{\tt ,} {\type}$_n$. The identifiers
    {\ident}$_1$ {\ldots} {\ident}$_n$ are the names of the
    coinduction hypotheses.

\end{Variants}

\section{Rewriting expressions}


These tactics use the equality {\tt eq:forall A:Type, A->A->Prop}
defined in file {\tt Logic.v} (see Section~\ref{Equality}). The
notation for {\tt eq}~$T~t~u$ is simply {\tt $t$=$u$} dropping the
implicit type of $t$ and $u$.

\subsection{\tt rewrite \term}
\label{rewrite}
\tacindex{rewrite}

This tactic applies to any goal. The type of {\term}
must have the form

\texttt{forall (x$_1$:A$_1$) \dots\ (x$_n$:A$_n$)}\texttt{eq} \term$_1$ \term$_2$.

\noindent where \texttt{eq} is the Leibniz equality or a registered
setoid equality.

\noindent Then {\tt rewrite \term} finds the first subterm matching
\term$_1$ in the goal, resulting in instances \term$_1'$ and \term$_2'$
and then replaces every occurrence of \term$_1'$ by \term$_2'$.
Hence, some of the variables x$_i$ are
solved by unification, and some of the types \texttt{A}$_1$, \dots,
\texttt{A}$_n$ become new subgoals.

% \Rem In case the type of
% \term$_1$ contains occurrences of variables bound in the
% type of \term, the tactic tries first to find a subterm of the goal
% which matches this term in order to find a closed instance \term$'_1$
% of \term$_1$, and then all instances of \term$'_1$ will be replaced.

\begin{ErrMsgs}
\item \errindex{The term provided does not end with an equation}

\item \errindex{Tactic generated a subgoal identical to the original goal}

This happens if \term$_1$ does not occur in the goal.
\end{ErrMsgs}

\begin{Variants}
\item {\tt rewrite -> \term}\tacindex{rewrite ->}

  Is equivalent to {\tt rewrite \term}

\item {\tt rewrite <- \term}\tacindex{rewrite <-}

  Uses the equality \term$_1${\tt=}\term$_2$ from right to left

\item {\tt rewrite {\term} in \nterm{clause}}
  \tacindex{rewrite \dots\ in}

  Analogous to {\tt rewrite {\term}} but rewriting is done following
  \nterm{clause} (similarly to \ref{Conversion-tactics}). For
  instance:
  \begin{itemize}
  \item \texttt{rewrite H in H1} will rewrite \texttt{H} in the hypothesis
    \texttt{H1} instead of the current goal.
  \item \texttt{rewrite H in H1 at 1, H2 at - 2 |- *} means \texttt{rewrite H; rewrite H in H1 at 1;
      rewrite H in H2 at - 2}. In particular a failure will happen if any of
    these three simpler tactics fails.
  \item \texttt{rewrite H in * |- } will do \texttt{rewrite H in
      H$_i$} for all hypothesis \texttt{H$_i$ <> H}. A success will happen
    as soon as at least one of these simpler tactics succeeds.
  \item \texttt{rewrite H in *} is a combination of \texttt{rewrite H}
    and \texttt{rewrite H in * |-} that succeeds if at
    least one of these two tactics succeeds.
  \end{itemize}
  Orientation {\tt ->} or {\tt <-} can be
  inserted before the term to rewrite.

\item {\tt rewrite {\term} at {\occlist}}
  \tacindex{rewrite \dots\ at}

  Rewrite only the given occurrences of \term$_1'$. Occurrences are
  specified from left to right as for \texttt{pattern} (\S
  \ref{pattern}). The rewrite is always performed using setoid
  rewriting, even for Leibniz's equality, so one has to
  \texttt{Import Setoid} to use this variant.

\item {\tt rewrite {\term} by {\tac}}
  \tacindex{rewrite \dots\ by}

  Use {\tac} to completely solve the side-conditions arising from the
  rewrite.

\item {\tt rewrite \term$_1$ , \mbox{\dots} , \term$_n$}

  Is equivalent to the $n$ successive tactics {\tt rewrite $\term_1$}
  up to {\tt rewrite $\term_n$}, each one working on the first subgoal
  generated by the previous one.
  Orientation {\tt ->} or {\tt <-} can be
  inserted before each term to rewrite. One unique \nterm{clause}
  can be added at the end after the keyword {\tt in}; it will
  then affect all rewrite operations.

\item In all forms of {\tt rewrite} described above, a term to rewrite
  can be immediately prefixed by one of the following modifiers:
  \begin{itemize}
  \item {\tt ?} : the tactic {\tt rewrite ?$\term$} performs the
    rewrite of $\term$  as many times as possible (perhaps zero time).
    This form never fails.
  \item {\tt $n$?} : works similarly, except that it will do at most
   $n$ rewrites.
  \item {\tt !} : works as {\tt ?}, except that at least one rewrite
    should succeed, otherwise the tactic fails.
  \item {\tt $n$!} (or simply {\tt $n$}) : precisely $n$ rewrites
    of $\term$ will be done, leading to failure if these $n$ rewrites are not possible.
  \end{itemize}

\item {\tt erewrite {\term}\tacindex{erewrite}}

This tactic works as {\tt rewrite {\term}} but turning unresolved
bindings into existential variables, if any, instead of failing. It has
the same variants as {\tt rewrite} has.

\end{Variants}

\subsection{\tt replace \term$_1$ with \term$_2$}
\label{tactic:replace}
\tacindex{replace \dots\ with}

This tactic applies to any goal. It replaces all free occurrences of
{\term$_1$} in the current goal with {\term$_2$} and generates the
equality {\term$_2$}{\tt =}{\term$_1$} as a subgoal. This equality is
automatically solved if it occurs among the assumption, or if its
symmetric form occurs.  It is equivalent to {\tt cut
\term$_2$=\term$_1$; [intro H{\sl n}; rewrite <- H{\sl n}; clear H{\sl
n}| assumption || symmetry; try assumption]}.

\begin{ErrMsgs}
\item \errindex{terms do not have convertible types}
\end{ErrMsgs}

\begin{Variants}

\item {\tt replace \term$_1$ with \term$_2$ by \tac}

  This acts as {\tt replace \term$_1$ with \term$_2$} but applies {\tt \tac}
  to solve the generated subgoal {\tt \term$_2$=\term$_1$}.

\item {\tt replace {\term}}

  Replaces {\term} with {\term'} using the
  first assumption whose type has the form {\tt \term=\term'} or {\tt
    \term'=\term}.

\item {\tt replace -> {\term}}

  Replaces {\term} with {\term'} using the
  first assumption whose type has the form {\tt \term=\term'}

\item {\tt replace <- {\term}}

  Replaces {\term} with {\term'} using the
  first assumption whose type has the form {\tt \term'=\term}

\item {\tt replace {\term$_1$} with {\term$_2$} in \nterm{clause} }\\
    {\tt replace {\term$_1$} with {\term$_2$} in \nterm{clause} by \tac }\\
    {\tt replace {\term} in \nterm{clause}}\\
    {\tt replace -> {\term} in \nterm{clause}}\\
    {\tt replace <- {\term} in \nterm{clause}}

    Acts as before but the replacements take place in
    \nterm{clause}~(see Section~\ref{Conversion-tactics}) and not only
    in the conclusion of the goal.
    The \nterm{clause} argument must not contain any \texttt{type of} nor \texttt{value of}.

\item {\tt cutrewrite <- (\term$_1$ = \term$_2$)}
%\label{cutrewrite}
\tacindex{cutrewrite}

This tactic is deprecated. It acts like {\tt replace {\term$_2$} with
  {\term$_1$}}, or, equivalently as {\tt enough} (\term$_1$ =
\term$_2$) {\tt as <-}.

\item {\tt cutrewrite -> (\term$_1$ = \term$_2$)}
%\label{cutrewrite}
\tacindex{cutrewrite}

This tactic is deprecated. It can be replaced by {\tt enough}
(\term$_1$ = \term$_2$) {\tt as ->}.

\end{Variants}

\subsection{\tt subst \ident}
\tacindex{subst}
\optindex{Regular Subst Tactic}

This tactic applies to a goal that has \ident\ in its context and
(at least) one hypothesis, say {\tt H}, of type {\tt
  \ident=t} or {\tt t=\ident}. Then it replaces
\ident\ by {\tt t} everywhere in the goal (in the hypotheses
and in the conclusion) and clears \ident\ and {\tt H} from the context.

\Rem
When several hypotheses have the form {\tt \ident=t} or {\tt
  t=\ident}, the first one is used.

\begin{Variants}
  \item {\tt subst \ident$_1$ \dots \ident$_n$}

    Is equivalent to {\tt subst \ident$_1$; \dots; subst \ident$_n$}.
  \item {\tt subst}

    Applies {\tt subst} repeatedly to all identifiers from the context
    for which an equality exists.

\noindent {\bf Remark: } The behavior of {\tt subst} can be controlled using option {\tt Set
    Regular Subst Tactic}. When this option is activated, {\tt subst}
  manages the following corner cases which otherwise it
  does not:
\begin{itemize}
\item A context with ordered hypotheses {\tt \ident$_1$ = \ident$_2$}
  and {\tt \ident$_1$ = $t$}, or {$t'$ = \ident$_1$} with $t'$ not a
  variable, and no other hypotheses of the form {\tt \ident$_2$ = $u$}
  or {\tt $u$ = \ident$_2$}
\item A context with cyclic dependencies as with hypotheses {\tt
  \ident$_1$ = f~\ident$_2$} and {\tt \ident$_2$ = g~\ident$_1$}
\end{itemize}
Additionally, it prevents a local definition such as {\tt \ident :=
  $t$} to be unfolded which otherwise it would exceptionally unfold in
configurations containing hypotheses of the form {\tt {\ident} = $u$},
or {\tt $u'$ = \ident} with $u'$ not a variable.

The option is off by default.

\end{Variants}

\subsection{\tt stepl \term}
\tacindex{stepl}

This tactic is for chaining rewriting steps. It assumes a goal of the
form ``$R$ {\term}$_1$ {\term}$_2$'' where $R$ is a binary relation
and relies on a database of lemmas of the form {\tt forall} $x$ $y$
$z$, $R$ $x$ $y$ {\tt ->} $eq$ $x$ $z$ {\tt ->} $R$ $z$ $y$ where $eq$
is typically a setoid equality. The application of {\tt stepl {\term}}
then replaces the goal by ``$R$ {\term} {\term}$_2$'' and adds a new
goal stating ``$eq$ {\term} {\term}$_1$''.

Lemmas are added to the database using the command
\comindex{Declare Left Step}
\begin{quote}
{\tt Declare Left Step {\term}.}
\end{quote}

The tactic is especially useful for parametric setoids which are not
accepted as regular setoids for {\tt rewrite} and {\tt
  setoid\_replace} (see Chapter~\ref{setoids}).

\begin{Variants}
\item{\tt stepl {\term} by {\tac}}

This applies {\tt stepl {\term}} then applies {\tac} to the second goal.

\item{\tt stepr {\term}}\\
     {\tt stepr {\term} by {\tac}}\tacindex{stepr}

This behaves as {\tt stepl} but on the right-hand-side of the binary relation.
Lemmas are expected to be of the form
``{\tt forall} $x$ $y$
$z$, $R$ $x$ $y$ {\tt ->} $eq$ $y$ $z$ {\tt ->} $R$ $x$ $z$''
and are registered using the command
\comindex{Declare Right Step}
\begin{quote}
{\tt Declare Right Step {\term}.}
\end{quote}
\end{Variants}

\subsection{\tt change \term}
\tacindex{change}
\label{change}

This tactic applies to any goal. It implements the rule
``Conv''\index{Typing rules!Conv} given in Section~\ref{Conv}.  {\tt
  change U} replaces the current goal \T\ with \U\ providing that
\U\ is well-formed and that \T\ and \U\ are convertible.

\begin{ErrMsgs}
\item \errindex{Not convertible}
\end{ErrMsgs}

\tacindex{change \dots\ in}
\begin{Variants}
\item {\tt change \term$_1$ with \term$_2$}

  This replaces the occurrences of \term$_1$ by \term$_2$ in the
  current goal.  The terms \term$_1$ and \term$_2$ must be
  convertible.

\item {\tt change \term$_1$ at \num$_1$ \dots\ \num$_i$ with \term$_2$}

  This replaces the occurrences numbered \num$_1$ \dots\ \num$_i$ of
  \term$_1$ by \term$_2$ in the current goal.
  The terms \term$_1$ and \term$_2$ must be convertible.

  \ErrMsg {\tt Too few occurrences}

\item {\tt change {\term} in {\ident}}

\item {\tt change \term$_1$ with \term$_2$ in {\ident}}

\item {\tt change \term$_1$ at  \num$_1$ \dots\ \num$_i$ with \term$_2$ in
    {\ident}}

  This applies the {\tt change} tactic not to the goal but to the
  hypothesis {\ident}.

\end{Variants}

\SeeAlso \ref{Conversion-tactics}


\section{Performing computations
\index{Conversion tactics}
\label{Conversion-tactics}}

This set of tactics implements different specialized usages of the
tactic \texttt{change}.

All conversion tactics (including \texttt{change}) can be
parameterized by the parts of the goal where the conversion can
occur. This is done using \emph{goal clauses} which consists in a list
of hypotheses and, optionally, of a reference to the conclusion of the
goal. For defined hypothesis it is possible to specify if the
conversion should occur on the type part, the body part or both
(default).

\index{Clauses}
\index{Goal clauses}
Goal clauses are written after a conversion tactic (tactics
\texttt{set}~\ref{tactic:set},          \texttt{rewrite}~\ref{rewrite},
\texttt{replace}~\ref{tactic:replace}                               and
\texttt{autorewrite}~\ref{tactic:autorewrite} also use goal clauses)  and
are introduced by  the keyword \texttt{in}. If no goal clause is provided,
the default is to perform the conversion only in the conclusion.

The syntax and description of the various goal clauses is the following:
\begin{description}
\item[]\texttt{in {\ident}$_1$ $\ldots$ {\ident}$_n$ |- } only in hypotheses {\ident}$_1$
  \ldots {\ident}$_n$
\item[]\texttt{in {\ident}$_1$ $\ldots$ {\ident}$_n$ |- *} in hypotheses {\ident}$_1$ \ldots
  {\ident}$_n$ and in the conclusion
\item[]\texttt{in * |-} in every hypothesis
\item[]\texttt{in *} (equivalent to \texttt{in * |- *}) everywhere
\item[]\texttt{in (type of {\ident}$_1$) (value of {\ident}$_2$) $\ldots$ |-} in
  type part of {\ident}$_1$, in the value part of {\ident}$_2$, etc.
\end{description}

For backward compatibility, the notation \texttt{in}~{\ident}$_1$\ldots {\ident}$_n$
performs the conversion in hypotheses {\ident}$_1$\ldots {\ident}$_n$.

%%%%%%%%%%%%%%%%%%%%%%%%%%%%%%%%%%
%voir reduction__conv_x : histoires d'univers.
%%%%%%%%%%%%%%%%%%%%%%%%%%%%%%%%%%

\subsection{{\tt cbv \flag$_1$ \mbox{\dots} \flag$_n$}, {\tt lazy \flag$_1$
\mbox{\dots} \flag$_n$}, and \tt compute}
\tacindex{cbv}
\tacindex{lazy}
\tacindex{compute}
\tacindex{vm\_compute}\label{vmcompute}
\tacindex{native\_compute}\label{nativecompute}

These parameterized reduction tactics apply to any goal and perform
the normalization of the goal according to the specified flags. In
correspondence with the kinds of reduction considered in \Coq\, namely
$\beta$ (reduction of functional application), $\delta$ (unfolding of
transparent constants, see \ref{Transparent}), $\iota$ (reduction of
pattern-matching over a constructed term, and unfolding of {\tt fix}
and {\tt cofix} expressions) and $\zeta$ (contraction of local
definitions), the flag are either {\tt beta}, {\tt delta}, {\tt iota}
or {\tt zeta}. The {\tt delta} flag itself can be refined into {\tt
delta [\qualid$_1$\ldots\qualid$_k$]} or {\tt delta
-[\qualid$_1$\ldots\qualid$_k$]}, restricting in the first case the
constants to unfold to the constants listed, and restricting in the
second case the constant to unfold to all but the ones explicitly
mentioned. Notice that the {\tt delta} flag does not apply to
variables bound by a let-in construction inside the term itself (use
here the {\tt zeta} flag). In any cases, opaque constants are not
unfolded (see Section~\ref{Opaque}).

The goal may be normalized with two strategies: {\em lazy} ({\tt lazy}
tactic), or {\em call-by-value} ({\tt cbv} tactic). The lazy strategy
is a call-by-need strategy, with sharing of reductions: the arguments of a
function call are partially evaluated only when necessary, and if an
argument is used several times then it is computed only once. This
reduction is efficient for reducing expressions with dead code. For
instance, the proofs of a proposition {\tt exists~$x$. $P(x)$} reduce to a
pair of a witness $t$, and a proof that $t$ satisfies the predicate
$P$. Most of the time, $t$ may be computed without computing the proof
of $P(t)$, thanks to the lazy strategy.

The call-by-value strategy is the one used in ML languages: the
arguments of a function call are evaluated first, using a weak
reduction (no reduction under the $\lambda$-abstractions). Despite the
lazy strategy always performs fewer reductions than the call-by-value
strategy, the latter is generally more efficient for evaluating purely
computational expressions (i.e. with few dead code).

\begin{Variants}
\item {\tt compute} \tacindex{compute}\\
      {\tt cbv}

  These are synonyms for {\tt cbv beta delta iota zeta}.

\item {\tt lazy}

  This is a synonym for {\tt lazy beta delta iota zeta}.

\item {\tt compute [\qualid$_1$\ldots\qualid$_k$]}\\
      {\tt cbv [\qualid$_1$\ldots\qualid$_k$]}

  These are synonyms of {\tt cbv beta delta
  [\qualid$_1$\ldots\qualid$_k$] iota zeta}.

\item {\tt compute -[\qualid$_1$\ldots\qualid$_k$]}\\
      {\tt cbv -[\qualid$_1$\ldots\qualid$_k$]}

  These are synonyms of {\tt cbv beta delta
  -[\qualid$_1$\ldots\qualid$_k$] iota zeta}.

\item {\tt lazy [\qualid$_1$\ldots\qualid$_k$]}\\
      {\tt lazy -[\qualid$_1$\ldots\qualid$_k$]}

  These are respectively synonyms of {\tt lazy beta delta
  [\qualid$_1$\ldots\qualid$_k$] iota zeta} and {\tt lazy beta delta
  -[\qualid$_1$\ldots\qualid$_k$] iota zeta}.

\item {\tt vm\_compute} \tacindex{vm\_compute}

  This tactic evaluates the goal using the optimized call-by-value evaluation
  bytecode-based virtual machine described in
  \cite{CompiledStrongReduction}. This algorithm is dramatically more efficient
  than the algorithm used for the {\tt cbv} tactic, but it cannot be
  fine-tuned. It is specially interesting for full evaluation of algebraic
  objects. This includes the case of reflection-based tactics.

\item {\tt native\_compute} \tacindex{native\_compute}

  This tactic evaluates the goal by compilation to \ocaml{} as described in
  \cite{FullReduction}. If \Coq{} is running in native code, it can be typically
  two to five times faster than {\tt vm\_compute}. Note however that the
  compilation cost is higher, so it is worth using only for intensive
  computations.

\end{Variants}

% Obsolete? Anyway not very important message
%\begin{ErrMsgs}
%\item \errindex{Delta must be specified before}
%
%  A list of constants appeared before the {\tt delta} flag.
%\end{ErrMsgs}


\subsection{\tt red}
\tacindex{red}

This tactic applies to a goal that has the form {\tt
  forall (x:T1)\dots(xk:Tk), t} with {\tt t}
$\beta\iota\zeta$-reducing to {\tt c t1 \dots\ tn} and {\tt c} a
constant.  If
{\tt c} is transparent then it replaces {\tt c} with its definition
(say {\tt t}) and then reduces {\tt (t t1 \dots\ tn)} according to
$\beta\iota\zeta$-reduction rules.

\begin{ErrMsgs}
\item \errindex{Not reducible}
\end{ErrMsgs}

\subsection{\tt hnf}
\tacindex{hnf}

This tactic applies to any goal. It replaces the current goal with its
head normal form according to the $\beta\delta\iota\zeta$-reduction
rules, i.e.  it reduces the head of the goal until it becomes a
product or an irreducible term. All inner $\beta\iota$-redexes are also
reduced.

\Example
The term \verb+forall n:nat, (plus (S n) (S n))+ is not reduced by {\tt hnf}.

\Rem The $\delta$ rule only applies to transparent constants
(see Section~\ref{Opaque} on transparency and opacity).

\subsection{\texorpdfstring{\texttt{cbn}}{cbn} and \texorpdfstring{\texttt{simpl}}{simpl}}
\tacindex{cbn} \tacindex{simpl}

These tactics apply to any goal. They try to reduce a term to
something still readable instead of fully normalizing it. They perform
a sort of strong normalization with two key differences:
\begin{itemize}
\item They unfold a constant if and only if it leads to a
  $\iota$-reduction, i.e. reducing a match or unfolding a fixpoint.
\item While reducing a constant unfolding to (co)fixpoints, 
  the tactics use the name of the
  constant the (co)fixpoint comes from instead of the (co)fixpoint
  definition in recursive calls.
\end{itemize}

The \texttt{cbn} tactic is claimed to be a more principled, faster and more
predictable replacement for \texttt{simpl}.

The \texttt{cbn} tactic accepts the same flags as \texttt{cbv} and
\texttt{lazy}.  The behavior of both \texttt{simpl} and \texttt{cbn}
can be tuned using the \texttt{Arguments} vernacular command as
follows: \comindex{Arguments}
\begin{itemize}
\item
A constant can be marked to be never unfolded by \texttt{cbn} or
\texttt{simpl}:
\begin{coq_example*}
Arguments minus n m : simpl never.
\end{coq_example*}
After that command an expression like \texttt{(minus (S x) y)} is left
untouched by the tactics \texttt{cbn} and \texttt{simpl}.
\item
A constant can be marked to be unfolded only if applied to enough arguments.
The number of arguments required can be specified using
the {\tt /} symbol in the arguments list of the {\tt Arguments} vernacular
command.
\begin{coq_example*}
Definition fcomp A B C f (g : A -> B) (x : A) : C := f (g x).
Notation "f \o g" := (fcomp f g) (at level 50).
Arguments fcomp {A B C} f g x /.
\end{coq_example*}
After that command the expression {\tt (f \verb+\+o g)} is left untouched by
{\tt simpl} while {\tt ((f \verb+\+o g) t)} is reduced to {\tt (f (g t))}.
The same mechanism can be used to make a constant volatile, i.e. always
unfolded.
\begin{coq_example*}
Definition volatile := fun x : nat => x.
Arguments volatile / x.
\end{coq_example*}
\item
A constant can be marked to be unfolded only if an entire set of arguments
evaluates to a constructor. The {\tt !} symbol can be used to mark such
arguments.
\begin{coq_example*}
Arguments minus !n !m.
\end{coq_example*}
After that command, the expression {\tt (minus (S x) y)} is left untouched by
{\tt simpl}, while {\tt (minus (S x) (S y))} is reduced to {\tt (minus x y)}.
\item
A special heuristic to determine if a constant has to be unfolded can be
activated with the following command:
\begin{coq_example*}
Arguments minus n m : simpl nomatch.
\end{coq_example*}
The heuristic avoids to perform a simplification step that would
expose a {\tt match} construct in head position. For example the
expression {\tt (minus (S (S x)) (S y))} is simplified to
{\tt (minus (S x) y)} even if an extra simplification is possible.
\end{itemize}

In detail, the tactic \texttt{simpl} first applies
$\beta\iota$-reduction.  Then, it expands transparent constants and
tries to reduce further using $\beta\iota$-reduction. But, when no
$\iota$ rule is applied after unfolding then $\delta$-reductions are
not applied. For instance trying to use \texttt{simpl} on
\texttt{(plus n O)=n} changes nothing.

Notice that only transparent constants whose name can be reused in the
recursive calls are possibly unfolded by \texttt{simpl}. For instance
a constant defined by \texttt{plus' := plus} is possibly unfolded and
reused in the recursive calls, but a constant such as \texttt{succ :=
  plus (S O)} is never unfolded. This is the main difference between
\texttt{simpl} and \texttt{cbn}. The tactic \texttt{cbn} reduces
whenever it will be able to reuse it or not: \texttt{succ t} is
reduced to \texttt{S t}.

\tacindex{simpl \dots\ in}
\begin{Variants}
\item {\tt cbn [\qualid$_1$\ldots\qualid$_k$]}\\
      {\tt cbn -[\qualid$_1$\ldots\qualid$_k$]}

  These are respectively synonyms of {\tt cbn beta delta
  [\qualid$_1$\ldots\qualid$_k$] iota zeta} and {\tt cbn beta delta
  -[\qualid$_1$\ldots\qualid$_k$] iota zeta} (see \ref{vmcompute}).

\item {\tt simpl {\pattern}}

  This applies {\tt simpl} only to the subterms matching {\pattern} in the
  current goal.

\item {\tt simpl {\pattern} at \num$_1$ \dots\ \num$_i$}

  This applies {\tt simpl} only to the \num$_1$, \dots, \num$_i$
  occurrences of the subterms matching {\pattern} in the current goal.

  \ErrMsg {\tt Too few occurrences}

\item {\tt simpl {\qualid}}\\
  {\tt simpl {\qstring}}

  This applies {\tt simpl} only to the applicative subterms whose head
  occurrence is the unfoldable constant {\qualid} (the constant can be
  referred to by its notation using {\qstring} if such a notation
  exists).

\item {\tt simpl {\qualid} at \num$_1$ \dots\ \num$_i$}\\
  {\tt simpl {\qstring} at \num$_1$ \dots\ \num$_i$}\\

  This applies {\tt simpl} only to the \num$_1$, \dots, \num$_i$
  applicative subterms whose head occurrence is {\qualid} (or
  {\qstring}).

\end{Variants}

\subsection{\tt unfold \qualid}
\tacindex{unfold}
\label{unfold}

This tactic applies to any goal. The argument {\qualid} must denote a
defined transparent constant or local definition (see Sections~\ref{Basic-definitions} and~\ref{Transparent}).  The tactic {\tt
  unfold} applies the $\delta$ rule to each occurrence of the constant
to which {\qualid} refers in the current goal and then replaces it
with its $\beta\iota$-normal form.

\begin{ErrMsgs}
\item {\qualid} \errindex{does not denote an evaluable constant}

\end{ErrMsgs}

\begin{Variants}
\item {\tt unfold {\qualid}$_1$, \dots, \qualid$_n$}
  \tacindex{unfold \dots\ in}

  Replaces {\em simultaneously} {\qualid}$_1$, \dots, {\qualid}$_n$
  with their definitions and replaces the current goal with its
  $\beta\iota$ normal form.

\item {\tt unfold {\qualid}$_1$ at \num$_1^1$, \dots, \num$_i^1$,
\dots,\ \qualid$_n$ at \num$_1^n$ \dots\ \num$_j^n$}

  The lists \num$_1^1$, \dots, \num$_i^1$ and \num$_1^n$, \dots,
  \num$_j^n$ specify the occurrences of {\qualid}$_1$, \dots,
  \qualid$_n$ to be unfolded. Occurrences are located from left to
  right.

  \ErrMsg {\tt bad occurrence number of {\qualid}$_i$}

  \ErrMsg {\qualid}$_i$ {\tt does not occur}

\item {\tt unfold {\qstring}}

  If {\qstring} denotes the discriminating symbol of a notation (e.g. {\tt
  "+"}) or an expression defining a notation (e.g. \verb!"_ + _"!), and
  this notation refers to an unfoldable constant, then the tactic
  unfolds it.

\item {\tt unfold {\qstring}\%{\delimkey}}

  This is variant of {\tt unfold {\qstring}} where {\qstring} gets its
  interpretation from the scope bound to the delimiting key
  {\delimkey} instead of its default interpretation (see
  Section~\ref{scopechange}).

\item {\tt unfold \qualidorstring$_1$ at \num$_1^1$, \dots, \num$_i^1$,
\dots,\ \qualidorstring$_n$ at \num$_1^n$ \dots\ \num$_j^n$}

  This is the most general form, where {\qualidorstring} is either a
  {\qualid} or a {\qstring} referring to a notation.

\end{Variants}

\subsection{\tt fold \term}
\tacindex{fold}

This tactic applies to any goal. The term \term\ is reduced using the {\tt red}
tactic. Every occurrence of the resulting term in the goal is then
replaced by \term.

\begin{Variants}
\item {\tt fold} \term$_1$ \dots\ \term$_n$

  Equivalent to {\tt fold} \term$_1${\tt;}\ldots{\tt; fold} \term$_n$.
\end{Variants}

\subsection{\tt pattern \term}
\tacindex{pattern}
\label{pattern}

This command applies to any goal. The argument {\term} must be a free
subterm of the current goal.  The command {\tt pattern} performs
$\beta$-expansion (the inverse of $\bt$-reduction) of the current goal
(say \T) by
\begin{enumerate}
\item replacing all occurrences of {\term} in {\T} with a fresh variable
\item abstracting this variable
\item applying the abstracted goal to {\term}
\end{enumerate}

For instance, if the current goal $T$ is expressible has $\phi(t)$
where the notation captures all the instances of $t$ in $\phi(t)$,
then {\tt pattern $t$} transforms it into {\tt (fun x:$A$ => $\phi(${\tt
x}$)$) $t$}.  This command can be used, for instance, when the tactic
{\tt apply} fails on matching.

\begin{Variants}
\item {\tt pattern {\term} at {\num$_1$} \dots\ {\num$_n$}}

  Only the occurrences {\num$_1$} \dots\ {\num$_n$} of {\term} are
  considered for $\beta$-expansion. Occurrences are located from left
  to right.

\item {\tt pattern {\term} at - {\num$_1$} \dots\ {\num$_n$}}

  All occurrences except the occurrences of indexes {\num$_1$} \dots\
  {\num$_n$} of {\term} are considered for
  $\beta$-expansion. Occurrences are located from left to right.

\item {\tt pattern {\term$_1$}, \dots, {\term$_m$}}

  Starting from a goal $\phi(t_1 \dots\ t_m)$, the tactic
   {\tt pattern $t_1$, \dots,\ $t_m$} generates the equivalent goal {\tt
   (fun (x$_1$:$A_1$) \dots\ (x$_m$:$A_m$) => $\phi(${\tt x$_1$\dots\
   x$_m$}$)$) $t_1$ \dots\ $t_m$}. If $t_i$ occurs in one of the
   generated types $A_j$ these occurrences will also be considered and
   possibly abstracted.

\item {\tt pattern {\term$_1$} at {\num$_1^1$} \dots\ {\num$_{n_1}^1$}, \dots,
    {\term$_m$} at {\num$_1^m$} \dots\ {\num$_{n_m}^m$}}

  This behaves as above but processing only the occurrences \num$_1^1$,
  \dots, \num$_i^1$ of \term$_1$, \dots, \num$_1^m$, \dots, \num$_j^m$
  of \term$_m$ starting from \term$_m$.

\item {\tt pattern} {\term$_1$} \zeroone{{\tt at \zeroone{-}} {\num$_1^1$} \dots\ {\num$_{n_1}^1$}} {\tt ,} \dots {\tt ,}
    {\term$_m$} \zeroone{{\tt at \zeroone{-}} {\num$_1^m$} \dots\ {\num$_{n_m}^m$}}

  This is the most general syntax that combines the different variants.

\end{Variants}

\subsection{Conversion tactics applied to hypotheses}

{\convtactic} {\tt in} \ident$_1$ \dots\ \ident$_n$

Applies the conversion tactic {\convtactic} to the
hypotheses \ident$_1$, \ldots, \ident$_n$. The tactic {\convtactic} is
any of the conversion tactics listed in this section.

If \ident$_i$ is a local definition, then \ident$_i$ can be replaced
by (Type of \ident$_i$) to address not the body but the type of the
local definition. Example: {\tt unfold not in (Type of H1) (Type of H3).}

\begin{ErrMsgs}
\item \errindex{No such hypothesis} : {\ident}.
\end{ErrMsgs}


\section{Automation}
\subsection{\tt auto}
\label{auto}
\tacindex{auto}

This tactic implements a Prolog-like resolution procedure to solve the
current goal. It first tries to solve the goal using the {\tt
  assumption} tactic, then it reduces the goal to an atomic one using
{\tt intros} and introduces the newly generated hypotheses as hints.
Then it looks at the list of tactics associated to the head symbol of
the goal and tries to apply one of them (starting from the tactics
with lower cost). This process is recursively applied to the generated
subgoals.

By default, \texttt{auto} only uses the hypotheses of the current goal and the
hints of the database named {\tt core}.

\begin{Variants}

\item  {\tt auto \num}

  Forces the search depth to be \num. The maximal search depth is 5 by
  default.

\item {\tt auto with \ident$_1$ \dots\ \ident$_n$}

  Uses the hint databases $\ident_1$ \dots\ $\ident_n$ in addition to
  the database {\tt core}. See Section~\ref{Hints-databases} for the
  list of pre-defined databases and the way to create or extend a
  database.  This option can be combined with the previous one.

\item {\tt auto with *}

  Uses all existing hint databases, minus the special database
  {\tt v62}. See Section~\ref{Hints-databases}

\item \texttt{auto using} \nterm{lemma}$_1$ {\tt ,} {\ldots} {\tt ,} \nterm{lemma}$_n$

  Uses \nterm{lemma}$_1$, \ldots, \nterm{lemma}$_n$ in addition to
  hints (can be combined with the \texttt{with \ident} option). If
  $lemma_i$ is an inductive type, it is the collection of its
  constructors which is added as hints.

\item \texttt{auto using} \nterm{lemma}$_1$ {\tt ,} {\ldots} {\tt ,} \nterm{lemma}$_n$ with \ident$_1$ {\ldots} \ident$_n$

  This combines the effects of the {\tt using} and {\tt with} options.

\item {\tt trivial}\tacindex{trivial}

  This tactic is a restriction of {\tt auto} that is not recursive and
  tries only hints that cost 0. Typically it solves trivial
  equalities like $X=X$.

\item \texttt{trivial with \ident$_1$ \dots\ \ident$_n$}

\item \texttt{trivial with *}

\end{Variants}

\Rem {\tt auto} either solves completely the goal or else leaves it
intact. \texttt{auto} and \texttt{trivial} never fail.

\SeeAlso Section~\ref{Hints-databases}

\subsection{\tt eauto}
\tacindex{eauto}
\label{eauto}

This tactic generalizes {\tt auto}. While {\tt auto} does not try
resolution hints which would leave existential variables in the goal,
{\tt eauto} does try them (informally speaking, it uses {\tt eapply}
where {\tt auto} uses {\tt apply}).
As a consequence, {\tt eauto} can solve such a goal:

\begin{coq_eval}
Reset Initial.
\end{coq_eval}
\begin{coq_example}
Hint Resolve ex_intro.
Goal forall P:nat -> Prop, P 0 ->  exists n, P n.
eauto.
\end{coq_example}
\begin{coq_eval}
Abort.
\end{coq_eval}

Note that {\tt ex\_intro} should be declared as an
hint.

\SeeAlso Section~\ref{Hints-databases}

\subsection{\tt autounfold with \ident$_1$ \mbox{\dots} \ident$_n$}
\tacindex{autounfold}
\label{autounfold}

This tactic unfolds constants that were declared through a {\tt Hint
  Unfold} in the given databases.

\begin{Variants}
\item {\tt autounfold with \ident$_1$ \dots\ \ident$_n$ in \nterm{clause}}

  Performs the unfolding in the given clause.

\item {\tt autounfold with *}

  Uses the unfold hints declared in all the hint databases.
\end{Variants}


\subsection{\tt autorewrite with \ident$_1$ \mbox{\dots} \ident$_n$}
\label{tactic:autorewrite}
\tacindex{autorewrite}

This tactic \footnote{The behavior of this tactic has much changed compared to
the versions available in the previous distributions (V6). This may cause
significant changes in your theories to obtain the same result. As a drawback
of the re-engineering of the code, this tactic has also been completely revised
to get a very compact and readable version.} carries out rewritings according
the rewriting rule bases {\tt \ident$_1$ \dots \ident$_n$}.

Each rewriting rule of a base \ident$_i$ is applied to the main subgoal until
it fails. Once all the rules have been processed, if the main subgoal has
progressed (e.g., if it is distinct from the initial main goal) then the rules
of this base are processed again. If the main subgoal has not progressed then
the next base is processed. For the bases, the behavior is exactly similar to
the processing of the rewriting rules.

The rewriting rule bases are built with the {\tt Hint~Rewrite} vernacular
command.

\Warning{} This tactic may loop if you build non terminating rewriting systems.

\begin{Variant}
\item {\tt autorewrite with \ident$_1$ \mbox{\dots} \ident$_n$ using \tac}

Performs, in the same way, all the rewritings of the bases {\tt \ident$_1$
\mbox{\dots} \ident$_n$} applying {\tt \tac} to the main subgoal after each
rewriting step.

\item {\tt autorewrite with \ident$_1$ \mbox{\dots} \ident$_n$ in \qualid}

  Performs all the rewritings in hypothesis {\qualid}.
\item {\tt autorewrite with \ident$_1$ \mbox{\dots} \ident$_n$ in {\qualid} using \tac}

  Performs all  the rewritings  in hypothesis {\qualid}  applying {\tt
    \tac} to the main subgoal after each rewriting step.

\item {\tt autorewrite with \ident$_1$ \mbox{\dots} \ident$_n$ in \nterm{clause}}

  Performs all the rewriting  in the clause \nterm{clause}.
  The \nterm{clause} argument must not contain any \texttt{type of} nor \texttt{value of}.

\end{Variant}

\SeeAlso Section~\ref{HintRewrite} for feeding the database of lemmas used by {\tt autorewrite}.

\SeeAlso Section~\ref{autorewrite-example} for examples showing the use of
this tactic.

% En attente d'un moyen de valoriser les fichiers de demos
%\SeeAlso file \texttt{contrib/Rocq/DEMOS/Demo\_AutoRewrite.v}

\section{Controlling automation}

\subsection{The hints databases for {\tt auto} and {\tt eauto}}
\index{Hints databases}
\label{Hints-databases}
\comindex{Hint}

The hints for \texttt{auto} and \texttt{eauto} are stored in
databases.  Each database maps head symbols to a list of hints. One can
use the command \texttt{Print Hint \ident} to display the hints
associated to the head symbol \ident{} (see \ref{PrintHint}). Each
hint has a cost that is a nonnegative integer, and an optional pattern.
The hints with lower cost are tried first. A hint is tried by
\texttt{auto} when the conclusion of the current goal
matches its pattern or when it has no pattern.

\subsubsection*{Creating Hint databases
  \label{CreateHintDb}\comindex{CreateHintDb}}

One can optionally declare a hint database using the command
\texttt{Create HintDb}. If a hint is added to an unknown database, it
will be automatically created.

\medskip
\texttt{Create HintDb} {\ident} [\texttt{discriminated}]
\medskip

This command creates a new database named \ident.
The database is implemented by a Discrimination Tree (DT) that serves as
an index of all the lemmas. The DT can use transparency information to decide
if a constant should be indexed or not (c.f. \ref{HintTransparency}),
making the retrieval more efficient.
The legacy implementation (the default one for new databases) uses the
DT only on goals without existentials (i.e., auto goals), for non-Immediate
hints and do not make use of transparency hints, putting more work on the
unification that is run after retrieval (it keeps a list of the lemmas
in case the DT is not used). The new implementation enabled by
the {\tt discriminated} option makes use of DTs in all cases and takes
transparency information into account. However, the order in which hints
are retrieved from the DT may differ from the order in which they were
inserted, making this implementation observationally different from the
legacy one.

The general
command to add a hint to some databases \ident$_1$, \dots, \ident$_n$ is
\begin{tabbing}
  {\tt Hint {\hintdef} :~\ident$_1$ \mbox{\dots} \ident$_n$}
\end{tabbing}

\begin{Variants}
\item {\tt Hint \hintdef}

  No database name is given: the hint is registered in the {\tt core}
    database.

\item {\tt Local Hint {\hintdef} :~\ident$_1$ \mbox{\dots} \ident$_n$}

  This is used to declare hints that must not be exported to the other
  modules that require and import the current module. Inside a
  section, the option {\tt Local} is useless since hints do not
  survive anyway to the closure of sections.

\item {\tt Local Hint \hintdef}

  Idem for the {\tt core} database.

\end{Variants}

The {\hintdef} is one of the following expressions:

\begin{itemize}
\item {\tt Resolve \term}
  \comindex{Hint Resolve}

  This command adds {\tt simple apply {\term}} to the hint list
  with the head symbol of the type of \term. The cost of that hint is
  the number of subgoals generated by {\tt simple apply {\term}}.
%{\tt auto} actually uses a slightly modified variant of {\tt simple apply} with use_metas_eagerly_in_conv_on_closed_terms set to false

  The cost of that hint is the number of subgoals generated by that
  tactic.

  % Is it really needed?
  %% In case the inferred type of \term\ does not start with a product
  %% the tactic added in the hint list is {\tt exact {\term}}. In case
  %% this type can however be reduced to a type starting with a product,
  %% the tactic {\tt apply {\term}} is also stored in the hints list.

  If the inferred type of \term\ contains a dependent quantification
  on a variable which occurs only in the premisses of the type and not
  in its conclusion, no instance could be inferred for the variable by
  unification with the goal. In this case, the hint is added to the
  hint list of {\tt eauto} (see \ref{eauto}) instead of the hint list
  of {\tt auto} and a warning is printed. A typical example of a hint
  that is used only by \texttt{eauto} is a transitivity lemma.

  \begin{ErrMsgs}
%%  \item \errindex{Bound head variable}

  \item \term\ \errindex{cannot be used as a hint}

    The head symbol of the type of {\term} is a bound variable such
    that this tactic cannot be associated to a constant.

    %% The type of {\term} contains products over variables that do not
    %% appear in the conclusion. A typical example is a transitivity axiom.
    %% In that case the {\tt simple apply} tactic fails, and thus is useless.

  \end{ErrMsgs}

  \begin{Variants}

  \item {\tt Resolve \term$_1$ \mbox{\dots} \term$_m$}

    Adds each \texttt{Resolve} {\term$_i$}.

  \end{Variants}

\item \texttt{Immediate {\term}}
\comindex{Hint Immediate}

  This command adds {\tt simple apply {\term}; trivial} to the hint list
  associated with the head symbol of the type of {\ident} in the given
  database. This tactic will fail if all the subgoals generated by
  {\tt simple apply {\term}} are not solved immediately by the {\tt trivial}
  tactic (which only tries tactics with cost $0$).

  This command is useful for theorems such as the symmetry of equality
  or $n+1=m+1 \to n=m$ that we may like to introduce with a
  limited use in order to avoid useless proof-search.

  The cost of this tactic (which never generates subgoals) is always 1,
  so that it is not used by {\tt trivial} itself.

  \begin{ErrMsgs}

%%  \item \errindex{Bound head variable}

  \item \term\ \errindex{cannot be used as a hint}

  \end{ErrMsgs}

  \begin{Variants}

    \item {\tt Immediate \term$_1$ \mbox{\dots} \term$_m$}

      Adds each \texttt{Immediate} {\term$_i$}.

  \end{Variants}

\item \texttt{Constructors} {\ident}
\comindex{Hint Constructors}

  If {\ident} is an inductive type, this command adds all its
  constructors as hints of type \texttt{Resolve}. Then, when the
  conclusion of current goal has the form \texttt{({\ident} \dots)},
  \texttt{auto} will try to apply each constructor.

  \begin{ErrMsgs}

    \item {\ident} \errindex{is not an inductive type}

% No need to have this message here, is is generic to all commands
% referring to globals
%%    \item {\ident} \errindex{not declared}

  \end{ErrMsgs}

  \begin{Variants}

    \item {\tt Constructors \ident$_1$ \mbox{\dots} \ident$_m$}

      Adds each \texttt{Constructors} {\ident$_i$}.

  \end{Variants}

\item \texttt{Unfold} {\qualid}
\comindex{Hint Unfold}

  This adds the tactic {\tt unfold {\qualid}} to the hint list that
  will only be used when the head constant of the goal is \ident.  Its
  cost is 4.

  \begin{Variants}

    \item {\tt Unfold \ident$_1$ \mbox{\dots} \ident$_m$}

      Adds each \texttt{Unfold} {\ident$_i$}.

  \end{Variants}

\item \texttt{Transparent}, \texttt{Opaque} {\qualid}
\label{HintTransparency}
\comindex{Hint Transparent}
\comindex{Hint Opaque}

  This adds a transparency hint to the database, making {\tt {\qualid}}
  a transparent or opaque constant during resolution. This information
  is used during unification of the goal with any lemma in the database
  and inside the discrimination network to relax or constrain it in the
  case of \texttt{discriminated} databases.

  \begin{Variants}

    \item \texttt{Transparent}, \texttt{Opaque} {\ident$_1$} \mbox{\dots} {\ident$_m$}

      Declares each {\ident$_i$} as a transparent or opaque constant.

  \end{Variants}

\item \texttt{Extern \num\ [\pattern]\ => }\textsl{tactic}
\comindex{Hint Extern}

  This hint type is to extend \texttt{auto} with tactics other than
  \texttt{apply} and \texttt{unfold}. For that, we must specify a
  cost, an optional pattern and a tactic to execute. Here is an example:

\begin{quotation}
\begin{verbatim}
Hint Extern 4 (~(_ = _)) => discriminate.
\end{verbatim}
\end{quotation}

  Now, when the head of the goal is a disequality, \texttt{auto} will
  try \texttt{discriminate} if it does not manage to solve the goal
  with hints with a cost less than 4.

  One can even use some sub-patterns of the pattern in the tactic
  script. A sub-pattern is a question mark followed by an identifier, like
  \texttt{?X1} or \texttt{?X2}. Here is an example:

% Require EqDecide.
\begin{coq_example*}
Require Import List.
\end{coq_example*}
\begin{coq_example}
Hint Extern 5   ({?X1 = ?X2} + {?X1 <> ?X2}) =>
 generalize X1, X2; decide equality : eqdec.
Goal
forall a b:list (nat * nat), {a = b} + {a <> b}.
Info 1 auto with eqdec.
\end{coq_example}
\begin{coq_eval}
Abort.
\end{coq_eval}

\item \texttt{Cut} {\textit{regexp}}
\label{HintCut}
\comindex{Hint Cut}

  \textit{Warning:} these hints currently only apply to typeclass proof search and
  the \texttt{typeclasses eauto} tactic.

  This command can be used to cut the proof-search tree according to a
  regular expression matching paths to be cut. The grammar for regular
  expressions is the following:
\[\begin{array}{lcll}
  e & ::= & \ident & \text{ hint or instance identifier } \\
  & & \texttt{*} & \text{ any hint } \\
  & & e | e' & \text{ disjunction } \\
  & & e ; e' & \text{ sequence } \\
  & & ! e & \text{ Kleene star } \\
  & & \texttt{emp} & \text{ empty } \\
  & & \texttt{eps} & \text{ epsilon } \\
  & & \texttt{(} e \texttt{)} & 
\end{array}\]

The \texttt{emp} regexp does not match any search path while
\texttt{eps} matches the empty path. During proof search, the path of
successive successful hints on a search branch is recorded, as a list of
identifiers for the hints (note \texttt{Hint Extern}'s do not have an
associated identifier). Before applying any hint $\ident$ the current
path $p$ extended with $\ident$ is matched against the current cut
expression $c$ associated to the hint database. If matching succeeds,
the hint is \emph{not} applied. The semantics of \texttt{Hint Cut} $e$
is to set the cut expression to $c | e$, the initial cut expression
being \texttt{emp}.




\end{itemize}

\Rem One can use an \texttt{Extern} hint with no pattern to do
pattern-matching on hypotheses using \texttt{match goal with} inside
the tactic.

\begin{Variants}
\item {\tt Hint \hintdef}

  No database name is given: the hint is registered in the {\tt core}
    database.

\item {\tt Hint Local {\hintdef} : \ident$_1$ \mbox{\dots} \ident$_n$}

  This is used to declare hints that must not be exported to the other
  modules that require and import the current module. Inside a
  section, the option {\tt Local} is useless since hints do not
  survive anyway to the closure of sections.

\item {\tt Hint Local \hintdef}

  Idem for the {\tt core} database.

\end{Variants}

% There are shortcuts that allow to define several goal at once:

% \begin{itemize}
% \item \comindex{Hints Resolve}\texttt{Hints Resolve \ident$_1$ \dots\ \ident$_n$ : \ident.}\\
%   This command is a shortcut for the following ones:
%   \begin{quotation}
%    \noindent\texttt{Hint \ident$_1$ : \ident\ := Resolve \ident$_1$}\\
%    \dots\\
%    \texttt{Hint \ident$_1$ : \ident := Resolve \ident$_1$}
%   \end{quotation}
%   Notice that the hint name is the same that the theorem given as
%   hint.
% \item \comindex{Hints Immediate}\texttt{Hints Immediate \ident$_1$ \dots\ \ident$_n$ : \ident.}\\
% \item \comindex{Hints Unfold}\texttt{Hints Unfold \qualid$_1$ \dots\ \qualid$_n$ : \ident.}\\
% \end{itemize}

%\begin{Warnings}
%  \item \texttt{Overriding hint named \dots\ in database \dots}
%\end{Warnings}



\subsection{Hint databases defined in the \Coq\ standard library}

Several hint databases are defined in the \Coq\ standard library.  The
actual content of a database is the collection of the hints declared
to belong to this database in each of the various modules currently
loaded.  Especially, requiring new modules potentially extend a
database. At {\Coq} startup, only the {\tt core} and {\tt v62}
databases are non empty and can be used.

\begin{description}

\item[\tt core] This special database is automatically used by
  \texttt{auto}, except when pseudo-database \texttt{nocore} is
  given to \texttt{auto}. The \texttt{core} database contains
  only basic lemmas about negation,
  conjunction, and so on from. Most of the hints in this database come
  from the \texttt{Init} and \texttt{Logic} directories.

\item[\tt arith] This database contains all lemmas about Peano's
  arithmetic proved in the directories \texttt{Init} and
  \texttt{Arith}

\item[\tt zarith] contains lemmas about binary signed integers from
  the directories \texttt{theories/ZArith}. When required, the module
  {\tt Omega} also extends the database {\tt zarith} with a high-cost
  hint that calls {\tt omega} on equations and inequalities in {\tt
  nat} or {\tt Z}.

\item[\tt bool] contains lemmas about booleans, mostly from directory
  \texttt{theories/Bool}.

\item[\tt datatypes] is for lemmas about lists, streams and so on that
  are mainly proved in the \texttt{Lists} subdirectory.

\item[\tt sets] contains lemmas about sets and relations from the
  directories \texttt{Sets} and \texttt{Relations}.

\item[\tt typeclass\_instances] contains all the type class instances
  declared in the environment, including those used for \texttt{setoid\_rewrite},
  from the \texttt{Classes} directory.
\end{description}

There is also a special database called {\tt v62}. It collects all
hints that were declared in the versions of {\Coq} prior to version
6.2.4 when the databases {\tt core}, {\tt arith}, and so on were
introduced.  The purpose of the database {\tt v62} is to ensure
compatibility with further versions of {\Coq} for developments done in
versions prior to 6.2.4 ({\tt auto} being replaced by {\tt auto with v62}).
The database {\tt v62} is intended not to be extended (!). It is not
included in the hint databases list used in the {\tt auto with *} tactic.

Furthermore, you are advised not to put your own hints in the
{\tt core} database, but use one or several databases specific to your
development.

\subsection{\tt Remove Hints \term$_1$ \mbox{\dots} \term$_n$ :~ \ident$_1$
  \mbox{\dots} \ident$_m$}
\label{RemoveHints}
\comindex{Remove Hints}

This command removes the hints associated to terms \term$_1$ \mbox{\dots}
\term$_n$ in databases \ident$_1$ \mbox{\dots} \ident$_m$.

\subsection{\tt Print Hint}
\label{PrintHint}
\comindex{Print Hint}

This command displays all hints that apply to the current goal. It
fails if no proof is being edited, while the two variants can be used at
every moment.

\begin{Variants}

\item {\tt Print Hint \ident}

 This command displays only tactics associated with \ident\ in the
 hints list. This is independent of the goal being edited, so this
 command will not fail if no goal is being edited.

\item {\tt Print Hint *}

  This command displays all declared hints.

\item {\tt Print HintDb \ident}
\label{PrintHintDb}
\comindex{Print HintDb}

 This command displays all hints from database \ident.

\end{Variants}

\subsection{\tt Hint Rewrite \term$_1$ \mbox{\dots} \term$_n$ :~ \ident$_1$ \mbox{\dots} \ident$_m$}
\label{HintRewrite}
\comindex{Hint Rewrite}

This vernacular command adds the terms {\tt \term$_1$ \mbox{\dots} \term$_n$}
(their types must be equalities) in the rewriting bases \ident$_1$, \dots, \ident$_m$
with the default orientation (left to right). Notice that the
rewriting bases are distinct from the {\tt auto} hint bases and that
{\tt auto} does not take them into account.

This command is synchronous with the section mechanism (see \ref{Section}):
when closing a section, all aliases created by \texttt{Hint Rewrite} in that
section are lost. Conversely, when loading a module, all \texttt{Hint Rewrite}
declarations at the global level of that module are loaded.

\begin{Variants}
\item {\tt Hint Rewrite -> \term$_1$ \mbox{\dots} \term$_n$ :~\ident$_1$ \mbox{\dots} \ident$_m$}

This is strictly equivalent to the command above (we only make explicit the
orientation which otherwise defaults to {\tt ->}).

\item {\tt Hint Rewrite <- \term$_1$ \mbox{\dots} \term$_n$ :~\ident$_1$ \mbox{\dots} \ident$_m$}

Adds the rewriting rules {\tt \term$_1$ \mbox{\dots} \term$_n$} with a right-to-left
orientation in the bases \ident$_1$, \dots, \ident$_m$.

\item {\tt Hint Rewrite \term$_1$ \mbox{\dots} \term$_n$ using {\tac} :~\ident$_1$ \mbox{\dots} \ident$_m$}

When the rewriting rules {\tt \term$_1$ \mbox{\dots} \term$_n$} in \ident$_1$, \dots, \ident$_m$ will
be used, the tactic {\tt \tac} will be applied to the generated subgoals, the
main subgoal excluded.

%% \item
%% {\tt Hint Rewrite [ \term$_1$ \dots \term$_n$ ] in \ident}\\
%% {\tt Hint Rewrite [ \term$_1$ \dots \term$_n$ ] in {\ident} using {\tac}}\\
%% These are deprecated syntactic variants for
%% {\tt Hint Rewrite \term$_1$ \dots \term$_n$ : \ident} and
%% {\tt Hint Rewrite \term$_1$ \dots \term$_n$ using {\tac} : {\ident}}.

\item \texttt{Print Rewrite HintDb {\ident}}

  This command displays all rewrite hints contained in {\ident}.

\end{Variants}

\subsection{Hint locality
\label{Hint-Locality}}
\optindex{Loose Hint Behavior}

Hints provided by the \texttt{Hint} commands are erased when closing a
section. Conversely, all hints of a module \texttt{A} that are not
defined inside a section (and not defined with option {\tt Local}) become
available when the module {\tt A} is imported (using
e.g. \texttt{Require Import A.}).

As of today, hints only have a binary behavior regarding locality, as described
above: either they disappear at the end of a section scope, or they remain
global forever. This causes a scalability issue, because hints coming from an
unrelated part of the code may badly influence another development. It can be
mitigated to some extent thanks to the {\tt Remove Hints} command
(see ~\ref{RemoveHints}), but this is a mere workaround and has some
limitations (for instance, external hints cannot be removed).

A proper way to fix this issue is to bind the hints to their module scope, as
for most of the other objects Coq uses. Hints should only made available when
the module they are defined in is imported, not just required. It is very
difficult to change the historical behavior, as it would break a lot of scripts.
We propose a smooth transitional path by providing the {\tt Loose Hint Behavior}
option which accepts three flags allowing for a fine-grained handling of
non-imported hints.

\begin{Variants}

\item {\tt Set Loose Hint Behavior "Lax"}

  This is the default, and corresponds to the historical behavior, that is,
  hints defined outside of a section have a global scope.

\item {\tt Set Loose Hint Behavior "Warn"}

  When set, it outputs a warning when a non-imported hint is used. Note that
  this is an over-approximation, because a hint may be triggered by a run that
  will eventually fail and backtrack, resulting in the hint not being actually
  useful for the proof.

\item {\tt Set Loose Hint Behavior "Strict"}

  When set, it changes the behavior of an unloaded hint to a immediate fail
  tactic, allowing to emulate an import-scoped hint mechanism.

\end{Variants}

\subsection{Setting implicit automation tactics}

\subsubsection{\tt Proof with {\tac}}
\label{ProofWith}
\comindex{Proof with}

  This command may be used to start a proof. It defines a default
  tactic to be used each time a tactic command {\tac$_1$} is ended by
  ``\verb#...#''. In this case the tactic command typed by the user is
  equivalent to \tac$_1$;{\tac}.

\SeeAlso {\tt Proof.} in Section~\ref{BeginProof}.

\begin{Variants}

\item {\tt Proof with {\tac} using \ident$_1$ \mbox{\dots} \ident$_n$}

  Combines in a single line {\tt Proof with} and {\tt Proof using},
  see~\ref{ProofUsing}

\item {\tt Proof using \ident$_1$ \mbox{\dots} \ident$_n$ with {\tac}}

  Combines in a single line {\tt Proof with} and {\tt Proof using},
  see~\ref{ProofUsing}

\end{Variants}

\subsubsection{\tt Declare Implicit Tactic {\tac}}\label{DeclareImplicit}
\comindex{Declare Implicit Tactic}

This command declares a tactic to be used to solve implicit arguments
that {\Coq} does not know how to solve by unification. It is used
every time the term argument of a tactic has one of its holes not
fully resolved.

Here is an example:

\begin{coq_example}
Parameter quo : nat -> forall n:nat, n<>0 -> nat.
Notation "x // y" := (quo x y _) (at level 40).

Declare Implicit Tactic assumption.
Goal forall n m, m<>0 -> { q:nat & { r | q * m + r = n } }.
intros.
exists (n // m).
\end{coq_example}
\begin{coq_eval}
Clear Implicit Tactic.
Reset Initial.
\end{coq_eval}

The tactic {\tt exists (n // m)} did not fail. The hole was solved by
{\tt assumption} so that it behaved as {\tt exists (quo n m H)}.

\section{Decision procedures}

\subsection{\tt tauto}
\tacindex{tauto}
\label{tauto}

This tactic implements a decision procedure for intuitionistic propositional
calculus based on the contraction-free sequent calculi LJT* of Roy Dyckhoff
\cite{Dyc92}. Note that {\tt tauto} succeeds on any instance of an
intuitionistic tautological proposition. {\tt tauto} unfolds negations
and logical equivalence but does not unfold any other definition.

The following goal can be proved by {\tt tauto} whereas {\tt auto}
would fail:

\begin{coq_example}
Goal forall (x:nat) (P:nat -> Prop), x = 0 \/ P x -> x <> 0 -> P x.
  intros.
  tauto.
\end{coq_example}
\begin{coq_eval}
Abort.
\end{coq_eval}

Moreover, if it has nothing else to do, {\tt tauto} performs
introductions. Therefore, the use of {\tt intros} in the previous
proof is unnecessary. {\tt tauto} can for instance prove the
following:
\begin{coq_example}
(* auto would fail *)
Goal forall (A:Prop) (P:nat -> Prop),
    A \/ (forall x:nat, ~ A -> P x) -> forall x:nat, ~ A -> P x.

  tauto.
\end{coq_example}
\begin{coq_eval}
Abort.
\end{coq_eval}

\Rem In contrast, {\tt tauto} cannot solve the following goal

\begin{coq_example*}
Goal forall (A:Prop) (P:nat -> Prop),
    A \/ (forall x:nat, ~ A -> P x) -> forall x:nat, ~ ~ (A \/ P x).
\end{coq_example*}
\begin{coq_eval}
Abort.
\end{coq_eval}

because \verb=(forall x:nat, ~ A -> P x)= cannot be treated as atomic and an
instantiation of \verb=x= is necessary.

\subsection{\tt intuition \tac}
\tacindex{intuition}
\label{intuition}

The tactic \texttt{intuition} takes advantage of the search-tree built
by the decision procedure involved in the tactic {\tt tauto}. It uses
this information to generate a set of subgoals equivalent to the
original one (but simpler than it) and applies the tactic
{\tac} to them \cite{Mun94}. If this tactic fails on some goals then
{\tt intuition} fails. In fact, {\tt tauto} is simply {\tt intuition
  fail}.

For instance, the tactic {\tt intuition auto} applied to the goal
\begin{verbatim}
(forall (x:nat), P x)/\B -> (forall (y:nat),P y)/\ P O \/B/\ P O
\end{verbatim}
internally replaces it by the equivalent one:
\begin{verbatim}
(forall (x:nat), P x), B |- P O
\end{verbatim}
and then uses {\tt auto} which completes the proof.

Originally due to C{\'e}sar~Mu{\~n}oz, these tactics ({\tt tauto} and {\tt intuition})
have been completely re-engineered by David~Delahaye using mainly the tactic
language (see Chapter~\ref{TacticLanguage}). The code is now much shorter and
a significant increase in performance has been noticed. The general behavior
with respect to dependent types, unfolding and introductions has
slightly changed to get clearer semantics. This may lead to some
incompatibilities.

\begin{Variants}
\item {\tt intuition}

  Is equivalent to {\tt intuition auto with *}.
\end{Variants}

% En attente d'un moyen de valoriser les fichiers de demos
%\SeeAlso file \texttt{contrib/Rocq/DEMOS/Demo\_tauto.v}


\subsection{\tt rtauto}
\tacindex{rtauto}
\label{rtauto}

The {\tt rtauto} tactic solves propositional tautologies similarly to what {\tt tauto} does. The main difference is that the proof term is built using a reflection scheme applied to a sequent calculus proof of the goal. The search procedure is also implemented using a different technique.

Users should be aware that this difference may result in faster proof-search but  slower proof-checking, and {\tt rtauto} might not solve goals that {\tt tauto} would be able to solve (e.g. goals involving universal quantifiers).

\subsection{\tt firstorder}
\tacindex{firstorder}
\label{firstorder}

The tactic \texttt{firstorder} is an {\it experimental} extension of
\texttt{tauto} to
first-order reasoning, written by Pierre Corbineau.
It is not restricted to usual logical connectives but
instead may reason about any first-order class inductive definition.

The default tactic used by \texttt{firstorder} when no rule applies is {\tt
  auto with *}, it can be reset locally or globally using the {\nobreak
  {\tt Set Firstorder Solver {\tac}}} \optindex{Firstorder Solver}
vernacular command and printed using {\nobreak {\tt Print Firstorder
    Solver}}.

\begin{Variants}
 \item {\tt firstorder {\tac}}
   \tacindex{firstorder {\tac}}

   Tries to solve the goal with {\tac} when no logical rule may apply.

 \item {\tt firstorder with \ident$_1$ \dots\ \ident$_n$ }
   \tacindex{firstorder with}

   Adds lemmas \ident$_1$ \dots\ \ident$_n$ to the proof-search
   environment.

 \item {\tt firstorder using {\qualid}$_1$ , \dots\ , {\qualid}$_n$ }
   \tacindex{firstorder using}

   Adds lemmas in {\tt auto} hints bases {\qualid}$_1$ \dots\ {\qualid}$_n$
   to the proof-search environment. If {\qualid}$_i$ refers to an inductive
   type, it is the collection of its constructors which is added as hints.

\item \texttt{firstorder using {\qualid}$_1$ , \dots\ , {\qualid}$_n$ with \ident$_1$ \dots\ \ident$_n$}

  This combines the effects of the {\tt using} and {\tt with} options.

\end{Variants}

Proof-search is bounded by a depth parameter which can be set by typing the
{\nobreak \tt Set Firstorder Depth $n$} \optindex{Firstorder Depth}
vernacular command.


\subsection{\tt congruence}
\tacindex{congruence}
\label{congruence}

The tactic {\tt congruence}, by Pierre Corbineau, implements the standard Nelson and Oppen
congruence closure algorithm, which is a decision procedure for ground
equalities with uninterpreted symbols. It also include the constructor theory
(see \ref{injection} and \ref{discriminate}).
If the goal is a non-quantified equality, {\tt congruence} tries to
prove it with non-quantified equalities in the context. Otherwise it
tries to infer a discriminable equality from those in the context. Alternatively, congruence tries to prove that a hypothesis is equal to the goal or to the negation of another hypothesis.

{\tt congruence} is also able to take advantage of hypotheses stating quantified equalities, you have to provide a bound for the number of extra equalities generated that way. Please note that one of the members of the equality must contain all the quantified variables in order for {\tt congruence} to match against it.

\begin{coq_eval}
Reset Initial.
Variable A:Set.
Variables a b:A.
Variable f:A->A.
Variable g:A->A->A.
\end{coq_eval}

\begin{coq_example}
Theorem T:
  a=(f a) -> (g b (f a))=(f (f a)) -> (g a b)=(f (g b a)) -> (g a b)=a.
intros.
congruence.
\end{coq_example}

\begin{coq_eval}
Reset Initial.
Variable A:Set.
Variables a c d:A.
Variable f:A->A*A.
\end{coq_eval}

\begin{coq_example}
Theorem inj : f = pair a -> Some (f c) = Some (f d) -> c=d.
intros.
congruence.
\end{coq_example}

\begin{Variants}
 \item {\tt congruence {\sl n}}

  Tries to add at most {\tt \sl n} instances of hypotheses stating quantified equalities to the problem in order to solve it. A bigger value of {\tt \sl n} does not make success slower, only failure. You might consider adding some lemmas as hypotheses using {\tt assert} in order for congruence to use them.

\item {\tt congruence with \term$_1$ \dots\ \term$_n$}

  Adds {\tt \term$_1$ \dots\ \term$_n$} to the pool of terms used by
  {\tt congruence}. This helps in case you have partially applied
  constructors in your goal.
\end{Variants}

\begin{ErrMsgs}
  \item \errindex{I don't know how to handle dependent equality}

    The decision procedure managed to find a proof of the goal or of
    a discriminable equality but this proof could not be built in {\Coq}
    because of dependently-typed functions.

  \item \errindex{Goal is solvable by congruence but some arguments are missing. Try "congruence with \dots", replacing metavariables by arbitrary terms.}

    The decision procedure could solve the goal with the provision
    that additional arguments are supplied for some partially applied
    constructors. Any term of an appropriate type will allow the
    tactic to successfully solve the goal. Those additional arguments
    can be given to {\tt congruence} by filling in the holes in the
    terms given in the error message, using the {\tt with} variant
    described above.
\end{ErrMsgs}





































\section{Everything after this point has yet to be sorted}


\subsection{\tt constr\_eq \term$_1$ \term$_2$}
\tacindex{constr\_eq}
\label{constreq}

This tactic applies to any goal. It checks whether its arguments are
equal modulo alpha conversion and casts.

\ErrMsg \errindex{Not equal}

\subsection{\tt unify \term$_1$ \term$_2$}
\tacindex{unify}
\label{unify}

This tactic applies to any goal. It checks whether its arguments are
unifiable, potentially instantiating existential variables.

\ErrMsg \errindex{Not unifiable}

\begin{Variants}
\item {\tt unify \term$_1$ \term$_2$ with \ident}

  Unification takes the transparency information defined in the
  hint database {\tt \ident} into account (see Section~\ref{HintTransparency}).
\end{Variants}

\subsection{\tt is\_evar \term}
\tacindex{is\_evar}
\label{isevar}

This tactic applies to any goal. It checks whether its argument is an
existential variable. Existential variables are uninstantiated
variables generated by e.g. {\tt eapply} (see Section~\ref{apply}).

\ErrMsg \errindex{Not an evar}

\subsection{\tt has\_evar \term}
\tacindex{has\_evar}
\label{hasevar}

This tactic applies to any goal. It checks whether its argument has an
existential variable as a subterm. Unlike {\tt context} patterns
combined with {\tt is\_evar}, this tactic scans all subterms,
including those under binders.

\ErrMsg \errindex{No evars}

\subsection{\tt is\_var \term}
\tacindex{is\_var}
\label{isvar}

This tactic applies to any goal. It checks whether its argument is a
variable or hypothesis in the current goal context or in the opened sections.

\ErrMsg \errindex{Not a variable or hypothesis}

\section{Equality}

\subsection{\tt f\_equal}
\label{f-equal}
\tacindex{f\_equal}

This tactic applies to a goal of the form $f\ a_1\ \ldots\ a_n = f'\
a'_1\ \ldots\ a'_n$. Using {\tt f\_equal} on such a goal leads to
subgoals $f=f'$ and $a_1=a'_1$ and so on up to $a_n=a'_n$. Amongst
these subgoals, the simple ones (e.g. provable by
reflexivity or congruence) are automatically solved by {\tt f\_equal}.

\subsection{\tt reflexivity}
\label{reflexivity}
\tacindex{reflexivity}

This tactic applies to a goal that has the form {\tt t=u}. It checks
that {\tt t} and {\tt u} are convertible and then solves the goal.
It is equivalent to {\tt apply refl\_equal}.

\begin{ErrMsgs}
\item \errindex{The conclusion is not a substitutive equation}
\item \errindex{Impossible to unify \dots\ with \dots}
\end{ErrMsgs}

\subsection{\tt symmetry}
\tacindex{symmetry}

This tactic applies to a goal that has the form {\tt t=u} and changes it
into {\tt u=t}.

\begin{Variants}
\item {\tt symmetry in \ident} \tacindex{symmetry in}

If the statement of the hypothesis {\ident} has the form {\tt t=u},
the tactic changes it to {\tt u=t}.
\end{Variants}

\subsection{\tt transitivity \term}
\tacindex{transitivity}

This tactic applies to a goal that has the form {\tt t=u}
and transforms it into the two subgoals
{\tt t={\term}} and {\tt {\term}=u}.

\section{Equality and inductive sets}

We describe in this section some special purpose tactics dealing with
equality and inductive sets or types. These tactics use the equality
{\tt eq:forall (A:Type), A->A->Prop}, simply written with the
infix symbol {\tt =}.

\subsection{\tt decide equality}
\label{decideequality}
\tacindex{decide equality}

This tactic solves a goal of the form
{\tt forall $x$ $y$:$R$, \{$x$=$y$\}+\{\verb|~|$x$=$y$\}}, where $R$
is an inductive type such that its constructors do not take proofs or
functions as arguments, nor objects in dependent types.
It solves goals of the form {\tt \{$x$=$y$\}+\{\verb|~|$x$=$y$\}} as well.

\subsection{\tt compare \term$_1$ \term$_2$}
\tacindex{compare}

This tactic compares two given objects \term$_1$ and \term$_2$
of an inductive datatype. If $G$ is the current goal, it leaves the sub-goals
\term$_1${\tt =}\term$_2$ {\tt ->} $G$ and \verb|~|\term$_1${\tt =}\term$_2$
{\tt ->} $G$. The type
of \term$_1$ and \term$_2$ must satisfy the same restrictions as in the tactic
\texttt{decide equality}.

\subsection{\tt simplify\_eq \term}
\tacindex{simplify\_eq}
\tacindex{esimplify\_eq}
\label{simplify-eq}

Let {\term} be the proof of a statement of conclusion {\tt
  {\term$_1$}={\term$_2$}}. If {\term$_1$} and
{\term$_2$} are structurally different (in the sense described for the
tactic {\tt discriminate}), then the tactic {\tt simplify\_eq} behaves as {\tt
  discriminate {\term}}, otherwise it behaves as {\tt injection
  {\term}}.

\Rem If some quantified hypothesis of the goal is named {\ident}, then
{\tt simplify\_eq {\ident}} first introduces the hypothesis in the local
context using \texttt{intros until \ident}.

\begin{Variants}
\item \texttt{simplify\_eq} \num

  This does the same thing as \texttt{intros until \num} then
\texttt{simplify\_eq \ident} where {\ident} is the identifier for the last
introduced hypothesis.

\item \texttt{simplify\_eq} \term{} {\tt with} {\bindinglist}

  This does the same as \texttt{simplify\_eq {\term}} but using
  the given bindings to instantiate parameters or hypotheses of {\term}.

\item \texttt{esimplify\_eq} \num\\
      \texttt{esimplify\_eq} \term{} \zeroone{{\tt with} {\bindinglist}}

  This works the same as {\tt simplify\_eq} but if the type of {\term},
  or the type of the hypothesis referred to by {\num}, has uninstantiated
  parameters, these parameters are left as existential variables.

\item{\tt simplify\_eq}

If the current goal has form $t_1\verb=<>=t_2$, it behaves as
\texttt{intro {\ident}; simplify\_eq {\ident}}.
\end{Variants}

\subsection{\tt dependent rewrite -> \ident}
\tacindex{dependent rewrite ->}
\label{dependent-rewrite}

This tactic applies to any goal.  If \ident\ has type
\verb+(existT B a b)=(existT B a' b')+
in the local context (i.e. each term of the
equality has a sigma type $\{ a:A~ \&~(B~a)\}$) this tactic rewrites
\verb+a+ into \verb+a'+ and \verb+b+ into \verb+b'+ in the current
goal. This tactic works even if $B$ is also a sigma type.  This kind
of equalities between dependent pairs may be derived by the injection
and inversion tactics.

\begin{Variants}
\item{\tt dependent rewrite <- {\ident}}
\tacindex{dependent rewrite <-}

Analogous to {\tt dependent rewrite ->} but uses the equality from
right to left.
\end{Variants}

\section{Inversion
\label{inversion}}

\subsection{\tt functional inversion \ident}
\tacindex{functional inversion}
\label{sec:functional-inversion}

\texttt{functional inversion} is a tactic
that performs inversion on hypothesis {\ident} of the form
\texttt{\qualid\ \term$_1$\dots\term$_n$\ = \term} or \texttt{\term\ =
  \qualid\ \term$_1$\dots\term$_n$} where \qualid\ must have been
defined using \texttt{Function} (see Section~\ref{Function}).

\begin{ErrMsgs}
\item \errindex{Hypothesis {\ident} must contain at least one Function}

\item \errindex{Cannot find inversion information for hypothesis \ident}

  This error may be raised when  some inversion lemma failed to be
  generated by Function.
\end{ErrMsgs}

\begin{Variants}
\item {\tt functional inversion \num}

  This does the same thing as \texttt{intros until \num} then
  \texttt{functional inversion \ident} where {\ident} is the
  identifier for the last introduced hypothesis.
\item {\tt functional inversion \ident\ \qualid}\\
  {\tt functional inversion \num\ \qualid}

  If the hypothesis {\ident} (or {\num}) has a type of the form
  \texttt{\qualid$_1$\ \term$_1$\dots\term$_n$\ =\ \qualid$_2$\
    \term$_{n+1}$\dots\term$_{n+m}$} where \qualid$_1$ and \qualid$_2$
  are valid candidates to functional inversion, this variant allows
  choosing which {\qualid} is inverted.
\end{Variants}



\subsection{\tt quote \ident}
\tacindex{quote}
\index{2-level approach}

This kind of inversion has nothing to do with the tactic
\texttt{inversion} above. This tactic does \texttt{change (\ident\
  t)}, where \texttt{t} is a term built in order to ensure the
convertibility. In other words, it does inversion of the function
\ident. This function must be a fixpoint on a simple recursive
datatype: see~\ref{quote-examples} for the full details.

\begin{ErrMsgs}
\item \errindex{quote: not a simple fixpoint}

  Happens when \texttt{quote} is not able to perform inversion properly.
\end{ErrMsgs}

\begin{Variants}
\item \texttt{quote {\ident} [ \ident$_1$ \dots \ident$_n$ ]}

  All terms that are built only with \ident$_1$ \dots \ident$_n$ will be
  considered by \texttt{quote} as constants rather than variables.
\end{Variants}

% En attente d'un moyen de valoriser les fichiers de demos
% \SeeAlso file \texttt{theories/DEMOS/DemoQuote.v} in the distribution

\section{Classical tactics}
\label{ClassicalTactics}

In order to ease the proving process, when the {\tt Classical} module is loaded. A few more tactics are available. Make sure to load the module using the \texttt{Require Import} command.

\subsection{{\tt classical\_left} and \tt classical\_right}
\tacindex{classical\_left}
\tacindex{classical\_right}

The tactics \texttt{classical\_left} and \texttt{classical\_right} are the analog of the \texttt{left} and \texttt{right} but using classical logic. They can only be used for disjunctions.
Use  \texttt{classical\_left} to prove the left part of the disjunction with the assumption that the negation of right part holds.
Use \texttt{classical\_right} to prove the right part of the disjunction with the assumption that the negation of left part holds.

\section{Automatizing
\label{Automatizing}}

% EXISTE ENCORE ?
%
% \subsection{\tt Prolog [ \term$_1$ \dots\ \term$_n$ ] \num}
% \tacindex{Prolog}\label{Prolog}
% This tactic, implemented by Chet Murthy, is based upon the concept of
% existential variables of Gilles Dowek, stating that resolution is a
% kind of unification. It tries to solve the current goal using the {\tt
%   Assumption} tactic, the {\tt intro} tactic, and applying hypotheses
% of the local context and terms of the given list {\tt [ \term$_1$
%   \dots\ \term$_n$\ ]}.  It is more powerful than {\tt auto} since it
% may apply to any theorem, even those of the form {\tt (x:A)(P x) -> Q}
% where {\tt x} does not appear free in {\tt Q}.  The maximal search
% depth is {\tt \num}.

% \begin{ErrMsgs}
% \item \errindex{Prolog failed}\\
%   The Prolog tactic was not able to prove the subgoal.
% \end{ErrMsgs}


%% \subsection{{\tt jp} {\em (Jprover)}
%% \tacindex{jp}
%% \label{jprover}}

%% The tactic \texttt{jp}, due to Huang Guan-Shieng, is an experimental
%% port of the {\em Jprover}\cite{SLKN01} semi-decision procedure for
%% first-order intuitionistic logic implemented in {\em
%%   NuPRL}\cite{Kre02}.

%% The tactic \texttt{jp}, due to Huang Guan-Shieng, is an {\it
%%   experimental} port of the {\em Jprover}\cite{SLKN01} semi-decision
%% procedure for first-order intuitionistic logic implemented in {\em
%%   NuPRL}\cite{Kre02}.

%% Search may optionally be bounded by a multiplicity parameter
%% indicating how many (at most) copies of a formula may be used in
%% the proof process, its absence may lead to non-termination of the tactic.

%% %\begin{coq_eval}
%% %Variable S:Set.
%% %Variables P Q:S->Prop.
%% %Variable f:S->S.
%% %\end{coq_eval}

%% %\begin{coq_example*}
%% %Lemma example: (exists x |P x\/Q x)->(exists x |P x)\/(exists x |Q x).
%% %jp.
%% %Qed.

%% %Lemma example2: (forall x ,P x->P (f x))->forall x,P x->P (f(f x)).
%% %jp.
%% %Qed.
%% %\end{coq_example*}

%% \begin{Variants}
%%  \item {\tt jp $n$}\\
%%    \tacindex{jp $n$}
%%    Tries the {\em Jprover} procedure with multiplicities up to $n$,
%%    starting from 1.
%%  \item {\tt jp}\\
%%    Tries the {\em Jprover} procedure without multiplicity bound,
%%    possibly running forever.
%% \end{Variants}

%% \begin{ErrMsgs}
%%  \item \errindex{multiplicity limit reached}\\
%%    The procedure tried all multiplicities below the limit and
%%    failed. Goal might be solved by increasing the multiplicity limit.
%%  \item \errindex{formula is not provable}\\
%%    The procedure determined that goal was not provable in
%%    intuitionistic first-order logic, no matter how big the
%%    multiplicity is.
%% \end{ErrMsgs}


% \subsection[\tt Linear]{\tt Linear\tacindex{Linear}\label{Linear}}
% The tactic \texttt{Linear}, due to Jean-Christophe Filli{\^a}atre
% \cite{Fil94}, implements a decision procedure for {\em Direct
%   Predicate Calculus}, that is first-order Gentzen's Sequent Calculus
% without contraction rules \cite{KeWe84,BeKe92}.  Intuitively, a
% first-order goal is provable in Direct Predicate Calculus if it can be
% proved using each hypothesis at most once.

% Unlike the previous tactics, the \texttt{Linear} tactic does not belong
% to the initial state of the system, and it must be loaded explicitly
% with the command

% \begin{coq_example*}
% Require Linear.
% \end{coq_example*}

% For instance, assuming that \texttt{even} and \texttt{odd} are two
% predicates on natural numbers, and \texttt{a} of type \texttt{nat}, the
% tactic \texttt{Linear} solves the following goal

% \begin{coq_eval}
% Variables even,odd : nat -> Prop.
% Variable a:nat.
% \end{coq_eval}

% \begin{coq_example*}
% Lemma example : (even a)
%               -> ((x:nat)((even x)->(odd (S x))))
%               -> (EX y | (odd y)).
% \end{coq_example*}

% You can find examples of the use of \texttt{Linear} in
% \texttt{theories/DEMOS/DemoLinear.v}.
% \begin{coq_eval}
% Abort.
% \end{coq_eval}

% \begin{Variants}
% \item {\tt Linear with \ident$_1$ \dots\ \ident$_n$}\\
%   \tacindex{Linear with}
%   Is equivalent to apply first {\tt generalize \ident$_1$ \dots
%     \ident$_n$} (see Section~\ref{generalize}) then the \texttt{Linear}
%   tactic.  So one can use axioms, lemmas or hypotheses of the local
%   context with \texttt{Linear} in this way.
% \end{Variants}

% \begin{ErrMsgs}
% \item \errindex{Not provable in Direct Predicate Calculus}
% \item \errindex{Found $n$ classical proof(s) but no intuitionistic one}\\
%   The decision procedure looks actually for classical proofs of the
%   goals, and then checks that they are intuitionistic.  In that case,
%   classical proofs have been found, which do not correspond to
%   intuitionistic ones.
% \end{ErrMsgs}


\subsection{\tt btauto}
\tacindex{btauto}
\label{btauto}

The tactic \texttt{btauto} implements a reflexive solver for boolean tautologies. It
solves goals of the form {\tt t = u} where {\tt t} and {\tt u} are constructed
over the following grammar:

$$\mathtt{t} ::= x \mid \mathtt{true} \mid \mathtt{false}\mid \mathtt{orb\ t_1\ t_2}
\mid \mathtt{andb\ t_1\ t_2} \mid\mathtt{xorb\ t_1\ t_2} \mid\mathtt{negb\ t}
\mid\mathtt{if\ t_1\ then\ t_2\ else\ t_3}
$$

Whenever the formula supplied is not a tautology, it also provides a counter-example.

Internally, it uses a system very similar to the one of the {\tt ring} tactic.

\subsection{\tt omega}
\tacindex{omega}
\label{omega}

The tactic \texttt{omega}, due to Pierre Cr{\'e}gut,
is an automatic decision procedure for Presburger
arithmetic. It solves quantifier-free
formulas built with \verb|~|, \verb|\/|, \verb|/\|,
\verb|->| on top of equalities, inequalities and disequalities on
both the type \texttt{nat} of natural numbers and \texttt{Z} of binary
integers. This tactic must be loaded by the command \texttt{Require Import
  Omega}. See the additional documentation about \texttt{omega}
(see Chapter~\ref{OmegaChapter}).

\subsection{{\tt ring} and \tt ring\_simplify \term$_1$ \mbox{\dots} \term$_n$}
\tacindex{ring}
\tacindex{ring\_simplify}
\comindex{Add Ring}
\comindex{Print Rings}

The {\tt ring} tactic solves equations upon polynomial expressions of
a ring (or semi-ring) structure. It proceeds by normalizing both hand
sides of the equation (w.r.t. associativity, commutativity and
distributivity, constant propagation) and comparing syntactically the
results.

{\tt ring\_simplify} applies the normalization procedure described
above to the terms given. The tactic then replaces all occurrences of
the terms given in the conclusion of the goal by their normal
forms. If no term is given, then the conclusion should be an equation
and both hand sides are normalized.

See Chapter~\ref{ring} for more information on the tactic and how to
declare new ring structures.  All declared field structures can be
printed with the {\tt Print Rings} command.

\subsection{{\tt field}, {\tt field\_simplify \term$_1$ \mbox{\dots}
  \term$_n$}, and \tt field\_simplify\_eq}
\tacindex{field}
\tacindex{field\_simplify}
\tacindex{field\_simplify\_eq}
\comindex{Add Field}
\comindex{Print Fields}

The {\tt field} tactic is built on the same ideas as {\tt ring}: this
is a reflexive tactic that solves or simplifies equations in a field
structure. The main idea is to reduce a field expression (which is an
extension of ring expressions with the inverse and division
operations) to a fraction made of two polynomial expressions.

Tactic {\tt field} is used to solve subgoals, whereas {\tt
  field\_simplify \term$_1$\dots\term$_n$} replaces the provided terms
by their reduced fraction. {\tt field\_simplify\_eq} applies when the
conclusion is an equation: it simplifies both hand sides and multiplies
so as to cancel denominators. So it produces an equation without
division nor inverse.

All of these 3 tactics may generate a subgoal in order to prove that
denominators are different from zero.

See Chapter~\ref{ring} for more information on the tactic and how to
declare new field structures.  All declared field structures can be
printed with the {\tt Print Fields} command.

\Example
\begin{coq_example*}
Require Import Reals.
Goal forall x y:R,
    (x * y > 0)%R ->
    (x * (1 / x + x / (x + y)))%R =
    ((- 1 / y) * y * (- x * (x / (x + y)) - 1))%R.
\end{coq_example*}

\begin{coq_example}
intros; field.
\end{coq_example}

\begin{coq_eval}
Reset Initial.
\end{coq_eval}

\SeeAlso file {\tt plugins/setoid\_ring/RealField.v} for an example of instantiation,\\
\phantom{\SeeAlso}theory {\tt theories/Reals} for many examples of use of {\tt
field}.

\subsection{\tt fourier}
\tacindex{fourier}

This tactic written by Lo{\"\i}c Pottier solves linear inequalities on
real numbers using Fourier's method~\cite{Fourier}. This tactic must
be loaded by {\tt Require Import Fourier}.

\Example
\begin{coq_example*}
Require Import Reals.
Require Import Fourier.
Goal forall x y:R, (x < y)%R -> (y + 1 >= x - 1)%R.
\end{coq_example*}

\begin{coq_example}
intros; fourier.
\end{coq_example}

\begin{coq_eval}
Reset Initial.
\end{coq_eval}

\section{Non-logical tactics}

\subsection[\tt cycle \num]{\tt cycle \num\tacindex{cycle}}

This tactic puts the {\num} first goals at the end of the list of
goals. If {\num} is negative, it will put the last $\left|\num\right|$ goals at
the beginning of the list.

\Example
\begin{coq_example*}
Parameter P : nat -> Prop.
Goal P 1 /\ P 2 /\ P 3 /\ P 4 /\ P 5.
\end{coq_example*}
\begin{coq_example}
repeat split.
all: cycle 2.
all: cycle -3.
\end{coq_example}

\begin{coq_eval}
Reset Initial.
\end{coq_eval}

\subsection[\tt swap \num$_1$ \num$_2$]{\tt swap \num$_1$ \num$_2$\tacindex{swap}}

This tactic switches the position of the goals of indices $\num_1$ and $\num_2$. If either $\num_1$ or $\num_2$ is negative then goals are counted from the end of the focused goal list. Goals are indexed from $1$, there is no goal with position $0$. 

\Example
\begin{coq_example*}
Parameter P : nat -> Prop.
Goal P 1 /\ P 2 /\ P 3 /\ P 4 /\ P 5.
\end{coq_example*}
\begin{coq_example}
repeat split.
all: swap 1 3.
all: swap 1 -1.
\end{coq_example}

\begin{coq_eval}
Reset Initial.
\end{coq_eval}

\subsection[\tt revgoals]{\tt revgoals\tacindex{revgoals}}

This tactics reverses the list of the focused goals.

\Example
\begin{coq_example*}
Parameter P : nat -> Prop.
Goal P 1 /\ P 2 /\ P 3 /\ P 4 /\ P 5.
\end{coq_example*}
\begin{coq_example}
repeat split.
all: revgoals.
\end{coq_example}

\begin{coq_eval}
Reset Initial.
\end{coq_eval}



\subsection[\tt shelve]{\tt shelve\tacindex{shelve}\label{shelve}}

This tactic moves all goals under focus to a shelf. While on the shelf, goals
will not be focused on. They can be solved by unification, or they can be called
back into focus with the command {\tt Unshelve} (Section~\ref{unshelve}).

\begin{Variants}
  \item \texttt{shelve\_unifiable}\tacindex{shelve\_unifiable}

    Shelves only the goals under focus that are mentioned in other goals.
    Goals that appear in the type of other goals can be solved by unification.

\Example
\begin{coq_example}
Goal exists n, n=0.
refine (ex_intro _ _ _).
all:shelve_unifiable.
reflexivity.
\end{coq_example}

\begin{coq_eval}
Reset Initial.
\end{coq_eval}

\end{Variants}

\subsection[\tt Unshelve]{\tt Unshelve\comindex{Unshelve}\label{unshelve}}

This command moves all the goals on the shelf (see Section~\ref{shelve}) from the
shelf into focus, by appending them to the end of the current list of focused goals.

\subsection[\tt give\_up]{\tt give\_up\tacindex{give\_up}}

This tactic removes the focused goals from the proof. They are not solved, and cannot
be solved later in the proof. As the goals are not solved, the proof cannot be closed.

The {\tt give\_up} tactic can be used while editing a proof, to choose to write the
proof script in a non-sequential order.

\section{Simple tactic macros}
\index{Tactic macros}
\label{TacticDefinition}

A simple example has more value than a long explanation:

\begin{coq_example}
Ltac Solve := simpl; intros; auto.
Ltac ElimBoolRewrite b H1 H2 :=
  elim b; [ intros; rewrite H1; eauto | intros; rewrite H2; eauto ].
\end{coq_example}

The tactics macros are synchronous with the \Coq\ section mechanism:
a tactic definition is deleted from the current environment
when you close the section (see also \ref{Section})
where it was defined. If you want that a
tactic macro defined in a module is usable in the modules that
require it, you should put it outside of any section.

Chapter~\ref{TacticLanguage} gives examples of more complex
user-defined tactics.


%%% Local Variables:
%%% mode: latex
%%% TeX-master: "Reference-Manual"
%%% TeX-master: "Reference-Manual"
%%% End:
