%\documentstyle[11pt,../tools/coq-tex/coq,fullpage]{article}

%\pagestyle{plain}

%\begin{document}
%\nocite{Augustsson85,wadler87,HuetLevy79,MaSi94,maranget94,Laville91,saidi94,dowek93,Leroy90,puel-suarez90}

%%%%%%%%%%%%%%%%%%%%%%%%%%%%%%%%%
% File title.tex
% Page formatting commands
% Macro \coverpage
%%%%%%%%%%%%%%%%%%%%%%%%%%%%%%%%

%\setlength{\marginparwidth}{0pt}
%\setlength{\oddsidemargin}{0pt}
%\setlength{\evensidemargin}{0pt}
%\setlength{\marginparsep}{0pt}
%\setlength{\topmargin}{0pt}
%\setlength{\textwidth}{16.9cm}
%\setlength{\textheight}{22cm}
%\usepackage{fullpage}

%\newcommand{\printingdate}{\today}
%\newcommand{\isdraft}{\Large\bf\today\\[20pt]}
%\newcommand{\isdraft}{\vspace{20pt}}

\newcommand{\coverpage}[3]{
\thispagestyle{empty}
\begin{center}
\bfseries % for the rest of this page, until \end{center}
\Huge
The Coq Proof Assistant\\[12pt]
#1\\[20pt]
\Large\today\\[20pt]
Version \coqversion%\footnote[1]{This research was partly supported by IST working group ``Types''}

\vspace{0pt plus .5fill}
#2
\par\vfill
The Coq Development Team

\vspace*{15pt}
\end{center}
\newpage

\thispagestyle{empty}
\hbox{}\vfill % without \hbox \vfill does not work at the top of the page
\begin{flushleft}
%BEGIN LATEX
V\coqversion, \today
\par\vspace{20pt}
%END LATEX
\copyright INRIA 1999-2004 ({\Coq} versions 7.x)

\copyright INRIA 2004-2010 ({\Coq} versions 8.x)

#3
\end{flushleft}
} % end of \coverpage definition


% \newcommand{\shorttitle}[1]{
% \begin{center}
% \begin{huge}
% \begin{bf}
% The Coq Proof Assistant\\
% \vspace{10pt}
%     #1\\
% \end{bf}
% \end{huge}
% \end{center}
% \vspace{5pt}
% }

% Local Variables: 
% mode: LaTeX
% TeX-master: ""
% End: 

% $Id$ 

%%%%%%%%%%%%%%%%%%%%%%%%%%%%%%%%%%%%%%%%%%%
% MACROS FOR THE REFERENCE MANUAL OF COQ %
%%%%%%%%%%%%%%%%%%%%%%%%%%%%%%%%%%%%%%%%%%

\newcommand{\coqversion}{7.3}

% For commentaries (define \com as {} for the release manual)
%\newcommand{\com}[1]{{\it(* #1 *)}}
\newcommand{\com}[1]{}

%%%%%%%%%%%%%%%%%%%%%%%
% Formatting commands %
%%%%%%%%%%%%%%%%%%%%%%%

\newcommand{\ErrMsg}{\medskip \noindent {\bf Error message: }}
\newcommand{\ErrMsgx}{\medskip \noindent {\bf Error messages: }}
\newcommand{\variant}{\medskip \noindent {\bf Variant: }}
\newcommand{\variants}{\medskip \noindent {\bf Variants: }}
\newcommand{\SeeAlso}{\medskip \noindent {\bf See also: }}
\newcommand{\Rem}{\medskip \noindent {\bf Remark: }}
\newcommand{\Rems}{\medskip \noindent {\bf Remarks: }}
\newcommand{\Example}{\medskip \noindent {\bf Example: }}
\newcommand{\Warning}{\medskip \noindent {\bf Warning: }}
\newcommand{\Warns}{\medskip \noindent {\bf Warnings: }}
\newcounter{ex}
\newcommand{\firstexample}{\setcounter{ex}{1}}
\newcommand{\example}[1]{
\medskip \noindent \textbf{Example \arabic{ex}: }\textit{#1}
\addtocounter{ex}{1}}

\newenvironment{Variant}{\variant\begin{enumerate}}{\end{enumerate}}
\newenvironment{Variants}{\variants\begin{enumerate}}{\end{enumerate}}
\newenvironment{ErrMsgs}{\ErrMsgx\begin{enumerate}}{\end{enumerate}}
\newenvironment{Remarks}{\Rems\begin{enumerate}}{\end{enumerate}}
\newenvironment{Warnings}{\Warns\begin{enumerate}}{\end{enumerate}}
\newenvironment{Examples}{\medskip\noindent{\bf Examples:}
\begin{enumerate}}{\end{enumerate}}

\newcommand{\rr}{\raggedright}

\newcommand{\tinyskip}{\rule{0mm}{4mm}}

\newcommand{\bd}{\noindent\bf}
\newcommand{\sbd}{\vspace{8pt}\noindent\bf}
\newcommand{\sdoll}[1]{\begin{small}$ #1~ $\end{small}}
\newcommand{\sdollnb}[1]{\begin{small}$ #1 $\end{small}}
\newcommand{\kw}[1]{\textsf{#1}}
\newcommand{\spec}[1]{\{\,#1\,\}}

% Building regular expressions
\newcommand{\zeroone}[1]{{\sl [}#1{\sl ]}}
%\newcommand{\zeroonemany}[1]{$\{$#1$\}$*}
%\newcommand{\onemany}[1]{$\{$#1$\}$+}
\newcommand{\nelist}[2]{{#1} {\tt #2} {\ldots} {\tt #2} {#1}}
\newcommand{\sequence}[2]{{\sl [}{#1} {\tt #2} {\ldots} {\tt #2} {#1}{\sl ]}}
\newcommand{\nelistwithoutblank}[2]{#1{\tt #2}\ldots{\tt #2}#1}
\newcommand{\sequencewithoutblank}[2]{$[$#1{\tt #2}\ldots{\tt #2}#1$]$}

% Used for RefMan-gal
\newcommand{\ml}[1]{\hbox{\tt{#1}}}
\newcommand{\op}{\,|\,}

%%%%%%%%%%%%%%%%%%%%%%%%
% Trademarks and so on %
%%%%%%%%%%%%%%%%%%%%%%%%

\newcommand{\Coq}{{\sf Coq}}
\newcommand{\ocaml}{{\sf Objective Caml}}
\newcommand{\camlpppp}{{\sf Camlp4}}
\newcommand{\emacs}{{\sf GNU Emacs}}
\newcommand{\gallina}{\textsf{Gallina}}
\newcommand{\CIC}{\mbox{\sc Cic}}
\newcommand{\FW}{\mbox{$F_{\omega}$}}
\newcommand{\bn}{{\sf BNF}}

%%%%%%%%%%%%%%%%%%%
% Name of tactics %
%%%%%%%%%%%%%%%%%%%

\newcommand{\Natural}{\mbox{\tt Natural}}

%%%%%%%%%%%%%%%%%
% \rm\sl series %
%%%%%%%%%%%%%%%%%

\newcommand{\Fwterm}{\textrm{\textsl{Fwterm}}}
\newcommand{\Index}{\textrm{\textsl{index}}}
\newcommand{\abbrev}{\textrm{\textsl{abbreviation}}}
\newcommand{\annotation}{\textrm{\textsl{annotation}}}
\newcommand{\atomictac}{\textrm{\textsl{atomic\_tactic}}}
\newcommand{\binders}{\textrm{\textsl{bindings}}}
\newcommand{\binder}{\textrm{\textsl{binding}}}
\newcommand{\bindinglist}{\textrm{\textsl{bindings\_list}}}
\newcommand{\cast}{\textrm{\textsl{cast}}}
\newcommand{\cofixpointbody}{\textrm{\textsl{cofix\_body}}}
\newcommand{\coinductivebody}{\textrm{\textsl{coind\_body}}}
\newcommand{\commandtac}{\textrm{\textsl{tactic\_invocation}}}
\newcommand{\constructor}{\textrm{\textsl{constructor}}}
\newcommand{\convtactic}{\textrm{\textsl{conv\_tactic}}}
\newcommand{\declarationkeyword}{\textrm{\textsl{declaration\_keyword}}}
\newcommand{\declaration}{\textrm{\textsl{declaration}}}
\newcommand{\definition}{\textrm{\textsl{definition}}}
\newcommand{\digit}{\textrm{\textsl{digit}}}
\newcommand{\eqn}{\textrm{\textsl{equation}}}
\newcommand{\exteqn}{\textrm{\textsl{ext\_eqn}}}
\newcommand{\field}{\textrm{\textsl{field}}}
\newcommand{\firstletter}{\textrm{\textsl{first\_letter}}}
\newcommand{\fixpg}{\textrm{\textsl{fix\_pgm}}}
\newcommand{\fixpointbody}{\textrm{\textsl{fix\_body}}}
\newcommand{\fixpoint}{\textrm{\textsl{fixpoint}}}
\newcommand{\flag}{\textrm{\textsl{flag}}}
\newcommand{\form}{\textrm{\textsl{form}}}
\newcommand{\gensymbol}{\textrm{\textsl{symbol}}}
\newcommand{\localassums}{\textrm{\textsl{local\_assums}}}
\newcommand{\localdef}{\textrm{\textsl{local\_def}}}
\newcommand{\localdecls}{\textrm{\textsl{local\_decls}}}
\newcommand{\ident}{\textrm{\textsl{ident}}}
\newcommand{\accessident}{\textrm{\textsl{access\_ident}}}
\newcommand{\inductivebody}{\textrm{\textsl{ind\_body}}}
\newcommand{\inductive}{\textrm{\textsl{inductive}}}
\newcommand{\naturalnumber}{\textrm{\textsl{natural}}}
\newcommand{\integer}{\textrm{\textsl{integer}}}
\newcommand{\multpattern}{\textrm{\textsl{mult\_pattern}}}
\newcommand{\mutualcoinductive}{\textrm{\textsl{mutual\_coinductive}}}
\newcommand{\mutualinductive}{\textrm{\textsl{mutual\_inductive}}}
\newcommand{\nestedpattern}{\textrm{\textsl{nested\_pattern}}}
\newcommand{\num}{\textrm{\textsl{num}}}
\newcommand{\params}{\textrm{\textsl{params}}}
\newcommand{\pattern}{\textrm{\textsl{pattern}}}
\newcommand{\pat}{\textrm{\textsl{pat}}}
\newcommand{\pgs}{\textrm{\textsl{pgms}}}
\newcommand{\pg}{\textrm{\textsl{pgm}}}
\newcommand{\proof}{\textrm{\textsl{proof}}}
\newcommand{\record}{\textrm{\textsl{record}}}
\newcommand{\rewrule}{\textrm{\textsl{rewriting\_rule}}}
\newcommand{\sentence}{\textrm{\textsl{sentence}}}
\newcommand{\simplepattern}{\textrm{\textsl{simple\_pattern}}}
\newcommand{\sort}{\textrm{\textsl{sort}}}
\newcommand{\specif}{\textrm{\textsl{specif}}}
\newcommand{\statement}{\textrm{\textsl{statement}}}
\newcommand{\str}{\textrm{\textsl{string}}}
\newcommand{\subsequentletter}{\textrm{\textsl{subsequent\_letter}}}
\newcommand{\switch}{\textrm{\textsl{switch}}}
\newcommand{\tac}{\textrm{\textsl{tactic}}}
\newcommand{\terms}{\textrm{\textsl{terms}}}
\newcommand{\term}{\textrm{\textsl{term}}}
\newcommand{\module}{\textrm{\textsl{module}}}
\newcommand{\modexpr}{\textrm{\textsl{module\_expression}}}
\newcommand{\modtype}{\textrm{\textsl{module\_type}}}
\newcommand{\onemodbinding}{\textrm{\textsl{module\_binding}}}
\newcommand{\modbindings}{\textrm{\textsl{module\_bindings}}}
\newcommand{\qualid}{\textrm{\textsl{qualid}}}
\newcommand{\class}{\textrm{\textsl{class}}}
\newcommand{\dirpath}{\textrm{\textsl{dirpath}}}
\newcommand{\typedidents}{\textrm{\textsl{typed\_idents}}}
\newcommand{\type}{\textrm{\textsl{type}}}
\newcommand{\vref}{\textrm{\textsl{ref}}}
\newcommand{\zarithformula}{\textrm{\textsl{zarith\_formula}}}
\newcommand{\zarith}{\textrm{\textsl{zarith}}}

%%%%%%%%%%%%%%%%%%%%%%%%%%%%%%%%%%%%%%%%%%%%%%%%%%%%%%%
% \mbox{\sf } series for roman text in maths formulas %
%%%%%%%%%%%%%%%%%%%%%%%%%%%%%%%%%%%%%%%%%%%%%%%%%%%%%%%

\newcommand{\alors}{\mbox{\textsf{then}}}
\newcommand{\alter}{\mbox{\textsf{alter}}}
\newcommand{\bool}{\mbox{\textsf{bool}}}
\newcommand{\conc}{\mbox{\textsf{conc}}}
\newcommand{\cons}{\mbox{\textsf{cons}}}
\newcommand{\consf}{\mbox{\textsf{consf}}}
\newcommand{\emptyf}{\mbox{\textsf{emptyf}}}
\newcommand{\EqSt}{\mbox{\textsf{EqSt}}}
\newcommand{\false}{\mbox{\textsf{false}}}
\newcommand{\filter}{\mbox{\textsf{filter}}}
\newcommand{\forest}{\mbox{\textsf{forest}}}
\newcommand{\from}{\mbox{\textsf{from}}}
\newcommand{\hd}{\mbox{\textsf{hd}}}
\newcommand{\Length}{\mbox{\textsf{Length}}}
\newcommand{\length}{\mbox{\textsf{length}}}
\newcommand{\LengthA}{\mbox {\textsf{Length\_A}}}
\newcommand{\List}{\mbox{\textsf{List}}}
\newcommand{\ListA}{\mbox{\textsf{List\_A}}}
\newcommand{\LNil}{\mbox{\textsf{Lnil}}}
\newcommand{\LCons}{\mbox{\textsf{Lcons}}}
\newcommand{\nat}{\mbox{\textsf{nat}}}
\newcommand{\nO}{\mbox{\textsf{O}}}
\newcommand{\nS}{\mbox{\textsf{S}}}
\newcommand{\node}{\mbox{\textsf{node}}}
\newcommand{\Nil}{\mbox{\textsf{nil}}}
\newcommand{\Prop}{\mbox{\textsf{Prop}}}
\newcommand{\Set}{\mbox{\textsf{Set}}}
\newcommand{\si}{\mbox{\textsf{if}}}
\newcommand{\sinon}{\mbox{\textsf{else}}}
\newcommand{\Str}{\mbox{\textsf{Stream}}}
\newcommand{\tl}{\mbox{\textsf{tl}}}
\newcommand{\tree}{\mbox{\textsf{tree}}}
\newcommand{\true}{\mbox{\textsf{true}}}
\newcommand{\Type}{\mbox{\textsf{Type}}}
\newcommand{\unfold}{\mbox{\textsf{unfold}}}
\newcommand{\zeros}{\mbox{\textsf{zeros}}}

%%%%%%%%%
% Misc. %
%%%%%%%%%
\newcommand{\T}{\texttt{T}}
\newcommand{\U}{\texttt{U}}
\newcommand{\real}{\textsf{Real}}
\newcommand{\Spec}{\textit{Spec}}
\newcommand{\Data}{\textit{Data}}
\newcommand{\In} {{\textbf{in }}}
\newcommand{\AND} {{\textbf{and}}}
\newcommand{\If}{{\textbf{if }}}
\newcommand{\Else}{{\textbf{else }}}
\newcommand{\Then} {{\textbf{then }}}
\newcommand{\Let}{{\textbf{let }}}
\newcommand{\Where}{{\textbf{where rec }}}
\newcommand{\Function}{{\textbf{function }}}
\newcommand{\Rec}{{\textbf{rec }}}
\newcommand{\cn}{\centering}

%%%%%%%%%%%%%%%%%%%%%%%%%%%%%
% Math commands and symbols %
%%%%%%%%%%%%%%%%%%%%%%%%%%%%%

\newcommand{\la}{\leftarrow}
\newcommand{\ra}{\rightarrow}
\newcommand{\Ra}{\Rightarrow}
\newcommand{\rt}{\Rightarrow}
\newcommand{\lla}{\longleftarrow}
\newcommand{\lra}{\longrightarrow}
\newcommand{\Llra}{\Longleftrightarrow}
\newcommand{\mt}{\mapsto}
\newcommand{\ov}{\overrightarrow}
\newcommand{\wh}{\widehat}
\newcommand{\up}{\uparrow}
\newcommand{\dw}{\downarrow}
\newcommand{\nr}{\nearrow}
\newcommand{\se}{\searrow}
\newcommand{\sw}{\swarrow}
\newcommand{\nw}{\nwarrow}

\newcommand{\vm}[1]{\vspace{#1em}}
\newcommand{\vx}[1]{\vspace{#1ex}}
\newcommand{\hm}[1]{\hspace{#1em}}
\newcommand{\hx}[1]{\hspace{#1ex}}
\newcommand{\sm}{\mbox{ }}
\newcommand{\mx}{\mbox}

\newcommand{\nq}{\neq}
\newcommand{\eq}{\equiv}
\newcommand{\fa}{\forall}
\newcommand{\ex}{\exists}
\newcommand{\impl}{\rightarrow}
\newcommand{\Or}{\vee}
\newcommand{\And}{\wedge}
\newcommand{\ms}{\models}
\newcommand{\bw}{\bigwedge}
\newcommand{\ts}{\times}
\newcommand{\cc}{\circ}
\newcommand{\es}{\emptyset}
\newcommand{\bs}{\backslash}
\newcommand{\vd}{\vdash}
\newcommand{\lan}{{\langle }}
\newcommand{\ran}{{\rangle }}

\newcommand{\al}{\alpha}
\newcommand{\bt}{\beta}
\newcommand{\io}{\iota}
\newcommand{\lb}{\lambda}
\newcommand{\sg}{\sigma}
\newcommand{\sa}{\Sigma}
\newcommand{\om}{\Omega}
\newcommand{\tu}{\tau}

%%%%%%%%%%%%%%%%%%%%%%%%%
% Custom maths commands %
%%%%%%%%%%%%%%%%%%%%%%%%%

\newcommand{\sumbool}[2]{\{#1\}+\{#2\}}
\newcommand{\myifthenelse}[3]{\kw{if} ~ #1 ~\kw{then} ~ #2 ~ \kw{else} ~ #3}
\newcommand{\fun}[2]{\item[]{\tt {#1}}. \quad\\ #2}
\newcommand{\WF}[2]{\ensuremath{{\cal W\!F}(#1)[#2]}}
\newcommand{\WFE}[1]{\WF{E}{#1}}
\newcommand{\WT}[4]{\ensuremath{#1[#2] \vdash #3 : #4}}
\newcommand{\WTE}[3]{\WT{E}{#1}{#2}{#3}}
\newcommand{\WTEG}[2]{\WTE{\Gamma}{#1}{#2}}

\newcommand{\WTM}[3]{\WT{#1}{}{#2}{#3}}
\newcommand{\WFT}[2]{\ensuremath{#1[] \vdash {\cal W\!F}(#2)}}
\newcommand{\WS}[3]{\ensuremath{#1[] \vdash #2 <: #3}}
\newcommand{\WSE}[2]{\WS{E}{#1}{#2}}

\newcommand{\WTRED}[5]{\mbox{$#1[#2] \vdash #3 #4 #5$}}
\newcommand{\WTERED}[4]{\mbox{$E[#1] \vdash #2 #3 #4$}}
\newcommand{\WTELECONV}[3]{\WTERED{#1}{#2}{\leconvert}{#3}}
\newcommand{\WTEGRED}[3]{\WTERED{\Gamma}{#1}{#2}{#3}}
\newcommand{\WTECONV}[3]{\WTERED{#1}{#2}{\convert}{#3}}
\newcommand{\WTEGCONV}[2]{\WTERED{\Gamma}{#1}{\convert}{#2}}
\newcommand{\WTEGLECONV}[2]{\WTERED{\Gamma}{#1}{\leconvert}{#2}}

\newcommand{\lab}[1]{\mathit{labels}(#1)}
\newcommand{\dom}[1]{\mathit{dom}(#1)}

\newcommand{\CI}[2]{\mbox{$\{#1\}^{#2}$}}
\newcommand{\CIP}[3]{\mbox{$\{#1\}_{#2}^{#3}$}}
\newcommand{\CIPV}[1]{\CIP{#1}{I_1.. I_k}{P_1.. P_k}}
\newcommand{\CIPI}[1]{\CIP{#1}{I}{P}}
\newcommand{\CIF}[1]{\mbox{$\{#1\}_{f_1.. f_n}$}}
\newcommand{\NInd}[3]{\mbox{{\sf Ind}$(#1)(\begin{array}[t]{l}#2:=#3
                                              \,)\end{array}$}}
\newcommand{\Ind}[4]{\mbox{{\sf Ind}$(#1)[#2](\begin{array}[t]{l}#3:=#4
                                                 \,)\end{array}$}}
\newcommand{\Indp}[5]{\mbox{{\sf Ind}$_{#5}(#1)[#2](\begin{array}[t]{l}#3:=#4
                                                 \,)\end{array}$}}
\newcommand{\Def}[4]{\mbox{{\sf Def}$(#1)(#2:=#3:#4)$}}
\newcommand{\Assum}[3]{\mbox{{\sf Assum}$(#1)(#2:#3)$}}
\newcommand{\Match}[3]{\mbox{$<\!#1\!>\!{\mbox{\tt Match}}~#2~{\mbox{\tt with}}~#3~{\mbox{\tt end}}$}}
\newcommand{\Case}[3]{\mbox{$<\!#1\!>\!{\mbox{\tt Cases}}~#2~{\mbox{\tt of}}~#3~{\mbox{\tt end}}$}}
\newcommand{\Fix}[2]{\mbox{\tt Fix}~#1\{#2\}}
\newcommand{\CoFix}[2]{\mbox{\tt CoFix}~#1\{#2\}}
\newcommand{\With}[2]{\mbox{\tt ~with~}}
\newcommand{\subst}[3]{#1\{#2/#3\}}
\newcommand{\substs}[4]{#1\{(#2/#3)_{#4}\}}
\newcommand{\Sort}{\mbox{$\cal S$}}
\newcommand{\convert}{=_{\beta\delta\iota\zeta}}
\newcommand{\leconvert}{\leq_{\beta\delta\iota\zeta}}
\newcommand{\NN}{\mbox{I\hspace{-7pt}N}}
\newcommand{\inference}[1]{$${#1}$$}
\newcommand{\compat}[2]{\mbox{$[#1|#2]$}}
\newcommand{\tristackrel}[3]{\mathrel{\mathop{#2}\limits_{#3}^{#1}}}

\newcommand{\Impl}{{\it Impl}}
\newcommand{\Mod}[3]{{\sf Mod}({#1}:{#2}:={#3})}
\newcommand{\ModType}[2]{{\sf ModType}({#1}:={#2})}
\newcommand{\ModS}[2]{{\sf ModS}({#1}:{#2})}
\newcommand{\ModSEq}[3]{{\sf ModSEq}({#1}:{#2}=={#3})}
\newcommand{\functor}[3]{\ensuremath{{\sf Functor}[#1:#2]\;#3}}
\newcommand{\funsig}[3]{\ensuremath{{\sf Funsig}(#1:#2)\;#3}}
\newcommand{\sig}[1]{\ensuremath{{\sf Sig}~#1~{\sf End}}}
\newcommand{\struct}[1]{\ensuremath{{\sf Struct}~#1~{\sf End}}}


\newbox\tempa
\newbox\tempb
\newdimen\tempc
\newcommand{\mud}[1]{\hfil $\displaystyle{\mathstrut #1}$\hfil}
\newcommand{\rig}[1]{\hfil $\displaystyle{#1}$}
\newcommand{\irulehelp}[3]{\setbox\tempa=\hbox{$\displaystyle{\mathstrut #2}$}%
                        \setbox\tempb=\vbox{\halign{##\cr
        \mud{#1}\cr
        \noalign{\vskip\the\lineskip}
        \noalign{\hrule height 0pt}
        \rig{\vbox to 0pt{\vss\hbox to 0pt{${\; #3}$\hss}\vss}}\cr
        \noalign{\hrule}
        \noalign{\vskip\the\lineskip}
        \mud{\copy\tempa}\cr}}
                      \tempc=\wd\tempb
                      \advance\tempc by \wd\tempa
                      \divide\tempc by 2 }
\newcommand{\irule}[3]{{\irulehelp{#1}{#2}{#3}
                     \hbox to \wd\tempa{\hss \box\tempb \hss}}}

\newcommand{\sverb}[1]{\tt #1}
\newcommand{\mover}[2]{{#1\over #2}}
\newcommand{\jd}[2]{#1 \vdash #2}
\newcommand{\mathline}[1]{\[#1\]}
\newcommand{\zrule}[2]{#2: #1}
\newcommand{\orule}[3]{#3: {\mover{#1}{#2}}}
\newcommand{\trule}[4]{#4: \mover{#1  \qquad #2} {#3}}
\newcommand{\thrule}[5]{#5: {\mover{#1  \qquad #2 \qquad #3}{#4}}}


% $Id$ 


%%% Local Variables: 
%%% mode: latex
%%% TeX-master: "Reference-Manual"
%%% End: 

%\coverpage{The Macro \verb+Cases+}{Cristina Cornes}
%\pagestyle{plain}
\chapter{The Macro {\tt Cases}}\label{Cases}\index{Cases@{\tt Cases}}

\marginparwidth 0pt \oddsidemargin 0pt \evensidemargin 0pt \marginparsep 0pt
\topmargin 0pt \textwidth 6.5in \textheight 8.5in


\verb+Cases+ is an extension to the concrete syntax of Coq that allows
to write case expressions using patterns in a syntax close to that of ML languages.  
This construction is just a  macro that is
expanded during parsing into a sequence of the primitive construction 
 \verb+Case+.
The current implementation contains two strategies, one for compiling
non-dependent case and another one for dependent case.
\section{Patterns}\label{implementation}
A pattern is a term that indicates the {\em shape} of a value, i.e. a
term where the variables can be seen as holes. When a value is 
matched against a pattern (this is called {\em pattern matching}) 
the pattern behaves as a filter, and associates a sub-term  of the value
to each hole (i.e. to each variable pattern). 
 

The syntax of patterns is presented in  figure \ref{grammar}\footnote{
Notation:  \{$P$\}$^*$ denotes zero or more repetitions of $P$ and 
 \{$P$\}$^+$ denotes  one or more repetitions of $P$. {\sl command} is the
non-terminal corresponding to terms in Coq.}.
Patterns are built up from constructors and variables. Any identifier
that is not a constructor of an inductive or coinductive type is
considered to be 
a variable. Identifiers in patterns should be linear except for 
the ``don't care''  pattern denoted by ``\verb+_+''.
We can use patterns to build more complex patterns. 
We call {\em simple pattern} a variable  or a pattern of the form
$(c~\vec{x})$ where $c$ is a constructor symbol and $\vec{x}$ is a
linear vector of  variables. If a pattern is
not simple we call it {\em nested}.  


A variable pattern matches any value, and the
identifier is bound to that value. The pattern ``\verb+_+'' also matches
any value, but it is not binding. Alias patterns written \verb+(+{\sl pattern} \verb+as+ {\sl
identifier}\verb+)+ are also accepted. This pattern matches the same values as
{\sl pattern} does   and
{\sl identifier} is bound to the matched value.
A list of patterns is also considered as a pattern and is called {\em
multiple pattern}.

\begin{figure}[t]
\begin{center}
\begin{sl}
\begin{tabular}{|l|}\hline  \\
\begin{tabular}{rcl}%\hline && \\
simple\_pattern  & := &  pattern \verb+as+  identifier \\
                  &$|$ &  pattern \verb+,+ pattern \\ 
                  &$|$ &  pattern pattern\_list \\ && \\

pattern  & := & identifier  $|$ \verb+(+ simple\_pattern \verb+)+ \\ &&\\


equation & := &  \{pattern\}$^+$ ~\verb+=>+ ~term \\ && \\

ne\_eqn\_list & := & \verb+|+$^{opt}$ equation~ \{\verb+|+ equation\}$^*$  \\ &&\\

eqn\_list & := & \{~equation~ \{\verb+|+ equation\}$^*$~\}$^*$\\ &&\\


term & := &  
\verb+Cases+ \{term \}$^+$ \verb+of+ ne\_eqn\_list \verb+end+ \\
&$|$ & \verb+<+term\verb+>+ \verb+Cases+ \{ term \}$^+$
\verb+of+ eqn\_list \verb+end+ \\&& %\\ \hline
\end{tabular} \\ \hline
\end{tabular} 
\end{sl} \end{center}
\caption{Macro Cases syntax.}
\label{grammar}
\end{figure}


Pattern matching  improves readability. Compare for example the term
of the  function {\em is\_zero} of natural 
numbers written with patterns and the one written in primitive
concrete syntax (note that the first bar \verb+|+ is optional)~:

\begin{center}
\begin{small}
\begin{tabular}{l}
\verb+[n:nat] Cases n of | O => true | _ => false end+,\\ 
\verb+[n:nat] Case  n of true [_:nat]false end+.
\end{tabular}
\end{small}
\end{center}

In Coq pattern matching is compiled into the primitive constructions,
thus the expressiveness of the theory remains the same. Once the stage
of parsing has finished patterns disappear. An easy way to see the
result of the expansion is by printing the term with \texttt{Print} if
the term is a constant, or
using the command \texttt{Check}  that displays
the term with its type :

\begin{coq_example}
Check (fun n:nat => match n with
                    | O => true
                    | _ => false
                    end).
\end{coq_example}


\verb+Cases+ accepts optionally an infix term enclosed between
brackets \verb+<>+ that we
call the {\em elimination predicate}. 
This term is the same argument as the one expected by the primitive
\verb+Case+. Given a pattern matching
expression, if all the right hand sides of \verb+=>+ ({\em rhs} in
short) have the same type, then this term
can be sometimes synthesized, and so we can omit the \verb+<>+.
Otherwise we have to
provide the predicate between \verb+<>+ as for the primitive \verb+Case+.

Let us illustrate through examples the different aspects of pattern matching.
Consider for example the function that computes the maximum of two
natural numbers.  We can write it in primitive syntax by:
\begin{coq_example}
Fixpoint max (n m:nat) {struct m} : nat :=
  match n with
  | O => 
      (*  O   *) m
      (* S n' *) 
  | S n' =>
      match m with
      | O => 
          (* O    *) S n'
          (* S m' *) 
      | S m' => S (max n' m')
      end
  end.
\end{coq_example}

Using patterns in the definitions gives:

\begin{coq_example}
Reset max.
Fixpoint max (n m:nat) {struct m} : nat :=
  match n with
  | O => m
  | S n' => match m with
            | O => S n'
            | S m' => S (max n' m')
            end
  end.
\end{coq_example}

Another way to write this definition is to use a multiple pattern to
 match    \verb+n+ and \verb+m+:

\begin{coq_example}
Reset max.
Fixpoint max (n m:nat) {struct m} : nat :=
  match n, m with
  | O, _ => m
  | S n', O => S n'
  | S n', S m' => S (max n' m')
  end.
\end{coq_example}


The strategy examines patterns 
from left to right. A case expression is generated {\bf only}  when there is at least one constructor in the column of patterns.
For example, 
\begin{coq_example}
Check (fun x:nat => match x return nat with
                    | y => y
                    end).
\end{coq_example}



We can also use ``\verb+as+ patterns'' to associate a name to a
sub-pattern:

\begin{coq_example}
Reset max.
Fixpoint max (n m:nat) {struct n} : nat :=
  match n, m with
  | O, _ => m
  | S n' as N, O => N
  | S n', S m' => S (max n' m')
  end.
\end{coq_example}


In the previous examples patterns do not conflict with, but
sometimes it is comfortable to write patterns that admits a non
trivial superposition. Consider
the boolean function $lef$ that given two natural numbers
yields \verb+true+ if the first one is less or equal than the second
one and \verb+false+ otherwise. We can write it as follows:

\begin{coq_example}
Fixpoint lef (n m:nat) {struct m} : bool :=
  match n, m with
  | O, x => true
  | x, O => false
  | S n, S m => lef n m
  end.
\end{coq_example}

Note that the first and the second  multiple pattern  superpose because the couple of
values \verb+O O+ matches both. Thus, what is the result of the
function on those values?
To eliminate ambiguity we use the {\em textual priority rule}: we
consider patterns ordered from top to bottom, then a value is matched
by the  pattern at the $ith$ row if and only if is not matched by some
pattern of a previous row. Thus in the example,
\verb+O O+ is matched by the first pattern, and so \verb+(lef O O)+
yields \verb+true+.

Another way to write  this function is:

\begin{coq_example}
Reset lef.
Fixpoint lef (n m:nat) {struct m} : bool :=
  match n, m with
  | O, x => true
  | S n, S m => lef n m
  | _, _ => false
  end.
\end{coq_example}


Here the last pattern superposes with the first two. Because
of the priority rule, the last pattern 
will be used only for values that do not match neither the  first nor
the second one.  

Terms with useless patterns are accepted by the
system. For example,
\begin{coq_example}
Check
  (fun x:nat => match x with
                | O => true
                | S _ => false
                | x => true
                end).
\end{coq_example}

is accepted even though the last pattern is never used.
Beware,  the
current implementation rises no warning message when there are unused
patterns in a term.




\subsection{About patterns of parametric types}
When matching objects of a parametric type, constructors in patterns
{\em do not expect} the parameter arguments. Their value is deduced
during expansion.

Consider for example the polymorphic lists:

\begin{coq_example}
Inductive List (A:Set) : Set :=
  | nil : List A
  | cons : A -> List A -> List A.
\end{coq_example}

We can check the function {\em tail}:

\begin{coq_example}
Check
  (fun l:List nat =>
     match l with
     | nil => nil nat
     | cons _ l' => l'
     end).
\end{coq_example}


When we use parameters in patterns there is an error message:
\begin{coq_example}
Check
  (fun l:List nat =>
     match l with
     | nil nat => nil nat
     | cons nat _ l' => l'
     end).
\end{coq_example}



\subsection{Matching objects of dependent types}
The previous examples illustrate pattern matching on objects of
non-dependent types, but we can also 
use the macro to destructure objects of dependent type.
Consider the type \verb+listn+ of lists of a certain length:
\label{listn}

\begin{coq_example}
Inductive listn : nat -> Set :=
  | niln : listn O
  | consn : forall n:nat, nat -> listn n -> listn (S n).
\end{coq_example}

\subsubsection{Understanding dependencies in patterns}
We can define the function \verb+length+ over \verb+listn+ by :

\begin{coq_example}
Definition length (n:nat) (l:listn n) := n.
\end{coq_example}

Just for illustrating pattern matching, 
we can define it by case analysis:
\begin{coq_example}
Reset length.
Definition length (n:nat) (l:listn n) :=
  match l with
  | niln => O
  | consn n _ _ => S n
  end.
\end{coq_example}

We can understand the meaning of this definition using the
same notions of usual pattern matching.

Now suppose we split the second pattern  of \verb+length+ into two 
cases so to give an
alternative definition using nested patterns:
\begin{coq_example}
Definition length1 (n:nat) (l:listn n) :=
  match l with
  | niln => O
  | consn n _ niln => S n
  | consn n _ (consn _ _ _) => S n
  end.
\end{coq_example}

It is obvious that \verb+length1+ is  another version of
\verb+length+. We can also give the following definition:
\begin{coq_example}
Definition length2 (n:nat) (l:listn n) :=
  match l with
  | niln => O
  | consn n _ niln => S O
  | consn n _ (consn m _ _) => S (S m)
  end.
\end{coq_example}

If we forget that \verb+listn+ is a dependent type and we read these
definitions using the usual semantics of pattern matching,  we can conclude
that \verb+length1+
and \verb+length2+ are different functions.
In fact, they are equivalent
because the pattern \verb+niln+ implies that \verb+n+ can only match
the value $0$ and analogously the pattern \verb+consn+ determines that \verb+n+ can
only match  values of the form  $(S~v)$ where $v$ is the value matched by
\verb+m+. 


The converse is also true. If
we destructure the  length  value with the pattern \verb+O+ then the list
value should be $niln$. 
Thus, the following term \verb+length3+ corresponds to the function
\verb+length+ but this time defined by case analysis on the dependencies instead of on the list:

\begin{coq_example}
Definition length3 (n:nat) (l:listn n) :=
  match l with
  | niln => O
  | consn O _ _ => S O
  | consn (S n) _ _ => S (S n)
  end.
\end{coq_example}

When we have nested patterns of dependent types, the semantics of
pattern matching becomes a little more difficult because
the set of values that are matched by a sub-pattern may be conditioned by the
values matched by another sub-pattern. Dependent nested patterns are
somehow constrained patterns. 
In the examples, the expansion of
\verb+length1+ and \verb+length2+ yields exactly the same term
 but the
expansion of \verb+length3+ is completely different. \verb+length1+ and
\verb+length2+ are expanded into two nested case analysis on
\verb+listn+ while \verb+length3+ is expanded into a case analysis on
\verb+listn+ containing a case analysis on natural numbers inside.


In practice the user can think about the patterns as independent and
it is the expansion algorithm that cares to relate them. \\


\subsubsection{When the elimination predicate must be provided}
The examples  given so far do not need an explicit elimination predicate
between \verb+<>+ because all the rhs have the same type and the
strategy succeeds to synthesize it.
Unfortunately when dealing with dependent patterns it often happens
that we need to write cases where the type of the rhs are 
different  instances of the elimination  predicate.
The function  \verb+concat+ for \verb+listn+
is an example where the branches have different type
and we need to provide the elimination predicate:

\begin{coq_example}
Fixpoint concat (n:nat) (l:listn n) (m:nat) (l':listn m) {struct l} :
 listn (n + m) :=
  match l in listn n return listn (n + m) with
  | niln => l'
  | consn n' a y => consn (n' + m) a (concat n' y m l')
  end.
\end{coq_example}

Recall that a list of patterns is also a pattern. So, when
we destructure several terms at the same time and the branches have
different type  we need to provide
the elimination predicate for this multiple pattern.

For example, an equivalent definition for \verb+concat+ (even though with a useless extra pattern) would have
been:

\begin{coq_example}
Reset concat.
Fixpoint concat (n:nat) (l:listn n) (m:nat) (l':listn m) {struct l} :
 listn (n + m) :=
  match l in listn n, l' return listn (n + m) with
  | niln, x => x
  | consn n' a y, x => consn (n' + m) a (concat n' y m x)
  end.
\end{coq_example}

Note that this time, the predicate \verb+[n,_:nat](listn (plus n m))+ is binary because we
destructure both \verb+l+ and \verb+l'+ whose types have arity one.
In general, if we destructure the terms $e_1\ldots e_n$
the predicate will be of arity $m$ where $m$ is the sum of the
number of dependencies of the type of $e_1, e_2,\ldots e_n$ (the $\lambda$-abstractions
should correspond from left to right to each dependent argument of the
type of $e_1\ldots e_n$).
When the arity of the predicate (i.e. number of abstractions) is not
correct Coq rises an error message. For example:

\begin{coq_example}
Fixpoint concat (n:nat) (l:listn n) (m:nat) (l':listn m) {struct l} :
 listn (n + m) :=
  match l, l' with
  | niln, x => x
  | consn n' a y, x => consn (n' + m) a (concat n' y m x)
  end.
\end{coq_example}


\subsection{Using pattern matching to write proofs}
In all the previous examples the  elimination predicate  does not depend on the object(s) matched. 
The typical case where this is not possible is when we write a proof by
induction or a function that yields an object of dependent type.

For example, we can write 
the function \verb+buildlist+ that given a natural number
$n$ builds a list length $n$ containing zeros as follows:

\begin{coq_example}
Fixpoint buildlist (n:nat) : listn n :=
  match n return listn n with
  | O => niln
  | S n => consn n 0 (buildlist n)
  end.
\end{coq_example}

We can also use multiple patterns whenever the elimination predicate has
the correct arity. 

Consider the following definition of the predicate less-equal
\verb+Le+:

\begin{coq_example}
Inductive Le : nat -> nat -> Prop :=
  | LeO : forall n:nat, Le O n
  | LeS : forall n m:nat, Le n m -> Le (S n) (S m).
\end{coq_example}

We can use multiple patterns to write  the proof of the lemma
 \verb+(n,m:nat) (Le n m)\/(Le m n)+:

\begin{coq_example}
Fixpoint dec (n m:nat) {struct n} : Le n m \/ Le m n :=
  match n, m return Le n m \/ Le m n with
  | O, x => or_introl (Le x 0) (LeO x)
  | x, O => or_intror (Le x 0) (LeO x)
  | S n as N, S m as M =>
      match dec n m with
      | or_introl h => or_introl (Le M N) (LeS n m h)
      | or_intror h => or_intror (Le N M) (LeS m n h)
      end
  end.
\end{coq_example}
In the example of \verb+dec+ the elimination predicate is binary
because we destructure two arguments of \verb+nat+ that is a
non-dependent type. Note the first \verb+Cases+ is dependent while the
second  is not.

In general, consider the terms $e_1\ldots e_n$,
where  the type of $e_i$ is an instance of a family type
$[\vec{d_i}:\vec{D_i}]T_i$  ($1\leq i
\leq n$). Then to  write \verb+<+${\cal P}$\verb+>Cases+  $e_1\ldots
e_n$ \verb+of+ \ldots \verb+end+, the 
elimination predicate ${\cal P}$ should be of the form:
$[\vec{d_1}:\vec{D_1}][x_1:T_1]\ldots [\vec{d_n}:\vec{D_n}][x_n:T_n]Q.$




\section{Extending the syntax of pattern}
The primitive syntax for patterns considers only those patterns containing
symbols of constructors and variables. Nevertheless, we
may define our own syntax for  constructors and may be interested in
using this syntax to write patterns. 
Because not  any term is a pattern, the fact of extending the terms
syntax does not imply the extension of pattern syntax. Thus,  
the grammar of patterns should be explicitly extended whenever we
want to use a particular syntax for a constructor.
The grammar rules for the macro  \verb+Cases+ (and thus for patterns)
are defined in the file \verb+Multcase.v+ in the directory
\verb+src/syntax+. To extend the grammar of patterns 
we need to extend the non-terminals corresponding to patterns
(we refer the reader to chapter of grammar extensions).

 
We have already extended the pattern syntax so as to note
the constructor \verb+pair+ of cartesian product  with "( , )" in patterns. 
This allows for example, to write the first projection
of pairs as follows:
\begin{coq_example}
Definition fst (A B:Set) (H:A * B) := match H with
                                      | pair x y => x
                                      end.
\end{coq_example}
The grammar presented in figure \ref{grammar} actually
contains this extension. 

\section{When does the expansion strategy fail?}\label{limitations}
The strategy works very like in ML languages when treating
patterns of non-dependent type.  
But there are new cases of failure that are due to the presence of 
dependencies. 

The error messages of the current implementation may be
sometimes confusing.
When the tactic fails because patterns are somehow incorrect  then 
error messages refer to the initial expression. But the strategy
may succeed to build an expression whose sub-expressions are well typed but
the whole expression is not. In this situation  the message makes 
reference to the expanded expression.
We encourage users, when they have patterns with the same outer constructor in different equations, to name the variable patterns in the same positions with the same name.
E.g. to write {\small\verb+(cons n O x) => e1+} and  {\small\verb+(cons n \_ x) => e2+} instead of
{\small\verb+(cons n O x) => e1+} and  {\small\verb+(cons n' \_ x') => e2+}. This helps to maintain certain name correspondence  between the generated expression and the original.


Here is a summary of the error messages corresponding to each situation:
\begin{itemize}
\item  patterns are incorrect (because constructors are not
applied to the correct number of the arguments, because they  are not linear or they are
wrongly typed)
\begin{itemize}
\item \sverb{In pattern } {\sl term} \sverb{the constructor } {\sl ident}
\sverb{expects } {\sl num} \sverb{arguments}

\item \sverb{The variable } {\sl ident} \sverb{is bound several times in pattern } {\sl term}

\item \sverb{Constructor pattern: } {\sl term}  \sverb{cannot match values of type } {\sl term}
\end{itemize}

\item the pattern matching is not exhaustive
\begin{itemize}
\item \sverb{This pattern-matching is not exhaustive} 
\end{itemize}
\item the elimination predicate provided to \verb+Cases+  has not the expected arity

\begin{itemize}
\item \sverb{The elimination predicate } {\sl term} \sverb{should be
of arity } {\sl num} \sverb{(for non dependent case) or }  {\sl num} \sverb{(for dependent case)}
\end{itemize}

 \item  the whole expression is wrongly typed, or the synthesis of implicit arguments fails (for example to find
the elimination predicate or to resolve implicit arguments in the rhs).


There are {\em nested patterns  of dependent type}, the
elimination predicate corresponds to non-dependent case and has the  form $[x_1:T_1]...[x_n:T_n]T$
and {\bf some} $x_i$ occurs {\bf free} in
$T$.  Then, the  strategy may fail to find out a correct elimination
predicate during some step of compilation.
In this situation we recommend the user to rewrite the nested
dependent patterns into several \verb+Cases+ with  {\em simple patterns}.

In all these cases we have the following error message:

  \begin{itemize}
  \item 
  {\tt Expansion strategy failed to build a well typed case expression.
  There is a branch that mismatches the expected type.
  The risen type error on the result of expansion was:}
  \end{itemize}

\item because of nested patterns, it may happen that even  though all
the rhs have  the same type, the strategy needs
dependent elimination and so an elimination predicate must be
provided. The system
warns about this situation, trying to compile anyway with the
non-dependent strategy. The risen message is:
\begin{itemize}
\item {\tt Warning: This pattern matching may need dependent elimination to be compiled.
I will try, but if fails try again giving dependent elimination predicate.}
\end{itemize}

\item there are {\em nested patterns of dependent type} and the strategy
builds a term that is well typed but recursive
calls in fix point are reported as illegal:
\begin{itemize}
\item {\tt Error: Recursive call applied to an illegal term ...}
\end{itemize}

This is because the strategy generates a term that is correct
w.r.t. to the initial term but which does  not pass the guard condition.
In this situation we recommend the user to transform the nested  dependent
patterns into {\em several \verb+Cases+ of simple patterns}.
Let us explain this with an example.
Consider the following defintion of a function that yields the last
element of a list and \verb+O+ if it is empty:

\begin{coq_example}
Fixpoint last (n:nat) (l:listn n) {struct l} : nat :=
  match l with
  | consn _ a niln => a
  | consn m _ x => last m x
  | niln => O
  end.
\end{coq_example}

It fails because of the priority between patterns, we know that this
definition is equivalent to the following more explicit one (which
fails too):

\begin{coq_example*}
Fixpoint last (n:nat) (l:listn n) {struct l} : nat :=
  match l with
  | consn _ a niln => a
  | consn n _ (consn m b x) => last n (consn m b x)
  | niln => O
  end.
\end{coq_example*}

Note that the recursive call  \sverb{(last n (consn m b x)) } is not
guarded. When treating with patterns of dependent types the strategy
interprets the first definition of \texttt{last} as the second
onefootnote{In languages of the ML family
the first definition would be translated into a term where the
variable \texttt{x} is shared in the expression.  When
patterns are of non-dependent types, Coq compiles as in ML languages
using sharing. When patterns are of dependent types the compilation
reconstructs the term as in the second definition of \texttt{last} so to
ensure the result of expansion is well typed.}.
Thus it generates a
term where the recursive call is rejected by the 
guard condition.

You can get rid of this problem by writing the definition with \emph{simple
patterns}:

\begin{coq_example}
Fixpoint last (n:nat) (l:listn n) {struct l} : nat :=
  match l return nat with
  | consn m a x => match x with
                   | niln => a
                   | _ => last m x
                   end
  | niln => O
  end.
\end{coq_example}


\end{itemize}

%\end{document}

