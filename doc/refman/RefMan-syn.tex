\chapter[Syntax extensions and interpretation scopes]{Syntax extensions and interpretation scopes\label{Addoc-syntax}}
%HEVEA\cutname{syntax-extensions.html}

In this chapter, we introduce advanced commands to modify the way
{\Coq} parses and prints objects, i.e. the translations between the
concrete and internal representations of terms and commands. The main
commands are {\tt Notation} and {\tt Infix} which are described in
section \ref{Notation}.  It also happens that the same symbolic
notation is expected in different contexts. To achieve this form of
overloading, {\Coq} offers a notion of interpretation scope. This is
described in Section~\ref{scopes}.

\Rem The commands {\tt Grammar}, {\tt Syntax} and {\tt Distfix} which
were present for a while in {\Coq} are no longer available from {\Coq}
version 8.0. The underlying AST structure is also no longer available.
The functionalities of the command {\tt Syntactic Definition} are
still available; see Section~\ref{Abbreviations}.

\section[Notations]{Notations\label{Notation}
\comindex{Notation}}

\subsection{Basic notations}

A {\em notation} is a symbolic abbreviation denoting some term
or term pattern.

A typical notation is the use of the infix symbol \verb=/\= to denote
the logical conjunction (\texttt{and}). Such a notation is declared
by

\begin{coq_example*}
Notation "A /\ B" := (and A B).
\end{coq_example*}

The expression \texttt{(and A B)} is the abbreviated term and the
string \verb="A /\ B"= (called a {\em notation}) tells how it is 
symbolically written.

A notation is always surrounded by double quotes (except when the
abbreviation is a single identifier; see \ref{Abbreviations}). The
notation is composed of {\em tokens} separated by spaces.  Identifiers
in the string (such as \texttt{A} and \texttt{B}) are the {\em
parameters} of the notation. They must occur at least once each in the
denoted term. The other elements of the string (such as \verb=/\=) are
the {\em symbols}.

An identifier can be used as a symbol but it must be surrounded by
simple quotes to avoid the confusion with a parameter. Similarly,
every symbol of at least 3 characters and starting with a simple quote
must be quoted (then it starts by two single quotes). Here is an example.

\begin{coq_example*}
Notation "'IF' c1 'then' c2 'else' c3" := (IF_then_else c1 c2 c3).
\end{coq_example*}

%TODO quote the identifier when not in front, not a keyword, as in "x 'U' y" ?

A notation binds a syntactic expression to a term. Unless the parser
and pretty-printer of {\Coq} already know how to deal with the
syntactic expression (see \ref{ReservedNotation}), explicit precedences and
associativity rules have to be given.

\Rem The right-hand side of a notation is interpreted at the time the
notation is given. In particular, implicit arguments (see
Section~\ref{Implicit Arguments}), coercions (see
Section~\ref{Coercions}), etc. are resolved at the time of the
declaration of the notation.

\subsection[Precedences and associativity]{Precedences and associativity\index{Precedences}
\index{Associativity}}

Mixing different symbolic notations in the same text may cause serious
parsing ambiguity. To deal with the ambiguity of notations, {\Coq}
uses precedence levels ranging from 0 to 100 (plus one extra level
numbered 200) and associativity rules.

Consider for example the new notation

\begin{coq_example*}
Notation "A \/ B" := (or A B).
\end{coq_example*}

Clearly, an expression such as {\tt forall A:Prop, True \verb=/\= A \verb=\/=
A \verb=\/= False} is ambiguous. To tell the {\Coq} parser how to
interpret the expression, a priority between the symbols \verb=/\= and
\verb=\/= has to be given. Assume for instance that we want conjunction
to bind more than disjunction. This is expressed by assigning a
precedence level to each notation, knowing that a lower level binds
more than a higher level.  Hence the level for disjunction must be
higher than the level for conjunction.

Since connectives are not tight articulation points of a text, it
is reasonable to choose levels not so far from the highest level which
is 100, for example 85 for disjunction and 80 for
conjunction\footnote{which are the levels effectively chosen in the
current implementation of {\Coq}}.

Similarly, an associativity is needed to decide whether {\tt True \verb=/\=
False \verb=/\= False} defaults to {\tt True \verb=/\= (False
\verb=/\= False)} (right associativity) or to {\tt (True
\verb=/\= False) \verb=/\= False} (left associativity). We may
even consider that the expression is not well-formed and that
parentheses are mandatory (this is a ``no associativity'')\footnote{
{\Coq} accepts notations declared as no associative but the parser on
which {\Coq} is built, namely {\camlpppp}, currently does not implement the
no-associativity and replaces it by a left associativity; hence it is
the same for {\Coq}: no-associativity is in fact left associativity}.
We don't know of a special convention of the associativity of
disjunction and conjunction, so let's apply for instance a right
associativity (which is the choice of {\Coq}).

Precedence levels and associativity rules of notations have to be
given between parentheses in a list of modifiers that the
\texttt{Notation} command understands. Here is how the previous
examples refine.

\begin{coq_example*}
Notation "A /\ B" := (and A B) (at level 80, right associativity).
Notation "A \/ B" := (or A B)  (at level 85, right associativity).
\end{coq_example*}

By default, a notation is considered non associative, but the
precedence level is mandatory (except for special cases whose level is
canonical). The level is either a number or the phrase {\tt next
level} whose meaning is obvious. The list of levels already assigned
is on Figure~\ref{init-notations}.

\subsection{Complex notations}

Notations can be made from arbitrarily complex symbols. One can for
instance define prefix notations.

\begin{coq_example*}
Notation "~ x" := (not x) (at level 75, right associativity).
\end{coq_example*}

One can also define notations for incomplete terms, with the hole
expected to be inferred at typing time.

\begin{coq_example*}
Notation "x = y" := (@eq _ x y) (at level 70, no associativity).
\end{coq_example*}

One can define {\em closed} notations whose both sides are symbols. In
this case, the default precedence level for inner subexpression is 200.

\begin{coq_eval}
Set Printing Depth 50.
(********** The following is correct but produces **********)
(**** an incompatibility with the reserved notation ********)
\end{coq_eval}
\begin{coq_example*}
Notation "( x , y )" := (@pair _ _ x y) (at level 0).
\end{coq_example*}

One can also define notations for binders.

\begin{coq_eval}
Set Printing Depth 50.
(********** The following is correct but produces **********)
(**** an incompatibility with the reserved notation ********)
\end{coq_eval}
\begin{coq_example*}
Notation "{ x : A  |  P }" := (sig A (fun x => P)) (at level 0).
\end{coq_example*}

In the last case though, there is a conflict with the notation for
type casts. This last notation, as shown by the command {\tt Print Grammar
constr} is at level 100. To avoid \verb=x : A= being parsed as a type cast,
it is necessary to put {\tt x} at a level below 100, typically 99. Hence, a
correct definition is 

\begin{coq_example*}
Notation "{ x : A  |  P }" := (sig A (fun x => P)) (at level 0, x at level 99).
\end{coq_example*}

%This change has retrospectively an effect on the notation for notation
%{\tt "{ A } + { B }"}. For the sake of factorization, {\tt A} must be
%put at level 99 too, which gives
%
%\begin{coq_example*}
%Notation "{ A } + { B }" := (sumbool A B) (at level 0, A at level 99).
%\end{coq_example*}

See the next section for more about factorization.

\subsection{Simple factorization rules}

{\Coq} extensible parsing is performed by Camlp5 which is essentially a
LL1 parser. Hence, some care has to be taken not to hide already
existing rules by new rules. Some simple left factorization work has
to be done. Here is an example.

\begin{coq_eval}
(********** The next rule for notation _ < _ < _  produces **********)
(*** Error: Notation _ < _ < _ is already defined at level 70 ... ***)
\end{coq_eval}
\begin{coq_example*}
Notation "x < y"     := (lt x y) (at level 70).
Notation "x < y < z" := (x < y /\ y < z) (at level 70).
\end{coq_example*}

In order to factorize the left part of the rules, the subexpression
referred by {\tt y} has to be at the same level in both rules. However
the default behavior puts {\tt y} at the next level below 70
in the first rule (no associativity is the default), and at the level
200 in the second rule (level 200 is the default for inner expressions).
To fix this, we need to force the parsing level of {\tt y},
as follows.

\begin{coq_example*}
Notation "x < y"     := (lt x y) (at level 70).
Notation "x < y < z" := (x < y /\ y < z) (at level 70, y at next level).
\end{coq_example*}

For the sake of factorization with {\Coq} predefined rules, simple
rules have to be observed for notations starting with a symbol:
e.g. rules starting with ``\{'' or ``('' should be put at level 0. The
list of {\Coq} predefined notations can be found in Chapter~\ref{Theories}.

The command to display the current state of the {\Coq} term parser is
\comindex{Print Grammar constr}

\begin{quote}
\tt Print Grammar constr.
\end{quote}

\variant

\comindex{Print Grammar pattern}
{\tt Print Grammar pattern.}\\

This displays the state of the subparser of patterns (the parser
used in the grammar of the {\tt match} {\tt with} constructions).

\subsection{Displaying symbolic notations}

The command \texttt{Notation} has an effect both on the {\Coq} parser and
on the {\Coq} printer. For example:

\begin{coq_example}
Check (and True True).
\end{coq_example}

However, printing, especially pretty-printing, requires
more care than parsing. We may want specific indentations,
line breaks, alignment if on several lines, etc. 

The default printing of notations is very rudimentary. For printing a
notation, a {\em formatting box} is opened in such a way that if the
notation and its arguments cannot fit on a single line, a line break
is inserted before the symbols of the notation and the arguments on
the next lines are aligned with the argument on the first line.

A first, simple control that a user can have on the printing of a
notation is the insertion of spaces at some places of the
notation. This is performed by adding extra spaces between the symbols
and parameters: each extra space (other than the single space needed
to separate the components) is interpreted as a space to be inserted
by the printer. Here is an example showing how to add spaces around
the bar of the notation.

\begin{coq_example}
Notation "{{ x : A  |  P }}" := (sig (fun x : A => P))
  (at level 0, x at level 99).
Check (sig (fun x : nat => x=x)).
\end{coq_example}

The second, more powerful control on printing is by using the {\tt
format} modifier. Here is an example

\begin{small}
\begin{coq_example}
Notation "'If' c1 'then' c2 'else' c3" := (IF_then_else c1 c2 c3)
(at level 200, right associativity, format
"'[v   ' 'If'  c1 '/' '[' 'then'  c2  ']' '/' '[' 'else'  c3 ']' ']'").
\end{coq_example}
\end{small}

A {\em format} is an extension of the string denoting the notation with
the possible following elements delimited by single quotes:

\begin{itemize}
\item extra spaces are translated into simple spaces
\item tokens of the form \verb='/  '= are translated into breaking point,
  in case a line break occurs, an indentation of the number of spaces
  after the ``\verb=/='' is applied (2 spaces in the given example)
\item token of the form \verb='//'= force writing on a new line
\item well-bracketed pairs of tokens of the form \verb='[    '= and \verb=']'=
  are translated into printing boxes; in case a line break occurs,
  an extra indentation of the number of spaces given after the ``\verb=[=''
  is applied (4 spaces in the example)
\item well-bracketed pairs of tokens of the form \verb='[hv   '= and \verb=']'=
  are translated into horizontal-orelse-vertical printing boxes; 
  if the content of the box does not fit on a single line, then every breaking
  point forces a newline and an extra  indentation of the number of spaces
  given after the ``\verb=[='' is applied at the beginning of each newline
  (3 spaces in the example)
\item well-bracketed pairs of tokens of the form \verb='[v '= and
  \verb=']'= are translated into vertical printing boxes; every
  breaking point forces a newline, even if the line is large enough to
  display the whole content of the box, and an extra indentation of the
  number of spaces given after the ``\verb=[='' is applied at the beginning
  of each newline
\end{itemize}

%Thus, for the previous example, we get
%\footnote{The ``@'' is here to shunt
%the notation "'IF' A 'then' B 'else' C" which is defined in {\Coq}
%initial state}:

Notations do not survive the end of sections. No typing of the denoted
expression is performed at definition time. Type-checking is done only
at the time of use of the notation.

\begin{coq_example}
Check 
 (IF_then_else (IF_then_else True False True) 
   (IF_then_else True False True)
   (IF_then_else True False True)).   
\end{coq_example}

\Rem
Sometimes, a notation is expected only for the parser.
%(e.g. because
%the underlying parser of {\Coq}, namely {\camlpppp}, is LL1 and some extra
%rules are needed to circumvent the absence of factorization).
To do so, the option {\em only parsing} is allowed in the list of modifiers of
\texttt{Notation}.

Conversely, the {\em only printing} can be used to declare
that a notation should only be used for printing and should not declare a
parsing rule. In particular, such notations do not modify the parser.

\subsection{The \texttt{Infix} command
\comindex{Infix}}

The \texttt{Infix} command is a shortening for declaring notations of
infix symbols. Its syntax is 

\begin{quote}
\noindent\texttt{Infix "{\symbolentry}" :=} {\qualid} {\tt (} \nelist{\em modifier}{,} {\tt )}.
\end{quote}

and it is equivalent to

\begin{quote}
\noindent\texttt{Notation "x {\symbolentry} y" := ({\qualid} x y)  (} \nelist{\em modifier}{,} {\tt )}.
\end{quote}

where {\tt x} and {\tt y} are fresh names distinct from {\qualid}. Here is an example.

\begin{coq_example*}
Infix "/\" := and (at level 80, right associativity).
\end{coq_example*}

\subsection{Reserving notations
\label{ReservedNotation}
\comindex{Reserved Notation}}

A given notation may be used in different contexts. {\Coq} expects all
uses of the notation to be defined at the same precedence and with the
same associativity. To avoid giving the precedence and associativity
every time, it is possible to declare a parsing rule in advance
without giving its interpretation. Here is an example from the initial
state of {\Coq}.

\begin{coq_example}
Reserved Notation "x = y" (at level 70, no associativity).
\end{coq_example}

Reserving a notation is also useful for simultaneously defining an
inductive type or a recursive constant and a notation for it.

\Rem The notations mentioned on Figure~\ref{init-notations} are
reserved. Hence their precedence and associativity cannot be changed.

\subsection{Simultaneous definition of terms and notations
\comindex{Fixpoint {\ldots} where {\ldots}}
\comindex{CoFixpoint {\ldots} where {\ldots}}
\comindex{Inductive {\ldots} where {\ldots}}}

Thanks to reserved notations, the inductive, co-inductive, recursive
and corecursive definitions can benefit of customized notations. To do
this, insert a {\tt where} notation clause after the definition of the
(co)inductive type or (co)recursive term (or after the definition of
each of them in case of mutual definitions). The exact syntax is given
on Figure~\ref{notation-syntax}. Here are examples:

\begin{coq_eval}
Set Printing Depth 50.
(********** The following is correct but produces an error **********)
(********** because the symbol /\ is already bound **********)
(**** Error: The conclusion of A -> B -> A /\ B is not valid *****)
\end{coq_eval}

\begin{coq_example*}
Inductive and (A B:Prop) : Prop := conj : A -> B -> A /\ B 
where "A /\ B" := (and A B).
\end{coq_example*}

\begin{coq_eval}
Set Printing Depth 50.
(********** The following is correct but produces an error **********)
(********** because the symbol + is already bound **********)
(**** Error: no recursive definition *****)
\end{coq_eval}

\begin{coq_example*}
Fixpoint plus (n m:nat) {struct n} : nat :=
  match n with
  | O => m
  | S p => S (p+m)
  end
where "n + m" := (plus n m).
\end{coq_example*}

\subsection{Displaying informations about notations
\optindex{Printing Notations}}

To deactivate the printing of all notations, use the command
\begin{quote}
\tt Unset Printing Notations.
\end{quote}
To reactivate it, use the command
\begin{quote}
\tt Set Printing Notations.
\end{quote}
The default is to use notations for printing terms wherever possible.

\SeeAlso {\tt Set Printing All} in Section~\ref{SetPrintingAll}.

\subsection{Locating notations
\comindex{Locate}
\label{LocateSymbol}}

To know to which notations a given symbol belongs to, use the command
\begin{quote}
\tt Locate {\symbolentry}
\end{quote}
where symbol is any (composite) symbol surrounded by double quotes. To locate
a particular notation, use a string where the variables of the
notation are replaced by ``\_'' and where possible single quotes
inserted around identifiers or tokens starting with a single quote are
dropped.

\Example
\begin{coq_example}
Locate "exists".
Locate "exists _ .. _ , _".
\end{coq_example}

\SeeAlso Section \ref{Locate}.

\begin{figure}
\begin{small}
\begin{centerframe}
\begin{tabular}{lcl}
{\sentence} & ::= & 
   \zeroone{\tt Local} \texttt{Notation} {\str} \texttt{:=} {\term} 
   \zeroone{\modifiers} \zeroone{:{\scope}} .\\
  & $|$ & 
   \zeroone{\tt Local} \texttt{Infix} {\str} \texttt{:=} {\qualid} 
   \zeroone{\modifiers} \zeroone{:{\scope}} .\\
  & $|$ & 
   \zeroone{\tt Local} \texttt{Reserved Notation} {\str}
   \zeroone{\modifiers} .\\
  & $|$ & {\tt Inductive}
   \nelist{{\inductivebody} \zeroone{\declnotation}}{with}{\tt .}\\
  & $|$ & {\tt CoInductive}
   \nelist{{\inductivebody} \zeroone{\declnotation}}{with}{\tt .}\\
  & $|$ & {\tt Fixpoint}
   \nelist{{\fixpointbody} \zeroone{\declnotation}}{with} {\tt .} \\
  & $|$ & {\tt CoFixpoint}
   \nelist{{\cofixpointbody} \zeroone{\declnotation}}{with} {\tt .} \\
\\
{\declnotation} & ::= & 
  \zeroone{{\tt where} \nelist{{\str} {\tt :=} {\term} \zeroone{:{\scope}}}{\tt and}}.
\\
\\
{\modifiers}
  & ::= & \nelist{\ident}{,} {\tt at level} {\naturalnumber} \\
  & $|$ & \nelist{\ident}{,} {\tt at next level} \\
  & $|$ & {\tt at level} {\naturalnumber} \\
  & $|$ & {\tt left associativity} \\
  & $|$ & {\tt right associativity} \\
  & $|$ & {\tt no associativity} \\
  & $|$ & {\ident} {\tt ident} \\
  & $|$ & {\ident} {\tt binder} \\
  & $|$ & {\ident} {\tt closed binder} \\
  & $|$ & {\ident} {\tt global} \\
  & $|$ & {\ident} {\tt bigint} \\
  & $|$ & {\tt only parsing} \\
  & $|$ & {\tt only printing} \\
  & $|$ & {\tt format} {\str} 
\end{tabular}
\end{centerframe}
\end{small}
\caption{Syntax of the variants of {\tt Notation}}
\label{notation-syntax}
\end{figure}

\subsection{Notations and simple binders}

Notations can be defined for binders as in the example:

\begin{coq_eval}
Set Printing Depth 50.
(********** The following is correct but produces **********)
(**** an incompatibility with the reserved notation ********)
\end{coq_eval}
\begin{coq_example*}
Notation "{ x : A  |  P  }" := (sig (fun x : A => P)) (at level 0).
\end{coq_example*}

The binding variables in the left-hand-side that occur as a parameter
of the notation naturally bind all their occurrences appearing in
their respective scope after instantiation of the parameters of the
notation.

Contrastingly, the binding variables that are not a parameter of the
notation do not capture the variables of same name that
could appear in their scope after instantiation of the
notation. E.g., for the notation

\begin{coq_example*}
Notation "'exists_different' n" := (exists p:nat, p<>n) (at level 200).
\end{coq_example*}
the next command fails because {\tt p} does not bind in 
the instance of {\tt n}.
\begin{coq_eval}
Set Printing Depth 50.
\end{coq_eval}
% (********** The following produces **********)
% (**** The reference p was not found in the current environment ********)
\begin{coq_example}
Fail Check (exists_different p).
\end{coq_example}

\Rem Binding variables must not necessarily be parsed using the
{\tt ident} entry. For factorization purposes, they can be said to be
parsed at another level (e.g. {\tt x} in \verb="{ x : A | P }"= must be
parsed at level 99 to be factorized with the notation
\verb="{ A } + { B }"= for which {\tt A} can be any term).  
However, even if parsed as a term, this term must at the end be effectively 
a single identifier.

\subsection{Notations with recursive patterns}
\label{RecursiveNotations}

A mechanism is provided for declaring elementary notations with
recursive patterns. The basic example is:

\begin{coq_example*}
Notation "[ x ; .. ; y ]" := (cons x .. (cons y nil) ..).
\end{coq_example*}

On the right-hand side, an extra construction of the form {\tt ..} $t$
{\tt ..} can be used. Notice that {\tt ..} is part of the {\Coq}
syntax and it must not be confused with the three-dots notation
$\ldots$ used in this manual to denote a sequence of arbitrary size.

On the left-hand side, the part ``$x$ $s$ {\tt ..} $s$ $y$'' of the
notation parses any number of time (but at least one time) a sequence
of expressions separated by the sequence of tokens $s$ (in the
example, $s$ is just ``{\tt ;}'').

In the right-hand side, the term enclosed within {\tt ..} must be a
pattern with two holes of the form $\phi([~]_E,[~]_I)$ where the first
hole is occupied either by $x$ or by $y$ and the second hole is
occupied by an arbitrary term $t$ called the {\it terminating}
expression of the recursive notation. The subterm {\tt ..} $\phi(x,t)$
{\tt ..} (or {\tt ..} $\phi(y,t)$ {\tt ..})  must itself occur at
second position of the same pattern where the first hole is occupied
by the other variable, $y$ or $x$. Otherwise said, the right-hand side
must contain a subterm of the form either $\phi(x,${\tt ..}
$\phi(y,t)$ {\tt ..}$)$ or $\phi(y,${\tt ..}  $\phi(x,t)$ {\tt ..}$)$.
The pattern $\phi$ is the {\em iterator} of the recursive notation
and, of course, the name $x$ and $y$ can be chosen arbitrarily.

The parsing phase produces a list of expressions which are used to
fill in order the first hole of the iterating pattern which is
repeatedly nested as many times as the length of the list, the second
hole being the nesting point. In the innermost occurrence of the
nested iterating pattern, the second hole is finally filled with the
terminating expression.

In the example above, the iterator $\phi([~]_E,[~]_I)$ is {\tt cons
  $[~]_E$ $[~]_I$} and the terminating expression is {\tt nil}. Here are
other examples:
\begin{coq_example*}
Notation "( x , y , .. , z )" := (pair .. (pair x y) .. z) (at level 0).
Notation "[| t * ( x , y , .. , z ) ; ( a , b , .. , c )  * u |]" :=
  (pair (pair .. (pair (pair t x) (pair t y)) .. (pair t z))
        (pair .. (pair (pair a u) (pair b u)) .. (pair c u)))
  (t at level 39).
\end{coq_example*}

Recursive patterns can occur several times on the right-hand side.
Here is an example:

\begin{coq_example*}
Notation "[> a , .. , b <]" :=
  (cons a .. (cons b nil) .., cons b .. (cons a nil) ..).
\end{coq_example*}

Notations with recursive patterns can be reserved like standard
notations, they can also be declared within interpretation scopes (see
section \ref{scopes}).

\subsection{Notations with recursive patterns involving binders}

Recursive notations can also be used with binders. The basic example is:

\begin{coq_example*}
Notation "'exists' x .. y , p" := (ex (fun x => .. (ex (fun y => p)) ..))
  (at level 200, x binder, y binder, right associativity).
\end{coq_example*}

The principle is the same as in Section~\ref{RecursiveNotations}
except that in the iterator $\phi([~]_E,[~]_I)$, the first hole is a
placeholder occurring at the position of the binding variable of a {\tt
  fun} or a {\tt forall}.

To specify that the part ``$x$ {\tt ..} $y$'' of the notation
parses a sequence of binders, $x$ and $y$ must be marked as {\tt
  binder} in the list of modifiers of the notation.  Then, the list of
binders produced at the parsing phase are used to fill in the first
hole of the iterating pattern which is repeatedly nested as many times
as the number of binders generated. If ever the generalization
operator {\tt `} (see Section~\ref{implicit-generalization}) is used
in the binding list, the added binders are taken into account too.

Binders parsing exist in two flavors. If $x$ and $y$ are marked as
{\tt binder}, then a sequence such as {\tt a b c : T} will be accepted
and interpreted as the sequence of binders {\tt (a:T) (b:T)
  (c:T)}. For instance, in the notation above, the syntax {\tt exists
  a b : nat, a = b} is provided.

The variables $x$ and $y$ can also be marked as {\tt closed binder} in
which case only well-bracketed binders of the form {\tt (a b c:T)} or
{\tt \{a b c:T\}} etc. are accepted.

With closed binders, the recursive sequence in the left-hand side can
be of the general form $x$ $s$ {\tt ..} $s$ $y$ where $s$ is an
arbitrary sequence of tokens. With open binders though, $s$ has to be
empty. Here is an example of recursive notation with closed binders:

\begin{coq_example*}
Notation "'mylet' f x .. y :=  t 'in' u":=
  (let f := fun x => .. (fun y => t) .. in u)
  (at level 200, x closed binder, y closed binder, right associativity).
\end{coq_example*}

A recursive pattern for binders can be used in position of a recursive
pattern for terms. Here is an example:

\begin{coq_example*}
Notation "'FUNAPP' x .. y , f" :=
  (fun x => .. (fun y => (.. (f x) ..) y ) ..)
  (at level 200, x binder, y binder, right associativity).
\end{coq_example*}

\subsection{Summary}

\paragraph{Syntax of notations}

The different syntactic variants of the command \texttt{Notation} are
given on Figure~\ref{notation-syntax}. The optional {\tt :{\scope}} is
described in the Section~\ref{scopes}.

\Rem No typing of the denoted expression is performed at definition
time. Type-checking is done only at the time of use of the notation.

\Rem Many examples of {\tt Notation} may be found in the files
composing the initial state of {\Coq} (see directory {\tt
\$COQLIB/theories/Init}).

\Rem The notation \verb="{ x }"= has a special status in such a way
that complex notations of the form \verb="x + { y }"= or
\verb="x * { y }"= can be nested with correct precedences. Especially,
every notation involving a pattern of the form \verb="{ x }"= is
parsed as a notation where the pattern \verb="{ x }"= has been simply
replaced by \verb="x"= and the curly brackets are parsed separately.
E.g. \verb="y + { z }"= is not parsed as a term of the given form but
as a term of the form \verb="y + z"= where \verb=z= has been parsed
using the rule parsing \verb="{ x }"=. Especially, level and
precedences for a rule including patterns of the form \verb="{ x }"=
are relative not to the textual notation but to the notation where the
curly brackets have been removed (e.g. the level and the associativity
given to some notation, say \verb="{ y } & { z }"= in fact applies to
the underlying \verb="{ x }"=-free rule which is \verb="y & z"=).

\paragraph{Persistence of notations}

Notations do not survive the end of sections. They survive modules
unless the command {\tt Local Notation} is used instead of {\tt
Notation}.

\section[Interpretation scopes]{Interpretation scopes\index{Interpretation scopes}
\label{scopes}}
% Introduction

An {\em interpretation scope} is a set of notations for terms with
their interpretation. Interpretation scopes provide a weak,
purely syntactical form of notation overloading: the same notation, for
instance the infix symbol \verb=+=, can be used to denote distinct
definitions of the additive operator. Depending on which interpretation
scope is currently open, the interpretation is different.
Interpretation scopes can include an interpretation for
numerals and strings. However, this is only made possible at the
{\ocaml} level.

See Figure \ref{notation-syntax} for the syntax of notations including
the possibility to declare them in a given scope.  Here is a typical
example which declares the notation for conjunction in the scope {\tt
type\_scope}.

\begin{verbatim}
Notation "A /\ B" := (and A B) : type_scope.
\end{verbatim}

\Rem A notation not defined in a scope is called a {\em lonely} notation.

\subsection{Global interpretation rules for notations}

At any time, the interpretation of a notation for term is done within
a {\em stack} of interpretation scopes and lonely notations. In case a
notation has several interpretations, the actual interpretation is the
one defined by (or in) the more recently declared (or open) lonely
notation (or interpretation scope) which defines this notation.
Typically if a given notation is defined in some scope {\scope} but
has also an interpretation not assigned to a scope, then, if {\scope}
is open before the lonely interpretation is declared, then the lonely
interpretation is used (and this is the case even if the
interpretation of the notation in {\scope} is given after the lonely
interpretation: otherwise said, only the order of lonely
interpretations and opening of scopes matters, and not the declaration
of interpretations within a scope).

The initial state of {\Coq} declares three interpretation scopes and
no lonely notations. These scopes, in opening order, are {\tt
core\_scope}, {\tt type\_scope} and {\tt nat\_scope}.

The command to add a scope to the interpretation scope stack is
\comindex{Open Scope}
\comindex{Close Scope}
\begin{quote}
{\tt Open Scope} {\scope}.
\end{quote}
It is also possible to remove a scope from the interpretation scope
stack by using the command
\begin{quote}
{\tt Close Scope} {\scope}.
\end{quote}
Notice that this command does not only cancel the last {\tt Open Scope
{\scope}} but all the invocation of it.

\Rem {\tt Open Scope} and {\tt Close Scope} do not survive the end of
sections where they occur. When defined outside of a section, they are
exported to the modules that import the module where they occur.

\begin{Variants}

\item {\tt Local Open Scope} {\scope}.

\item {\tt Local Close Scope} {\scope}.

These variants are not exported to the modules that import the module
where they occur, even if outside a section.

\item {\tt Global Open Scope} {\scope}.

\item {\tt Global Close Scope} {\scope}.

These variants survive sections. They behave as if {\tt Global} were
absent when not inside a section.

\end{Variants}

\subsection{Local interpretation rules for notations}

In addition to the global rules of interpretation of notations, some
ways to change the interpretation of subterms are available.

\subsubsection{Local opening of an interpretation scope 
\label{scopechange}
\index{\%}
\comindex{Delimit Scope}
\comindex{Undelimit Scope}}

It is possible to locally extend the interpretation scope stack using
the syntax ({\term})\%{\delimkey} (or simply {\term}\%{\delimkey}
for atomic terms), where {\delimkey} is a special identifier called
{\em delimiting key} and bound to a given scope.

In such a situation, the term {\term}, and all its subterms, are
interpreted in the scope stack extended with the scope bound to
{\delimkey}.

To bind a delimiting key to a scope, use the command

\begin{quote}
\texttt{Delimit Scope} {\scope} \texttt{with} {\ident} 
\end{quote}

To remove a delimiting key of a scope, use the command

\begin{quote}
\texttt{Undelimit Scope} {\scope}
\end{quote}

\subsubsection{Binding arguments of a constant to an interpretation scope
\comindex{Arguments}}

It is possible to set in advance that some arguments of a given
constant have to be interpreted in a given scope. The command is
\begin{quote}
{\tt Arguments} {\qualid} \nelist{\name {\tt \%}\scope}{}
\end{quote}
where the list is a prefix of the list of the arguments of {\qualid} eventually
annotated with their {\scope}. Grouping round parentheses can be used to
decorate multiple arguments with the same scope.  {\scope} can be either a scope
name or its delimiting key. For example the following command puts the first two
arguments of {\tt plus\_fct} in the scope delimited by the key {\tt F} ({\tt
  Rfun\_scope}) and the last argument in the scope delimited by the key {\tt R}
({\tt R\_scope}).

\begin{coq_example*}
Arguments plus_fct (f1 f2)%F x%R.
\end{coq_example*}

The {\tt Arguments} command accepts scopes decoration to all grouping
parentheses. In the following example arguments {\tt A} and {\tt B} 
are marked as maximally inserted implicit arguments and are
put into the {\tt type\_scope} scope.

\begin{coq_example*}
Arguments respectful {A B}%type (R R')%signature _ _.
\end{coq_example*}

When interpreting a term, if some of the arguments of {\qualid} are
built from a notation, then this notation is interpreted in the scope
stack extended by the scope bound (if any) to this argument. The
effect of the scope is limited to the argument itself. It does not propagate 
to subterms but the subterms that, after interpretation of the
notation, turn to be themselves arguments of a reference are
interpreted accordingly to the arguments scopes bound to this reference.

Arguments scopes can be cleared with the following command:

\begin{quote}
{\tt Arguments {\qualid} : clear scopes}
\end{quote}

\begin{Variants}
\item {\tt Global Arguments} {\qualid} \nelist{\name {\tt \%}\scope}{}

This behaves like {\tt Arguments} {\qualid} \nelist{\name {\tt \%}\scope}{}
but survives when a section is closed instead
of stopping working at section closing. Without the {\tt Global} modifier,
the effect of the command stops when the section it belongs to ends.

\item {\tt Local Arguments} {\qualid} \nelist{\name {\tt \%}\scope}{}

This behaves like {\tt Arguments} {\qualid} \nelist{\name {\tt \%}\scope}{}
but does not survive modules and files.
Without the {\tt Local} modifier, the effect of the command is
visible from within other modules or files.

\end{Variants}

\SeeAlso The command to show the scopes bound to the arguments of a
function is described in Section~\ref{About}.

\subsubsection{Binding types of arguments to an interpretation scope}

When an interpretation scope is naturally associated to a type
(e.g. the scope of operations on the natural numbers), it may be
convenient to bind it to this type. When a scope {\scope} is bound to
a type {\type}, any new function defined later on gets its arguments
of type {\type} interpreted by default in scope {\scope} (this default
behavior can however be overwritten by explicitly using the command
{\tt Arguments}).

Whether the argument of a function has some type {\type} is determined
statically. For instance, if {\tt f} is a polymorphic function of type
{\tt forall X:Type, X -> X} and type {\tt t} is bound to a scope
{\scope}, then {\tt a} of type {\tt t} in {\tt f~t~a} is not
recognized as an argument to be interpreted in scope {\scope}.

\comindex{Bind Scope}
\label{bindscope}
More generally, any coercion {\class} (see Chapter~\ref{Coercions-full}) can be
bound to an interpretation scope. The command to do it is
\begin{quote}
{\tt Bind Scope} {\scope} \texttt{with} {\class}
\end{quote}

\Example
\begin{coq_example}
Parameter U : Set.
Bind Scope U_scope with U.
Parameter Uplus : U -> U -> U.
Parameter P : forall T:Set, T -> U -> Prop.
Parameter f : forall T:Set, T -> U.
Infix "+" := Uplus : U_scope.
Unset Printing Notations.
Open Scope nat_scope. (* Define + on the nat as the default for + *)
Check (fun x y1 y2 z t => P _ (x + t) ((f _ (y1 + y2) + z))).
\end{coq_example}

\Rem The scopes {\tt type\_scope} and {\tt function\_scope} also have a local effect on
interpretation. See the next section.

\SeeAlso The command to show the scopes bound to the arguments of a
function is described in Section~\ref{About}.

\Rem In notations, the subterms matching the identifiers of the
notations are interpreted in the scope in which the identifiers
occurred at the time of the declaration of the notation. Here is an
example:

\begin{coq_example}
Parameter g : bool -> bool.
Notation "@@" := true (only parsing) : bool_scope.
Notation "@@" := false (only parsing): mybool_scope.

(* Defining a notation while the argument of g is bound to bool_scope *)
Bind Scope bool_scope with bool.
Notation "# x #" := (g x) (at level 40).
Check # @@ #.
(* Rebinding the argument of g to mybool_scope has no effect on the notation *)
Arguments g _%mybool_scope.
Check # @@ #.
(* But we can force the scope *)
Delimit Scope mybool_scope with mybool.
Check # @@%mybool #.
\end{coq_example}

\subsection[The {\tt type\_scope} interpretation scope]{The {\tt type\_scope} interpretation scope\index{type\_scope@\texttt{type\_scope}}}

The scope {\tt type\_scope} has a special status. It is a primitive
interpretation scope which is temporarily activated each time a
subterm of an expression is expected to be a type.  It is delimited by
the key {\tt type}, and bound to the coercion class {\tt Sortclass}. It is also
used in certain situations where an expression is statically known to
be a type, including the conclusion and the type of hypotheses within
an {\tt Ltac} goal match (see Section~\ref{ltac-match-goal})
the statement of a theorem, the type of
a definition, the type of a binder, the domain and codomain of
implication, the codomain of products, and more generally any type
argument of a declared or defined constant.

\subsection[The {\tt function\_scope} interpretation scope]{The {\tt function\_scope} interpretation scope\index{function\_scope@\texttt{function\_scope}}}

The scope {\tt function\_scope} also has a special status. 
It is temporarily activated each time the argument of a global reference is
recognized to be a {\tt Funclass instance}, i.e., of type {\tt forall x:A, B} or {\tt A -> B}.

\subsection{Interpretation scopes used in the standard library of {\Coq}}

We give an overview of the scopes used in the standard library of
{\Coq}. For a complete list of notations in each scope, use the
commands {\tt Print Scopes} or {\tt Print Scope {\scope}}.

\subsubsection{\tt type\_scope}

This scope includes infix {\tt *} for product types and infix {\tt +} for
sum types. It is delimited by key {\tt type}, and bound to the coercion class 
{\tt Sortclass}, as described at \ref{bindscope}.

\subsubsection{\tt nat\_scope}

This scope includes the standard arithmetical operators and relations on
type {\tt nat}. Positive numerals in this scope are mapped to their
canonical representent built from {\tt O} and {\tt S}. The scope is
delimited by key {\tt nat}, and bound to the type {\tt nat} (see \ref{bindscope}).

\subsubsection{\tt N\_scope}

This scope includes the standard arithmetical operators and relations on
type {\tt N} (binary natural numbers). It is delimited by key {\tt N}
and comes with an interpretation for numerals as closed term of type {\tt N}.

\subsubsection{\tt Z\_scope}

This scope includes the standard arithmetical operators and relations on
type {\tt Z} (binary integer numbers). It is delimited by key {\tt Z} 
and comes with an interpretation for numerals as closed term of type {\tt Z}.

\subsubsection{\tt positive\_scope}

This scope includes the standard arithmetical operators and relations on
type {\tt positive} (binary strictly positive numbers). It is
delimited by key {\tt positive} and comes with an interpretation for
numerals as closed term of type {\tt positive}.

\subsubsection{\tt Q\_scope}

This scope includes the standard arithmetical operators and relations on
type {\tt Q} (rational numbers defined as fractions of an integer and
a strictly positive integer modulo the equality of the
numerator-denominator cross-product). As for numerals, only $0$ and
$1$ have an interpretation in scope {\tt Q\_scope} (their
interpretations are $\frac{0}{1}$ and $\frac{1}{1}$ respectively).

\subsubsection{\tt Qc\_scope}

This scope includes the standard arithmetical operators and relations on the
type {\tt Qc} of rational numbers defined as the type of irreducible
fractions of an integer and a strictly positive integer.

\subsubsection{\tt real\_scope}

This scope includes the standard arithmetical operators and relations on
type {\tt R} (axiomatic real numbers). It is delimited by key {\tt R}
and comes with an interpretation for numerals using the {\tt IZR}
morphism from binary integer numbers to {\tt R}.

\subsubsection{\tt bool\_scope}

This scope includes notations for the boolean operators. It is
delimited by key {\tt bool}, and bound to the type {\tt bool} (see \ref{bindscope}).

\subsubsection{\tt list\_scope}

This scope includes notations for the list operators. It is
delimited by key {\tt list}, and bound to the type {\tt list} (see \ref{bindscope}).

\subsubsection{\tt function\_scope}

This scope is delimited by the key {\tt function}, and bound to the coercion class {\tt Funclass}, 
as described at \ref{bindscope}.

\subsubsection{\tt core\_scope}

This scope includes the notation for pairs. It is delimited by key {\tt core}.

\subsubsection{\tt string\_scope}

This scope includes notation for strings as elements of the type {\tt
string}.  Special characters and escaping follow {\Coq} conventions
on strings (see Section~\ref{strings}). Especially, there is no
convention to visualize non printable characters of a string.  The
file {\tt String.v} shows an example that contains quotes, a newline
and a beep (i.e. the ASCII character of code 7).

\subsubsection{\tt char\_scope}

This scope includes interpretation for all strings of the form
\verb!"!$c$\verb!"! where $c$ is an ASCII character, or of the form
\verb!"!$nnn$\verb!"! where $nnn$ is a three-digits number (possibly
with leading 0's), or of the form \verb!""""!. Their respective
denotations are the ASCII code of $c$, the decimal ASCII code $nnn$,
or the ASCII code of the character \verb!"! (i.e. the ASCII code
34), all of them being represented in the type {\tt ascii}.

\subsection{Displaying informations about scopes}

\subsubsection{\tt Print Visibility\comindex{Print Visibility}}

This displays the current stack of notations in scopes and lonely
notations that is used to interpret a notation. The top of the stack
is displayed last. Notations in scopes whose interpretation is hidden
by the same notation in a more recently open scope are not
displayed. Hence each notation is displayed only once.

\variant

{\tt Print Visibility {\scope}}\\

This displays the current stack of notations in scopes and lonely
notations assuming that {\scope} is pushed on top of the stack.  This
is useful to know how a subterm locally occurring in the scope of
{\scope} is interpreted.

\subsubsection{\tt Print Scope {\scope}\comindex{Print Scope}}

This displays all the notations defined in interpretation scope
{\scope}.  It also displays the delimiting key if any and the class to
which the scope is bound, if any.

\subsubsection{\tt Print Scopes\comindex{Print Scopes}}

This displays all the notations, delimiting keys and corresponding
class of all the existing interpretation scopes.
It also displays the lonely notations.

\section[Abbreviations]{Abbreviations\index{Abbreviations}
\label{Abbreviations}
\comindex{Notation}}

An {\em abbreviation} is a name, possibly applied to arguments, that
denotes a (presumably) more complex expression. Here are examples:

\begin{coq_eval}
Require Import List.
Require Import Relations.
Set Printing Notations.
\end{coq_eval}
\begin{coq_example}
Notation Nlist := (list nat).
Check 1 :: 2 :: 3 :: nil.
Notation reflexive R := (forall x, R x x).
Check forall A:Prop, A <-> A.
Check reflexive iff.
\end{coq_example}

An abbreviation expects no precedence nor associativity, since it
follows the usual syntax of application. Abbreviations are used as
much as possible by the {\Coq} printers unless the modifier
\verb=(only parsing)= is given.

Abbreviations are bound to an absolute name as an ordinary
definition is, and they can be referred by qualified names too.

Abbreviations are syntactic in the sense that they are bound to
expressions which are not typed at the time of the definition of the
abbreviation but at the time it is used. Especially, abbreviations can
be bound to terms with holes (i.e. with ``\_''). The general syntax
for abbreviations is
\begin{quote}
\zeroone{{\tt Local}} \texttt{Notation} {\ident} \sequence{\ident} {\ident} \texttt{:=} {\term} 
 \zeroone{{\tt (only parsing)}}~\verb=.=
\end{quote}

\Example
\begin{coq_eval}
Set Strict Implicit.
Reset Initial.
\end{coq_eval}
\begin{coq_example}
Definition explicit_id (A:Set) (a:A) := a.
Notation id := (explicit_id _).
Check (id 0).
\end{coq_example}

Abbreviations do not survive the end of sections. No typing of the denoted
expression is performed at definition time. Type-checking is done only
at the time of use of the abbreviation.

%\Rem \index{Syntactic Definition} %
%Abbreviations are similar to the {\em syntactic
%definitions} available in versions of {\Coq} prior to version 8.0,
%except that abbreviations are used for printing (unless the modifier
%\verb=(only parsing)= is given) while syntactic definitions were not.

\section{Tactic Notations
\comindex{Tactic Notation}}

Tactic notations allow to customize the syntax of the tactics of the
tactic language\footnote{Tactic notations are just a simplification of
the {\tt Grammar tactic simple\_tactic} command that existed in
versions prior to version 8.0.}. Tactic notations obey the following
syntax
\medskip

\noindent
\begin{tabular}{lcl}
{\sentence} & ::= & \zeroone{\tt Local} \texttt{Tactic Notation} \zeroone{\taclevel} \sequence{\proditem}{} \\
& & \texttt{:= {\tac} .}\\
{\proditem} & ::= & {\str} $|$ {\tacargtype}{\tt ({\ident})} \\ 
{\taclevel} & ::= & {\tt (at level} {\naturalnumber}{\tt )} \\
{\tacargtype}\!\! & ::= &
%{\tt preident} $|$
{\tt ident} $|$
{\tt simple\_intropattern} $|$
{\tt reference} \\ & $|$ &
{\tt hyp} $|$
{\tt hyp\_list} $|$
{\tt ne\_hyp\_list} \\ & $|$ &
% {\tt quantified\_hypothesis} \\ & $|$ &
{\tt constr} $|$ {\tt uconstr} $|$
{\tt constr\_list} $|$
{\tt ne\_constr\_list} \\ & $|$ &
%{\tt castedopenconstr} $|$
{\tt integer} $|$
{\tt integer\_list} $|$
{\tt ne\_integer\_list} \\ & $|$ &
{\tt int\_or\_var} $|$
{\tt int\_or\_var\_list} $|$
{\tt ne\_int\_or\_var\_list} \\ & $|$ &
{\tt tactic} $|$ {\tt tactic$n$} \qquad\mbox{(for $0\leq n\leq 5$)}

\end{tabular}
\medskip

A tactic notation {\tt Tactic Notation {\taclevel}
{\sequence{\proditem}{}} := {\tac}} extends the parser and
pretty-printer of tactics with a new rule made of the list of
production items. It then evaluates into the tactic expression
{\tac}. For simple tactics, it is recommended to use a terminal
symbol, i.e. a {\str}, for the first production item.  The tactic
level indicates the parsing precedence of the tactic notation. This
information is particularly relevant for notations of tacticals.
Levels 0 to 5 are available (default is 0). 
To know the parsing precedences of the
existing tacticals, use the command {\tt Print Grammar tactic.}

Each type of tactic argument has a specific semantic regarding how it
is parsed and how it is interpreted. The semantic is described in the
following table. The last command gives examples of tactics which
use the corresponding kind of argument.

\medskip
\noindent
\begin{tabular}{l|l|l|l}
Tactic argument type & parsed as & interpreted as & as in tactic \\
\hline & & & \\
{\tt\small ident} & identifier & a user-given name & {\tt intro} \\
{\tt\small simple\_intropattern} & intro\_pattern & an intro\_pattern & {\tt intros}\\
{\tt\small hyp} & identifier & an hypothesis defined in context & {\tt clear}\\
%% quantified_hypothesis actually not supported
%%{\tt\small quantified\_hypothesis} & identifier or integer & a named or non dep. hyp. of the goal & {\tt intros until}\\
{\tt\small reference} & qualified identifier & a global reference of term & {\tt unfold}\\
{\tt\small constr} & term & a term & {\tt exact} \\
{\tt\small uconstr} & term & an untyped term & {\tt refine} \\
%% castedopenconstr actually not supported
%%{\tt\small castedopenconstr} & term & a term with its sign. of exist. var. & {\tt refine}\\
{\tt\small integer} & integer & an integer &  \\
{\tt\small int\_or\_var} & identifier or integer & an integer & {\tt do} \\
{\tt\small tactic} & tactic at level 5 & a tactic &  \\
{\tt\small tactic$n$} & tactic at level $n$ & a tactic & \\
{\tt\small {\nterm{entry}}\_list} & list of {\nterm{entry}} & a list of how {\nterm{entry}} is interpreted & \\
{\tt\small ne\_{\nterm{entry}}\_list} & non-empty list of {\nterm{entry}} & a list of how {\nterm{entry}} is interpreted& \\
\end{tabular}

\Rem In order to be bound in tactic definitions, each syntactic entry
for argument type must include the case of simple {\ltac} identifier
as part of what it parses. This is naturally the case for {\tt ident},
{\tt simple\_intropattern}, {\tt reference}, {\tt constr}, ... but not
for {\tt integer}. This is the reason for introducing a special entry
{\tt int\_or\_var} which evaluates to integers only but which
syntactically includes identifiers in order to be usable in tactic
definitions.

\Rem The {\tt {\nterm{entry}}\_list} and {\tt ne\_{\nterm{entry}}\_list}
entries can be used in primitive tactics or in other notations at
places where a list of the underlying entry can be used: {\nterm{entry}} is
either {\tt\small constr}, {\tt\small hyp}, {\tt\small integer} or
{\tt\small int\_or\_var}.

Tactic notations do not survive the end of sections. They survive
modules unless the command {\tt Local Tactic Notation} is used instead
of {\tt Tactic Notation}.

%%% Local Variables: 
%%% mode: latex
%%% TeX-master: "Reference-Manual"
%%% End: 
