\chapter[The \Coq~commands]{The \Coq~commands\label{Addoc-coqc}
\ttindex{coqtop}
\ttindex{coqc}}

There are three \Coq~commands: 
\begin{itemize}
\item {\tt coqtop}: The \Coq\ toplevel (interactive mode) ; 
\item {\tt coqc} : The \Coq\ compiler (batch compilation).
\item {\tt coqchk} : The \Coq\ checker (validation of compiled libraries)
\end{itemize}
The options are (basically) the same for the first two commands, and
roughly described below. You can also look at the \verb!man! pages of
\verb!coqtop! and \verb!coqc! for more details.


\section{Interactive use ({\tt coqtop})}

In the interactive mode, also known as the \Coq~toplevel, the user can
develop his theories and proofs step by step.  The \Coq~toplevel is
run by the command {\tt coqtop}. 

\index{byte-code}
\index{native code}
\label{binary-images}
They are two different binary images of \Coq: the byte-code one and
the native-code one (if {\ocaml} provides a native-code compiler
for your platform, which is supposed in the following). By default,
\verb!coqc! executes the native-code version; this can be overridden
using the \verb!-byte! option.

The byte-code toplevel is based on a Caml
toplevel (to allow the dynamic link of tactics).  You can switch to
the Caml toplevel with the command \verb!Drop.!, and come back to the
\Coq~toplevel with the command \verb!Toplevel.loop();;!.

\section{Batch compilation ({\tt coqc})}
The {\tt coqc} command takes a name {\em file} as argument.  Then it
looks for a vernacular file named {\em file}{\tt .v}, and tries to
compile it into a {\em file}{\tt .vo} file (See ~\ref{compiled}).

\Warning The name {\em file} must be a regular {\Coq} identifier, as
defined in the Section~\ref{lexical}. It
must only contain letters, digits or underscores
(\_). Thus it can be \verb+/bar/foo/toto.v+ but cannot be
\verb+/bar/foo/to-to.v+.

\section[Customization]{Customization at launch time}

\subsection{By resource file\index{Resource file}}

When \Coq\ is launched, with either {\tt coqtop} or {\tt coqc}, the
resource file \verb:$XDG_CONFIG_HOME/coq/coqrc.xxx: is loaded, where
\verb:$XDG_CONFIG_HOME: is the configuration directory of the user (by
default its home directory \verb!/.config! and \verb:xxx: is the version
number (e.g. 8.3).  If this file is not found, then the file
\verb:$XDG_CONFIG_HOME/coqrc: is searched. You can also specify an
arbitrary name for the resource file (see option \verb:-init-file:
below).


This file may contain, for instance, \verb:Add LoadPath: commands to add
directories to the load path of \Coq.
It is possible to skip the loading of the resource file with the
option \verb:-q:.

\section{By environment variables\label{EnvVariables}
\index{Environment variables}\label{envars}}

Load path can be specified to the \Coq\ system by setting up
\verb:$COQPATH: environment variable. It is a list of directories
separated by \verb|:| (\verb|;| on windows). {\Coq} will also honour
\verb:$XDG_DATA_HOME: and \verb:$XDG_DATA_DIRS: (see section
\ref{loadpath}).

Some {\Coq} commands call other {\Coq} commands. In this case, they
look for the commands in directory specified by \verb:$COQBIN:. If
this variable is not set, they look for the command in the executable
path.

\subsection{By command line options\index{Options of the command line}
\label{vmoption}
\label{coqoptions}}

The following command-line options are recognized by the commands {\tt
  coqc} and {\tt coqtop}, unless stated otherwise:

\begin{description}
\item[{\tt -I} {\em directory}, {\tt -include} {\em directory}]\

Add physical path {\em directory} to the {\ocaml} loadpath.

  \SeeAlso Section~\ref{Libraries} and the command {\tt Declare ML Module} Section \ref{compiled}.

\item[\texttt{-Q} \emph{directory} {\dirpath}]\

  Add physical path \emph{directory} to the list of directories where
  {\Coq} looks for a file and bind it to the the logical directory
  \emph{dirpath}. The sub-directory structure of \emph{directory} is
  recursively available from {\Coq} using absolute names (extending
  the {\dirpath} prefix) (see Section~\ref{LongNames}).
  
  \SeeAlso Section~\ref{Libraries}.

\item[{\tt -R} {\em directory} {\dirpath}, {\tt -R} {\em directory} [{\tt -as} {\dirpath}]]\

  Do as \texttt{-Q} \emph{directory} {\dirpath} but make the
  sub-directory structure of \emph{directory} recursively visible so
  that the recursive contents of physical \emph{directory} is available
  from {\Coq} using short or partially qualified names.
  
  \SeeAlso Section~\ref{Libraries}.

\item[{\tt -top} {\dirpath}, {\tt -notop}]\ 

  This sets the toplevel module name to {\dirpath}/the empty logical path instead
  of {\tt Top}. Not valid for {\tt coqc}.

\item[{\tt -exclude-dir} {\em subdirectory}]\ 

  This tells to exclude any sub-directory named {\em subdirectory}
  while processing option {\tt -R}. Without this option only the
  conventional version control management sub-directories named {\tt
  CVS} and {\tt \_darcs} are excluded.

\item[{\tt -nois}]\ 

  Cause \Coq~to begin with an empty state.

\item[{\tt -init-file} {\em file}, {\tt -q}]\ 

  Take {\em file} as the resource file. /
  Cause \Coq~not to load the resource file.

\item[{\tt -load-ml-source} {\em file}]\ 

  Load the Caml source file {\em file}.

\item[{\tt -load-ml-object} {\em file}]\ 

  Load the Caml object file {\em file}.

\item[{\tt -l[v]} {\em file}, {\tt -load-vernac-source[-verbose]} {\em file}]\ 

  Load \Coq~file {\em file}{\tt .v} optionally with copy it contents on the
  standard input.

\item[{\tt -load-vernac-object} {\em file}]\ 

  Load \Coq~compiled file {\em file}{\tt .vo}

\item[{\tt -require} {\em file}]\ 

  Load \Coq~compiled file {\em file}{\tt .vo} and import it ({\tt
    Require} {\em file}).

\item[{\tt -compile} {\em file},{\tt -compile-verbose} {\em file}, {\tt -batch}]\ 

  {\tt coqtop} options only used internally by {\tt coqc}.

  This compiles file {\em file}{\tt .v} into {\em file}{\tt .vo} without/with a
  copy of the contents of the file on standard input.  This option implies options
  {\tt -batch} (exit just after arguments parsing). It is only
  available for {\tt coqtop}.

\item[{\tt -verbose}]\ 

  This option is only for {\tt coqc}. It tells to compile the file with
  a copy of its contents on standard input.

%Mostly unused in the code
%\item[{\tt -debug}]\ 
%
%  Switch on the debug flag.

\item[{\tt -with-geoproof} (yes|no)]\ 

  Activate or not special functions for Geoproof within {\CoqIDE} (default is yes).

\item[{\tt -beautify}]\ 

  While compiling {\em file}, pretty prints each command just after having parsing
  it in {\em file}{\tt .beautified} in order to get old-fashion
  syntax/definitions/notations.

\item[{\tt -emacs}, {\tt -ide-slave}]\ 

  Start a special main loop to communicate with ide.

\item[{\tt -impredicative-set}]\ 

  Change the logical theory of {\Coq} by declaring the sort {\tt Set}
  impredicative; warning: this is known to be inconsistent with
  some standard axioms of classical mathematics such as the functional
  axiom of choice or the principle of description

\item[{\tt -type-in-type}]\

  This collapses the universe hierarchy of {\Coq} making the logic inconsistent.

\item[{\tt -compat} {\em version}] \mbox{}

  Attempt to maintain some of the incompatible changes in their {\em version}
  behavior.

\item[{\tt -dump-glob} {\em file}]\ 

  This dumps references for global names in file {\em file}
  (to be used by coqdoc, see~\ref{coqdoc})

\item[{\tt -no-hash-consing}] \mbox{}

\item[{\tt -vm}]\ 

  This activates the use of the bytecode-based conversion algorithm
  for the current session (see Section~\ref{SetVirtualMachine}).

\item[{\tt -image} {\em file}]\ 

  This option sets the binary image to be used by {\tt coqc} to be {\em file}
  instead of the standard one. Not of general use.

\item[{\tt -bindir} {\em directory}]\ 

  Set for {\tt coqc} the directory containing \Coq\ binaries.
  It is equivalent to do \texttt{export COQBIN=}{\em directory}
  before launching {\tt coqc}.

\item[{\tt -where}, {\tt -config}, {\tt -filteropts}]\ 

  Print the \Coq's standard library location or \Coq's binaries, dependencies,
  libraries locations or the list of command line arguments that {\tt coqtop} has
  recognize as options and exit.

\item[{\tt -v}]\ 

  Print the \Coq's version and exit.

\item[{\tt -h}, {\tt --help}]\ 

  Print a short usage and exit.

\end{description}


\section{Compiled libraries checker ({\tt coqchk})}

The {\tt coqchk} command takes a list of library paths as argument.
The corresponding compiled libraries (.vo files) are searched in the
path, recursively processing the libraries they depend on. The content
of all these libraries is then type-checked. The effect of {\tt
  coqchk} is only to return with normal exit code in case of success,
and with positive exit code if an error has been found. Error messages
are not deemed to help the user understand what is wrong. In the
current version, it does not modify the compiled libraries to mark
them as successfully checked.

Note that non-logical information is not checked. By logical
information, we mean the type and optional body associated to names.
It excludes for instance anything related to the concrete syntax of
objects (customized syntax rules, association between short and long
names), implicit arguments, etc.

This tool can be used for several purposes. One is to check that a
compiled library provided by a third-party has not been forged and
that loading it cannot introduce inconsistencies.\footnote{Ill-formed
  non-logical information might for instance bind {\tt
    Coq.Init.Logic.True} to short name {\tt False}, so apparently {\tt
    False} is inhabited, but using fully qualified names, {\tt
    Coq.Init.Logic.False} will always refer to the absurd proposition,
  what we guarantee is that there is no proof of this latter
  constant.}
Another point is to get an even higher level of security. Since {\tt
  coqtop} can be extended with custom tactics, possibly ill-typed
code, it cannot be guaranteed that the produced compiled libraries are
correct. {\tt coqchk} is a standalone verifier, and thus it cannot be
tainted by such malicious code.

Command-line options {\tt -I}, {\tt -R}, {\tt -where} and
{\tt -impredicative-set} are supported by {\tt coqchk} and have the
same meaning as for {\tt coqtop}. Extra options are:
\begin{description}
\item[{\tt -norec} $module$]\ 

  Check $module$ but do not force check of its dependencies.
\item[{\tt -admit} $module$] \

  Do not check $module$ and any of its dependencies, unless
  explicitly required.
\item[{\tt -o}]\ 

  At exit, print a summary about the context. List the names of all
  assumptions and variables (constants without body).
\item[{\tt -silent}]\ 

  Do not write progress information in standard output.
\end{description}

Environment variable \verb:$COQLIB: can be set to override the
location of the standard library.

The algorithm for deciding which modules are checked or admitted is
the following: assuming that {\tt coqchk} is called with argument $M$,
option {\tt -norec} $N$, and {\tt -admit} $A$. Let us write
$\overline{S}$ the set of reflexive transitive dependencies of set
$S$. Then:
\begin{itemize}
\item Modules $C=\overline{M}\backslash\overline{A}\cup M\cup N$ are
  loaded and type-checked before being added to the context.
\item And $\overline{M}\cup\overline{N}\backslash C$ is the set of
  modules that are loaded and added to the context without
  type-checking. Basic integrity checks (checksums) are nonetheless
  performed.
\end{itemize}

As a rule of thumb, the {\tt -admit} can be used to tell that some
libraries have already been checked. So {\tt coqchk A B} can be split
in {\tt coqchk A \&\& coqchk B -admit A} without type-checking any
definition twice. Of course, the latter is slightly slower since it
makes more disk access. It is also less secure since an attacker might
have replaced the compiled library $A$ after it has been read by the
first command, but before it has been read by the second command.

%%% Local Variables: 
%%% mode: latex
%%% TeX-master: "Reference-Manual"
%%% End: 
