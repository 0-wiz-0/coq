\chapter[Detailed examples of tactics]{Detailed examples of tactics\label{Tactics-examples}}

This chapter presents detailed examples of certain tactics, to
illustrate their behavior.

\section[\tt dependent induction]{\tt dependent induction\label{dependent-induction-example}}
\def\depind{{\tt dependent induction}~}
\def\depdestr{{\tt dependent destruction}~}

The tactics \depind and \depdestr are another solution for inverting
inductive predicate instances and potentially doing induction at the
same time. It is based on the \texttt{BasicElim} tactic of Conor McBride which
works by abstracting each argument of an inductive instance by a variable
and constraining it by equalities afterwards. This way, the usual 
{\tt induction} and {\tt destruct} tactics can be applied to the
abstracted instance and after simplification of the equalities we get
the expected goals.

The abstracting tactic is called {\tt generalize\_eqs} and it takes as
argument an hypothesis to generalize. It uses the {\tt JMeq} datatype
defined in {\tt Coq.Logic.JMeq}, hence we need to require it before.
For example, revisiting the first example of the inversion documentation above:

\begin{coq_example*}
Require Import Coq.Logic.JMeq.
\end{coq_example*}
\begin{coq_eval}
Require Import Coq.Program.Equality.
\end{coq_eval}

\begin{coq_eval}
Inductive Le : nat -> nat -> Set :=
  | LeO : forall n:nat, Le 0 n
  | LeS : forall n m:nat, Le n m -> Le (S n) (S m).
Variable P : nat -> nat -> Prop.
Variable Q : forall n m:nat, Le n m -> Prop.
\end{coq_eval}

\begin{coq_example*}
Goal forall n m:nat, Le (S n) m -> P n m.
intros n m H.
\end{coq_example*}
\begin{coq_example}
generalize_eqs H.
\end{coq_example}

The index {\tt S n} gets abstracted by a variable here, but a
corresponding equality is added under the abstract instance so that no
information is actually lost. The goal is now almost amenable to do induction
or case analysis. One should indeed first move {\tt n} into the goal to
strengthen it before doing induction, or {\tt n} will be fixed in
the inductive hypotheses (this does not matter for case analysis). 
As a rule of thumb, all the variables that appear inside constructors in
the indices of the hypothesis should be generalized. This is exactly
what the \texttt{generalize\_eqs\_vars} variant does:

\begin{coq_eval} 
Undo 1.
\end{coq_eval}
\begin{coq_example}
generalize_eqs_vars H.
induction H.
\end{coq_example}

As the hypothesis itself did not appear in the goal, we did not need to
use an heterogeneous equality to relate the new hypothesis to the old
one (which just disappeared here). However, the tactic works just as well
in this case, e.g.:

\begin{coq_eval}
Admitted.
\end{coq_eval}

\begin{coq_example}
Goal forall n m (p : Le (S n) m), Q (S n) m p.
intros n m p ; generalize_eqs_vars p.
\end{coq_example}

One drawback of this approach is that in the branches one will have to
substitute the equalities back into the instance to get the right
assumptions. Sometimes injection of constructors will also be needed to
recover the needed equalities. Also, some subgoals should be directly
solved because of inconsistent contexts arising from the constraints on 
indexes. The nice thing is that we can make a tactic based on
discriminate, injection and variants of substitution to automatically 
do such simplifications (which may involve the K axiom). 
This is what the {\tt simplify\_dep\_elim} tactic from
{\tt Coq.Program.Equality} does. For example, we might simplify the
previous goals considerably:
% \begin{coq_eval} 
% Abort.
% Goal forall n m:nat, Le (S n) m -> P n m.
% intros n m H ; generalize_eqs_vars H.
% \end{coq_eval}

\begin{coq_example}
induction p ; simplify_dep_elim.
\end{coq_example}

The higher-order tactic {\tt do\_depind} defined in {\tt
  Coq.Program.Equality} takes a tactic and combines the
building blocks we have seen with it: generalizing by equalities
calling the given tactic with the
generalized induction hypothesis as argument and cleaning the subgoals
with respect to equalities. Its most important instantiations are
\depind and \depdestr that do induction or simply case analysis on the
generalized hypothesis. For example we can redo what we've done manually
with \depdestr:

\begin{coq_eval}
Abort.
\end{coq_eval}
\begin{coq_example*}
Require Import Coq.Program.Equality.
Lemma ex : forall n m:nat, Le (S n) m -> P n m.
intros n m H.
\end{coq_example*}
\begin{coq_example}
dependent destruction H.
\end{coq_example}
\begin{coq_eval}
Abort.
\end{coq_eval}

This gives essentially the same result as inversion. Now if the
destructed hypothesis actually appeared in the goal, the tactic would
still be able to invert it, contrary to {\tt dependent
 inversion}. Consider the following example on vectors:

\begin{coq_example*}
Require Import Coq.Program.Equality.
Set Implicit Arguments.
Variable A : Set.
Inductive vector : nat -> Type := 
| vnil : vector 0 
| vcons : A -> forall n, vector n -> vector (S n).
Goal forall n, forall v : vector (S n), 
  exists v' : vector n, exists a : A, v = vcons a v'.
  intros n v.
\end{coq_example*}
\begin{coq_example}
  dependent destruction v.
\end{coq_example}
\begin{coq_eval}
Abort.
\end{coq_eval}

In this case, the {\tt v} variable can be replaced in the goal by the
generalized hypothesis only when it has a type of the form {\tt vector
 (S n)}, that is only in the second case of the {\tt destruct}. The
first one is dismissed because {\tt S n <> 0}.

\subsection{A larger example}

Let's see how the technique works with {\tt induction} on inductive
predicates on a real example. We will develop an example application to the
theory of simply-typed lambda-calculus formalized in a dependently-typed style:

\begin{coq_example*}
Inductive type : Type :=
| base : type
| arrow : type -> type -> type.
Notation " t --> t' " := (arrow t t') (at level 20, t' at next level).
Inductive ctx : Type :=
| empty : ctx
| snoc : ctx -> type -> ctx.
Notation " G , tau " := (snoc G tau) (at level 20, tau at next level).
Fixpoint conc (G D : ctx) : ctx :=
  match D with
    | empty => G
    | snoc D' x => snoc (conc G D') x
  end.
Notation " G ; D " := (conc G D) (at level 20).
Inductive term : ctx -> type -> Type :=
| ax : forall G tau, term (G, tau) tau
| weak : forall G tau, 
  term G tau -> forall tau', term (G, tau') tau
| abs : forall G tau tau', 
  term (G , tau) tau' -> term G (tau --> tau')
| app : forall G tau tau', 
  term G (tau --> tau') -> term G tau -> term G tau'.
\end{coq_example*}

We have defined types and contexts which are snoc-lists of types. We
also have a {\tt conc} operation that concatenates two contexts.
The {\tt term} datatype represents in fact the possible typing
derivations of the calculus, which are isomorphic to the well-typed
terms, hence the name. A term is either an application of:
\begin{itemize}
\item the axiom rule to type a reference to the first variable in a context,
\item the weakening rule to type an object in a larger context
\item the abstraction or lambda rule to type a function
\item the application to type an application of a function to an argument
\end{itemize}

Once we have this datatype we want to do proofs on it, like weakening:

\begin{coq_example*}
Lemma weakening : forall G D tau, term (G ; D) tau -> 
  forall tau', term (G , tau' ; D) tau.
\end{coq_example*}
\begin{coq_eval}
  Abort.
\end{coq_eval}

The problem here is that we can't just use {\tt induction} on the typing
derivation because it will forget about the {\tt G ; D} constraint
appearing in the instance. A solution would be to rewrite the goal as:
\begin{coq_example*}
Lemma weakening' : forall G' tau, term G' tau -> 
  forall G D, (G ; D) = G' ->
  forall tau', term (G, tau' ; D) tau.
\end{coq_example*}
\begin{coq_eval}
  Abort.
\end{coq_eval}

With this proper separation of the index from the instance and the right
induction loading (putting {\tt G} and {\tt D} after the inducted-on
hypothesis), the proof will go through, but it is a very tedious
process. One is also forced to make a wrapper lemma to get back the
more natural statement. The \depind tactic alleviates this trouble by
doing all of this plumbing of generalizing and substituting back automatically.
Indeed we can simply write:

\begin{coq_example*}
Require Import Coq.Program.Tactics.
Lemma weakening : forall G D tau, term (G ; D) tau -> 
  forall tau', term (G , tau' ; D) tau.
Proof with simpl in * ; simpl_depind ; auto.
  intros G D tau H. dependent induction H generalizing G D ; intros.
\end{coq_example*}

This call to \depind has an additional arguments which is a list of
variables appearing in the instance that should be generalized in the
goal, so that they can vary in the induction hypotheses. By default, all
variables appearing inside constructors (except in a parameter position)
of the instantiated hypothesis will be generalized automatically but
one can always give the list explicitly.

\begin{coq_example}
  Show.
\end{coq_example}

The {\tt simpl\_depind} tactic includes an automatic tactic that tries
to simplify equalities appearing at the beginning of induction
hypotheses, generally using trivial applications of
reflexivity. In cases where the equality is not between constructor
forms though, one must help the automation by giving
some arguments, using the {\tt specialize} tactic for example.

\begin{coq_example*}
destruct D... apply weak ; apply ax. apply ax.
destruct D...
\end{coq_example*}
\begin{coq_example}
Show.
\end{coq_example}
\begin{coq_example}
  specialize (IHterm G0 empty eq_refl).
\end{coq_example}

Once the induction hypothesis has been narrowed to the right equality,
it can be used directly. 

\begin{coq_example}
  apply weak, IHterm.
\end{coq_example}

If there is an easy first-order solution to these equations as in this subgoal, the
{\tt specialize\_eqs} tactic can be used instead of giving explicit proof
terms:

\begin{coq_example}
  specialize_eqs IHterm.
\end{coq_example}
This concludes our example.
\SeeAlso The induction \ref{elim}, case \ref{case} and inversion \ref{inversion} tactics.

\section[\tt autorewrite]{\tt autorewrite\label{autorewrite-example}}

Here are two examples of {\tt autorewrite} use. The first one ({\em Ackermann
function}) shows actually a quite basic use where there is no conditional
rewriting. The second one ({\em Mac Carthy function}) involves conditional
rewritings and shows how to deal with them using the optional tactic of the
{\tt Hint~Rewrite} command.

\firstexample
\example{Ackermann function}
%Here is a basic use of {\tt AutoRewrite} with the Ackermann function:

\begin{coq_example*}
Reset Initial.
Require Import Arith.
Variable Ack : 
           nat -> nat -> nat.
Axiom Ack0 : 
        forall m:nat, Ack 0 m = S m.
Axiom Ack1 : forall n:nat, Ack (S n) 0 = Ack n 1.
Axiom Ack2 : forall n m:nat, Ack (S n) (S m) = Ack n (Ack (S n) m).
\end{coq_example*}

\begin{coq_example}
Hint Rewrite Ack0 Ack1 Ack2 : base0.
Lemma ResAck0 : 
 Ack 3 2 = 29.
autorewrite with base0 using try reflexivity.
\end{coq_example}

\begin{coq_eval}
Reset Initial.
\end{coq_eval}

\example{Mac Carthy function}
%The Mac Carthy function shows a more complex case:

\begin{coq_example*}
Require Import Omega.
Variable g :   
           nat -> nat -> nat.
Axiom g0 : 
        forall m:nat, g 0 m = m.
Axiom
  g1 :
    forall n m:nat,
      (n > 0) -> (m > 100) -> g n m = g (pred n) (m - 10).
Axiom
  g2 :
    forall n m:nat,
      (n > 0) -> (m <= 100) -> g n m = g (S n) (m + 11).
\end{coq_example*}

\begin{coq_example}
Hint Rewrite g0 g1 g2 using omega : base1.
Lemma Resg0 : 
 g 1 110 = 100.
autorewrite with base1 using reflexivity || simpl.
\end{coq_example}

\begin{coq_eval}
Abort.
\end{coq_eval}

\begin{coq_example}
Lemma Resg1 : g 1 95 = 91.
autorewrite with base1 using reflexivity || simpl.
\end{coq_example}

\begin{coq_eval}
Reset Initial.
\end{coq_eval}

\section[\tt quote]{\tt quote\tacindex{quote}
\label{quote-examples}}

The tactic \texttt{quote} allows using Barendregt's so-called
2-level approach without writing any ML code. Suppose you have a
language \texttt{L} of 
'abstract terms' and a type \texttt{A} of 'concrete terms' 
and a function \texttt{f : L -> A}. If \texttt{L} is a simple
inductive datatype and \texttt{f} a simple fixpoint, \texttt{quote f}
will replace the head of current goal by a convertible term of the form 
\texttt{(f t)}. \texttt{L} must have a constructor of type: \texttt{A
  -> L}. 

Here is an example:

\begin{coq_example}
Require Import Quote.
Parameters A B C : Prop.
Inductive formula : Type :=
  | f_and : formula -> formula -> formula (* binary constructor *)
  | f_or : formula -> formula -> formula
  | f_not : formula -> formula (* unary constructor *)
  | f_true : formula (* 0-ary constructor *)
  | f_const : Prop -> formula (* constructor for constants *).
Fixpoint interp_f (f:
                   formula) : Prop :=
  match f with
  | f_and f1 f2 => interp_f f1 /\ interp_f f2
  | f_or f1 f2 => interp_f f1 \/ interp_f f2
  | f_not f1 => ~ interp_f f1
  | f_true => True
  | f_const c => c
  end.
Goal A /\ (A \/ True) /\ ~ B /\ (A <-> A).
quote interp_f.
\end{coq_example}

The algorithm to perform this inversion is: try to match the
term with right-hand sides expression of \texttt{f}. If there is a
match, apply the corresponding left-hand side and call yourself
recursively on sub-terms. If there is no match, we are at a leaf:
return the corresponding constructor (here \texttt{f\_const}) applied
to the term. 

\begin{ErrMsgs}
\item \errindex{quote: not a simple fixpoint} \\
  Happens when \texttt{quote} is not able to perform inversion properly.
\end{ErrMsgs}

\subsection{Introducing variables map}

The normal use of \texttt{quote} is to make proofs by reflection: one
defines a function \texttt{simplify : formula -> formula} and proves a 
theorem \texttt{simplify\_ok: (f:formula)(interp\_f (simplify f)) ->
  (interp\_f f)}. Then, one can simplify formulas by doing:
\begin{verbatim}
   quote interp_f.
   apply simplify_ok.
   compute.
\end{verbatim}
But there is a problem with leafs: in the example above one cannot
write a function that implements, for example, the logical simplifications 
$A \land A \ra A$ or $A \land \lnot A \ra \texttt{False}$. This is
because the \Prop{} is impredicative.

It is better to use that type of formulas:

\begin{coq_eval}
Reset formula.
\end{coq_eval}
\begin{coq_example}
Inductive formula : Set :=
  | f_and : formula -> formula -> formula
  | f_or : formula -> formula -> formula
  | f_not : formula -> formula
  | f_true : formula
  | f_atom : index -> formula.
\end{coq_example*}

\texttt{index} is defined in module \texttt{quote}. Equality on that
type is decidable so we are able to simplify $A \land A$ into $A$ at
the abstract level. 

When there are variables, there are bindings, and \texttt{quote}
provides also a type \texttt{(varmap A)} of bindings from
\texttt{index} to any set \texttt{A}, and a function
\texttt{varmap\_find} to search in such maps. The interpretation
function has now another argument, a variables map:

\begin{coq_example}
Fixpoint interp_f (vm:
                    varmap Prop) (f:formula) {struct f} : Prop :=
  match f with
  | f_and f1 f2 => interp_f vm f1 /\ interp_f vm f2
  | f_or f1 f2 => interp_f vm f1 \/ interp_f vm f2
  | f_not f1 => ~ interp_f vm f1
  | f_true => True
  | f_atom i => varmap_find True i vm
  end.
\end{coq_example}

\noindent\texttt{quote} handles this second case properly:

\begin{coq_example}
Goal A /\ (B \/ A) /\ (A \/ ~ B).
quote interp_f.
\end{coq_example}

It builds \texttt{vm} and \texttt{t} such that \texttt{(f vm t)} is
convertible with the conclusion of current goal.

\subsection{Combining variables and constants}

One can have both variables and constants in abstracts terms; that is
the case, for example, for the \texttt{ring} tactic (chapter
\ref{ring}). Then one must provide to \texttt{quote} a list of
\emph{constructors of constants}. For example, if the list is
\texttt{[O S]} then closed natural numbers will be considered as
constants and other terms as variables. 

Example: 

\begin{coq_eval}
Reset formula.
\end{coq_eval}
\begin{coq_example*}
Inductive formula : Type :=
  | f_and : formula -> formula -> formula
  | f_or : formula -> formula -> formula
  | f_not : formula -> formula
  | f_true : formula
  | f_const : Prop -> formula (* constructor for constants *)
  | f_atom : index -> formula.
Fixpoint interp_f
 (vm:            (* constructor for variables *)
  varmap Prop) (f:formula) {struct f} : Prop :=
  match f with
  | f_and f1 f2 => interp_f vm f1 /\ interp_f vm f2
  | f_or f1 f2 => interp_f vm f1 \/ interp_f vm f2
  | f_not f1 => ~ interp_f vm f1
  | f_true => True
  | f_const c => c
  | f_atom i => varmap_find True i vm
  end.
Goal 
A /\ (A \/ True) /\ ~ B /\ (C <-> C).
\end{coq_example*}

\begin{coq_example}
quote interp_f [ A B ].
Undo.
  quote interp_f [ B C iff ].
\end{coq_example}

\Warning Since function inversion
is undecidable in general case, don't expect miracles from it!

\begin{Variants}

\item {\tt quote {\ident} in {\term} using {\tac}}

  \tac\ must be a functional tactic (starting with {\tt fun x =>})
  and will be called with the quoted version of \term\ according to
  \ident.

\item {\tt quote {\ident} [ \ident$_1$ \dots\ \ident$_n$ ] in {\term} using {\tac}}

  Same as above, but will use \ident$_1$, \dots, \ident$_n$ to
  chose which subterms are constants (see above).

\end{Variants}

% \SeeAlso file \texttt{theories/DEMOS/DemoQuote.v}

\SeeAlso comments of source file \texttt{plugins/quote/quote.ml}

\SeeAlso the \texttt{ring} tactic (Chapter~\ref{ring})



\section{Using the tactical language}

\subsection{About the cardinality of the set of natural numbers}

A first example which shows how to use the pattern matching over the proof
contexts is the proof that natural numbers have more than two elements. The
proof of such a lemma can be done as %shown on Figure~\ref{cnatltac}.
follows:
%\begin{figure}
%\begin{centerframe}
\begin{coq_eval}
Reset Initial.
Require Import Arith.
Require Import List.
\end{coq_eval}
\begin{coq_example*}
Lemma card_nat :
 ~ (exists x : nat, exists y : nat, forall z:nat, x = z \/ y = z).
Proof.
red; intros (x, (y, Hy)).
elim (Hy 0); elim (Hy 1); elim (Hy 2); intros;
 match goal with
 | [_:(?a = ?b),_:(?a = ?c) |- _ ] =>
     cut (b = c); [ discriminate | transitivity a; auto ]
 end.
Qed.
\end{coq_example*}
%\end{centerframe}
%\caption{A proof on cardinality of natural numbers}
%\label{cnatltac}
%\end{figure}

We can notice that all the (very similar) cases coming from the three
eliminations (with three distinct natural numbers) are successfully solved by
a {\tt match goal} structure and, in particular, with only one pattern (use
of non-linear matching).

\subsection{Permutation on closed lists}

Another more complex example is the problem of permutation on closed lists. The
aim is to show that a closed list is a permutation of another one.

First, we define the permutation predicate as shown in table~\ref{permutpred}.

\begin{figure}
\begin{centerframe}
\begin{coq_example*}
Section Sort.
Variable A : Set.
Inductive permut : list A -> list A -> Prop :=
  | permut_refl   : forall l, permut l l
  | permut_cons   :
      forall a l0 l1, permut l0 l1 -> permut (a :: l0) (a :: l1)
  | permut_append : forall a l, permut (a :: l) (l ++ a :: nil)
  | permut_trans  :
      forall l0 l1 l2, permut l0 l1 -> permut l1 l2 -> permut l0 l2.
End Sort.
\end{coq_example*}
\end{centerframe}
\caption{Definition of the permutation predicate}
\label{permutpred}
\end{figure}

A more complex example is the problem of permutation on closed lists.
The aim is to show that a closed list is a permutation of another one.
First, we define the permutation predicate as shown on
Figure~\ref{permutpred}.

\begin{figure}
\begin{centerframe}
\begin{coq_example}
Ltac Permut n :=
  match goal with
  | |- (permut _ ?l ?l) => apply permut_refl
  | |- (permut _ (?a :: ?l1) (?a :: ?l2)) =>
      let newn := eval compute in (length l1) in
      (apply permut_cons; Permut newn)
  | |- (permut ?A (?a :: ?l1) ?l2) =>
      match eval compute in n with
      | 1 => fail
      | _ =>
          let l1' := constr:(l1 ++ a :: nil) in
          (apply (permut_trans A (a :: l1) l1' l2);
            [ apply permut_append | compute; Permut (pred n) ])
      end
  end.
Ltac PermutProve :=
  match goal with
  | |- (permut _ ?l1 ?l2) =>
      match eval compute in (length l1 = length l2) with
      | (?n = ?n) => Permut n
      end
  end.
\end{coq_example}
\end{centerframe}
\caption{Permutation tactic}
\label{permutltac}
\end{figure}

Next, we can write naturally the tactic and the result can be seen on
Figure~\ref{permutltac}. We can notice that we use two toplevel
definitions {\tt PermutProve} and {\tt Permut}. The function to be
called is {\tt PermutProve} which computes the lengths of the two
lists and calls {\tt Permut} with the length if the two lists have the
same length. {\tt Permut} works as expected.  If the two lists are
equal, it concludes. Otherwise, if the lists have identical first
elements, it applies {\tt Permut} on the tail of the lists.  Finally,
if the lists have different first elements, it puts the first element
of one of the lists (here the second one which appears in the {\tt
  permut} predicate) at the end if that is possible, i.e., if the new
first element has been at this place previously. To verify that all
rotations have been done for a list, we use the length of the list as
an argument for {\tt Permut} and this length is decremented for each
rotation down to, but not including, 1 because for a list of length
$n$, we can make exactly $n-1$ rotations to generate at most $n$
distinct lists. Here, it must be noticed that we use the natural
numbers of {\Coq} for the rotation counter. On Figure~\ref{ltac}, we
can see that it is possible to use usual natural numbers but they are
only used as arguments for primitive tactics and they cannot be
handled, in particular, we cannot make computations with them. So, a
natural choice is to use {\Coq} data structures so that {\Coq} makes
the computations (reductions) by {\tt eval compute in} and we can get
the terms back by {\tt match}.
 
With {\tt PermutProve}, we can now prove lemmas as 
% shown on Figure~\ref{permutlem}.
follows:
%\begin{figure}
%\begin{centerframe}

\begin{coq_example*}
Lemma permut_ex1 :
  permut nat (1 :: 2 :: 3 :: nil) (3 :: 2 :: 1 :: nil).
Proof. PermutProve. Qed.
Lemma permut_ex2 :
  permut nat
    (0 :: 1 :: 2 :: 3 :: 4 :: 5 :: 6 :: 7 :: 8 :: 9 :: nil)
    (0 :: 2 :: 4 :: 6 :: 8 :: 9 :: 7 :: 5 :: 3 :: 1 :: nil).
Proof. PermutProve. Qed.
\end{coq_example*}
%\end{centerframe}
%\caption{Examples of {\tt PermutProve} use}
%\label{permutlem}
%\end{figure}


\subsection{Deciding intuitionistic propositional logic}

\begin{figure}[b]
\begin{centerframe}
\begin{coq_example}
Ltac Axioms :=
  match goal with
  | |- True => trivial
  | _:False |- _  => elimtype False; assumption
  | _:?A |- ?A  => auto
  end.
\end{coq_example}
\end{centerframe}
\caption{Deciding intuitionistic propositions (1)}
\label{tautoltaca}
\end{figure}


\begin{figure}
\begin{centerframe}
\begin{coq_example}
Ltac DSimplif :=
  repeat
   (intros;
    match goal with
     | id:(~ _) |- _ => red in id
     | id:(_ /\ _) |- _ =>
         elim id; do 2 intro; clear id
     | id:(_ \/ _) |- _ =>
         elim id; intro; clear id
     | id:(?A /\ ?B -> ?C) |- _ =>
         cut (A -> B -> C);
          [ intro | intros; apply id; split; assumption ]
     | id:(?A \/ ?B -> ?C) |- _ =>
         cut (B -> C);
          [ cut (A -> C);
             [ intros; clear id
             | intro; apply id; left; assumption ]
          | intro; apply id; right; assumption ]
     | id0:(?A -> ?B),id1:?A |- _ =>
         cut B; [ intro; clear id0 | apply id0; assumption ]
     | |- (_ /\ _) => split
     | |- (~ _) => red
     end).
Ltac TautoProp :=
  DSimplif;
   Axioms ||
     match goal with
     | id:((?A -> ?B) -> ?C) |- _ =>
          cut (B -> C);
          [ intro; cut (A -> B);
             [ intro; cut C;
                [ intro; clear id | apply id; assumption ]
             | clear id ]
          | intro; apply id; intro; assumption ]; TautoProp
     | id:(~ ?A -> ?B) |- _ =>
         cut (False -> B);
          [ intro; cut (A -> False);
             [ intro; cut B;
                [ intro; clear id | apply id; assumption ]
             | clear id ]
          | intro; apply id; red; intro; assumption ]; TautoProp
     | |- (_ \/ _) => (left; TautoProp) || (right; TautoProp)
     end.
\end{coq_example}
\end{centerframe}
\caption{Deciding intuitionistic propositions (2)}
\label{tautoltacb}
\end{figure}

The pattern matching on goals allows a complete and so a powerful
backtracking when returning tactic values. An interesting application
is the problem of deciding intuitionistic propositional logic.
Considering the contraction-free sequent calculi {\tt LJT*} of
Roy~Dyckhoff (\cite{Dyc92}), it is quite natural to code such a tactic
using the tactic language as shown on Figures~\ref{tautoltaca}
and~\ref{tautoltacb}. The tactic {\tt Axioms} tries to conclude using
usual axioms. The tactic {\tt DSimplif} applies all the reversible
rules of Dyckhoff's system. Finally, the tactic {\tt TautoProp} (the
main tactic to be called) simplifies with {\tt DSimplif}, tries to
conclude with {\tt Axioms} and tries several paths using the
backtracking rules (one of the four Dyckhoff's rules for the left
implication to get rid of the contraction and the right or).

For example, with {\tt TautoProp}, we can prove tautologies like
 those:
% on Figure~\ref{tautolem}.
%\begin{figure}[tbp]
%\begin{centerframe}
\begin{coq_example*}
Lemma tauto_ex1 : forall A B:Prop, A /\ B -> A \/ B.
Proof. TautoProp. Qed.
Lemma tauto_ex2 :
   forall A B:Prop, (~ ~ B -> B) -> (A -> B) -> ~ ~ A -> B.
Proof. TautoProp. Qed.
\end{coq_example*}
%\end{centerframe}
%\caption{Proofs of tautologies with {\tt TautoProp}}
%\label{tautolem}
%\end{figure}

\subsection{Deciding type isomorphisms}

A more tricky problem is to decide equalities between types and modulo
isomorphisms. Here, we choose to use the isomorphisms of the simply typed
$\lb{}$-calculus with Cartesian product and $unit$ type (see, for example,
\cite{RC95}). The axioms of this $\lb{}$-calculus are given by
table~\ref{isosax}.

\begin{figure}
\begin{centerframe}
\begin{coq_eval}
Reset Initial.
\end{coq_eval}
\begin{coq_example*}
Open Scope type_scope.
Section Iso_axioms.
Variables A B C : Set.
Axiom Com : A * B = B * A.
Axiom Ass : A * (B * C) = A * B * C.
Axiom Cur : (A * B -> C) = (A -> B -> C).
Axiom Dis : (A -> B * C) = (A -> B) * (A -> C).
Axiom P_unit : A * unit = A.
Axiom AR_unit : (A -> unit) = unit.
Axiom AL_unit : (unit -> A) = A.
Lemma Cons : B = C -> A * B = A * C.
Proof.
intro Heq; rewrite Heq; reflexivity.
Qed.
End Iso_axioms.
\end{coq_example*}
\end{centerframe}
\caption{Type isomorphism axioms}
\label{isosax}
\end{figure}

A more tricky problem is to decide equalities between types and modulo
isomorphisms. Here, we choose to use the isomorphisms of the simply typed
$\lb{}$-calculus with Cartesian product and $unit$ type (see, for example,
\cite{RC95}). The axioms of this $\lb{}$-calculus are given on
Figure~\ref{isosax}.

\begin{figure}[ht]
\begin{centerframe}
\begin{coq_example}
Ltac DSimplif trm :=
  match trm with
  | (?A * ?B * ?C) =>
      rewrite <- (Ass A B C); try MainSimplif
  | (?A * ?B -> ?C) =>
      rewrite (Cur A B C); try MainSimplif
  | (?A -> ?B * ?C) =>
      rewrite (Dis A B C); try MainSimplif
  | (?A * unit) =>
      rewrite (P_unit A); try MainSimplif
  | (unit * ?B) =>
      rewrite (Com unit B); try MainSimplif
  | (?A -> unit) =>
      rewrite (AR_unit A); try MainSimplif
  | (unit -> ?B) =>
      rewrite (AL_unit B); try MainSimplif
  | (?A * ?B) =>
      (DSimplif A; try MainSimplif) || (DSimplif B; try MainSimplif)
  | (?A -> ?B) =>
      (DSimplif A; try MainSimplif) || (DSimplif B; try MainSimplif)
  end
 with MainSimplif :=
  match goal with
  | |- (?A = ?B) => try DSimplif A; try DSimplif B
  end.
Ltac Length trm :=
  match trm with
  | (_ * ?B) => let succ := Length B in constr:(S succ)
  | _ => constr:(1)
  end.
Ltac assoc := repeat rewrite <- Ass.
\end{coq_example}
\end{centerframe}
\caption{Type isomorphism tactic (1)}
\label{isosltac1}
\end{figure}

\begin{figure}[ht]
\begin{centerframe}
\begin{coq_example}
Ltac DoCompare n :=
  match goal with
  | [ |- (?A = ?A) ] => reflexivity
  | [ |- (?A * ?B = ?A * ?C) ] =>
      apply Cons; let newn := Length B in
                  DoCompare newn
  | [ |- (?A * ?B = ?C) ] =>
      match eval compute in n with
      | 1 => fail
      | _ =>
          pattern (A * B) at 1; rewrite Com; assoc; DoCompare (pred n)
      end
  end.
Ltac CompareStruct :=
  match goal with
  | [ |- (?A = ?B) ] =>
      let l1 := Length A
      with l2 := Length B in
      match eval compute in (l1 = l2) with
      | (?n = ?n) => DoCompare n
      end
  end.
Ltac IsoProve := MainSimplif; CompareStruct.
\end{coq_example}
\end{centerframe}
\caption{Type isomorphism tactic (2)}
\label{isosltac2}
\end{figure}

The tactic to judge equalities modulo this axiomatization can be written as
shown on Figures~\ref{isosltac1} and~\ref{isosltac2}. The algorithm is quite
simple. Types are reduced using axioms that can be oriented (this done by {\tt
MainSimplif}). The normal forms are sequences of Cartesian
products without Cartesian product in the left component. These normal forms
are then compared modulo permutation of the components (this is done by {\tt
CompareStruct}). The main tactic to be called and realizing this algorithm is
{\tt IsoProve}.

% Figure~\ref{isoslem} gives 
Here are examples of what can be solved by {\tt IsoProve}.
%\begin{figure}[ht]
%\begin{centerframe}
\begin{coq_example*}
Lemma isos_ex1 : 
  forall A B:Set, A * unit * B = B * (unit * A).
Proof.
intros; IsoProve.
Qed.

Lemma isos_ex2 :
  forall A B C:Set,
    (A * unit -> B * (C * unit)) =
    (A * unit -> (C -> unit) * C) * (unit -> A -> B).
Proof.
intros; IsoProve.
Qed.
\end{coq_example*}
%\end{centerframe}
%\caption{Type equalities solved by {\tt IsoProve}}
%\label{isoslem}
%\end{figure}

%%% Local Variables: 
%%% mode: latex
%%% TeX-master: "Reference-Manual"
%%% End: 
