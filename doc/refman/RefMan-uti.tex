\chapter[Utilities]{Utilities\label{Utilities}}

The distribution provides utilities to simplify some tedious works
beside proof development, tactics writing or documentation.

\section[Building a toplevel extended with user tactics]{Building a toplevel extended with user tactics\label{Coqmktop}\ttindex{coqmktop}}

The native-code version of \Coq\ cannot dynamically load user tactics
using {\ocaml} code. It is possible to build a toplevel of \Coq,
with {\ocaml} code statically linked, with the tool {\tt
  coqmktop}.

For example, one can build a native-code \Coq\ toplevel extended with a tactic
which source is in {\tt tactic.ml} with the command
\begin{verbatim}
     % coqmktop -opt -o mytop.out tactic.cmx
\end{verbatim}
where {\tt tactic.ml} has been compiled with the native-code
compiler {\tt ocamlopt}. This command generates an executable
called {\tt mytop.out}. To use this executable to compile your \Coq\
files, use {\tt coqc -image mytop.out}.

A basic example is the native-code version of \Coq\ ({\tt coqtop.opt}),
which can be generated by {\tt coqmktop -opt -o coqopt.opt}.


\paragraph[Application: how to use the {\ocaml} debugger with Coq.]{Application: how to use the {\ocaml} debugger with Coq.\index{Debugger}}

One useful application of \texttt{coqmktop} is to build a \Coq\ toplevel in
order to debug your tactics with the {\ocaml} debugger.
You need to have configured and compiled \Coq\ for debugging
(see the file \texttt{INSTALL} included in the distribution).
Then, you must compile the Caml modules of your tactic with the
option \texttt{-g} (with the bytecode compiler) and build a stand-alone
bytecode toplevel with the following command:

\begin{quotation}
\texttt{\% coqmktop -g -o coq-debug}~\emph{<your \texttt{.cmo} files>}
\end{quotation}


To launch the \ocaml\ debugger with the image you need to execute it in
an environment which correctly sets the \texttt{COQLIB} variable.
Moreover, you have to indicate the directories in which
\texttt{ocamldebug} should search for Caml modules.

A possible solution is to use a wrapper around \texttt{ocamldebug}
which detects the executables containing the word \texttt{coq}. In
this case, the debugger is called with the required additional
arguments. In other cases, the debugger is simply called without additional
arguments. Such a wrapper can be found in the \texttt{dev/}
subdirectory of the sources.

\section[Modules dependencies]{Modules dependencies\label{Dependencies}\index{Dependencies}
  \ttindex{coqdep}}

In order to compute modules dependencies (so to use {\tt make}),
\Coq\ comes with an appropriate tool, {\tt coqdep}.

{\tt coqdep} computes inter-module dependencies for \Coq\ and
\ocaml\ programs, and prints the dependencies on the standard
output in a format readable by make.  When a directory is given as
argument, it is recursively looked at.

Dependencies of \Coq\ modules are computed by looking at {\tt Require}
commands ({\tt Require}, {\tt Requi\-re Export}, {\tt Require Import},
but also at the command {\tt Declare ML Module}.

Dependencies of \ocaml\ modules are computed by looking at
\verb!open! commands and the dot notation {\em module.value}. However,
this is done approximately and you are advised to use {\tt ocamldep}
instead for the \ocaml\ modules dependencies.

See the man page of {\tt coqdep} for more details and options.


\section[Creating a {\tt Makefile} for \Coq\ modules]
{Creating a {\tt Makefile} for \Coq\ modules
\label{Makefile}
\ttindex{Makefile}
\ttindex{coq\_Makefile}
\ttindex{\_CoqProject}}

A project is a proof development split into several files, possibly
including the sources of some {\ocaml} plugins, that are located (in
various sub-directories of) a certain directory. The
\texttt{coq\_makefile} command allows to generate generic and complete
\texttt{Makefile} files, that can be used to compile the different
components of the project. A \_CoqProject file
specifies both the list of target files relevant to the project
and the common options that should be passed to each executable at
compilation times, for the IDE, etc.

\paragraph{\_CoqProject file as an argument to  coq\_Makefile.}
In particular, a \_CoqProject file contains the relevant
arguments to be passed to the \texttt{coq\_makefile} makefile
generator using for instance the command:

\begin{quotation}
\texttt{\% coq\_makefile -f \_CoqProject -o Makefile}
\end{quotation}

This command generates a file \texttt{Makefile} that can be used to
compile all the sources of the current project. It follows the
syntax described by the output of \texttt{\% coq\_makefile ----help}.
Once the \texttt{Makefile} file has been generated a first time, it
can be used by the \texttt{make} command to compile part or all of
the project. Note that once it has been generated once, as soon as
\texttt{\_CoqProject} file is updated, the \texttt{Makefile} file is
automatically regenerated by an invocation of \texttt{make}.

The following command generates a minimal example of
\texttt{\_CoqProject} file:
\begin{quotation}
\texttt{\% \{ echo '-R .} \textit{MyFancyLib} \texttt{' ; find . -name
  '*.v' -print \} > \_CoqProject}
\end{quotation}
when executed at the root of the directory containing the
project. Here the \texttt{\_CoqProject} lists all the \texttt{.v} files
that are present in the current directory and its sub-directories. But no
plugin sources is listed. If a \texttt{Makefile} is generated from
this \texttt{\_CoqProject}, the command \texttt{make install} will
install the compiled project in a sub-directory \texttt{MyFancyLib} of
the \texttt{user-contrib} directory  (of the user's {\Coq} libraries
location). This sub-directory is created if it does not already exist.

\paragraph{\_CoqProject file as a configuration for IDEs.}

A \texttt{\_CoqProject} file can also be used to configure the options
of the \texttt{coqtop} process executed by a user interface. This
allows to import the libraries of the project under a correct name,
both as a developer of the project, working in the directory
containing its sources, and as a user, using the project after
the installation of its libraries. Currently, both \CoqIDE{} and Proof
General (version $\geq$ 4.3pre) support configuration via
\texttt{\_CoqProject} files.

\paragraph{Remarks.}

\begin{itemize}
\item Every {\Coq} files must use a \texttt{.v} file extension.
 The {\ocaml} modules must use a \texttt{.ml4} file extension
 if they require camlp preprocessing (and in \texttt{.ml} otherwise).
 The {\ocaml} module signatures, library
 description and packing files must use respectively \texttt{.mli},
 \texttt{.mllib} and \texttt{.mlpack} file extension.

\item Any argument that is not a valid option is considered as a
  sub-directory. Any sub-directory specified in the list of targets must
  itself contain a \texttt{Makefile}.

\item The phony targets \texttt{all} and \texttt{clean} recursively
  call their target in every sub-directory.

\item \texttt{-R} and \texttt{-Q} options are for {\Coq} files, \texttt{-I}
  for {\ocaml} ones. A same directory can be passed to both nature of
  options, in the same \texttt{\_CoqProject}.

\item Using \texttt{-R} or \texttt{-Q} is the appropriate way to
  obtain both a correct logical path and a correct installation location to
  the libraries of a given project.

\item Dependencies on external libraries to the project must be
  declared with care. If in the \texttt{\_CoqProject} file an external
  library \texttt{foo} is passed to a \texttt{-Q} option, like in
  \texttt{-Q foo}, the subsequent \textit{make clean} command can
  erase \textit{foo}. It is hence safer to customize the
  \texttt{COQPATH} variable (see \ref{envars}), to include the
  location of the required external libraries.

\item Using \texttt{-extra-phony} with no command adds extra
  dependencies on already defined rules. For example the following
  skeleton appends ``something'' to the \texttt{install} rule:
\begin{quotation}
\texttt{-extra-phony "install" "install-my-stuff" ""
  -extra-phony "install-my-stuff" "" "something"}
\end{quotation}
\end{itemize}


\section[Documenting \Coq\ files with coqdoc]{Documenting \Coq\ files with coqdoc\label{coqdoc}
\ttindex{coqdoc}}


%\newcommand{\Coq}{\textsf{Coq}}
\newcommand{\javadoc}{\textsf{javadoc}}
\newcommand{\ocamldoc}{\textsf{ocamldoc}}
\newcommand{\coqdoc}{\textsf{coqdoc}}
\newcommand{\texmacs}{\TeX{}macs}
\newcommand{\monurl}[1]{#1}
%HEVEA\renewcommand{\monurl}[1]{\ahref{#1}{#1}}
%\newcommand{\lnot}{not} % Hevea handles these symbols nicely
%\newcommand{\lor}{or}
%\newcommand{\land}{\&}
%%% attention : -- dans un argument de \texttt est affich� comme un
%%% seul - d'o� l'utilisation de la macro suivante
\newcommand{\mm}{\symbol{45}\symbol{45}}


\coqdoc\ is a documentation tool for the proof assistant
\Coq, similar to \javadoc\ or \ocamldoc. 
The task of \coqdoc\ is
\begin{enumerate}
\item to produce a nice \LaTeX\ and/or HTML document from the \Coq\ 
  sources, readable for a human and not only for the proof assistant;
\item to help the user navigating in his own (or third-party) sources.
\end{enumerate}


%%%%%%%%%%%%%%%%%%%%%%%%%%%%%%%%%%%%%%%%%%%%%%%%%%%%%%%%%%%%

\subsection{Principles}

Documentation is inserted into \Coq\ files as \emph{special comments}.  
Thus your files will compile as usual, whether you use \coqdoc\ or not.
\coqdoc\ presupposes that the given \Coq\ files are well-formed (at
least lexically).  Documentation starts with
\texttt{(**}, followed by a space, and ends with the pending \texttt{*)}. 
The documentation format is inspired
  by Todd~A.~Coram's \emph{Almost Free Text (AFT)} tool: it is mainly
ASCII text with some syntax-light controls, described below.
\coqdoc\ is robust: it shouldn't fail, whatever the input is. But
remember: ``garbage in, garbage out''.

\paragraph{\Coq\ material inside documentation.}
\Coq\ material is quoted between the
delimiters \texttt{[} and \texttt{]}. Square brackets may be nested,
the inner ones being understood as being part of the quoted code (thus
you can quote a term like $[x:T]u$ by writing
\texttt{[[x:T]u]}). Inside quotations, the code is pretty-printed in
the same way as it is in code parts.

Pre-formatted vernacular is enclosed by \texttt{[[} and
\texttt{]]}. The former must be followed by a newline and the latter
must follow a newline.

\paragraph{Pretty-printing.}
\coqdoc\ uses different faces for identifiers and keywords.  
The pretty-printing of \Coq\ tokens (identifiers or symbols) can be
controlled using one of the following commands:
\begin{alltt}
(** printing \emph{token} %...\LaTeX...% #...HTML...# *)
\end{alltt}
or
\begin{alltt}
(** printing \emph{token} $...\LaTeX\ math...$ #...HTML...# *)
\end{alltt}
It gives the \LaTeX\ and HTML texts to be produced for the given \Coq\
token. One of the \LaTeX\ or HTML text may be ommitted, causing the
default pretty-printing to be used for this token.

The printing for one token can be removed with
\begin{alltt}
(** remove printing \emph{token} *)
\end{alltt}

Initially, the pretty-printing table contains the following mapping:
\begin{center}
  \begin{tabular}{ll@{\qquad\qquad}ll@{\qquad\qquad}ll@{\qquad\qquad}}
    \verb!->!            & $\rightarrow$   &
    \verb!<-!            & $\leftarrow$    &
    \verb|*|             & $\times$        \\
    \verb|<=|            & $\le$           &
    \verb|>=|            & $\ge$           &
    \verb|=>|            & $\Rightarrow$   \\
    \verb|<>|            & $\not=$         &
    \verb|<->|           & $\leftrightarrow$ &
    \verb!|-!            & $\vdash$        \\
    \verb|\/|            & $\lor$          &
    \verb|/\|            & $\land$         &
    \verb|~|             & $\lnot$ 
  \end{tabular}
\end{center}
Any of these can be overwritten or suppressed using the
\texttt{printing} commands.

Important note: the recognition of tokens is done by a (ocaml)lex
automaton and thus applies the longest-match rule. For instance,
\verb!->~! is recognized as a single token, where \Coq\ sees two
tokens. It is the responsability of the user to insert space between
tokens \emph{or} to give pretty-printing rules for the possible
combinations, e.g. 
\begin{verbatim}
(** printing ->~ %\ensuremath{\rightarrow\lnot}% *)
\end{verbatim}


\paragraph{Sections.}
Sections are introduced by 1 to 4 leading stars (i.e. at the beginning of the
line) followed by a space. One star is a section, two stars a sub-section, etc.
The section title is given on the remaining of the line.
Example:
\begin{verbatim}
    (** * Well-founded relations
  
        In this section, we introduce...  *)
\end{verbatim}


%TODO \paragraph{Fonts.}


\paragraph{Lists.}
List items are introduced by 1 to 4 leading dashes.
Deepness of the list is indicated by the number of dashes.
List ends with a blank line.
Example:
\begin{verbatim}
    This module defines
        - the predecessor [pred]
        - the addition [plus]
        - order relations:
          -- less or equal [le]
          -- less [lt]
\end{verbatim}

\paragraph{Rules.}
More than 4 leading dashes produce an horizontal rule.


\paragraph{Escapings to \LaTeX\ and HTML.}
Pure \LaTeX\ or HTML material can be inserted using the following
escape sequences:
\begin{itemize}
\item \verb+$...LaTeX stuff...$+ inserts some \LaTeX\ material in math mode.
  Simply discarded in HTML output.

\item \verb+%...LaTeX stuff...%+ inserts some \LaTeX\ material.
  Simply discarded in HTML output.

\item \verb+#...HTML stuff...#+ inserts some HTML material. Simply
  discarded in \LaTeX\ output.
\end{itemize}


\paragraph{Verbatim.} 
Verbatim material is introduced by a leading \verb+<<+ and closed by
\verb+>>+ at the beginning of a line. Example:
\begin{verbatim}
Here is the corresponding caml code:
<<
  let rec fact n = 
    if n <= 1 then 1 else n * fact (n-1)
>>
\end{verbatim}


\paragraph{Hyperlinks.}
Hyperlinks can be inserted into the HTML output, so that any
identifier is linked to the place of its definition.

In order to get hyperlinks you need to first compile your \Coq\ file
using \texttt{coqc \mm{}dump-glob \emph{file}}; this appends 
\Coq\ names resolutions done during the compilation to file
\texttt{\emph{file}}. Take care of erasing this file, if any, when
starting the whole compilation process.

Then invoke \texttt{coqdoc \mm{}glob-from \emph{file}} to tell
\coqdoc\ to look for name resolutions into the file \texttt{\emph{file}}.

Identifiers from the \Coq\ standard library are linked to the \Coq\
web site at \url{http://coq.inria.fr/library/}. This behavior can be
changed using command line options \url{--no-externals} and
\url{--coqlib}; see below.


\paragraph{Hiding / Showing parts of the source.}
Some parts of the source can be hidden using command line options
\texttt{-g} and \texttt{-l} (see below), or using such comments:
\begin{alltt}
(* begin hide *)
\emph{some Coq material}
(* end hide *)
\end{alltt}
Conversely, some parts of the source which would be hidden can be
shown using such comments: 
\begin{alltt}
(* begin show *)
\emph{some Coq material}
(* end show *)
\end{alltt}
The latter cannot be used around some inner parts of a proof, but can
be used around a whole proof.


%%%%%%%%%%%%%%%%%%%%%%%%%%%%%%%%%%%%%%%%%%%%%%%%%%%%%%%%%%%%

\subsection{Usage}

\coqdoc\ is invoked on a shell command line as follows:
\begin{displaymath}
  \texttt{coqdoc }<\textit{options and files}>
\end{displaymath}
Any command line argument which is not an option is considered to be a
file (even if it starts with a \verb!-!). \Coq\ files are identified
by the suffixes \verb!.v! and \verb!.g! and \LaTeX\ files by the
suffix \verb!.tex!. 

\begin{description}
\item[HTML output] ~\par
  This is the default output.
  One HTML file is created for each \Coq\ file given on the command line,
  together with a file \texttt{index.html} (unless option
  \texttt{-no-index} is passed). The HTML pages use a style sheet
  named \texttt{style.css}. Such a file is distributed with \coqdoc.

\item[\LaTeX\ output] ~\par
  A single \LaTeX\ file is created, on standard output. It can be
  redirected to a file with option \texttt{-o}. 
  The order of files on the command line is kept in the final
  document. \LaTeX\ files given on the command line are copied `as is'
  in the final document .
  DVI and PostScript can be produced directly with the options
  \texttt{-dvi} and \texttt{-ps} respectively.

\item[\texmacs\ output] ~\par
  To translate the input files to \texmacs\ format, to be used by
  the \texmacs\ Coq interface 
  (see \url{http://www-sop.inria.fr/lemme/Philippe.Audebaud/tmcoq/}).
\end{description}


\subsubsection*{Command line options}


\paragraph{Overall options}

\begin{description}

\item[\texttt{\mm{}html}] ~\par
  
  Select a HTML output.

\item[\texttt{\mm{}latex}] ~\par
  
  Select a \LaTeX\ output.

\item[\texttt{\mm{}dvi}] ~\par
  
  Select a DVI output.

\item[\texttt{\mm{}ps}] ~\par
  
  Select a PostScript output.

\item[\texttt{\mm{}texmacs}] ~\par
  
  Select a \texmacs\ output.

\item[\texttt{--stdout}] ~\par

  Write output to stdout.

\item[\texttt{-o }\textit{file}, \texttt{\mm{}output }\textit{file}] ~\par
  
  Redirect the output into the file `\textit{file}' (meaningless with
  \texttt{-html}).

\item[\texttt{-d }\textit{dir}, \texttt{\mm{}directory }\textit{dir}] ~\par

  Output files into directory `\textit{dir}' instead of current
  directory (option \texttt{-d} does not change the filename specified
  with option \texttt{-o}, if any).

\item[\texttt{-s }, \texttt{\mm{}short}] ~\par
  
  Do not insert titles for the files. The default behavior is to
  insert a title like ``Library Foo'' for each file.

\item[\texttt{-t }\textit{string}, 
      \texttt{\mm{}title }\textit{string}] ~\par
  
  Set the document title.      

\item[\texttt{\mm{}body-only}] ~\par

  Suppress the header and trailer of the final document. Thus, you can
  insert the resulting document into a larger one.

\item[\texttt{-p} \textit{string}, \texttt{\mm{}preamble} \textit{string}]~\par

  Insert some material in the \LaTeX\ preamble, right before
  \verb!\begin{document}! (meaningless with \texttt{-html}).

\item[\texttt{\mm{}vernac-file }\textit{file},
      \texttt{\mm{}tex-file }\textit{file}] ~\par
      
      Considers the file `\textit{file}' respectively as a \verb!.v!
      (or \verb!.g!) file or a \verb!.tex! file.

\item[\texttt{\mm{}files-from }\textit{file}] ~\par

  Read file names to process in file `\textit{file}' as if they were
  given on the command line. Useful for program sources splitted in
  several directories.
  
\item[\texttt{-q}, \texttt{\mm{}quiet}] ~\par

  Be quiet. Do not print anything except errors.

\item[\texttt{-h}, \texttt{\mm{}help}] ~\par

  Give a short summary of the options and exit.

\item[\texttt{-v}, \texttt{\mm{}version}] ~\par

  Print the version and exit.

\end{description}

\paragraph{Index options} ~\par

Default behavior is to build an index, for the HTML output only, into
\texttt{index.html}.

\begin{description}

\item[\texttt{\mm{}no-index}] ~\par
  
  Do not output the index.

\item[\texttt{\mm{}multi-index}] ~\par
  
  Generate one page for each category and each letter in the index,
  together with a top page \texttt{index.html}.

\end{description}

\paragraph{Table of contents option} ~\par

\begin{description}

\item[\texttt{-toc}, \texttt{\mm{}table-of-contents}] ~\par

  Insert a table of contents.
  For a \LaTeX\ output, it inserts a \verb!\tableofcontents! at the
  beginning of the document. For a HTML output, it builds a table of
  contents into \texttt{toc.html}.

\end{description}

\paragraph{Hyperlinks options}
\begin{description}

\item[\texttt{\mm{}glob-from }\textit{file}] ~\par
  
  Make references using \Coq\ globalizations from file \textit{file}. 
  (Such globalizations are obtained with \Coq\ option \texttt{-dump-glob}).

\item[\texttt{\mm{}no-externals}] ~\par
  
  Do not insert links to the \Coq\ standard library.

\item[\texttt{\mm{}coqlib }\textit{url}] ~\par

  Set base URL for the \Coq\ standard library (default is 
  \url{http://coq.inria.fr/library/}).

\item[\texttt{-R }\textit{dir }\textit{coqdir}] ~\par

  Map physical directory \textit{dir} to \Coq\ logical directory
  \textit{coqdir} (similarly to \Coq\ option \texttt{-R}).

  Note: option \texttt{-R} only has effect on the files
  \emph{following} it on the command line, so you will probably need
  to put this option first.

\end{description}

\paragraph{Contents options}
\begin{description}

\item[\texttt{-g}, \texttt{\mm{}gallina}] ~\par

  Do not print proofs.

\item[\texttt{-l}, \texttt{\mm{}light}] ~\par
  
  Light mode. Suppress proofs (as with \texttt{-g}) and the following commands:
  \begin{itemize}
  \item {}[\texttt{Recursive}] \texttt{Tactic Definition}
  \item \texttt{Hint / Hints} 
  \item \texttt{Require}
  \item \texttt{Transparent / Opaque}
  \item \texttt{Implicit Argument / Implicits}
  \item \texttt{Section / Variable / Hypothesis / End}
  \end{itemize}

\end{description}
The behavior of options \texttt{-g} and \texttt{-l} can be locally
overridden using the \texttt{(* begin show *)} \dots\ \texttt{(* end
  show *)} environment (see above).

\paragraph{Language options} ~\par

Default behavior is to assume ASCII 7 bits input files.

\begin{description}

\item[\texttt{-latin1}, \texttt{\mm{}latin1}] ~\par

  Select ISO-8859-1 input files. It is equivalent to
  \texttt{--inputenc latin1 --charset iso-8859-1}.

\item[\texttt{-utf8}, \texttt{\mm{}utf8}] ~\par

  Select UTF-8 (Unicode) input files. It is equivalent to
  \texttt{--inputenc utf8 --charset utf-8}.
  \LaTeX\ UTF-8 support can be found at
 \url{http://www.ctan.org/tex-archive/macros/latex/contrib/supported/unicode/}.

\item[\texttt{\mm{}inputenc} \textit{string}] ~\par

  Give a \LaTeX\ input encoding, as an option to \LaTeX\ package
  \texttt{inputenc}. 

\item[\texttt{\mm{}charset} \textit{string}] ~\par

  Specify the HTML character set, to be inserted in the HTML header.

\end{description}


%%%%%%%%%%%%%%%%%%%%%%%%%%%%%%%%%%%%%%%%%%%%%%%%%%%%%%%%%%%%

\subsection[The coqdoc \LaTeX{} style file]{The coqdoc \LaTeX{} style file\label{section:coqdoc.sty}}

In case you choose to produce a document without the default \LaTeX{}
preamble (by using option \verb|--no-preamble|), then you must insert
into your own preamble the command
\begin{quote}
  \verb|\usepackage{coqdoc}|
\end{quote}
Then you may alter the rendering of the document by
redefining some macros:
\begin{description}

\item[\texttt{coqdockw}, \texttt{coqdocid}] ~ 
  
  The one-argument macros for typesetting keywords and identifiers.
  Defaults are sans-serif for keywords and italic for identifiers.

  For example, if you would like a slanted font for keywords, you
  may insert  
\begin{verbatim}
     \renewcommand{\coqdockw}[1]{\textsl{#1}}
\end{verbatim}
  anywhere between \verb|\usepackage{coqdoc}| and
  \verb|\begin{document}|. 

\item[\texttt{coqdocmodule}] ~ 
  
  One-argument macro for typesetting the title of a \verb|.v| file.
  Default is
\begin{verbatim}
\newcommand{\coqdocmodule}[1]{\section*{Module #1}}
\end{verbatim}
  and you may redefine it using \verb|\renewcommand|.

\end{description}




\section[Embedded \Coq\ phrases inside \LaTeX\ documents]{Embedded \Coq\ phrases inside \LaTeX\ documents\label{Latex}
  \ttindex{coq-tex}\index{Latex@{\LaTeX}}}

When writing a documentation about a proof development, one may want
to insert \Coq\ phrases inside a \LaTeX\ document, possibly together with
the corresponding answers of the system. We provide a
mechanical way to process such \Coq\ phrases embedded in \LaTeX\ files: the
{\tt coq-tex} filter.  This filter extracts Coq phrases embedded in
LaTeX files, evaluates them, and insert the outcome of the evaluation
after each phrase.

Starting with a file {\em file}{\tt.tex} containing \Coq\ phrases,
the {\tt coq-tex} filter produces a file named {\em file}{\tt.v.tex} with
the \Coq\ outcome.

There are options to produce the \Coq\ parts in smaller font, italic,
between horizontal rules, etc.
See the man page of {\tt coq-tex} for more details.

\medskip\noindent {\bf Remark.} This Reference Manual and the Tutorial
have been completely produced with {\tt coq-tex}.


\section[\Coq\ and \emacs]{\Coq\ and \emacs\label{Emacs}\index{Emacs}}

\subsection{The \Coq\ Emacs mode}

\Coq\ comes with a Major mode for \emacs, {\tt gallina.el}. This mode provides
syntax highlighting
and also a rudimentary indentation facility
in the style of the Caml \emacs\ mode.

Add the following lines to your \verb!.emacs! file:

\begin{verbatim}
  (setq auto-mode-alist (cons '("\\.v$" . coq-mode) auto-mode-alist))
  (autoload 'coq-mode "gallina" "Major mode for editing Coq vernacular." t)
\end{verbatim}

The \Coq\ major mode is triggered by visiting a file with extension {\tt .v},
or manually with the command \verb!M-x coq-mode!.
It gives you the correct syntax table for
the \Coq\ language, and also a rudimentary indentation facility:
\begin{itemize}
  \item pressing {\sc Tab} at the beginning of a line indents the line like
    the line above;

  \item extra {\sc Tab}s increase the indentation level
    (by 2 spaces by default);

  \item M-{\sc Tab} decreases the indentation level.
\end{itemize}

An inferior mode to run \Coq\ under Emacs, by Marco Maggesi, is also
included in the distribution, in file \texttt{coq-inferior.el}.
Instructions to use it are contained in this file.

\subsection[{\ProofGeneral}]{{\ProofGeneral}\index{Proof General@{\ProofGeneral}}}

{\ProofGeneral} is a generic interface for proof assistants based on
Emacs. The main idea is that the \Coq\ commands you are
editing are sent to a \Coq\ toplevel running behind Emacs and the
answers of the system automatically inserted into other Emacs buffers.
Thus you don't need to copy-paste the \Coq\ material from your files
to the \Coq\ toplevel or conversely from the \Coq\ toplevel to some
files.

{\ProofGeneral} is developed and distributed independently of the
system \Coq. It is freely available at \verb!proofgeneral.inf.ed.ac.uk!.


\section[Module specification]{Module specification\label{gallina}\ttindex{gallina}}

Given a \Coq\ vernacular file, the {\tt gallina} filter extracts its
specification (inductive types declarations, definitions, type of
lemmas and theorems), removing the proofs parts of the file. The \Coq\
file {\em file}{\tt.v} gives birth to the specification file
{\em file}{\tt.g} (where the suffix {\tt.g} stands for \gallina).

See the man page of {\tt gallina} for more details and options.


\section[Man pages]{Man pages\label{ManPages}\index{Man pages}}

There are man pages for the commands {\tt coqdep}, {\tt gallina} and
{\tt coq-tex}. Man pages are installed at installation time
(see installation instructions in file {\tt INSTALL}, step 6).

%BEGIN LATEX
\RefManCutCommand{ENDREFMAN=\thepage}
%END LATEX

%%% Local Variables:
%%% mode: latex
%%% TeX-master: t
%%% End:
