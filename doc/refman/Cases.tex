\achapter{Extended pattern-matching}
%HEVEA\cutname{cases.html}
%BEGIN LATEX
\defaultheaders
%END LATEX
\aauthor{Cristina Cornes and Hugo Herbelin}

\label{Mult-match-full}
\ttindex{Cases}
\index{ML-like patterns}

This section describes the full form of pattern-matching in {\Coq} terms.

\asection{Patterns}\label{implementation} The full syntax of {\tt
match} is presented in Figures~\ref{term-syntax}
and~\ref{term-syntax-aux}.  Identifiers in patterns are either
constructor names or variables. Any identifier that is not the
constructor of an inductive or co-inductive type is considered to be a
variable. A variable name cannot occur more than once in a given
pattern. It is recommended to start variable names by a lowercase
letter.

If a pattern has the form $(c~\vec{x})$ where $c$ is a constructor
symbol and $\vec{x}$ is a linear vector of (distinct) variables, it is
called {\em simple}: it is the kind of pattern recognized by the basic
version of {\tt match}. On the opposite, if it is a variable $x$ or
has the form $(c~\vec{p})$ with $p$ not only made of variables, the
pattern is called {\em nested}.

A variable pattern matches any value, and the identifier is bound to
that value. The pattern ``\texttt{\_}'' (called ``don't care'' or
``wildcard'' symbol) also matches any value, but does not bind
anything. It may occur an arbitrary number of times in a
pattern. Alias patterns written \texttt{(}{\sl pattern} \texttt{as}
{\sl identifier}\texttt{)} are also accepted. This pattern matches the
same values as {\sl pattern} does and {\sl identifier} is bound to the
matched value.
A pattern of the form {\pattern}{\tt |}{\pattern} is called
disjunctive. A list of patterns separated with commas is also
considered as a pattern and is called {\em multiple pattern}. However
multiple patterns can only occur at the root of pattern-matching
equations. Disjunctions of {\em multiple pattern} are allowed though.

Since extended {\tt match} expressions are compiled into the primitive
ones, the expressiveness of the theory remains the same. Once the
stage of parsing has finished only simple patterns remain.  Re-nesting
of pattern is performed at printing time. An easy way to see the
result of the expansion is to toggle off the nesting performed at
printing (use here {\tt Set Printing Matching}), then by printing the term
with \texttt{Print} if the term is a constant, or using the command
\texttt{Check}.

The extended \texttt{match} still accepts an optional {\em elimination
predicate} given after the keyword \texttt{return}.  Given a pattern
matching expression, if all the right-hand-sides of \texttt{=>} ({\em
rhs} in short) have the same type, then this type can be sometimes
synthesized, and so we can omit the \texttt{return} part. Otherwise
the predicate after \texttt{return} has to be provided, like for the basic
\texttt{match}.

Let us illustrate through examples the different aspects of extended
pattern matching. Consider for example the function that computes the
maximum of two natural numbers. We can write it in primitive syntax
by:

\begin{coq_example}
Fixpoint max (n m:nat) {struct m} : nat :=
  match n with
  | O => m
  | S n' => match m with
            | O => S n'
            | S m' => S (max n' m')
            end
  end.
\end{coq_example}

\paragraph{Multiple patterns}

Using multiple patterns in the definition of {\tt max} lets us write:

\begin{coq_eval}
Reset max.
\end{coq_eval}
\begin{coq_example}
Fixpoint max (n m:nat) {struct m} : nat :=
  match n, m with
  | O, _ => m
  | S n', O => S n'
  | S n', S m' => S (max n' m')
  end.
\end{coq_example}

which will be compiled into the previous form.

The pattern-matching compilation strategy examines patterns from left
to right. A \texttt{match} expression is generated {\bf only} when
there is at least one constructor in the column of patterns. E.g. the
following example does not build a \texttt{match} expression.

\begin{coq_example}
Check (fun x:nat => match x return nat with
                    | y => y
                    end).
\end{coq_example}

\paragraph{Aliasing subpatterns}

We can also use ``\texttt{as} {\ident}'' to associate a name to a
sub-pattern:

\begin{coq_eval}
Reset max.
\end{coq_eval}
\begin{coq_example}
Fixpoint max (n m:nat) {struct n} : nat :=
  match n, m with
  | O, _ => m
  | S n' as p, O => p
  | S n', S m' => S (max n' m')
  end.
\end{coq_example}

\paragraph{Nested patterns}

Here is now an example of nested patterns:

\begin{coq_example}
Fixpoint even (n:nat) : bool :=
  match n with
  | O => true
  | S O => false
  | S (S n') => even n'
  end.
\end{coq_example}

This is compiled into:

\begin{coq_example}
Unset Printing Matching.
Print even.
\end{coq_example}
\begin{coq_eval}
Set Printing Matching.
\end{coq_eval}

In the previous examples patterns do not conflict with, but
sometimes it is comfortable to write patterns that admit a non
trivial superposition. Consider
the boolean function \texttt{lef} that given two natural numbers
yields \texttt{true} if the first one is less or equal than the second
one and \texttt{false} otherwise. We can write it as follows:

\begin{coq_example}
Fixpoint lef (n m:nat) {struct m} : bool :=
  match n, m with
  | O, x => true
  | x, O => false
  | S n, S m => lef n m
  end.
\end{coq_example}

Note that the first and the second multiple pattern superpose because
the couple of values \texttt{O O} matches both. Thus, what is the result
of the function on those values?  To eliminate ambiguity we use the
{\em textual priority rule}: we consider patterns ordered from top to
bottom, then a value is matched by the pattern at the $ith$ row if and
only if it is not matched by some pattern of a previous row. Thus in the
example,
\texttt{O O} is matched by the first pattern, and so \texttt{(lef O O)}
yields \texttt{true}.

Another way to write  this function is:

\begin{coq_eval}
Reset lef.
\end{coq_eval}
\begin{coq_example}
Fixpoint lef (n m:nat) {struct m} : bool :=
  match n, m with
  | O, x => true
  | S n, S m => lef n m
  | _, _ => false
  end.
\end{coq_example}

Here the last pattern superposes with the first two. Because
of the priority rule, the last pattern
will be used only for values that do not match neither the  first nor
the second one.

Terms with useless patterns are not accepted by the
system. Here is an example:
% Test failure: "This clause is redundant."
\begin{coq_eval}
Set Printing Depth 50.
\end{coq_eval}
\begin{coq_example}
Fail Check (fun x:nat =>
         match x with
         | O => true
         | S _ => false
         | x => true
         end).
\end{coq_example}

\paragraph{Disjunctive patterns}

Multiple patterns that share the same right-hand-side can be
factorized using the notation \nelist{\multpattern}{\tt |}. For instance,
{\tt max} can be rewritten as follows:

\begin{coq_eval}
Reset max.
\end{coq_eval}
\begin{coq_example}
Fixpoint max (n m:nat) {struct m} : nat :=
  match n, m with
  | S n', S m' => S (max n' m')
  | 0, p | p, 0 => p
  end.
\end{coq_example}

Similarly, factorization of (non necessary multiple) patterns
that share the same variables is possible by using the notation
\nelist{\pattern}{\tt |}. Here is an example:

\begin{coq_example}
Definition filter_2_4 (n:nat) : nat :=
  match n with
  | 2 as m | 4 as m => m
  | _ => 0
  end.
\end{coq_example}

Here is another example using disjunctive subpatterns.

\begin{coq_example}
Definition filter_some_square_corners (p:nat*nat) : nat*nat :=
  match p with
  | ((2 as m | 4 as m), (3 as n | 5 as n)) => (m,n)
  | _ => (0,0)
  end.
\end{coq_example}

\asection{About patterns of parametric types}
\paragraph{Parameters in patterns}
When matching objects of a parametric type, parameters do not bind in patterns.
They must be substituted by ``\_''.
Consider for example the type of polymorphic lists:

\begin{coq_example}
Inductive List (A:Set) : Set :=
  | nil : List A
  | cons : A -> List A -> List A.
\end{coq_example}

We can check the function {\em tail}:

\begin{coq_example}
Check
  (fun l:List nat =>
     match l with
     | nil _ => nil nat
     | cons _ _ l' => l'
     end).
\end{coq_example}


When we use parameters in patterns there is an error message:
% Test failure: "The parameters do not bind in patterns."
\begin{coq_eval}
Set Printing Depth 50.
\end{coq_eval}
\begin{coq_example}
Fail Check
  (fun l:List nat =>
     match l with
     | nil A => nil nat
     | cons A _ l' => l'
     end).
\end{coq_example}

The option {\tt Set Asymmetric Patterns} \optindex{Asymmetric Patterns}
(off by default) removes parameters from constructors in patterns:
\begin{coq_example}
  Set Asymmetric Patterns.
  Check (fun l:List nat =>
    match l with
    | nil => nil
    | cons _ l' => l'
    end)
  Unset Asymmetric Patterns.
\end{coq_example}

\paragraph{Implicit arguments in patterns}
By default, implicit arguments are omitted in patterns. So we write:

\begin{coq_example}
Arguments nil [A].
Arguments cons [A] _ _.
Check
  (fun l:List nat =>
     match l with
     | nil => nil
     | cons _ l' => l'
     end).
\end{coq_example}

But the possibility to use all the arguments is given by ``{\tt @}'' implicit
explicitations (as for terms~\ref{Implicits-explicitation}).

\begin{coq_example}
Check
  (fun l:List nat =>
     match l with
     | @nil _ => @nil nat
     | @cons _ _ l' => l'
     end).
\end{coq_example}

\asection{Matching objects of dependent types}
The previous examples illustrate pattern matching on objects of
non-dependent types, but we can also
use the expansion strategy to destructure objects of dependent type.
Consider the type \texttt{listn} of lists of a certain length:
\label{listn}

\begin{coq_example}
Inductive listn : nat -> Set :=
  | niln : listn 0
  | consn : forall n:nat, nat -> listn n -> listn (S n).
\end{coq_example}

\asubsection{Understanding dependencies in patterns}
We can define the function \texttt{length} over \texttt{listn} by:

\begin{coq_example}
Definition length (n:nat) (l:listn n) := n.
\end{coq_example}

Just for illustrating pattern matching,
we can define it by case analysis:

\begin{coq_eval}
Reset length.
\end{coq_eval}
\begin{coq_example}
Definition length (n:nat) (l:listn n) :=
  match l with
  | niln => 0
  | consn n _ _ => S n
  end.
\end{coq_example}

We can understand the meaning of this definition using the
same notions of usual pattern matching.

%
% Constraining of dependencies is not longer valid in V7
%
\iffalse
Now suppose we split the second pattern  of \texttt{length} into two
cases so to give an
alternative definition using nested patterns:
\begin{coq_example}
Definition length1 (n:nat) (l:listn n) :=
  match l with
  | niln => 0
  | consn n _ niln => S n
  | consn n _ (consn _ _ _) => S n
  end.
\end{coq_example}

It is obvious that \texttt{length1} is  another version of
\texttt{length}. We can also give the following definition:
\begin{coq_example}
Definition length2 (n:nat) (l:listn n) :=
  match l with
  | niln => 0
  | consn n _ niln => 1
  | consn n _ (consn m _ _) => S (S m)
  end.
\end{coq_example}

If we forget that \texttt{listn} is a dependent type and we read these
definitions using the usual semantics of pattern matching,  we can conclude
that \texttt{length1}
and \texttt{length2} are different functions.
In fact, they are equivalent
because the pattern \texttt{niln} implies that \texttt{n} can only match
the value $0$ and analogously the pattern \texttt{consn} determines that \texttt{n} can
only match  values of the form  $(S~v)$ where $v$ is the value matched by
\texttt{m}.

The converse is also true. If
we destructure the  length  value with the pattern \texttt{O} then the list
value should be $niln$.
Thus, the following term \texttt{length3} corresponds to the function
\texttt{length} but this time defined by case analysis on the dependencies instead of on the list:

\begin{coq_example}
Definition length3 (n:nat) (l:listn n) :=
  match l with
  | niln => 0
  | consn O _ _ => 1
  | consn (S n) _ _ => S (S n)
  end.
\end{coq_example}

When we have nested patterns of dependent types, the semantics of
pattern matching becomes a little more difficult because
the set of values that are matched by a sub-pattern may be conditioned by the
values matched by another sub-pattern. Dependent nested patterns are
somehow constrained patterns.
In the examples, the expansion of
\texttt{length1} and \texttt{length2} yields exactly the same term
 but the
expansion of \texttt{length3} is completely different. \texttt{length1} and
\texttt{length2} are expanded into two nested case analysis on
\texttt{listn} while \texttt{length3} is expanded into a case analysis on
\texttt{listn} containing a case analysis on natural numbers inside.


In practice the user can think about the patterns as independent and
it is the expansion algorithm that cares to relate them. \\
\fi
%
%
%

\asubsection{When the elimination predicate must be provided}
\paragraph{Dependent pattern matching}
The examples  given so far do not need an explicit elimination predicate
 because all the rhs have the same type and the
strategy succeeds to synthesize it.
Unfortunately when dealing with dependent patterns it often happens
that we need to write cases where the type of the rhs are
different  instances of the elimination  predicate.
The function  \texttt{concat} for \texttt{listn}
is an example where the branches have different type
and we need to provide the elimination predicate:

\begin{coq_example}
Fixpoint concat (n:nat) (l:listn n) (m:nat) (l':listn m) {struct l} :
 listn (n + m) :=
  match l in listn n return listn (n + m) with
  | niln => l'
  | consn n' a y => consn (n' + m) a (concat n' y m l')
  end.
\end{coq_example}
The elimination predicate is {\tt fun (n:nat) (l:listn n) => listn~(n+m)}.
In general if $m$ has type {\tt (}$I$ $q_1$ {\ldots} $q_r$ $t_1$ {\ldots} $t_s${\tt )} where
$q_1$, {\ldots}, $q_r$ are parameters, the elimination predicate should be of
the form~:
{\tt fun} $y_1$ {\ldots} $y_s$ $x${\tt :}($I$~$q_1$ {\ldots} $q_r$ $y_1$ {\ldots}
  $y_s${\tt ) =>} $Q$.

In the concrete syntax, it should be written~:
\[ \kw{match}~m~\kw{as}~x~\kw{in}~(I~\_~\mbox{\ldots}~\_~y_1~\mbox{\ldots}~y_s)~\kw{return}~Q~\kw{with}~\mbox{\ldots}~\kw{end}\]

The variables which appear in the \kw{in} and \kw{as} clause are new
and bounded in the property $Q$ in the \kw{return} clause. The
parameters of the inductive definitions should not be mentioned and
are replaced by \kw{\_}.

\paragraph{Multiple dependent pattern matching}
Recall that a list of patterns is also a pattern. So, when we destructure several
terms at the same time and the branches have different types we need to provide the
elimination predicate for this multiple pattern. It is done using the same
scheme, each term may be associated to an \kw{as} and \kw{in} clause in order to
introduce a dependent product.

For example, an equivalent definition for \texttt{concat} (even though the
matching on the second term is trivial) would have been:

\begin{coq_eval}
Reset concat.
\end{coq_eval}
\begin{coq_example}
Fixpoint concat (n:nat) (l:listn n) (m:nat) (l':listn m) {struct l} :
 listn (n + m) :=
  match l in listn n, l' return listn (n + m) with
  | niln, x => x
  | consn n' a y, x => consn (n' + m) a (concat n' y m x)
  end.
\end{coq_example}

Even without real matching over the second term, this construction can be used to
keep types linked.  If {\tt a} and {\tt b} are two {\tt listn} of the same length,
by writing
\begin{coq_eval}
  Unset Printing Matching.
\end{coq_eval}
\begin{coq_example}
Check (fun n (a b: listn n) => match a,b with
 |niln,b0 => tt
 |consn n' a y, bS => tt
end).
\end{coq_example}
\begin{coq_eval}
  Set Printing Matching.
\end{coq_eval}

I have a copy of {\tt b} in type {\tt listn 0} resp {\tt listn (S n')}.

% Notice that this time, the predicate \texttt{[n,\_:nat](listn (plus n
%   m))}  is binary because we
% destructure both \texttt{l} and \texttt{l'} whose types have arity one.
% In general, if we destructure the terms $e_1\ldots e_n$
% the predicate will be of arity $m$ where $m$ is the sum of the
% number of dependencies of the type of $e_1, e_2,\ldots e_n$
% (the $\lambda$-abstractions
% should correspond from left to right to each dependent argument of the
% type of $e_1\ldots e_n$).
% When the arity of the predicate (i.e. number of abstractions) is not
% correct Coq raises an error message. For example:

% % Test failure
% \begin{coq_eval}
% Reset concat.
% Set Printing Depth 50.
% (********** The following is not correct and should produce ***********)
% (** Error: the term l' has type listn m while it is expected to have **)
% (** type listn (?31 + ?32)                                           **)
% \end{coq_eval}
% \begin{coq_example}
% Fixpoint concat
%  (n:nat) (l:listn n) (m:nat)
%  (l':listn m) {struct l} : listn (n + m) :=
%   match l, l' with
%   | niln, x => x
%   | consn n' a y, x => consn (n' + m) a (concat n' y m x)
%   end.
% \end{coq_example}

\paragraph{Patterns in {\tt in}}
\label{match-in-patterns}

If the type of the matched term is more precise than an inductive applied to
variables, arguments of the inductive in the {\tt in} branch can be more
complicated patterns than a variable.

Moreover, constructors whose type do not follow the same pattern will
become impossible branches. In an impossible branch, you can answer
anything but {\tt False\_rect unit} has the advantage to be subterm of
anything. % ???

To be concrete: the {\tt tail} function can be written:
\begin{coq_example}
Definition tail n (v: listn (S n)) :=
  match v in listn (S m) return listn m with
  | niln => False_rect unit
  | consn n' a y => y
  end.
\end{coq_example}
and {\tt tail n v} will be subterm of {\tt v}.

\asection{Using pattern matching to write proofs}
In all the previous examples the elimination predicate does not depend
on the object(s) matched. But it may depend and the typical case
is when we write a proof by induction or a function that yields an
object of dependent type. An example of proof using \texttt{match} in
given in Section~\ref{refine-example}.

For example, we can write
the function \texttt{buildlist} that given a natural number
$n$ builds a list of length $n$ containing zeros as follows:

\begin{coq_example}
Fixpoint buildlist (n:nat) : listn n :=
  match n return listn n with
  | O => niln
  | S n => consn n 0 (buildlist n)
  end.
\end{coq_example}

We can also use multiple patterns.
Consider the following definition of the predicate less-equal
\texttt{Le}:

\begin{coq_example}
Inductive LE : nat -> nat -> Prop :=
  | LEO : forall n:nat, LE 0 n
  | LES : forall n m:nat, LE n m -> LE (S n) (S m).
\end{coq_example}

We can use multiple patterns to write  the proof of the lemma
 \texttt{forall (n m:nat), (LE n m)}\verb=\/=\texttt{(LE m n)}:

\begin{coq_example}
Fixpoint dec (n m:nat) {struct n} : LE n m \/ LE m n :=
  match n, m return LE n m \/ LE m n with
  | O, x => or_introl (LE x 0) (LEO x)
  | x, O => or_intror (LE x 0) (LEO x)
  | S n as n', S m as m' =>
      match dec n m with
      | or_introl h => or_introl (LE m' n') (LES n m h)
      | or_intror h => or_intror (LE n' m') (LES m n h)
      end
  end.
\end{coq_example}
In the example of \texttt{dec},
the first \texttt{match} is dependent while
the second is not.

% In general, consider the terms $e_1\ldots e_n$,
% where  the type of $e_i$ is an instance of a family type
% $\lb (\vec{d_i}:\vec{D_i}) \mto T_i$  ($1\leq i
% \leq n$). Then, in expression \texttt{match}  $e_1,\ldots,
% e_n$ \texttt{of} \ldots \texttt{end}, the
% elimination predicate ${\cal P}$ should be of the form:
% $[\vec{d_1}:\vec{D_1}][x_1:T_1]\ldots [\vec{d_n}:\vec{D_n}][x_n:T_n]Q.$

The user can also use \texttt{match} in combination with the tactic
\texttt{refine} (see Section~\ref{refine}) to build incomplete proofs
beginning with a \texttt{match} construction.

\asection{Pattern-matching on inductive objects involving local
definitions}

If local definitions occur in the type of a constructor, then there are two ways
to match on this constructor. Either the local definitions are skipped and
matching is done only on the true arguments of the constructors, or the bindings
for local definitions can also be caught in the matching.

Example.

\begin{coq_eval}
Reset Initial.
Require Import Arith.
\end{coq_eval}

\begin{coq_example*}
Inductive list : nat -> Set :=
  | nil : list 0
  | cons : forall n:nat, let m := (2 * n) in list m -> list (S (S m)).
\end{coq_example*}

In the next example, the local definition is not caught.

\begin{coq_example}
Fixpoint length n (l:list n) {struct l} : nat :=
  match l with
  | nil => 0
  | cons n l0 => S (length (2 * n) l0)
  end.
\end{coq_example}

But in this example, it is.

\begin{coq_example}
Fixpoint length' n (l:list n) {struct l} : nat :=
  match l with
  | nil => 0
  | @cons _ m l0 => S (length' m l0)
  end.
\end{coq_example}

\Rem for a given matching clause, either none of the local definitions or all of
them can be caught.

\Rem you can only catch {\tt let} bindings in mode where you bind all variables and so you
have to use @ syntax.

\Rem this feature is incoherent with the fact that parameters cannot be caught and
consequently is somehow hidden. For example, there is no mention of it in error messages.

\asection{Pattern-matching and coercions}

If a mismatch occurs between the expected type of a pattern and its
actual type, a coercion made from constructors is sought. If such a
coercion can be found, it is automatically inserted around the
pattern.

Example:

\begin{coq_example}
Inductive I : Set :=
  | C1 : nat -> I
  | C2 : I -> I.
Coercion C1 : nat >-> I.
Check (fun x => match x with
                | C2 O => 0
                | _ => 0
                end).
\end{coq_example}


\asection{When does the expansion strategy fail ?}\label{limitations}
The strategy works very like in ML languages when treating
patterns of non-dependent type.
But there are new cases of failure that are due to the presence of
dependencies.

The error messages of the current implementation may be sometimes
confusing.  When the tactic fails because patterns are somehow
incorrect then error messages refer to the initial expression. But the
strategy may succeed to build an expression whose sub-expressions are
well typed when the whole expression is not. In this situation the
message makes reference to the expanded expression.  We encourage
users, when they have patterns with the same outer constructor in
different equations, to name the variable patterns in the same
positions with the same name.
E.g. to write {\small\texttt{(cons n O x) => e1}}
and {\small\texttt{(cons n \_ x) => e2}} instead of
{\small\texttt{(cons n O x) => e1}} and
{\small\texttt{(cons n' \_ x') => e2}}.
This helps to maintain certain name correspondence between the
generated expression and the original.

Here is a summary of the error messages corresponding to each situation:

\begin{ErrMsgs}
\item \sverb{The constructor } {\sl
    ident} \sverb{ expects } {\sl num} \sverb{ arguments}

 \sverb{The variable } {\sl ident} \sverb{ is bound several times
    in pattern } {\sl term}

 \sverb{Found a constructor of inductive type } {\term}
 \sverb{ while a constructor of } {\term} \sverb{ is expected}

 Patterns are incorrect (because constructors are not applied to
  the correct number of the arguments, because they are not linear or
  they are wrongly typed).

\item \errindex{Non exhaustive pattern-matching}

The pattern matching is not exhaustive.

\item \sverb{The elimination predicate } {\sl term} \sverb{ should be
    of arity } {\sl num} \sverb{ (for non dependent case) or } {\sl
    num} \sverb{ (for dependent case)}

The elimination predicate provided to \texttt{match} has not the
  expected arity.


%\item the whole expression is wrongly typed

% CADUC ?
% , or the synthesis of
%   implicit arguments fails (for example to find the elimination
%   predicate or to resolve implicit arguments in the rhs).

%   There are {\em nested patterns of dependent type}, the elimination
%   predicate corresponds to non-dependent case and has the form
%   $[x_1:T_1]...[x_n:T_n]T$ and {\bf some} $x_i$ occurs {\bf free} in
%   $T$.  Then, the strategy may fail to find out a correct elimination
%   predicate during some step of compilation.  In this situation we
%   recommend the user to rewrite the nested dependent patterns into
%   several \texttt{match} with {\em simple patterns}.

\item {\tt Unable to infer a match predicate\\
    Either there is a type incompatibility or the problem involves\\
    dependencies}

  There is a type mismatch between the different branches.
  The user should provide an elimination predicate.

% Obsolete ?
% \item because of nested patterns, it may happen that even though all
%   the rhs have the same type, the strategy needs dependent elimination
%   and so an elimination predicate must be provided. The system warns
%   about this situation, trying to compile anyway with the
%   non-dependent strategy. The risen message is:

% \begin{itemize}
% \item {\tt Warning: This pattern matching may need dependent
%     elimination to be compiled.  I will try, but if fails try again
%     giving dependent elimination predicate.}
% \end{itemize}


%%%%%%%%%%%%%%%%%%%%%%%%%%%%%%%%%%%%%%%%%%%%%%%%%%%%%%%%%%%%%%%%%%%%%%
% % LA PROPAGATION DES CONTRAINTES ARRIERE N'EST PAS FAITE DANS LA V7
% TODO
% \item there are {\em nested patterns of dependent type} and the
%   strategy builds a term that is well typed but recursive calls in fix
%   point are reported as illegal:
% \begin{itemize}
% \item {\tt Error: Recursive call applied to an illegal term ...}
% \end{itemize}

% This is because the strategy generates a term that is correct w.r.t.
% the initial term but which does not pass the guard condition.  In
% this situation we recommend the user to transform the nested dependent
% patterns into {\em several \texttt{match} of simple patterns}.  Let us
% explain this with an example.  Consider the following definition of a
% function that yields the last element of a list and \texttt{O} if it is
% empty:

% \begin{coq_example}
%   Fixpoint last [n:nat; l:(listn n)] : nat :=
%    match l of
%      (consn _ a niln) => a
%    | (consn m _ x) => (last m x) | niln => O
%    end.
% \end{coq_example}

% It fails because of the priority between patterns, we know that this
% definition is equivalent to the following more explicit one (which
% fails too):

% \begin{coq_example*}
%   Fixpoint last [n:nat; l:(listn n)] : nat :=
%    match l of
%      (consn _ a niln) => a
%    | (consn n _ (consn m b x)) => (last n (consn m b x))
%    | niln => O
%    end.
% \end{coq_example*}

% Note that the recursive call {\tt (last n (consn m b x))} is not
% guarded. When treating with patterns of dependent types the strategy
% interprets the first definition of \texttt{last} as the second
% one\footnote{In languages of the ML family the first definition would
%   be translated into a term where the variable \texttt{x} is shared in
%   the expression.  When patterns are of non-dependent types, Coq
%   compiles as in ML languages using sharing. When patterns are of
%   dependent types the compilation reconstructs the term as in the
%   second definition of \texttt{last} so to ensure the result of
%   expansion is well typed.}.  Thus it generates a term where the
% recursive call is rejected by the guard condition.

% You can get rid of this problem by writing the definition with
% \emph{simple patterns}:

% \begin{coq_example}
%   Fixpoint last [n:nat; l:(listn n)] : nat :=
%   <[_:nat]nat>match l of
%     (consn m a x) => Cases x of niln => a | _ => (last m x) end
%   | niln => O
%   end.
% \end{coq_example}

\end{ErrMsgs}


%%% Local Variables:
%%% mode: latex
%%% TeX-master: "Reference-Manual"
%%% End:
