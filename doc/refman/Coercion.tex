\achapter{Implicit Coercions}
\aauthor{Amokrane Saïbi}

\label{Coercions-full}
\index{Coercions!presentation}

\asection{General Presentation}

This section describes the inheritance mechanism of {\Coq}. In {\Coq} with
inheritance, we are not interested in adding any expressive power to
our theory, but only convenience. Given a term, possibly not typable,
we are interested in the problem of determining if it can be well
typed modulo insertion of appropriate coercions.  We allow to write:

\begin{itemize}
\item $f~a$ where $f:forall~ x:A, B$ and $a:A'$ when $A'$ can 
      be seen in some sense as a subtype of $A$.
\item $x:A$ when $A$ is not a type, but can be seen in 
      a certain sense as a type: set, group, category etc.
\item $f~a$ when $f$ is not a function, but can be seen in a certain sense
      as a function: bijection, functor, any structure morphism etc.
\end{itemize}

\asection{Classes}
\index{Coercions!classes}
 A class with $n$ parameters is any defined name with a type
$forall~ (x_1:A_1)..(x_n:A_n), s$ where $s$ is a sort.  Thus a class with
parameters is considered as a single class and not as a family of
classes.  An object of a class $C$ is any term of type $C~t_1
.. t_n$.  In addition to these user-classes, we have two abstract
classes:

\begin{itemize}
\item {\tt Sortclass}, the class of sorts; 
  its objects are the terms whose type is a sort.
\item {\tt Funclass}, the class of functions; 
  its objects are all the terms with a functional 
  type, i.e. of form $forall~ x:A, B$.
\end{itemize}

Formally, the syntax of a classes is defined on Figure~\ref{fig:classes}.
\begin{figure}
\begin{centerframe}
\begin{tabular}{lcl}
{\class} & ::= & {\qualid} \\
  & $|$ & {\tt Sortclass} \\
  & $|$ & {\tt Funclass} 
\end{tabular}
\end{centerframe}
\caption{Syntax of classes}
\label{fig:classes}
\end{figure}

\asection{Coercions}
\index{Coercions!Funclass}
\index{Coercions!Sortclass}
  A name $f$ can be declared as a coercion between a source user-class
$C$ with $n$ parameters and a target class $D$ if one of these
conditions holds:

\newcommand{\oftype}{\!:\!}

\begin{itemize}
\item $D$ is a user-class, then the type of $f$ must have the form
      $forall~ (x_1 \oftype A_1)..(x_n \oftype A_n)(y\oftype C~x_1..x_n), D~u_1..u_m$ where $m$
      is the number of parameters of $D$.
\item $D$ is {\tt Funclass}, then the type of $f$ must have the form
      $forall~ (x_1\oftype A_1)..(x_n\oftype A_n)(y\oftype C~x_1..x_n)(x:A), B$. 
\item $D$ is {\tt Sortclass}, then the type of $f$ must have the form
      $forall~ (x_1\oftype A_1)..(x_n\oftype A_n)(y\oftype C~x_1..x_n), s$ with $s$ a sort. 
\end{itemize}

We then write $f:C \mbox{\texttt{>->}} D$. The restriction on the type
of coercions is called {\em the uniform inheritance condition}.
Remark that the abstract classes {\tt Funclass} and {\tt Sortclass}
cannot be source classes.

To coerce an object $t:C~t_1..t_n$ of $C$ towards $D$, we have to
apply the coercion $f$ to it; the obtained term $f~t_1..t_n~t$ is
then an object of $D$.

\asection{Identity Coercions}
\index{Coercions!identity}

  Identity coercions are special cases of coercions used to go around
the uniform inheritance condition.  Let $C$ and $D$ be two classes
with respectively $n$ and $m$ parameters and
$f:forall~(x_1:T_1)..(x_k:T_k)(y:C~u_1..u_n), D~v_1..v_m$ a function which
does not verify the uniform inheritance condition. To declare $f$ as
coercion, one has first to declare a subclass $C'$ of $C$:

$$C' := fun~ (x_1:T_1)..(x_k:T_k) => C~u_1..u_n$$

\noindent We then define an {\em identity coercion} between $C'$ and $C$:
\begin{eqnarray*}
Id\_C'\_C  & := & fun~ (x_1:T_1)..(x_k:T_k)(y:C'~x_1..x_k) => (y:C~u_1..u_n)\\
\end{eqnarray*}

We can now declare $f$ as coercion from $C'$ to $D$, since we can
``cast'' its type as
$forall~ (x_1:T_1)..(x_k:T_k)(y:C'~x_1..x_k),D~v_1..v_m$.\\ The identity
coercions have a special status: to coerce an object $t:C'~t_1..t_k$
of $C'$ towards $C$, we does not have to insert explicitly $Id\_C'\_C$
since $Id\_C'\_C~t_1..t_k~t$ is convertible with $t$.  However we
``rewrite'' the type of $t$ to become an object of $C$; in this case,
it becomes $C~u_1^*..u_k^*$ where each $u_i^*$ is the result of the
substitution in $u_i$ of the variables $x_j$ by $t_j$.


\asection{Inheritance Graph}
\index{Coercions!inheritance graph}
Coercions form an inheritance graph with classes as nodes.  We call
{\em coercion path} an ordered list of coercions between two nodes of
the graph.  A class $C$ is said to be a subclass of $D$ if there is a
coercion path in the graph from $C$ to $D$; we also say that $C$
inherits from $D$. Our mechanism supports multiple inheritance since a
class may inherit from several classes, contrary to simple inheritance
where a class inherits from at most one class.  However there must be
at most one path between two classes.  If this is not the case, only
the {\em oldest} one is valid and the others are ignored. So the order
of declaration of coercions is important.

We extend notations for coercions to coercion paths. For instance
$[f_1;..;f_k]:C \mbox{\texttt{>->}} D$ is the coercion path composed
by the coercions $f_1..f_k$.  The application of a coercion path to a
term consists of the successive application of its coercions.

\asection{Declaration of Coercions}

%%%%% "Class" is useless, since classes are implicitely defined via coercions.

% \asubsection{\tt Class {\qualid}.}\comindex{Class}
% Declares {\qualid} as a new class.

% \begin{ErrMsgs}
% \item {\qualid} \errindex{not declared}
% \item {\qualid} \errindex{is already a class}
% \item \errindex{Type of {\qualid} does not end with a sort}
% \end{ErrMsgs}

% \begin{Variant}
% \item {\tt Class Local {\qualid}.} \\
% Declares the construction denoted by {\qualid} as a new local class to 
% the current section.
% \end{Variant}

% END "Class" is useless

\asubsection{\tt Coercion {\qualid} : {\class$_1$} >-> {\class$_2$}.}
\comindex{Coercion}

Declares the construction denoted by {\qualid} as a coercion between
{\class$_1$} and {\class$_2$}.

% Useless information
% The classes {\class$_1$} and {\class$_2$} are first declared if necessary.
   
\begin{ErrMsgs}
\item {\qualid} \errindex{not declared}
\item {\qualid} \errindex{is already a coercion}
\item \errindex{Funclass cannot be a source class}
\item \errindex{Sortclass cannot be a source class}
\item {\qualid} \errindex{is not a function}
\item \errindex{Cannot find the source class of {\qualid}}
\item \errindex{Cannot recognize {\class$_1$} as a source class of {\qualid}}
\item {\qualid} \errindex{does not respect the uniform inheritance condition}
\item \errindex{Found target class {\class} instead of {\class$_2$}}

\end{ErrMsgs}

When the coercion {\qualid} is added to the inheritance graph, non
valid coercion paths are ignored; they are signaled by a warning.
\\[0.3cm]
\noindent {\bf Warning :}
\begin{enumerate}
\item \begin{tabbing}
{\tt Ambiguous paths: }\= $[f_1^1;..;f_{n_1}^1] : C_1\mbox{\tt >->}D_1$\\
                       \> {\ldots} \\
                       \>$[f_1^m;..;f_{n_m}^m] : C_m\mbox{\tt >->}D_m$
      \end{tabbing}
\end{enumerate}

\begin{Variants}
\item {\tt Local Coercion {\qualid} : {\class$_1$} >-> {\class$_2$}.}
\comindex{Local Coercion}\\
  Declares the construction denoted by {\qualid} as a coercion local to
  the current section.

\item {\tt Coercion {\ident} := {\term}}\comindex{Coercion}\\
  This defines {\ident} just like \texttt{Definition {\ident} :=
    {\term}}, and then declares {\ident} as a coercion between it
  source and its target.

\item {\tt Coercion {\ident} := {\term} : {\type}}\\
  This defines {\ident} just like 
  \texttt{Definition {\ident} : {\type} := {\term}}, and then
  declares {\ident} as a coercion between it source and its target. 

\item {\tt Local Coercion {\ident} := {\term}}\comindex{Local Coercion}\\
  This defines {\ident} just like \texttt{Let {\ident} :=
    {\term}}, and then declares {\ident} as a coercion between it
  source and its target.

\item Assumptions can be declared as coercions at declaration
time. This extends the grammar of assumptions from 
Figure~\ref{sentences-syntax} as follows:
\comindex{Variable \mbox{\rm (and coercions)}}
\comindex{Axiom \mbox{\rm (and coercions)}}
\comindex{Parameter \mbox{\rm (and coercions)}}
\comindex{Hypothesis \mbox{\rm (and coercions)}}

\begin{tabular}{lcl}
%% Declarations
{\assumption} & ::= & {\assumptionkeyword} {\assums} {\tt .} \\
&&\\
{\assums} & ::= & {\simpleassums} \\
          & $|$ & \nelist{{\tt (} \simpleassums {\tt )}}{} \\
&&\\
{\simpleassums} & ::= &  \nelist{\ident}{} {\tt :}\zeroone{{\tt >}} {\term}\\  
\end{tabular}

If the extra {\tt >} is present before the type of some assumptions, these
assumptions are declared as coercions.

\item Constructors of inductive types can be declared as coercions at
definition time of the inductive type. This extends and modifies the
grammar of inductive types from Figure \ref{sentences-syntax} as follows: 
\comindex{Inductive \mbox{\rm (and coercions)}}
\comindex{CoInductive \mbox{\rm (and coercions)}}

\begin{center}
\begin{tabular}{lcl}
%% Inductives
{\inductive} & ::= & 
           {\tt Inductive} \nelist{\inductivebody}{with} {\tt .} \\
 & $|$ & {\tt CoInductive} \nelist{\inductivebody}{with} {\tt .} \\
           & & \\
{\inductivebody} & ::= & 
  {\ident} \zeroone{\binders} {\tt :} {\term} {\tt :=} \\
   && ~~~\zeroone{\zeroone{\tt |} \nelist{\constructor}{|}} \\
           & & \\
{\constructor} & ::= &  {\ident} \zeroone{\binders} \zeroone{{\tt :}\zeroone{\tt >} {\term}} \\
\end{tabular}
\end{center}

Especially, if the extra {\tt >} is present in a constructor
declaration, this constructor is declared as a coercion.
\end{Variants}

\asubsection{\tt Identity Coercion {\ident}:{\class$_1$} >-> {\class$_2$}.} 
\comindex{Identity Coercion}

We check that {\class$_1$} is a constant with a value of the form
$fun~ (x_1:T_1)..(x_n:T_n) => (\mbox{\class}_2~t_1..t_m)$ where $m$ is the
number of parameters of \class$_2$.  Then we define an identity
function with the type
$forall~ (x_1:T_1)..(x_n:T_n)(y:\mbox{\class}_1~x_1..x_n),
{\mbox{\class}_2}~t_1..t_m$, and we declare it as an identity
coercion between {\class$_1$} and {\class$_2$}.

\begin{ErrMsgs}
\item {\class$_1$} \errindex{must be a transparent constant} 
\end{ErrMsgs}

\begin{Variants}
\item {\tt Local Identity Coercion {\ident}:{\ident$_1$} >-> {\ident$_2$}.} \\
Idem but locally to the current section.

\item {\tt SubClass {\ident} := {\type}.} \\
\comindex{SubClass}
 If {\type} is a class
{\ident'} applied to some arguments then {\ident} is defined and an
identity coercion of name {\tt Id\_{\ident}\_{\ident'}} is
declared. Otherwise said, this is an abbreviation for 

{\tt Definition {\ident} := {\type}.} 

 followed by

{\tt Identity Coercion Id\_{\ident}\_{\ident'}:{\ident} >-> {\ident'}}.

\item {\tt Local SubClass {\ident} := {\type}.} \\
Same as before but locally to the current section.

\end{Variants}

\asection{Displaying Available Coercions}

\asubsection{\tt Print Classes.} 
\comindex{Print Classes}
Print the list of declared classes in the current context.

\asubsection{\tt Print Coercions.}
\comindex{Print Coercions}
Print the list of declared coercions in the current context.

\asubsection{\tt Print Graph.} 
\comindex{Print Graph}
Print the list of valid coercion paths in the current context.

\asubsection{\tt Print Coercion Paths {\class$_1$} {\class$_2$}.} 
\comindex{Print Coercion Paths}
Print the list of valid coercion paths from {\class$_1$} to {\class$_2$}.

\asection{Activating the Printing of Coercions}

\asubsection{\tt Set Printing Coercions.}
\optindex{Printing Coercions}

This command forces all the coercions to be printed.
Conversely, to skip the printing of coercions, use
 {\tt Unset Printing Coercions}.
By default, coercions are not printed.

\asubsection{\tt Set Printing Coercion {\qualid}.}
\optindex{Printing Coercion}

This command forces coercion denoted by {\qualid} to be printed.
To skip the printing of coercion {\qualid}, use
 {\tt Unset Printing Coercion {\qualid}}.
By default, a coercion is never printed.
 
\asection{Classes as Records}
\label{Coercions-and-records}
\index{Coercions!and records}
We allow the definition of {\em Structures with Inheritance} (or
classes as records) by extending the existing {\tt Record} macro
(see Section~\ref{Record}). Its new syntax is:

\begin{center}
\begin{tabular}{l}
{\tt Record \zeroone{>}~{\ident} \zeroone{\binders} : {\sort} := \zeroone{\ident$_0$} \verb+{+} \\
~~~~\begin{tabular}{l}
        {\tt \ident$_1$ $[$:$|$:>$]$ \term$_1$ ;} \\
        ... \\
        {\tt \ident$_n$ $[$:$|$:>$]$ \term$_n$ \verb+}+. }
    \end{tabular}
\end{tabular}
\end{center}
The identifier {\ident} is the name of the defined record and {\sort}
is its type. The identifier {\ident$_0$} is the name of its
constructor. The identifiers {\ident$_1$}, .., {\ident$_n$} are the
names of its fields and {\term$_1$}, .., {\term$_n$} their respective
types. The alternative {\tt $[$:$|$:>$]$} is ``{\tt :}'' or ``{\tt
:>}''. If {\tt {\ident$_i$}:>{\term$_i$}}, then {\ident$_i$} is
automatically declared as coercion from {\ident} to the class of
{\term$_i$}.  Remark that {\ident$_i$} always verifies the uniform
inheritance condition.  If the optional ``{\tt >}'' before {\ident} is
present, then {\ident$_0$} (or the default name {\tt Build\_{\ident}}
if {\ident$_0$} is omitted) is automatically declared as a coercion
from the class of {\term$_n$} to {\ident} (this may fail if the
uniform inheritance condition is not satisfied).

\Rem The keyword {\tt Structure}\comindex{Structure} is a synonym of {\tt
Record}.

\asection{Coercions and Sections}
\index{Coercions!and sections}
  The inheritance mechanism is compatible with the section
mechanism. The global classes and coercions defined inside a section
are redefined after its closing, using their new value and new
type. The classes and coercions which are local to the section are
simply forgotten.
Coercions with a local source class or a local target class, and 
coercions which do not verify the uniform inheritance condition any longer
are also forgotten.

\asection{Coercions and Modules}
\index{Coercions!and modules}

From Coq version 8.3, the coercions present in a module are activated
only when the module is explicitly imported. Formerly, the coercions
were activated as soon as the module was required, whatever it was
imported or not.

To recover the behavior of the versions of Coq prior to 8.3, use the
following command:

\optindex{Automatic Coercions Import}
\begin{verbatim}
Set Automatic Coercions Import.
\end{verbatim}

To cancel the effect of the option, use instead:

\begin{verbatim}
Unset Automatic Coercions Import.
\end{verbatim}

\asection{Examples}

  There are three situations:

\begin{itemize}
\item $f~a$ is ill-typed where $f:forall~x:A,B$ and $a:A'$. If there is a
      coercion path between $A'$ and $A$, $f~a$ is transformed into
      $f~a'$ where $a'$ is the result of the application of this
      coercion path to $a$.

We first give an example of coercion between atomic inductive types

%\begin{\small}
\begin{coq_example}
Definition bool_in_nat (b:bool) := if b then 0 else 1.
Coercion bool_in_nat : bool >-> nat.
Check (0 = true).
Set Printing Coercions.
Check (0 = true).
\end{coq_example}
%\end{small}

\begin{coq_eval}
Unset Printing Coercions.
\end{coq_eval}

\Warning ``\verb|Check true=O.|'' fails. This is ``normal'' behaviour of
coercions. To validate \verb|true=O|, the coercion is searched from
\verb=nat= to \verb=bool=. There is none.

We give an example of coercion between classes with parameters.

%\begin{\small}
\begin{coq_example}
Parameters
   (C : nat -> Set) (D : nat -> bool -> Set) (E : bool -> Set).
Parameter f : forall n:nat, C n -> D (S n) true.
Coercion f : C >-> D.
Parameter g : forall (n:nat) (b:bool), D n b -> E b.
Coercion g : D >-> E.
Parameter c : C 0.
Parameter T : E true -> nat.
Check (T c).
Set Printing Coercions.
Check (T c).
\end{coq_example}
%\end{small}

\begin{coq_eval}
Unset Printing Coercions.
\end{coq_eval}

We give now an example using identity coercions.

%\begin{small}
\begin{coq_example}
Definition D' (b:bool) := D 1 b.
Identity Coercion IdD'D : D' >-> D.
Print IdD'D.
Parameter d' : D' true.
Check (T d').
Set Printing Coercions.
Check (T d').
\end{coq_example}
%\end{small}

\begin{coq_eval}
Unset Printing Coercions.
\end{coq_eval}


  In the case of functional arguments, we use the monotonic rule of
sub-typing.  Approximatively, to coerce $t:forall~x:A, B$ towards
$forall~x:A',B'$, one have to coerce $A'$ towards $A$ and $B$ towards
$B'$. An example is given below:

%\begin{small}
\begin{coq_example}
Parameters (A B : Set) (h : A -> B).
Coercion h : A >-> B.
Parameter U : (A -> E true) -> nat.
Parameter t : B -> C 0.
Check (U t).
Set Printing Coercions.
Check (U t).
\end{coq_example}
%\end{small}

\begin{coq_eval}
Unset Printing Coercions.
\end{coq_eval}

  Remark the changes in the result following the modification of the
previous example.

%\begin{small}
\begin{coq_example}
Parameter U' : (C 0 -> B) -> nat.
Parameter t' : E true -> A.
Check (U' t').
Set Printing Coercions.
Check (U' t').
\end{coq_example}
%\end{small}

\begin{coq_eval}
Unset Printing Coercions.
\end{coq_eval}

\item An assumption $x:A$ when $A$ is not a type, is ill-typed.  It is
      replaced by $x:A'$ where $A'$ is the result of the application
      to $A$ of the coercion path between the class of $A$ and {\tt
      Sortclass} if it exists.  This case occurs in the abstraction
      $fun~ x:A => t$, universal quantification $forall~x:A, B$,
      global variables and parameters of (co-)inductive definitions
      and functions. In $forall~x:A, B$, such a coercion path may be
      applied to $B$ also if necessary.

%\begin{small}
\begin{coq_example}
Parameter Graph : Type.
Parameter Node : Graph -> Type.
Coercion Node : Graph >-> Sortclass.
Parameter G : Graph.
Parameter Arrows : G -> G -> Type.
Check Arrows.
Parameter fg : G -> G.
Check fg.
Set Printing Coercions.
Check fg.
\end{coq_example}
%\end{small}

\begin{coq_eval}
Unset Printing Coercions.
\end{coq_eval}

\item $f~a$ is ill-typed because $f:A$ is not a function. The term
      $f$ is replaced by the term obtained by applying to $f$ the
      coercion path between $A$ and {\tt Funclass} if it exists.

%\begin{small}
\begin{coq_example}
Parameter bij : Set -> Set -> Set.
Parameter ap : forall A B:Set, bij A B -> A -> B.
Coercion ap : bij >-> Funclass.
Parameter b : bij nat nat.
Check (b 0).
Set Printing Coercions.
Check (b 0).
\end{coq_example}
%\end{small}

\begin{coq_eval}
Unset Printing Coercions.
\end{coq_eval}

Let us see the resulting graph of this session.

%\begin{small}
\begin{coq_example}
Print Graph.
\end{coq_example}
%\end{small}

\end{itemize}


%%% Local Variables: 
%%% mode: latex
%%% TeX-master: "Reference-Manual"
%%% End: 
