\achapter{Polymorphic Universes}
\aauthor{Matthieu Sozeau}

\label{Universes-full}
\index{Universes!presentation}

\asection{General Presentation}

This section describes the universe polymorphic extension of Coq.
Universe polymorphism allows to write generic definitions making use of
universes and reuse them at different and sometimes incompatible levels.

A standard example of the difference between universe \emph{polymorphic} and 
\emph{monormorphic} definitions is given by the identity function:

\begin{coq_example*}
Definition identity {A : Type} (a : A) := a.
\end{coq_example*}

By default, constant declarations are monomorphic, hence the identity
function declares a global universe (say \texttt{Top.1}) for its
domain. Subsequently, if we try to self-apply the identity, we will get
an error:

\begin{coq_eval}
Set Printing Universes.
\end{coq_eval}
\begin{coq_example}
Fail Definition selfid := identity (@identity).
\end{coq_example}

Indeed, the global level \texttt{Top.1} would have to be strictly smaller than itself
for this self-application to typecheck, as the type of (@identity) is
\texttt{forall (A : Type@{Top.1}), A -> A} whose type is itself \texttt{Type@{Top.1+1}}.

A universe polymorphic identity function binds its domain universe level
at the definition level instead of making it global.

\begin{coq_example}
Polymorphic Definition pidentity {A : Type} (a : A) := a.
About pidentity.
\end{coq_example}

It is then possible to reuse the constant at different levels, like so:

\begin{coq_example}
Definition selfpid := pidentity (@pidentity).
\end{coq_example}

Of course, the two instances of \texttt{pidentity} in this definition
are different. This can be seen when \texttt{Set Printing Universes} is
on:

\begin{coq_example}
Print selfpid.
\end{coq_example}

Now \texttt{pidentity} is used at two different levels: at the head of
the application is is instantiated at \texttt{Top.3} while in the
argument position it is instantiated at \texttt{Top.4}. This definition
is only valid as long as \texttt{Top.4} is strictly smaller than
\texttt{Top.3}, as show by the constraints. Not that this definition is
monomorphic (not universe polymorphic), so in turn the two universes are
actually global levels.

Polymorphic constants, inductive 



\asubsection{\tt Polymorphic, Monomorphic}
\comindex{Polymorphic}
\comindex{Monomorphic}


%%% Local Variables: 
%%% mode: latex
%%% TeX-master: "Reference-Manual"
%%% End: 
