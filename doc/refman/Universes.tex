\achapter{Polymorphic Universes}
\aauthor{Matthieu Sozeau}

\label{Universes-full}
\index{Universes!presentation}

\asection{General Presentation}

This section describes the universe polymorphic extension of Coq.
Universe polymorphism allows writing generic definitions making use of
universes and reuse them at different and sometimes incompatible levels.

A standard example of the difference between universe \emph{polymorphic} and
\emph{monomorphic} definitions is given by the identity function:

\begin{coq_example*}
Definition identity {A : Type} (a : A) := a.
\end{coq_example*}

By default, constant declarations are monomorphic, hence the identity
function declares a global universe (say \texttt{Top.1}) for its
domain. Subsequently, if we try to self-apply the identity, we will get
an error:

\begin{coq_eval}
Set Printing Universes.
\end{coq_eval}
\begin{coq_example}
Fail Definition selfid := identity (@identity).
\end{coq_example}

Indeed, the global level \texttt{Top.1} would have to be strictly smaller than itself
for this self-application to typecheck, as the type of \texttt{(@identity)} is
\texttt{forall (A : Type@{Top.1}), A -> A} whose type is itself \texttt{Type@{Top.1+1}}.

A universe polymorphic identity function binds its domain universe level
at the definition level instead of making it global.

\begin{coq_example}
Polymorphic Definition pidentity {A : Type} (a : A) := a.
About pidentity.
\end{coq_example}

It is then possible to reuse the constant at different levels, like so:

\begin{coq_example}
Definition selfpid := pidentity (@pidentity).
\end{coq_example}

Of course, the two instances of \texttt{pidentity} in this definition
are different. This can be seen when \texttt{Set Printing Universes} is
on:

\begin{coq_example}
Print selfpid.
\end{coq_example}

Now \texttt{pidentity} is used at two different levels: at the head of
the application it is instantiated at \texttt{Top.3} while in the
argument position it is instantiated at \texttt{Top.4}. This definition
is only valid as long as \texttt{Top.4} is strictly smaller than
\texttt{Top.3}, as show by the constraints. Not that this definition is
monomorphic (not universe polymorphic), so in turn the two universes are
actually global levels.

Inductive types can also be declared universes polymorphic, on universes
appearing in their parameters or fields. A typical example is given by
monoids:

\begin{coq_example}
Polymorphic Record Monoid := { mon_car :> Type; mon_unit : mon_car; 
  mon_op : mon_car -> mon_car -> mon_car }.
Print Monoid.
\end{coq_example}

The \texttt{Monoid}'s carrier universe is polymorphic, hence it is
possible to instantiate it for example with \texttt{Monoid} itself.
First we build the trivial unit monoid, in \texttt{Set}:
\begin{coq_example}
Definition unit_monoid : Monoid := 
  {| mon_car := unit; mon_unit := tt; mon_op x y := tt |}.
\end{coq_example} 

From this we can build a definition for the monoid of
\texttt{Set}-monoids (where multiplication would be given by the product
of monoids).

\begin{coq_example*}
Polymorphic Definition monoid_monoid : Monoid.
  refine (@Build_Monoid Monoid unit_monoid _) ; admit.
Defined.
\end{coq_example*}
\begin{coq_example}
Print monoid_monoid.
\end{coq_example} 

As one can see from the constraints, this monoid is ``large'', it lives
in a universe strictly higher than \texttt{Set}.

\asection{\tt Polymorphic, Monomorphic}
\comindex{Polymorphic}
\comindex{Monomorphic}
\comindex{Set Universe Polymorphism}

As shown in the examples, polymorphic definitions and inductives can be
declared using the \texttt{Polymorphic} prefix. There also exists an
option \texttt{Set Universe Polymorphism} which will implicitly prepend
it to any definition of the user. In that case, to make a definition
producing global universe constraints, one can use the
\texttt{Monomorphic} prefix. Many other commands support the
\texttt{Polymorphic} flag, including:

\begin{itemize}
\item \texttt{Lemma}, \texttt{Axiom}, and all the other ``definition''
  keywords support polymorphism.
\item \texttt{Variables}, \texttt{Context} in a section support polymorphism.
  This means that the
  variables are discharged polymorphically over definitions that use
  them. In other words, two definitions in the section sharing a common
  variable will both get parameterized by the universes produced by the 
  variable declaration. This is in contrast to a ``mononorphic'' variable
  which introduce global universes, making the two definitions depend on
  the \emph{same} universes associated to the variable.
\item \texttt{Hint \{Resolve, Rewrite\}} will use the auto/rewrite hint
  polymorphically, not at a single instance.
\end{itemize}

\asection{Conversion and unification}

The semantics of conversion and unification have to be modified a little
to account for the new universe instance arguments to polymorphic
references. The semantics respect the fact that definitions are
transparent, so indistinguishable from their bodies during conversion.

This is accomplished by changing one rule of unification, the
first-order approximation rule:


TODO



\asection{Explicit Universes}

TODO


%%% Local Variables: 
%%% mode: latex
%%% TeX-master: "Reference-Manual"
%%% End: 
