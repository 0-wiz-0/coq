\achapter{Polymorphic Universes}
%HEVEA\cutname{universes.html}
\aauthor{Matthieu Sozeau}

\label{Universes-full}
\index{Universes!presentation}

\asection{General Presentation}

\begin{flushleft}
  \em The status of Universe Polymorphism is experimental.
\end{flushleft}

This section describes the universe polymorphic extension of Coq.
Universe polymorphism makes it possible to write generic definitions making use of
universes and reuse them at different and sometimes incompatible universe levels.

A standard example of the difference between universe \emph{polymorphic} and
\emph{monomorphic} definitions is given by the identity function:

\begin{coq_example*}
Definition identity {A : Type} (a : A) := a.
\end{coq_example*}

By default, constant declarations are monomorphic, hence the identity
function declares a global universe (say \texttt{Top.1}) for its
domain. Subsequently, if we try to self-apply the identity, we will get
an error:

\begin{coq_eval}
Set Printing Universes.
\end{coq_eval}
\begin{coq_example}
Fail Definition selfid := identity (@identity).
\end{coq_example}

Indeed, the global level \texttt{Top.1} would have to be strictly smaller than itself
for this self-application to typecheck, as the type of \texttt{(@identity)} is
\texttt{forall (A : Type@{Top.1}), A -> A} whose type is itself \texttt{Type@{Top.1+1}}.

A universe polymorphic identity function binds its domain universe level
at the definition level instead of making it global.

\begin{coq_example}
Polymorphic Definition pidentity {A : Type} (a : A) := a.
About pidentity.
\end{coq_example}

It is then possible to reuse the constant at different levels, like so:

\begin{coq_example}
Definition selfpid := pidentity (@pidentity).
\end{coq_example}

Of course, the two instances of \texttt{pidentity} in this definition
are different. This can be seen when \texttt{Set Printing Universes} is
on:

\begin{coq_example}
Print selfpid.
\end{coq_example}

Now \texttt{pidentity} is used at two different levels: at the head of
the application it is instantiated at \texttt{Top.3} while in the
argument position it is instantiated at \texttt{Top.4}. This definition
is only valid as long as \texttt{Top.4} is strictly smaller than
\texttt{Top.3}, as show by the constraints. Note that this definition is
monomorphic (not universe polymorphic), so the two universes
(in this case \texttt{Top.3} and \texttt{Top.4}) are actually global levels.

When printing \texttt{pidentity}, we can see the universes it binds in
the annotation \texttt{@\{Top.2\}}. Additionally, when \texttt{Set
  Printing Universes} is on we print the ``universe context'' of
\texttt{pidentity} consisting of the bound universes and the
constraints they must verify (for \texttt{pidentity} there are no
constraints).

Inductive types can also be declared universes polymorphic on universes
appearing in their parameters or fields. A typical example is given by
monoids:

\begin{coq_example}
Polymorphic Record Monoid := { mon_car :> Type; mon_unit : mon_car; 
  mon_op : mon_car -> mon_car -> mon_car }.
Print Monoid.
\end{coq_example}

The \texttt{Monoid}'s carrier universe is polymorphic, hence it is
possible to instantiate it for example with \texttt{Monoid} itself.
First we build the trivial unit monoid in \texttt{Set}:
\begin{coq_example}
Definition unit_monoid : Monoid := 
  {| mon_car := unit; mon_unit := tt; mon_op x y := tt |}.
\end{coq_example} 

From this we can build a definition for the monoid of
\texttt{Set}-monoids (where multiplication would be given by the product
of monoids).

\begin{coq_example*}
Polymorphic Definition monoid_monoid : Monoid.
  refine (@Build_Monoid Monoid unit_monoid (fun x y => x)).
Defined.
\end{coq_example*}
\begin{coq_example}
Print monoid_monoid.
\end{coq_example} 

As one can see from the constraints, this monoid is ``large'', it lives
in a universe strictly higher than \texttt{Set}.

\asection{\tt Polymorphic, Monomorphic}
\comindex{Polymorphic}
\comindex{Monomorphic}
\optindex{Universe Polymorphism}

As shown in the examples, polymorphic definitions and inductives can be
declared using the \texttt{Polymorphic} prefix. There also exists an
option \texttt{Set Universe Polymorphism} which will implicitly prepend
it to any definition of the user. In that case, to make a definition
producing global universe constraints, one can use the
\texttt{Monomorphic} prefix. Many other commands support the
\texttt{Polymorphic} flag, including:

\begin{itemize}
\item \texttt{Lemma}, \texttt{Axiom}, and all the other ``definition''
  keywords support polymorphism.
\item \texttt{Variables}, \texttt{Context}, \texttt{Universe} and
  \texttt{Constraint} in a section support polymorphism.  This means
  that the universe variables (and associated constraints) are
  discharged polymorphically over definitions that use them. In other
  words, two definitions in the section sharing a common variable will
  both get parameterized by the universes produced by the variable
  declaration. This is in contrast to a ``mononorphic'' variable which
  introduces global universes and constraints, making the two
  definitions depend on the \emph{same} global universes associated to
  the variable.
\item \texttt{Hint \{Resolve, Rewrite\}} will use the auto/rewrite hint
  polymorphically, not at a single instance.
\end{itemize}

\asection{{\tt Cumulative, NonCumulative}}
\comindex{Cumulative}
\comindex{NonCumulative}
\optindex{Polymorphic Inductive Cumulativity}

Polymorphic inductive types, coinductive types, variants and records can be
declared cumulative using the \texttt{Cumulative} prefix. Alternatively,
there is an option \texttt{Set Polymorphic Inductive Cumulativity} which when set,
makes all subsequent \emph{polymorphic} inductive definitions cumulative.  When set,
inductive types and the like can be enforced to be
\emph{non-cumulative} using the \texttt{NonCumulative} prefix. Consider the examples below.
\begin{coq_example*}
Polymorphic Cumulative Inductive list {A : Type} :=
| nil : list
| cons : A -> list -> list.
\end{coq_example*}
\begin{coq_example}
Print list.
\end{coq_example}
When printing \texttt{list}, the universe context indicates the
subtyping constraints by prefixing the level names with symbols.

Because inductive subtypings are only produced by comparing inductives
to themselves with universes changed, they amount to variance
information: each universe is either invariant, covariant or
irrelevant (there are no contravariant subtypings in Coq),
respectively represented by the symbols \texttt{=}, \texttt{+} and
\texttt{*}.

Here we see that \texttt{list} binds an irrelevant universe, so any
two instances of \texttt{list} are convertible:
$\WTEGCONV{\mathtt{list@\{i\}} A}{\mathtt{list@\{j\}} B}$ whenever
$\WTEGCONV{A}{B}$ and furthermore their corresponding (when fully
applied to convertible arguments) constructors.

See Chapter~\ref{Cic} for more details on convertibility and subtyping.
The following is an example of a record with non-trivial subtyping relation:
\begin{coq_example*}
Polymorphic Cumulative Record packType := {pk : Type}.
\end{coq_example*}
\begin{coq_example}
Print packType.
\end{coq_example}
\texttt{packType} binds a covariant universe, i.e.
$\WTEGCONV{\mathtt{packType@\{i\}}}{\mathtt{packType@\{j\}}}$ whenever
\texttt{i $\leq$ j}.

Cumulative inductive types, coninductive types, variants and records
only make sense when they are universe polymorphic. Therefore, an
error is issued whenever the user uses the \texttt{Cumulative} or
\texttt{NonCumulative} prefix in a monomorphic context.
Notice that this is not the case for the option \texttt{Set Polymorphic Inductive Cumulativity}.
That is, this option, when set, makes all subsequent \emph{polymorphic}
inductive declarations cumulative (unless, of course the \texttt{NonCumulative} prefix is used)
but has no effect on \emph{monomorphic} inductive declarations.
Consider the following examples.
\begin{coq_example}
Monomorphic Cumulative Inductive Unit := unit.
\end{coq_example}
\begin{coq_example}
Monomorphic NonCumulative Inductive Unit := unit.
\end{coq_example}
\begin{coq_example*}
Set Polymorphic Inductive Cumulativity.
Inductive Unit := unit.
\end{coq_example*}
\begin{coq_example}
Print Unit.
\end{coq_example}

\subsection*{An example of a proof using cumulativity}

\begin{coq_example}
Set Universe Polymorphism.
Set Polymorphic Inductive Cumulativity.

Inductive eq@{i} {A : Type@{i}} (x : A) : A -> Type@{i} := eq_refl : eq x x.

Definition funext_type@{a b e} (A : Type@{a}) (B : A -> Type@{b})
  := forall f g : (forall a, B a),
    (forall x, eq@{e} (f x) (g x))
    -> eq@{e} f g.

Section down.
  Universes a b e e'.
  Constraint e' < e.
  Lemma funext_down {A B}
    (H : @funext_type@{a b e} A B) : @funext_type@{a b e'} A B.
  Proof.
    exact H.
  Defined.
\end{coq_example}

\subsection{\tt Cumulativity Weak Constraints}
\optindex{Cumulativity Weak Constraints}

This option, on by default, causes ``weak'' constraints to be produced
when comparing universes in an irrelevant position. Processing weak
constraints is delayed until minimization time. A weak constraint
between {\tt u} and {\tt v} when neither is smaller than the other and
one is flexible causes them to be unified. Otherwise the constraint is
silently discarded.

This heuristic is experimental and may change in future versions.
Disabling weak constraints is more predictable but may produce
arbitrary numbers of universes.

\asection{Global and local universes}

Each universe is declared in a global or local environment before it can
be used. To ensure compatibility, every \emph{global} universe is set to
be strictly greater than \Set~when it is introduced, while every
\emph{local} (i.e. polymorphically quantified) universe is introduced as
greater or equal to \Set.

\asection{Conversion and unification}

The semantics of conversion and unification have to be modified a little
to account for the new universe instance arguments to polymorphic
references. The semantics respect the fact that definitions are
transparent, so indistinguishable from their bodies during conversion.

This is accomplished by changing one rule of unification, the
first-order approximation rule, which applies when two applicative terms
with the same head are compared. It tries to short-cut unfolding by
comparing the arguments directly. In case the constant is universe
polymorphic, we allow this rule to fire only when unifying the universes
results in instantiating a so-called flexible universe variables (not
given by the user). Similarly for conversion, if such an equation of
applicative terms fail due to a universe comparison not being satisfied,
the terms are unfolded. This change implies that conversion and
unification can have different unfolding behaviors on the same
development with universe polymorphism switched on or off.

\asection{Minimization}
\optindex{Universe Minimization ToSet}

Universe polymorphism with cumulativity tends to generate many useless
inclusion constraints in general. Typically at each application of a
polymorphic constant $f$, if an argument has expected type
\verb|Type@{i}| and is given a term of type \verb|Type@{j}|, a $j \le i$
constraint will be generated. It is however often the case that an
equation $j = i$ would be more appropriate, when $f$'s
universes are fresh for example. Consider the following example:

\begin{coq_eval}
Set Printing Universes.
\end{coq_eval}
\begin{coq_example}
Definition id0 := @pidentity nat 0.
Print id0.
\end{coq_example}

This definition is elaborated by minimizing the universe of id to level
\Set~while the more general definition would keep the fresh level i
generated at the application of id and a constraint that $\Set \le i$.
This minimization process is applied only to fresh universe
variables. It simply adds an equation between the variable and its lower
bound if it is an atomic universe (i.e. not an algebraic \texttt{max()}
universe).

The option \texttt{Unset Universe Minimization ToSet} disallows
minimization to the sort $\Set$ and only collapses floating universes
between themselves.

\asection{Explicit Universes}

The syntax has been extended to allow users to explicitly bind names to
universes and explicitly instantiate polymorphic definitions.

\subsection{\tt Universe {\ident}.
  \comindex{Universe}
  \label{UniverseCmd}}

In the monorphic case, this command declares a new global universe named
{\ident}, which can be referred to using its qualified name as
well. Global universe names live in a separate namespace. The command
supports the polymorphic flag only in sections, meaning the universe
quantification will be discharged on each section definition
independently. One cannot mix polymorphic and monomorphic declarations
in the same section.

\subsection{\tt Constraint {\ident} {\textit{ord}} {\ident}.
  \comindex{Constraint}
  \label{ConstraintCmd}}

This command declares a new constraint between named universes.
The order relation can be one of $<$, $\le$ or $=$. If consistent, 
the constraint is then enforced in the global environment. Like
\texttt{Universe}, it can be used with the \texttt{Polymorphic} prefix
in sections only to declare constraints discharged at section closing time.
One cannot declare a global constraint on polymorphic universes.

\begin{ErrMsgs}
\item \errindex{Undeclared universe {\ident}}.
\item \errindex{Universe inconsistency}
\end{ErrMsgs}

\subsection{Polymorphic definitions}
For polymorphic definitions, the declaration of (all) universe levels
introduced by a definition uses the following syntax:

\begin{coq_example*}
Polymorphic Definition le@{i j} (A : Type@{i}) : Type@{j} := A.
\end{coq_example*}
\begin{coq_example}
Print le.
\end{coq_example}

During refinement we find that $j$ must be larger or equal than $i$, as
we are using $A : Type@{i} <= Type@{j}$, hence the generated
constraint. At the end of a definition or proof, we check that the only
remaining universes are the ones declared. In the term and in general in
proof mode, introduced universe names can be referred to in
terms. Note that local universe names shadow global universe names.
During a proof, one can use \texttt{Show Universes} to display
the current context of universes.

Definitions can also be instantiated explicitly, giving their full instance:
\begin{coq_example}
Check (pidentity@{Set}).
Universes k l.
Check (le@{k l}).
\end{coq_example}

User-named universes and the anonymous universe implicitly attached to
an explicit $Type$ are considered rigid for unification and are never
minimized. Flexible anonymous universes can be produced with an
underscore or by omitting the annotation to a polymorphic definition.

\begin{coq_example}
  Check (fun x => x) : Type -> Type.
  Check (fun x => x) : Type -> Type@{_}.

  Check le@{k _}.
  Check le.
\end{coq_example}

\subsection{\tt Unset Strict Universe Declaration.
  \optindex{Strict Universe Declaration}
  \label{StrictUniverseDeclaration}}

The command \texttt{Unset Strict Universe Declaration} allows one to
freely use identifiers for universes without declaring them first, with
the semantics that the first use declares it. In this mode, the universe
names are not associated with the definition or proof once it has been
defined. This is meant mainly for debugging purposes.

%%% Local Variables: 
%%% mode: latex
%%% TeX-master: "Reference-Manual"
%%% End: 
