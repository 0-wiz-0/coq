\section{Module system
\index{Modules}
\label{section:Modules}}

The module system provides a way of packaging related elements
together, as well as a mean of massive abstraction.

\begin{figure}[t]
\begin{centerframe}
\begin{tabular}{rcl}
{\modtype}  & ::= & {\ident} \\
 & $|$ & {\modtype} \texttt{ with Definition }{\ident} := {\term} \\
 & $|$ & {\modtype} \texttt{ with Module }{\ident} := {\qualid} \\
 &&\\

{\onemodbinding}  & ::= & {\tt ( \nelist{\ident}{} : {\modtype} )}\\
 &&\\

{\modbindings} & ::= & \nelist{\onemodbinding}{}\\
 &&\\

{\modexpr} & ::= & \nelist{\qualid}{} 
\end{tabular}
\end{centerframe}
\caption{Syntax of modules}
\end{figure}

\subsection{\tt Module {\ident}
\comindex{Module}}

This command is used to start an interactive module named {\ident}.

\begin{Variants}

\item{\tt Module {\ident} {\modbindings}}

  Starts an interactive functor with parameters given by {\modbindings}.

\item{\tt Module {\ident} \verb.:. {\modtype}}

  Starts an interactive module specifying its module type. 

\item{\tt Module {\ident} {\modbindings} \verb.:. {\modtype}}

  Starts an interactive functor with parameters given by
  {\modbindings}, and output module type {\modtype}.

\item{\tt Module {\ident} \verb.<:. {\modtype}}

  Starts an interactive module satisfying {\modtype}. 

\item{\tt Module {\ident} {\modbindings} \verb.<:. {\modtype}}

  Starts an interactive functor with parameters given by
  {\modbindings}. The output module type is verified against the
  module type {\modtype}.

\item\texttt{Module [Import|Export]}

  Behaves like \texttt{Module}, but automatically imports or exports
  the module.

\end{Variants}

\subsection{\tt End {\ident}
\comindex{End}}

This command closes the interactive module {\ident}. If the module type
was given the content of the module is matched against it and an error
is signaled if the matching fails. If the module is basic (is not a
functor) its components (constants, inductive types, submodules etc) are
now available through the dot notation.

\begin{ErrMsgs}
\item \errindex{No such label {\ident}}
\item \errindex{Signature components for label {\ident} do not match}
\item \errindex{This is not the last opened module}
\end{ErrMsgs}


\subsection{\tt Module {\ident} := {\modexpr}
\comindex{Module}}

This command defines the module identifier {\ident} to be equal to
{\modexpr}. 

\begin{Variants}
\item{\tt Module {\ident} {\modbindings} := {\modexpr}}

 Defines a functor with parameters given by {\modbindings} and body {\modexpr}.

% Particular cases of the next 2 items
%\item{\tt Module {\ident} \verb.:. {\modtype} := {\modexpr}}
%
%  Defines a module with body {\modexpr} and interface {\modtype}.
%\item{\tt Module {\ident} \verb.<:. {\modtype} := {\modexpr}}
%
%  Defines a module with body {\modexpr}, satisfying {\modtype}.

\item{\tt Module {\ident} {\modbindings} \verb.:. {\modtype} :=
    {\modexpr}}

  Defines a functor with parameters given by {\modbindings} (possibly none),
  and output module type {\modtype}, with body {\modexpr}. 

\item{\tt Module {\ident} {\modbindings} \verb.<:. {\modtype} :=
    {\modexpr}}

  Defines a functor with parameters given by {\modbindings} (possibly none) 
  with body {\modexpr}. The body is checked against {\modtype}.

\end{Variants}

\subsection{\tt Module Type {\ident}
\comindex{Module Type}}

This command is used to start an interactive module type {\ident}.

\begin{Variants}

\item{\tt Module Type {\ident} {\modbindings}}

  Starts an interactive functor type with parameters given by {\modbindings}.

\end{Variants}

\subsection{\tt End {\ident}
\comindex{End}}

This command closes the interactive module type {\ident}.

\begin{ErrMsgs}
\item \errindex{This is not the last opened module type}
\end{ErrMsgs}

\subsection{\tt Module Type {\ident} := {\modtype}}

Defines a module type {\ident} equal to {\modtype}.

\begin{Variants}
\item {\tt Module Type {\ident} {\modbindings} := {\modtype}}

  Defines a functor type {\ident} specifying functors taking arguments
  {\modbindings} and returning {\modtype}.
\end{Variants}

\subsection{\tt Declare Module {\ident} : {\modtype}}

Declares a module {\ident} of type {\modtype}. This command is available
only in module types. 

\begin{Variants}

\item{\tt Declare Module {\ident} {\modbindings} \verb.:. {\modtype}}

  Declares a functor with parameters {\modbindings} and output module
  type {\modtype}.

\item{\tt Declare Module {\ident} := {\qualid}}

  Declares a module equal to the module {\qualid}.

\item{\tt Declare Module {\ident} \verb.<:. {\modtype} := {\qualid}}

  Declares a module equal to the module {\qualid}, verifying that the
  module type of the latter is a subtype of {\modtype}.

\item\texttt{Declare Module [Import|Export] {\ident} := {\qualid}}

  Declares a modules {\ident} of type {\modtype}, and imports or
  exports it directly.

\end{Variants}


\subsubsection{Example}

Let us define a simple module.
\begin{coq_example}
Module M.
  Definition T := nat.
  Definition x := 0.
  Definition y : bool.
    exact true.
  Defined.
End M.
\end{coq_example}
Inside a module one can define constants, prove theorems and do any
other things that can be done in the toplevel. Components of a closed
module can be accessed using the dot notation:
\begin{coq_example}
Print M.x.
\end{coq_example}
A simple module type:
\begin{coq_example}
Module Type SIG.
  Parameter T : Set.
  Parameter x : T.
End SIG.
\end{coq_example}
Inside a module type the proof editing mode is not available.
Consequently commands like \texttt{Definition}\ without body,
\texttt{Lemma}, \texttt{Theorem} are not allowed.  In order to declare
constants, use \texttt{Axiom} and \texttt{Parameter}.

Now we can create a new module from \texttt{M}, giving it a less
precise specification: the \texttt{y} component is dropped as well
as the body of \texttt{x}.

\begin{coq_eval}
Set Printing Depth 50.
(********** The following is not correct and should produce **********)
(***************** Error: N.y not a defined object *******************)
\end{coq_eval}
\begin{coq_example}
Module N  :  SIG with Definition T := nat  :=  M.
Print N.T.
Print N.x.
Print N.y.
\end{coq_example}
\begin{coq_eval}
Reset N.
\end{coq_eval}

\noindent
The definition of \texttt{N} using the module type expression
\texttt{SIG with Definition T:=nat} is equivalent to the following
one:

\begin{coq_example*}
Module Type SIG'.
  Definition T : Set := nat.
  Parameter x : T.
End SIG'.
Module N : SIG' := M.
\end{coq_example*}
If we just want to be sure that the our implementation satisfies a
given module type without restricting the interface, we can use a
transparent constraint
\begin{coq_example}
Module P <: SIG := M.
Print P.y.
\end{coq_example}
Now let us create a functor, i.e. a parametric module
\begin{coq_example}
Module Two (X Y: SIG).
\end{coq_example}
\begin{coq_example*}
  Definition T := (X.T * Y.T)%type.
  Definition x := (X.x, Y.x).
\end{coq_example*}
\begin{coq_example}
End Two.
\end{coq_example}
and apply it to our modules and do some computations
\begin{coq_example}
Module Q := Two M N.
Eval compute in (fst Q.x + snd Q.x).
\end{coq_example}
In the end, let us define a module type with two sub-modules, sharing
some of the fields and give one of its possible implementations:
\begin{coq_example}
Module Type SIG2.
  Declare Module M1 : SIG.
  Declare Module M2 <: SIG.
    Definition T := M1.T.
    Parameter x : T.
  End M2.
End SIG2.
\end{coq_example}
\begin{coq_example*}
Module Mod <: SIG2.
  Module M1.
    Definition T := nat.
    Definition x := 1.
  End M1.
  Module M2 := M.
\end{coq_example*}
\begin{coq_example}
End Mod.
\end{coq_example}
Notice that \texttt{M} is a correct body for the component \texttt{M2}
since its \texttt{T} component is equal \texttt{nat} and hence
\texttt{M1.T} as specified.
\begin{coq_eval}
Reset Initial.
\end{coq_eval}

\begin{Remarks}
\item Modules and module types can be nested components of each other.
\item When a module declaration is started inside a module type,
  the proof editing mode is still unavailable.
\item One can have sections inside a module or a module type, but
  not a module or a module type inside a section.
\item Commands like \texttt{Hint} or \texttt{Notation} can
  also appear inside modules and module types. Note that in case of a
  module definition like:

    \medskip
    \noindent
    {\tt Module N : SIG := M.} 
    \medskip

    or

    \medskip
    {\tt Module N : SIG.\\
      \ \ \dots\\
      End N.}
    \medskip 
    
    hints and the like valid for \texttt{N} are not those defined in
    \texttt{M} (or the module body) but the ones defined in
    \texttt{SIG}.

\end{Remarks}

\subsection{Import {\qualid}
\comindex{Import}
\label{Import}}

If {\qualid} denotes a valid basic module (i.e. its module type is a
signature), makes its components available by their short names.

Example:

\begin{coq_example}
Module Mod.
\end{coq_example}
\begin{coq_example}
  Definition T:=nat.
  Check T.
\end{coq_example}
\begin{coq_example}
End Mod.
Check Mod.T.
Check T. (* Incorrect ! *)
Import Mod.
Check T. (* Now correct *)
\end{coq_example}
\begin{coq_eval}
Reset Mod.
\end{coq_eval}


\begin{Variants}
\item{\tt Export {\qualid}}\comindex{Export}

  When the module containing the command {\tt Export {\qualid}} is
  imported, {\qualid} is imported as well.
\end{Variants}

\begin{ErrMsgs}
  \item \errindexbis{{\qualid} is not a module}{is not a module}
% this error is impossible in the import command
%  \item \errindex{Cannot mask the absolute name {\qualid} !}
\end{ErrMsgs}

\begin{Warnings}
  \item Warning: Trying to mask the absolute name {\qualid} !
\end{Warnings}

\subsection{\tt Print Module {\ident}
\comindex{Print Module}}

Prints the module type and (optionally) the body of the module {\ident}.

\subsection{\tt Print Module Type {\ident}
\comindex{Print Module Type}}

Prints the module type corresponding to {\ident}.

\subsection{\texttt{Locate Module {\qualid}}
\comindex{Locate Module}}

Prints the full name of the module {\qualid}.


%%% Local Variables: 
%%% mode: latex
%%% TeX-master: "Reference-Manual"
%%% End: 
