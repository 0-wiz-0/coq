\chapter{The {\Coq} library}
\index{Theories}\label{Theories}

The \Coq\ library is structured into three parts:

\begin{description}
\item[The initial library:] it contains
  elementary logical notions and datatypes. It constitutes the
  basic state of the system directly available when running
  \Coq;

\item[The standard library:] general-purpose libraries containing
  various developments of \Coq\ axiomatizations about sets, lists,
  sorting, arithmetic, etc. This library comes with the system and its
  modules are directly accessible through the \verb!Require! command
  (see section~\ref{Require});

\item[User contributions:] Other specification and proof developments
  coming from the \Coq\ users' community. These libraries are no
  longer distributed with the system.  They are available by anonymous
  FTP (see section~\ref{Contributions}).
\end{description}

This chapter briefly reviews these libraries.

\section{The basic library}
\label{Prelude}

This section lists the basic notions and results which are directly
available in the standard \Coq\ system
\footnote{Most of these constructions are defined in the
{\tt Prelude} module in directory {\tt theories/Init} at the {\Coq}
root directory; this includes the modules
{\tt Notations},
{\tt Logic},
{\tt Datatypes},
{\tt Specif},
{\tt Peano},
and {\tt Wf}.
Module {\tt Logic\_Type} also makes it in the initial state}.

\subsection{Notations} \label{Notations}

This module defines the parsing and pretty-printing of many symbols
(infixes, prefixes, etc.). However, it does not assign a meaning to these
notations. The purpose of this is to define precedence and
associativity of very common notations, and avoid users to use them
with other precedence, which may be confusing.

\begin{figure}
\begin{center}
\begin{tabular}{|cll|}
\hline
Notation & Precedence & Associativity \\
\hline
\verb!_ <-> _! & 95 & no \\
\verb!_ \/ _!  & 85 & right \\
\verb!_ /\ _!  & 80 & right \\
\verb!~ _!   & 75 & right \\
\verb!_ = _!   & 70 & no \\
\verb!_ = _ = _!   & 70 & no \\
\verb!_ = _ :> _!   & 70 & no \\
\verb!_ <> _!  & 70 & no \\
\verb!_ <> _ :> _!  & 70 & no \\
\verb!_ < _!   & 70 & no \\
\verb!_ > _!   & 70 & no \\
\verb!_ <= _!  & 70 & no \\
\verb!_ >= _!  & 70 & no \\
\verb!_ < _ < _! & 70 & no \\
\verb!_ < _ <= _! & 70 & no \\
\verb!_ <= _ < _! & 70 & no \\
\verb!_ <= _ <= _! & 70 & no \\
\verb!_ + _!   & 50 & left \\
\verb!_ - _!   & 50 & left \\
\verb!_ * _!   & 40 & left \\
\verb!_ / _!   & 40 & left \\
\verb!- _!  & 35 & right \\
\verb!/ _!  & 35 & right \\
\verb!_ ^ _!   & 30 & right \\
\hline
\end{tabular}
\end{center}
\caption{Notations in the initial state}
\label{init-notations}
\end{figure}

\subsection{Logic} 
\label{Logic}

\begin{figure}
\begin{centerframe}
\begin{tabular}{lclr}
{\form} & ::= & {\tt True} & ({\tt True})\\
  & $|$ & {\tt False} & ({\tt False})\\
  & $|$ & {\tt\char'176} {\form} & ({\tt not})\\
  & $|$ & {\form} {\tt /$\backslash$} {\form} & ({\tt and})\\
  & $|$ & {\form} {\tt $\backslash$/} {\form} & ({\tt or})\\
  & $|$ & {\form} {\tt ->} {\form} & (\em{primitive implication})\\
  & $|$ & {\form} {\tt <->} {\form} & ({\tt iff})\\
  & $|$ & {\tt forall} {\ident} {\tt :} {\type} {\tt ,}
  {\form} & (\em{primitive for all})\\
  & $|$ & {\tt exists} {\ident} \zeroone{{\tt :} {\specif}} {\tt
  ,} {\form}  & ({\tt ex})\\
  & $|$ & {\tt exists2} {\ident} \zeroone{{\tt :} {\specif}} {\tt
  ,} {\form}  {\tt \&} {\form} & ({\tt ex2})\\
  & $|$ & {\term} {\tt =} {\term} & ({\tt eq})\\
  & $|$ & {\term} {\tt =} {\term} {\tt :>} {\specif} & ({\tt eq})
\end{tabular}
\end{centerframe}
\caption{Syntax of formulas}
\label{formulas-syntax}
\end{figure}

The basic library of {\Coq} comes with the definitions of standard
(intuitionistic) logical connectives (they are defined as inductive
constructions). They are equipped with an appealing syntax enriching the
(subclass {\form}) of the syntactic class {\term}. The syntax
extension is shown on figure \ref{formulas-syntax}.

% The basic library of {\Coq} comes with the definitions of standard
% (intuitionistic) logical connectives (they are defined as inductive
% constructions). They are equipped with an appealing syntax enriching
% the (subclass {\form}) of the syntactic class {\term}. The syntax
% extension \footnote{This syntax is defined in module {\tt
%     LogicSyntax}} is shown on Figure~\ref{formulas-syntax}.
 
\Rem Implication is not defined but primitive (it is a non-dependent
product of a proposition over another proposition).  There is also a
primitive universal quantification (it is a dependent product over a
proposition). The primitive universal quantification allows both
first-order and higher-order quantification.

\subsubsection{Propositional Connectives} \label{Connectives}
\index{Connectives}

First, we find propositional calculus connectives:
\ttindex{True}
\ttindex{I}
\ttindex{False}
\ttindex{not}
\ttindex{and}
\ttindex{conj}
\ttindex{proj1}
\ttindex{proj2}

\begin{coq_eval}
Set Printing Depth 50.
\end{coq_eval}
\begin{coq_example*}
Inductive True : Prop := I.
Inductive False :  Prop := .
Definition not (A: Prop) := A -> False.
Inductive and (A B:Prop) : Prop := conj (_:A) (_:B).
Section Projections.
Variables A B : Prop.
Theorem proj1 : A /\ B -> A.
Theorem proj2 : A /\ B -> B.
\end{coq_example*}
\begin{coq_eval}
Abort All.
\end{coq_eval}
\ttindex{or}
\ttindex{or\_introl}
\ttindex{or\_intror}
\ttindex{iff}
\ttindex{IF\_then\_else}
\begin{coq_example*}
End Projections.
Inductive or (A B:Prop) : Prop :=
  | or_introl (_:A)
  | or_intror (_:B).
Definition iff (P Q:Prop) := (P -> Q) /\ (Q -> P).
Definition IF_then_else (P Q R:Prop) := P /\ Q \/ ~ P /\ R.
\end{coq_example*}

\subsubsection{Quantifiers} \label{Quantifiers}
\index{Quantifiers}

Then we find first-order quantifiers:
\ttindex{all}
\ttindex{ex}
\ttindex{exists}
\ttindex{ex\_intro}
\ttindex{ex2}
\ttindex{exists2}
\ttindex{ex\_intro2}

\begin{coq_example*}
Definition all (A:Set) (P:A -> Prop) := forall x:A, P x.
Inductive ex (A: Set) (P:A -> Prop) : Prop :=
    ex_intro (x:A) (_:P x).
Inductive ex2 (A:Set) (P Q:A -> Prop) : Prop :=
    ex_intro2 (x:A) (_:P x) (_:Q x).
\end{coq_example*}

The following abbreviations are allowed:
\begin{center}
  \begin{tabular}[h]{|l|l|}
    \hline
    \verb+exists x:A, P+     & \verb+ex A (fun x:A => P)+ \\
    \verb+exists x, P+       & \verb+ex _ (fun x => P)+ \\
    \verb+exists2 x:A, P & Q+ & \verb+ex2 A (fun x:A => P) (fun x:A => Q)+ \\
    \verb+exists2 x, P & Q+   & \verb+ex2 _ (fun x => P) (fun x => Q)+ \\
    \hline
  \end{tabular}
\end{center}

The type annotation \texttt{:A} can be omitted when \texttt{A} can be
synthesized by the system.

\subsubsection{Equality} \label{Equality}
\index{Equality}

Then, we find equality, defined as an inductive relation. That is,
given a \verb:Type: \verb:A: and an \verb:x: of type \verb:A:, the
predicate \verb:(eq A x): is the smallest one which contains \verb:x:.
This definition, due to Christine Paulin-Mohring, is equivalent to
define \verb:eq: as the smallest reflexive relation, and it is also
equivalent to Leibniz' equality.

\ttindex{eq}
\ttindex{refl\_equal}

\begin{coq_example*}
Inductive eq (A:Type) (x:A) : A -> Prop :=
    refl_equal : eq A x x.
\end{coq_example*}

\subsubsection{Lemmas} 
\label{PreludeLemmas}

Finally, a few easy lemmas are provided.

\ttindex{absurd}

\begin{coq_example*}
Theorem absurd : forall A C:Prop, A -> ~ A -> C.
\end{coq_example*}
\begin{coq_eval}
Abort.
\end{coq_eval}
\ttindex{sym\_eq}
\ttindex{trans\_eq}
\ttindex{f\_equal}
\ttindex{sym\_not\_eq}
\begin{coq_example*}
Section equality.
Variables A B : Type.
Variable f : A -> B.
Variables x y z : A.
Theorem sym_eq : x = y -> y = x.
Theorem trans_eq : x = y -> y = z -> x = z.
Theorem f_equal : x = y -> f x = f y.
Theorem sym_not_eq : x <> y -> y <> x.
\end{coq_example*}
\begin{coq_eval}
Abort.
Abort.
Abort.
Abort.
\end{coq_eval}
\ttindex{eq\_ind\_r}
\ttindex{eq\_rec\_r}
\ttindex{eq\_rect}
\ttindex{eq\_rect\_r}
%Definition eq_rect: (A:Set)(x:A)(P:A->Type)(P x)->(y:A)(x=y)->(P y).
\begin{coq_example*}
End equality.
Definition eq_ind_r :
  forall (A:Type) (x:A) (P:A -> Prop), P x -> forall y:A, y = x -> P y.
Definition eq_rec_r :
  forall (A:Type) (x:A) (P:A -> Set), P x -> forall y:A, y = x -> P y.
Definition eq_rect_r :
  forall (A:Type) (x:A) (P:A -> Type), P x -> forall y:A, y = x -> P y.
\end{coq_example*}
\begin{coq_eval}
Abort.
Abort.
Abort.
\end{coq_eval}
%Abort (for now predefined eq_rect)
\begin{coq_example*}
Hint Immediate sym_eq sym_not_eq : core.
\end{coq_example*}
\ttindex{f\_equal$i$}

The theorem {\tt f\_equal} is extended to functions with two to five
arguments. The theorem are names {\tt f\_equal2}, {\tt f\_equal3}, 
{\tt f\_equal4} and {\tt f\_equal5}.
For instance {\tt f\_equal3} is defined the following way.
\begin{coq_example*}
Theorem f_equal3 :
 forall (A1 A2 A3 B:Type) (f:A1 -> A2 -> A3 -> B) (x1 y1:A1) (x2 y2:A2)
   (x3 y3:A3), x1 = y1 -> x2 = y2 -> x3 = y3 -> f x1 x2 x3 = f y1 y2 y3.
\end{coq_example*}
\begin{coq_eval}
Abort.
\end{coq_eval}

\subsection{Datatypes}
\label{Datatypes}
\index{Datatypes}

\begin{figure}
\begin{centerframe}
\begin{tabular}{rclr}
{\specif} & ::= & {\specif} {\tt *} {\specif} & ({\tt prod})\\
  & $|$ & {\specif} {\tt +} {\specif} & ({\tt sum})\\
  & $|$ & {\specif} {\tt + \{} {\specif} {\tt \}} & ({\tt sumor})\\
  & $|$ & {\tt \{} {\specif} {\tt \} + \{} {\specif} {\tt \}} &
  ({\tt sumbool})\\  
  & $|$ & {\tt \{} {\ident} {\tt :} {\specif} {\tt |} {\form} {\tt \}}
  & ({\tt sig})\\
  & $|$ & {\tt \{} {\ident} {\tt :} {\specif} {\tt |} {\form}  {\tt \&}
  {\form} {\tt \}} & ({\tt sig2})\\
  & $|$ & {\tt \{} {\ident} {\tt :} {\specif} {\tt \&} {\specif} {\tt
    \}} & ({\tt sigS})\\
  & $|$ & {\tt \{} {\ident} {\tt :} {\specif} {\tt \&} {\specif} {\tt
    \&} {\specif} {\tt \}} & ({\tt sigS2})\\
  &  & & \\
{\term} & ::= & {\tt (} {\term} {\tt ,} {\term} {\tt )} & ({\tt pair})
\end{tabular}
\end{centerframe}
\caption{Syntax of datatypes and specifications}
\label{specif-syntax}
\end{figure}


In the basic library, we find the definition\footnote{They are in {\tt
    Datatypes.v}} of the basic data-types of programming, again
defined as inductive constructions over the sort \verb:Set:. Some of
them come with a special syntax shown on Figure~\ref{specif-syntax}.

\subsubsection{Programming} 
\label{Programming}
\index{Programming}
\label{libnats}

\ttindex{unit}
\ttindex{tt}
\ttindex{bool}
\ttindex{true}
\ttindex{false}
\ttindex{nat}
\ttindex{O}
\ttindex{S}
\ttindex{option}
\ttindex{Some}
\ttindex{None}
\ttindex{identity}
\ttindex{refl\_identity}

\begin{coq_example*}
Inductive unit : Set := tt.
Inductive bool : Set := true | false.
Inductive nat : Set := O | S (n:nat).
Inductive option (A:Set) : Set := Some (_:A) | None.
Inductive identity (A:Type) (a:A) : A -> Type :=
    refl_identity : identity A a a.
\end{coq_example*}

Note that zero is the letter \verb:O:, and {\sl not} the numeral
\verb:0:.

{\tt identity} is logically equivalent to equality but it lives in
sort {\tt Set}. Computationaly, it behaves like {\tt unit}.

We then define the disjoint sum of \verb:A+B: of two sets \verb:A: and
\verb:B:, and their product \verb:A*B:.
\ttindex{sum}
\ttindex{A+B}
\ttindex{+}
\ttindex{inl}
\ttindex{inr}
\ttindex{prod}
\ttindex{A*B}
\ttindex{*}
\ttindex{pair}
\ttindex{fst}
\ttindex{snd}

\begin{coq_example*}
Inductive sum (A B:Set) : Set := inl (_:A) | inr (_:B).
Inductive prod (A B:Set) : Set := pair (_:A) (_:B).
Section projections.
Variables A B : Set.
Definition fst (H: prod A B) := match H with
                                | pair x y => x
                                end.
Definition snd (H: prod A B) := match H with
                                | pair x y => y
                                end.
End projections.
\end{coq_example*}

\subsection{Specification}

The following notions\footnote{They are defined in module {\tt
Specif.v}} allows to build new datatypes and specifications. 
They are available with the syntax shown on
Figure~\ref{specif-syntax}\footnote{This syntax can be found in the module
{\tt SpecifSyntax.v}}.

For instance, given \verb|A:Set| and \verb|P:A->Prop|, the construct
\verb+{x:A | P x}+ (in abstract syntax \verb+(sig A P)+) is a
\verb:Set:. We may build elements of this set as \verb:(exist x p):
whenever we have a witness \verb|x:A| with its justification
\verb|p:P x|.

From such a \verb:(exist x p): we may in turn extract its witness
\verb|x:A| (using an elimination construct such as \verb:match:) but
{\sl not} its justification, which stays hidden, like in an abstract
data type. In technical terms, one says that \verb:sig: is a ``weak
(dependent) sum''.  A variant \verb:sig2: with two predicates is also
provided.

\index{\{x:A "| (P x)\}}
\index{"|}
\ttindex{sig}
\ttindex{exist}
\ttindex{sig2}
\ttindex{exist2}

\begin{coq_example*}
Inductive sig (A:Set) (P:A -> Prop) : Set := exist (x:A) (_:P x).
Inductive sig2 (A:Set) (P Q:A -> Prop) : Set := 
  exist2 (x:A) (_:P x) (_:Q x).
\end{coq_example*}

A ``strong (dependent) sum'' \verb+{x:A & (P x)}+ may be also defined,
when the predicate \verb:P: is now defined as a \verb:Set:
constructor.

\ttindex{\{x:A \& (P x)\}}
\ttindex{\&}
\ttindex{sigS}
\ttindex{existS}
\ttindex{projS1}
\ttindex{projS2}
\ttindex{sigS2}
\ttindex{existS2}

\begin{coq_example*}
Inductive sigS (A:Set) (P:A -> Set) : Set := existS (x:A) (_:P x).
Section sigSprojections.
Variable A : Set.
Variable P : A -> Set.
Definition projS1 (H:sigS A P) := let (x, h) := H in x.
Definition projS2 (H:sigS A P) :=
  match H return P (projS1 H) with
    existS x h => h
  end.
End sigSprojections.
Inductive sigS2 (A: Set) (P Q:A -> Set) : Set :=
    existS2 (x:A) (_:P x) (_:Q x).
\end{coq_example*}

A related non-dependent construct is the constructive sum
\verb"{A}+{B}" of two propositions \verb:A: and \verb:B:.
\label{sumbool}
\ttindex{sumbool}
\ttindex{left}
\ttindex{right}
\ttindex{\{A\}+\{B\}}

\begin{coq_example*}
Inductive sumbool (A B:Prop) : Set := left (_:A) | right (_:B).
\end{coq_example*}

This \verb"sumbool" construct may be used as a kind of indexed boolean
data type. An intermediate between \verb"sumbool" and \verb"sum" is
the mixed \verb"sumor" which combines \verb"A:Set" and \verb"B:Prop"
in the \verb"Set" \verb"A+{B}".
\ttindex{sumor}
\ttindex{inleft}
\ttindex{inright}
\ttindex{A+\{B\}}

\begin{coq_example*}
Inductive sumor (A:Set) (B:Prop) : Set := inleft (_:A) | inright (_:B).
\end{coq_example*}

We may define variants of the axiom of choice, like in Martin-L�f's
Intuitionistic Type Theory.
\ttindex{Choice}
\ttindex{Choice2}
\ttindex{bool\_choice}

\begin{coq_example*}
Lemma Choice :
 forall (S S':Set) (R:S -> S' -> Prop),
   (forall x:S, {y : S' | R x y}) ->
   {f : S -> S' | forall z:S, R z (f z)}.
Lemma Choice2 :
 forall (S S':Set) (R:S -> S' -> Set),
   (forall x:S, {y : S' &  R x y}) ->
   {f : S -> S' &  forall z:S, R z (f z)}.
Lemma bool_choice :
 forall (S:Set) (R1 R2:S -> Prop),
   (forall x:S, {R1 x} + {R2 x}) ->
   {f : S -> bool |
   forall x:S, f x = true /\ R1 x \/ f x = false /\ R2 x}.
\end{coq_example*}
\begin{coq_eval}
Abort.
Abort.
Abort.
\end{coq_eval}

The next constructs builds a sum between a data type \verb|A:Set| and
an exceptional value encoding errors:

\ttindex{Exc}
\ttindex{value}
\ttindex{error}

\begin{coq_example*}
Definition Exc := option.
Definition value := Some.
Definition error := None.
\end{coq_example*}


This module ends with theorems, 
relating the sorts \verb:Set: and
\verb:Prop: in a way which is consistent with the realizability
interpretation.
\ttindex{False\_rec}
\ttindex{eq\_rec}
\ttindex{Except}
\ttindex{absurd\_set}
\ttindex{and\_rec}

%Lemma False_rec : (P:Set)False->P.
%Lemma False_rect : (P:Type)False->P.
\begin{coq_example*}
Definition except := False_rec.
Notation Except := (except _).
Theorem absurd_set : forall (A:Prop) (C:Set), A -> ~ A -> C.
Theorem and_rec :
 forall (A B:Prop) (P:Set), (A -> B -> P) -> A /\ B -> P.
\end{coq_example*}
%\begin{coq_eval}
%Abort.
%Abort.
%\end{coq_eval}

\subsection{Basic Arithmetics}

The basic library includes a few elementary properties of natural
numbers, together with the definitions of predecessor, addition and
multiplication\footnote{This is in module {\tt Peano.v}}. It also
provides a scope {\tt nat\_scope} gathering standard notations for
common operations (+,*) and a decimal notation for numbers. That is he
can write \texttt{3} for \texttt{(S (S (S O)))}. This also works on
the left hand side of a \texttt{match} expression (see for example
section~\ref{refine-example}). This scope is opened by default.

%Remove the redefinition of nat
\begin{coq_eval}
Reset Initial.
\end{coq_eval}

The following example is not part of the standard library, but it
shows the usage of the notations:

\begin{coq_example*}
Fixpoint even (n:nat) : bool :=
  match n with
  | 0 => true
  | 1 => false
  | S (S n) => even n
  end.
\end{coq_example*}


\ttindex{eq\_S}
\ttindex{pred}
\ttindex{pred\_Sn}
\ttindex{eq\_add\_S}
\ttindex{not\_eq\_S}
\ttindex{IsSucc}
\ttindex{O\_S}
\ttindex{n\_Sn}
\ttindex{plus}
\ttindex{plus\_n\_O}
\ttindex{plus\_n\_Sm}
\ttindex{mult}
\ttindex{mult\_n\_O}
\ttindex{mult\_n\_Sm}

\begin{coq_example*}
Theorem eq_S : forall x y:nat, x = y -> S x = S y.
\end{coq_example*}
\begin{coq_eval}
Abort.
\end{coq_eval}
\begin{coq_example*}
Definition pred (n:nat) : nat :=
  match n with
  | 0 => 0
  | S u => u
  end.
Theorem pred_Sn : forall m:nat, m = pred (S m).
Theorem eq_add_S : forall n m:nat, S n = S m -> n = m.
Hint Immediate eq_add_S : core.
Theorem not_eq_S : forall n m:nat, n <> m -> S n <> S m.
\end{coq_example*}
\begin{coq_eval}
Abort All.
\end{coq_eval}
\begin{coq_example*}
Definition IsSucc (n:nat) : Prop :=
  match n with
  | 0 => False
  | S p => True
  end.
Theorem O_S : forall n:nat, 0 <> S n.
Theorem n_Sn : forall n:nat, n <> S n.
\end{coq_example*}
\begin{coq_eval}
Abort All.
\end{coq_eval}
\begin{coq_example*}
Fixpoint plus (n m:nat) {struct n} : nat :=
  match n with
  | 0 => m
  | S p => S (plus p m)
  end.
Lemma plus_n_O : forall n:nat, n = plus n 0.
Lemma plus_n_Sm : forall n m:nat, S (plus n m) = plus n (S m).
\end{coq_example*}
\begin{coq_eval}
Abort All.
\end{coq_eval}
\begin{coq_example*}
Fixpoint mult (n m:nat) {struct n} : nat :=
  match n with
  | 0 => 0
  | S p => m + mult p m
  end.
Lemma mult_n_O : forall n:nat, 0 = mult n 0.
Lemma mult_n_Sm : forall n m:nat, plus (mult n m) n = mult n (S m).
\end{coq_example*}
\begin{coq_eval}
Abort All.
\end{coq_eval}

Finally, it gives the definition of the usual orderings \verb:le:,
\verb:lt:, \verb:ge:, and \verb:gt:.
\ttindex{le}
\ttindex{le\_n}
\ttindex{le\_S}
\ttindex{lt}
\ttindex{ge}
\ttindex{gt}

\begin{coq_example*}
Inductive le (n:nat) : nat -> Prop :=
  | le_n : le n n
  | le_S : forall m:nat, le n m -> le n (S m).
Infix "+" := plus : nat_scope.
Definition lt (n m:nat) := S n <= m.
Definition ge (n m:nat) := m <= n.
Definition gt (n m:nat) := m < n.
\end{coq_example*}

Properties of these relations are not initially known, but may be
required by the user from modules \verb:Le: and \verb:Lt:.  Finally,
\verb:Peano: gives some lemmas allowing pattern-matching, and a double
induction principle.

\ttindex{nat\_case}
\ttindex{nat\_double\_ind}

\begin{coq_example*}
Theorem nat_case :
 forall (n:nat) (P:nat -> Prop), P 0 -> (forall m:nat, P (S m)) -> P n.
\end{coq_example*}
\begin{coq_eval}
Abort All.
\end{coq_eval}
\begin{coq_example*}
Theorem nat_double_ind :
 forall R:nat -> nat -> Prop,
   (forall n:nat, R 0 n) ->
   (forall n:nat, R (S n) 0) ->
   (forall n m:nat, R n m -> R (S n) (S m)) -> forall n m:nat, R n m.
\end{coq_example*}
\begin{coq_eval}
Abort All.
\end{coq_eval}

\subsection{Well-founded recursion}

The basic library contains the basics of well-founded recursion and 
well-founded induction\footnote{This is defined in module {\tt Wf.v}}.
\index{Well foundedness}
\index{Recursion}
\index{Well founded induction}
\ttindex{Acc}
\ttindex{Acc\_inv}
\ttindex{Acc\_rec}
\ttindex{well\_founded}

\begin{coq_example*}
Section Well_founded.
Variable A : Set.
Variable R : A -> A -> Prop.
Inductive Acc : A -> Prop :=
    Acc_intro : forall x:A, (forall y:A, R y x -> Acc y) -> Acc x.
Lemma Acc_inv : forall x:A, Acc x -> forall y:A, R y x -> Acc y.
\end{coq_example*}
\begin{coq_eval}
simple destruct 1; trivial.
Defined.
\end{coq_eval}
\begin{coq_example*}
Section AccRec.
Variable P : A -> Set.
Variable F :
    forall x:A,
      (forall y:A, R y x -> Acc y) -> (forall y:A, R y x -> P y) -> P x.
Fixpoint Acc_rec (x:A) (a:Acc x) {struct a} : P x :=
  F x (Acc_inv x a)
    (fun (y:A) (h:R y x) => Acc_rec y (Acc_inv x a y h)).
End AccRec.
Definition well_founded := forall a:A, Acc a.
Hypothesis Rwf : well_founded.
Theorem well_founded_induction :
 forall P:A -> Set,
   (forall x:A, (forall y:A, R y x -> P y) -> P x) -> forall a:A, P a.
Theorem well_founded_ind :
 forall P:A -> Prop,
   (forall x:A, (forall y:A, R y x -> P y) -> P x) -> forall a:A, P a.
\end{coq_example*}
\begin{coq_eval}
Abort All.
\end{coq_eval}
{\tt Acc\_rec} can be used to define functions by fixpoints using
well-founded relations to  justify termination. Assuming
extensionality of the functional used for the recursive call, the
fixpoint equation can be proved.
\ttindex{Fix\_F}
\ttindex{fix\_eq}
\ttindex{Fix\_F\_inv}
\ttindex{Fix\_F\_eq}
\begin{coq_example*}
Section FixPoint.
Variable P : A -> Set.
Variable F : forall x:A, (forall y:A, R y x -> P y) -> P x.
Fixpoint Fix_F (x:A) (r:Acc x) {struct r} : P x :=
  F x (fun (y:A) (p:R y x) => Fix_F y (Acc_inv x r y p)).
Definition Fix (x:A) := Fix_F x (Rwf x).
Hypothesis F_ext :
    forall (x:A) (f g:forall y:A, R y x -> P y),
      (forall (y:A) (p:R y x), f y p = g y p) -> F x f = F x g.
Lemma Fix_F_eq :
 forall (x:A) (r:Acc x),
   F x (fun (y:A) (p:R y x) => Fix_F y (Acc_inv x r y p)) = Fix_F x r.
Lemma Fix_F_inv : forall (x:A) (r s:Acc x), Fix_F x r = Fix_F x s.
Lemma fix_eq : forall x:A, Fix x = F x (fun (y:A) (p:R y x) => Fix y).
\end{coq_example*}
\begin{coq_eval}
Abort All.
\end{coq_eval}
\begin{coq_example*}
End FixPoint.
End Well_founded.
\end{coq_example*}

\subsection{Accessing the {\Type} level}

The basic library includes the definitions\footnote{This is in module
{\tt Logic\_Type.v}} of the counterparts of some datatypes and logical
quantifiers at the \verb:Type: level: negation, pair, and properties
of {\tt identity}.

\ttindex{notT}
\ttindex{prodT}
\ttindex{pairT}
\begin{coq_eval}
Reset Initial.
\end{coq_eval}
\begin{coq_example*}
Definition notT (A:Type) := A -> False.
Inductive prodT (A B:Type) : Type := pairT (_:A) (_:B).
\end{coq_example*}


At the end, it defines datatypes at the {\Type} level.

\section{The standard library}

\subsection{Survey}

The rest of the standard library is structured into the following 
subdirectories:

\begin{tabular}{lp{12cm}}
  {\bf Logic}   & Classical logic and dependent equality \\
  {\bf Arith}   & Basic Peano arithmetic \\
  {\bf ZArith}  & Basic integer arithmetic \\
  {\bf Bool}    &  Booleans (basic functions and results) \\
  {\bf Lists}   & Monomorphic and polymorphic lists (basic functions and
            results), Streams (infinite sequences defined with co-inductive
            types) \\
  {\bf Sets}    & Sets (classical, constructive, finite, infinite, power set,
            etc.) \\
  {\bf IntMap}    & Representation of finite sets by an efficient
  structure of map (trees indexed by binary integers).\\
 {\bf Reals}   & Axiomatization of Real Numbers (classical, basic functions, 
                 integer part, fractional part, limit, derivative, Cauchy 
                 series, power series and results,... Requires the 
                 \textbf{ZArith} library).\\
 {\bf Relations} & Relations (definitions and basic results). \\
 {\bf Sorting} & Sorted list (basic definitions and heapsort correctness). \\
 {\bf Wellfounded} & Well-founded relations (basic results). \\

\end{tabular}
\medskip

These directories belong to the initial load path of the system, and
the modules they provide are compiled at installation time. So they
are directly accessible with the command \verb!Require! (see
chapter~\ref{Other-commands}). 

The different modules of the \Coq\ standard library are described in the
additional document \verb!Library.dvi!. They are also accessible on the WWW
through the \Coq\ homepage
\footnote{\texttt{http://coq.inria.fr}}.

\subsection{Notations for integer arithmetics}
\index{Arithmetical notations}

On figure \ref{zarith-syntax} is described the syntax of expressions
for integer arithmetics. It is provided by requiring and opening the
module {\tt ZArith} and opening scope {\tt Z\_scope}.

\ttindex{+}
\ttindex{*}
\ttindex{-}
\ttindex{/}
\ttindex{<=}
\ttindex{>=}
\ttindex{<}
\ttindex{>}
\ttindex{?=}
\ttindex{mod}

\begin{figure}
\begin{center}
\begin{tabular}{l|l|l|l}
Notation & Interpretation & Precedence & Associativity\\
\hline
\verb!_ < _! & {\tt Zlt} &&\\
\verb!x <= y! & {\tt Zle} &&\\
\verb!_ > _! & {\tt Zgt} &&\\
\verb!x >= y! & {\tt Zge} &&\\
\verb!x < y < z! & {\tt x < y \verb!/\! y < z} &&\\
\verb!x < y <= z! & {\tt x < y \verb!/\! y <= z} &&\\
\verb!x <= y < z! & {\tt x <= y \verb!/\! y < z} &&\\
\verb!x <= y <= z! & {\tt x <= y \verb!/\! y <= z} &&\\
\verb!_ ?= _! & {\tt Zcompare} & 70 & no\\
\verb!_ + _! & {\tt Zplus} &&\\
\verb!_ - _! & {\tt Zminus} &&\\
\verb!_ * _! & {\tt Zmult} &&\\
\verb!_ / _! & {\tt Zdiv} &&\\
\verb!_ mod _! & {\tt Zmod} & 40 & no \\
\verb!- _!  & {\tt Zopp} &&\\
\verb!_ ^ _! & {\tt Zpower} &&\\
\end{tabular}
\end{center}
\label{zarith-syntax}
\caption{Definition of the scope for integer arithmetics ({\tt Z\_scope})}
\end{figure}

Figure~\ref{zarith-syntax} shows the notations provided by {\tt
Z\_scope}. It specifies how notations are interpreted and, when not
already reserved, the precedence and associativity.

\begin{coq_example}
Require Import ZArith.
Check  (2 + 3)%Z.
Open Scope Z_scope.
Check 2 + 3.
\end{coq_example}

\subsection{Peano's arithmetic (\texttt{nat})}
\index{Peano's arithmetic}
\ttindex{nat\_scope}

While in the initial state, many operations and predicates of Peano's
arithmetic are defined, further operations and results belong to other
modules. For instance, the decidability of the basic predicates are
defined here. This is provided by requiring the module {\tt Arith}.

Figure~\ref{nat-syntax} describes notation available in scope {\tt
nat\_scope}.

\begin{figure}
\begin{center}
\begin{tabular}{l|l}
Notation & Interpretation \\
\hline
\verb!_ < _! & {\tt lt} \\
\verb!x <= y! & {\tt le} \\
\verb!_ > _! & {\tt gt} \\
\verb!x >= y! & {\tt ge} \\
\verb!x < y < z! & {\tt x < y \verb!/\! y < z} \\
\verb!x < y <= z! & {\tt x < y \verb!/\! y <= z} \\
\verb!x <= y < z! & {\tt x <= y \verb!/\! y < z} \\
\verb!x <= y <= z! & {\tt x <= y \verb!/\! y <= z} \\
\verb!_ + _! & {\tt plus} \\
\verb!_ - _! & {\tt minus} \\
\verb!_ * _! & {\tt mult} \\
\end{tabular}
\end{center}
\label{nat-syntax}
\caption{Definition of the scope for natural numbers ({\tt nat\_scope})}
\end{figure}

\subsection{Real numbers library}

\subsubsection{Notations for real numbers}
\index{Notations for real numbers}

This is provided by requiring and opening the module {\tt Reals} and
opening scope {\tt R\_scope}. This set of notations is very similar to
the notation for integer arithmetics. The inverse function was added.
\begin{figure}
\begin{center}
\begin{tabular}{l|l}
Notation & Interpretation \\
\hline
\verb!_ < _! & {\tt Rlt} \\
\verb!x <= y! & {\tt Rle} \\
\verb!_ > _! & {\tt Rgt} \\
\verb!x >= y! & {\tt Rge} \\
\verb!x < y < z! & {\tt x < y \verb!/\! y < z} \\
\verb!x < y <= z! & {\tt x < y \verb!/\! y <= z} \\
\verb!x <= y < z! & {\tt x <= y \verb!/\! y < z} \\
\verb!x <= y <= z! & {\tt x <= y \verb!/\! y <= z} \\
\verb!_ + _! & {\tt Rplus} \\
\verb!_ - _! & {\tt Rminus} \\
\verb!_ * _! & {\tt Rmult} \\
\verb!_ / _! & {\tt Rdiv} \\
\verb!- _!  & {\tt Ropp} \\
\verb!/ _!  & {\tt Rinv} \\
\verb!_ ^ _! & {\tt pow} \\
\end{tabular}
\end{center}
\label{reals-syntax}
\caption{Definition of the scope for real arithmetics ({\tt R\_scope})}
\end{figure}

\begin{coq_eval}
Reset Initial.
\end{coq_eval}
\begin{coq_example}
Require Import Reals.
Check  (2 + 3)%R.
Open Scope R_scope.
Check 2 + 3.
\end{coq_example}

\subsubsection{Some tactics}

In addition to the \verb|ring|, \verb|field| and \verb|fourier|
tactics (see Chapter~\ref{Tactics}) there are:
\begin{itemize}
\item {\tt discrR} \tacindex{discrR}
  
  Proves that a real integer constant $c_1$ is different from another
  real integer constant $c_2$.  

\begin{coq_example*}
Require Import DiscrR.
Goal 5 <> 0.
\end{coq_example*}

\begin{coq_example}
discrR.
\end{coq_example}

\begin{coq_eval}
Abort.
\end{coq_eval}

\item {\tt split\_Rabs} allows to unfold {\tt Rabs} constant and splits 
corresponding conjonctions.
\tacindex{split\_Rabs}

\begin{coq_example*}
Require Import SplitAbsolu.
Goal forall x:R, x <= Rabs x.
\end{coq_example*}

\begin{coq_example}
intro; split_Rabs.
\end{coq_example}

\begin{coq_eval}
Abort.
\end{coq_eval}

\item {\tt split\_Rmult} allows to split a condition that a product is
  non null into subgoals corresponding to the condition on each
  operand of the product. 
\tacindex{split\_Rmult}

\begin{coq_example*}
Require Import SplitRmult.
Goal forall x y z:R, x * y * z <> 0.
\end{coq_example*}

\begin{coq_example}
intros; split_Rmult.
\end{coq_example}

\end{itemize}

All this tactics has been written with the tactic language Ltac
described in Chapter~\ref{TacticLanguage}.  More details are available
in document \url{http://coq.inria.fr/~desmettr/Reals.ps}.

\begin{coq_eval}
Reset Initial.
\end{coq_eval}

\subsection{List library}
\index{Notations for lists}
\ttindex{length}
\ttindex{head}
\ttindex{tail}
\ttindex{app}
\ttindex{rev}
\ttindex{nth}
\ttindex{map}
\ttindex{flat\_map}
\ttindex{fold\_left}
\ttindex{fold\_right}

Some elementary operations on polymorphic lists are defined here. They
can be accessed by requiring module {\tt List}.

It defines the following notions:
\begin{center}
\begin{tabular}{l|l}
\hline
{\tt length} & length \\
{\tt head} & first element (with default) \\
{\tt tail} & all but first element \\
{\tt app} & concatenation \\
{\tt rev} & reverse \\
{\tt nth} & accessing $n$-th element (with default) \\
{\tt map} & applying a function \\
{\tt flat\_map} & applying a function returning lists \\
{\tt fold\_left} & iterator (from head to tail) \\
{\tt fold\_right} & iterator (from tail to head) \\
\hline
\end{tabular}
\end{center}

Table show notations available when opening scope {\tt list\_scope}.

\begin{figure}
\begin{center}
\begin{tabular}{l|l|l|l}
Notation & Interpretation & Precedence & Associativity\\
\hline
\verb!_ ++ _! & {\tt app} & 60 & right \\
\verb!_ :: _! & {\tt cons} & 60 & right \\
\end{tabular}
\end{center}
\label{list-syntax}
\caption{Definition of the scope for lists ({\tt list\_scope})}
\end{figure}


\section{Users' contributions}
\index{Contributions}
\label{Contributions}

Numerous users' contributions have been collected and are available at
URL \url{coq.inria.fr/contribs/}.  On this web page, you have a list
of all contributions with informations (author, institution, quick
description, etc.) and the possibility to download them one by one.
There is a small search engine to look for keywords in all
contributions.  You will also find informations on how to submit a new
contribution.

The users' contributions may also be obtained by anonymous FTP from site
\verb:ftp.inria.fr:, in directory \verb:INRIA/coq/: and
searchable on-line at \url{http://coq.inria.fr/contribs-eng.html}

% $Id$ 

%%% Local Variables: 
%%% mode: latex
%%% TeX-master: "Reference-Manual"
%%% End: 
