\chapter[Vernacular commands]{Vernacular commands\label{Vernacular-commands}
\label{Other-commands}}

\section{Displaying}

\subsection[\tt Print {\qualid}.]{\tt Print {\qualid}.\comindex{Print}}
This command displays on the screen informations about the declared or
defined object referred by {\qualid}.

\begin{ErrMsgs}
\item {\qualid} \errindex{not a defined object}
\end{ErrMsgs}

\begin{Variants}
\item {\tt Print Term {\qualid}.}
\comindex{Print Term}\\ 
This is a synonym to {\tt Print {\qualid}} when {\qualid} denotes a
global constant. 

\item {\tt About {\qualid}.}
\label{About}
\comindex{About}\\ 
This displays various informations about the object denoted by {\qualid}:
its kind (module, constant, assumption, inductive,
constructor, abbreviation, \ldots), long name, type, implicit
arguments and argument scopes. It does not print the body of
definitions or proofs.

%\item {\tt Print Proof {\qualid}.}\comindex{Print Proof}\\
%In case \qualid\ denotes an opaque theorem defined in a section,
%it is stored on a special unprintable form and displayed as 
%{\tt <recipe>}. {\tt Print Proof} forces the printable form of \qualid\
%to be computed and displays it.
\end{Variants}

\subsection[\tt Print All.]{\tt Print All.\comindex{Print All}}
This command displays informations about the current state of the
environment, including sections and modules.

\begin{Variants}
\item {\tt Inspect \num.}\comindex{Inspect}\\
This command displays the {\num} last objects of the current
environment, including sections and modules.
\item {\tt Print Section {\ident}.}\comindex{Print Section}\\
should correspond to a currently open section, this command
displays the objects defined since the beginning of this section.
% Discontinued
%% \item {\tt Print.}\comindex{Print}\\
%% This command displays the axioms and variables declarations in the
%% environment as well as the constants defined since the last variable
%% was introduced.
\end{Variants}

\section{Options and Flags}
\subsection[\tt Set {\rm\sl option} {\rm\sl value}.]{\tt Set {\rm\sl option} {\rm\sl value}.\comindex{Set}}
This command sets {\rm\sl option} to {\rm\sl value}. The original value of
{\rm\sl option} is restored when the current module ends.

\begin{Variants}
\item {\tt Set {\rm\sl flag}.}\\
This command switches {\rm\sl flag} on. The original state of
{\rm\sl flag} is restored when the current module ends.
\item {\tt Local Set {\rm\sl option} {\rm\sl value}.\comindex{Local Set}}
This command sets {\rm\sl option} to {\rm\sl value}. The original value of
{\rm\sl option} is restored when the current \emph{section} ends.
\item {\tt Local Set {\rm\sl flag}.}\\
This command switches {\rm\sl flag} on. The original state of
{\rm\sl flag} is restored when the current \emph{section} ends.
\item {\tt Global Set {\rm\sl option} {\rm\sl value}.\comindex{Global Set}}
This command sets {\rm\sl option} to {\rm\sl value}. The original value of
{\rm\sl option} is \emph{not} restored at the end of the module. Additionally,
if set in a file, {\rm\sl option} is set to {\rm\sl value} when the file is
{\tt Require}-d.
\item {\tt Global Set {\rm\sl flag}.}\\
This command switches {\rm\sl flag} on. The original state of
{\rm\sl flag} is \emph{not} restored at the end of the module. Additionally,
if set in a file, {\rm\sl flag} is switched on when the file is
{\tt Require}-d.
\end{Variants}

\subsection[\tt Unset {\rm\sl flag}.]{\tt Unset {\rm\sl flag}.\comindex{Unset}}
This command switches {\rm\sl flag} off. The original state of {\rm\sl flag}
is restored when the current module ends.

\begin{Variants}
\item {\tt Local Unset {\rm\sl flag}.\comindex{Local Unset}}\\
This command switches {\rm\sl flag} off. The original state of {\rm\sl flag}
is restored when the current \emph{section} ends.
\item {\tt Global Unset {\rm\sl flag}.\comindex{Global Unset}}\\
This command switches {\rm\sl flag} off.  The original state of
{\rm\sl flag} is \emph{not} restored at the end of the module. Additionally,
if set in a file, {\rm\sl flag} is switched on when the file is
{\tt Require}-d.
\end{Variants}

\subsection[\tt Test {\rm\sl option}.]{\tt Test {\rm\sl option}.\comindex{Test}}
This command prints the current value of {\rm\sl option}.

\begin{Variants}
\item {\tt Test {\rm\sl flag}.}\\
This command prints whether {\rm\sl flag} is on or off.
\end{Variants}

\section{Requests to the environment}

\subsection[\tt Check {\term}.]{\tt Check {\term}.\label{Check}
\comindex{Check}}
This command displays the type of {\term}. When called in proof mode, 
the term is checked in the local context of the current subgoal.

\subsection[\tt Eval {\rm\sl convtactic} in {\term}.]{\tt Eval {\rm\sl convtactic} in {\term}.\comindex{Eval}}

This command performs the specified reduction on {\term}, and displays
the resulting term with its type. The term to be reduced may depend on
hypothesis introduced in the first subgoal (if a proof is in
progress).

\SeeAlso Section~\ref{Conversion-tactics}.

\subsection[\tt Compute {\term}.]{\tt Compute {\term}.\comindex{Compute}}

This command performs a call-by-value evaluation of {\term} by using
the bytecode-based virtual machine. It is a shortcut for
{\tt Eval vm\_compute in {\term}}.

\SeeAlso Section~\ref{Conversion-tactics}.

\subsection[\tt Extraction \term.]{\tt Extraction \term.\label{ExtractionTerm}
\comindex{Extraction}} 
This command displays the extracted term from
{\term}. The extraction is processed according to the distinction
between {\Set} and {\Prop}; that is to say, between logical and
computational content (see Section~\ref{Sorts}). The extracted term is
displayed in Objective Caml syntax, where global identifiers are still
displayed as in \Coq\ terms.

\begin{Variants}
\item \texttt{Recursive Extraction} {\qualid$_1$} \ldots{} {\qualid$_n$}{\tt .}\\
  Recursively extracts all the material needed for the extraction of 
  globals {\qualid$_1$}, \ldots, {\qualid$_n$}.
\end{Variants}

\SeeAlso Chapter~\ref{Extraction}.

\subsection[\tt Print Assumptions {\qualid}.]{\tt Print Assumptions {\qualid}.\comindex{Print Assumptions}}
\label{PrintAssumptions}

This commands display all the assumptions (axioms, parameters and
variables) a theorem or definition depends on.  Especially, it informs
on the assumptions with respect to which the validity of a theorem
relies.

\begin{Variants}
\item \texttt{\tt Print Opaque Dependencies {\qualid}.
  \comindex{Print Opaque Dependencies}}\\
  Displays the set of opaque constants {\qualid} relies on in addition
  to the assumptions.
\end{Variants}

\subsection[\tt Search {\term}.]{\tt Search {\term}.\comindex{Search}}
This command displays the name and type of all theorems of the current
context whose statement's conclusion has the form {\tt ({\term} t1 ..
  tn)}.  This command is useful to remind the user of the name of
library lemmas.

\begin{coq_example}
Search le.
Search (@eq bool).
\end{coq_example}

\begin{Variants}
\item
{\tt Search} {\term} {\tt inside} {\module$_1$} \ldots{} {\module$_n$}{\tt .}

This restricts the search to constructions defined in modules
{\module$_1$} \ldots{} {\module$_n$}.

\item {\tt Search} {\term} {\tt outside} {\module$_1$} \ldots{} {\module$_n$}{\tt .}

This restricts the search to constructions not defined in modules
{\module$_1$} \ldots{} {\module$_n$}.

\begin{ErrMsgs}
\item \errindex{Module/section \module{} not found}
No module \module{} has been required (see Section~\ref{Require}).
\end{ErrMsgs}

\end{Variants}

\subsection[\tt SearchAbout {\qualid}.]{\tt SearchAbout {\qualid}.\comindex{SearchAbout}}
This command displays the name and type of all objects (theorems,
axioms, etc) of the current context whose statement contains \qualid.
This command is useful to remind the user of the name of library
lemmas.

\begin{ErrMsgs}
\item \errindex{The reference \qualid\ was not found in the current
environment}\\
    There is no constant in the environment named \qualid.
\end{ErrMsgs}

\newcommand{\termpatternorstr}{{\termpattern}\textrm{\textsl{-}}{\str}}

\begin{Variants}
\item {\tt SearchAbout {\str}.}

If {\str} is a valid identifier, this command displays the name and type
of all objects (theorems, axioms, etc) of the current context whose
name contains {\str}. If {\str} is a notation's string denoting some
reference {\qualid} (referred to by its main symbol as in \verb="+"=
or by its notation's string as in \verb="_ + _"= or \verb="_ 'U' _"=, see
Section~\ref{Notation}), the command works like {\tt SearchAbout
{\qualid}}.

\item {\tt SearchAbout {\str}\%{\delimkey}.}

The string {\str} must be a notation or the main symbol of a notation
which is then interpreted in the scope bound to the delimiting key
{\delimkey} (see Section~\ref{scopechange}).

\item {\tt SearchAbout {\termpattern}.}

This searches for all statements or types of definition that contains
a subterm that matches the pattern {\termpattern} (holes of the
pattern are either denoted by ``{\texttt \_}'' or
by ``{\texttt ?{\ident}}'' when non linear patterns are expected).

\item {\tt SearchAbout \nelist{\zeroone{-}{\termpatternorstr}}{}.}\\

\noindent where {\termpatternorstr} is a
{\termpattern} or a {\str}, or a {\str} followed by a scope
delimiting key {\tt \%{\delimkey}}.

This generalization of {\tt SearchAbout} searches for all objects
whose statement or type contains a subterm matching {\termpattern} (or
{\qualid} if {\str} is the notation for a reference {\qualid}) and
whose name contains all {\str} of the request that correspond to valid
identifiers. If a {\termpattern} or a {\str} is prefixed by ``-'', the
search excludes the objects that mention that {\termpattern} or that
{\str}.

\item
  {\tt SearchAbout} \nelist{{\termpatternorstr}}{}
    {\tt inside} {\module$_1$} \ldots{} {\module$_n$}{\tt .}

This restricts the search to constructions defined in modules
{\module$_1$} \ldots{} {\module$_n$}.

\item
  {\tt SearchAbout \nelist{{\termpatternorstr}}{}
     outside {\module$_1$}...{\module$_n$}.}

This restricts the search to constructions not defined in modules
{\module$_1$} \ldots{} {\module$_n$}.

\item {\tt SearchAbout [ ... ]. }

For compatibility with older versions, the list of objects to search
may be enclosed by optional {\tt [  ]} delimiters.

\end{Variants}

\examples

\begin{coq_example*}
Require Import ZArith.
\end{coq_example*}
\begin{coq_example}
SearchAbout Z.mul Z.add "distr".
SearchAbout "+"%Z "*"%Z "distr" -positive -Prop.
SearchAbout (?x * _ + ?x * _)%Z outside OmegaLemmas.
\end{coq_example}

\subsection[\tt SearchPattern {\termpattern}.]{\tt SearchPattern {\term}.\comindex{SearchPattern}}

This command displays the name and type of all theorems of the current
context whose statement's conclusion or last hypothesis and conclusion
matches the expression {\term} where holes in the latter are denoted
by ``{\texttt \_}''. It is a variant of {\tt SearchAbout
  {\termpattern}} that does not look for subterms but searches for
statements whose conclusion has exactly the expected form, or whose
statement finishes by the given series of hypothesis/conclusion.

\begin{coq_example}
Require Import Arith.
SearchPattern (_ + _ = _ + _).
SearchPattern (nat -> bool).
SearchPattern (forall l : list _, _ l l).
\end{coq_example}

Patterns need not be linear: you can express that the same expression
must occur in two places by using pattern variables `{\texttt
?{\ident}}''.

\begin{coq_example}
Require Import Arith.
SearchPattern (?X1 + _ = _ + ?X1).
\end{coq_example}

\begin{Variants}
\item {\tt SearchPattern {\term} inside
{\module$_1$} \ldots{} {\module$_n$}.}

This restricts the search to constructions defined in modules
{\module$_1$} \ldots{} {\module$_n$}.

\item {\tt SearchPattern {\term} outside {\module$_1$} \ldots{} {\module$_n$}.}

This restricts the search to constructions not defined in modules
{\module$_1$} \ldots{} {\module$_n$}.

\end{Variants}

\subsection[\tt SearchRewrite {\term}.]{\tt SearchRewrite {\term}.\comindex{SearchRewrite}}

This command displays the name and type of all theorems of the current
context whose statement's conclusion is an equality of which one side matches
the expression {\term}. Holes in {\term} are denoted by ``{\texttt \_}''.

\begin{coq_example}
Require Import Arith.
SearchRewrite (_ + _ + _).
\end{coq_example}

\begin{Variants}
\item {\tt SearchRewrite {\term} inside
{\module$_1$} \ldots{} {\module$_n$}.}

This restricts the search to constructions defined in modules
{\module$_1$} \ldots{} {\module$_n$}.

\item {\tt SearchRewrite {\term} outside {\module$_1$} \ldots{} {\module$_n$}.}

This restricts the search to constructions not defined in modules
{\module$_1$} \ldots{} {\module$_n$}.

\end{Variants}

\subsubsection{Nota Bene:}
For the {\tt Search}, {\tt SearchAbout}, {\tt SearchPattern} and
{\tt SearchRewrite} queries, it is possible to globally filter
the search results via the command
{\tt Add Search Blacklist "substring1"}.
A lemma whose fully-qualified name contains any of the declared substrings
will be removed from the search results.
The default blacklisted substrings are {\tt "\_admitted"
  "\_subproof" "Private\_"}. The command {\tt Remove Search Blacklist
  ...} allows to expunge this blacklist.

% \begin{tabbing}
% \ \ \ \ \=11.\ \=\kill
% \>1.\>$A=B\mx{ if }A\stackrel{\bt{}\io{}}{\lra{}}B$\\
% \>2.\>$\sa{}x:A.B=\sa{}y:A.B[x\la{}y]\mx{ if }y\not\in{}FV(\sa{}x:A.B)$\\
% \>3.\>$\Pi{}x:A.B=\Pi{}y:A.B[x\la{}y]\mx{ if }y\not\in{}FV(\Pi{}x:A.B)$\\
% \>4.\>$\sa{}x:A.B=\sa{}x:B.A\mx{ if }x\not\in{}FV(A,B)$\\
% \>5.\>$\sa{}x:(\sa{}y:A.B).C=\sa{}x:A.\sa{}y:B[y\la{}x].C[x\la{}(x,y)]$\\
% \>6.\>$\Pi{}x:(\sa{}y:A.B).C=\Pi{}x:A.\Pi{}y:B[y\la{}x].C[x\la{}(x,y)]$\\
% \>7.\>$\Pi{}x:A.\sa{}y:B.C=\sa{}y:(\Pi{}x:A.B).(\Pi{}x:A.C[y\la{}(y\sm{}x)]$\\
% \>8.\>$\sa{}x:A.unit=A$\\
% \>9.\>$\sa{}x:unit.A=A[x\la{}tt]$\\
% \>10.\>$\Pi{}x:A.unit=unit$\\
% \>11.\>$\Pi{}x:unit.A=A[x\la{}tt]$
% \end{tabbing}

% For more informations about the exact working of this command, see
% \cite{Del97}.

\subsection[\tt Locate {\qualid}.]{\tt Locate {\qualid}.\comindex{Locate}
\label{Locate}}
This command displays the full name of the qualified identifier {\qualid}
and consequently the \Coq\ module in which it is defined.

\begin{coq_eval}
(*************** The last line should produce **************************)
(*********** Error: I.Dont.Exist not a defined object ******************)
\end{coq_eval}
\begin{coq_eval}
Set Printing Depth 50.
\end{coq_eval}
\begin{coq_example}
Locate nat.
Locate Datatypes.O.
Locate Init.Datatypes.O.
Locate Coq.Init.Datatypes.O.
Locate I.Dont.Exist.
\end{coq_example}

\SeeAlso Section \ref{LocateSymbol}

\subsection{The {\sc Whelp} searching tool
\label{Whelp}}

{\sc Whelp} is an experimental searching and browsing tool for the
whole {\Coq} library and the whole set of {\Coq} user contributions.
{\sc Whelp} requires a browser to work. {\sc Whelp} has been developed
at the University of Bologna as part of the HELM\footnote{Hypertextual
Electronic Library of Mathematics} and MoWGLI\footnote{Mathematics on
the Web, Get it by Logics and Interfaces} projects.  It can be invoked
directly from the {\Coq} toplevel or from {\CoqIDE}, assuming a
graphical environment is also running. The browser to use can be
selected by setting the environment variable {\tt
COQREMOTEBROWSER}. If not explicitly set, it defaults to
\verb!firefox -remote \"OpenURL(%s,new-tab)\" || firefox %s &"!  or
\verb!C:\\PROGRA~1\\INTERN~1\\IEXPLORE %s!, depending on the
underlying operating system (in the command, the string \verb!%s!
serves as metavariable for the url to open).
The Whelp tool relies on a dedicated Whelp server and on another server
called Getter that retrieves formal documents. The default Whelp server name
can be obtained using the command {\tt Test Whelp Server}
\comindex{Test Whelp Server} and the default Getter can be obtained
using the command: {\tt Test Whelp Getter} \comindex{Test Whelp
Getter}. The Whelp server name can be changed using the command:

\smallskip
\noindent {\tt Set Whelp Server {\str}}.\\
where {\str} is a URL (e.g. {\tt http://mowgli.cs.unibo.it:58080}).
\comindex{Set Whelp Server}
\smallskip

\noindent The Getter can be changed using the command:
\smallskip

\noindent {\tt Set Whelp Getter {\str}}.\\
where {\str} is a URL (e.g. {\tt http://mowgli.cs.unibo.it:58081}).  
\comindex{Set Whelp Getter}

\bigskip

The {\sc Whelp} commands are:

\subsubsection{\tt Whelp Locate "{\sl reg\_expr}".
\comindex{Whelp Locate}}

This command opens a browser window and displays the result of seeking
for all names that match the regular expression {\sl reg\_expr} in the
{\Coq} library and user contributions. The regular expression can
contain the special operators are * and ? that respectively stand for
an arbitrary substring and for exactly one character.

\variant {\tt Whelp Locate {\ident}.}\\
This is equivalent to {\tt Whelp Locate "{\ident}"}.

\subsubsection{\tt Whelp Match {\pattern}.
\comindex{Whelp Match}}

This command opens a browser window and displays the result of seeking
for all statements that match the pattern {\pattern}. Holes in the
pattern are represented by the wildcard character ``\_''.

\subsubsection[\tt Whelp Instance {\pattern}.]{\tt Whelp Instance {\pattern}.\comindex{Whelp Instance}}

This command opens a browser window and displays the result of seeking
for all statements that are instances of the pattern {\pattern}. The
pattern is here assumed to be an universally quantified expression.

\subsubsection[\tt Whelp Elim {\qualid}.]{\tt Whelp Elim {\qualid}.\comindex{Whelp Elim}}

This command opens a browser window and displays the result of seeking
for all statements that have the ``form'' of an elimination scheme
over the type denoted by {\qualid}.

\subsubsection[\tt Whelp Hint {\term}.]{\tt Whelp Hint {\term}.\comindex{Whelp Hint}}

This command opens a browser window and displays the result of seeking
for all statements that can be instantiated so that to prove the
statement {\term}.

\variant {\tt Whelp Hint.}\\ This is equivalent to {\tt Whelp Hint
{\sl goal}} where {\sl goal} is the current goal to prove. Notice that
{\Coq} does not send the local environment of definitions to the {\sc
Whelp} tool so that it only works on requests strictly based on, only,
definitions of the standard library and user contributions.

\section{Loading files}

\Coq\ offers the possibility of loading different
parts of a whole development stored in separate files. Their contents
will be loaded as if they were entered from the keyboard. This means
that the loaded files are ASCII files containing sequences of commands
for \Coq's toplevel. This kind of file is called a {\em script} for
\Coq\index{Script file}. The standard (and default) extension of
\Coq's script files is {\tt .v}.

\subsection[\tt Load {\ident}.]{\tt Load {\ident}.\comindex{Load}\label{Load}}
This command loads the file named {\ident}{\tt .v}, searching
successively in each of the directories specified in the {\em
  loadpath}. (see Section~\ref{loadpath})

\begin{Variants}
\item {\tt Load {\str}.}\label{Load-str}\\
  Loads the file denoted by the string {\str}, where {\str} is any
  complete filename. Then the \verb.~. and {\tt ..}
  abbreviations are allowed as well as shell variables. If no
  extension is specified, \Coq\ will use the default extension {\tt
    .v}
\item {\tt Load Verbose {\ident}.}, 
  {\tt Load Verbose {\str}}\\
  \comindex{Load Verbose}
  Display, while loading, the answers of \Coq\ to each command
  (including tactics) contained in the loaded file
  \SeeAlso Section~\ref{Begin-Silent}
\end{Variants}

\begin{ErrMsgs}
\item \errindex{Can't find file {\ident} on loadpath}
\end{ErrMsgs}

\section[Compiled files]{Compiled files\label{compiled}\index{Compiled files}}

This section describes the commands used to load compiled files (see
Chapter~\ref{Addoc-coqc} for documentation on how to compile a file).
A compiled file is a particular case of module called {\em library file}.

%%%%%%%%%%%%
% Import and Export described in RefMan-mod.tex
% the minor difference (to avoid multiple Exporting of libraries) in
% the treatment of normal modules and libraries by Export omitted

\subsection[\tt Require {\qualid}.]{\tt Require {\qualid}.\label{Require}
\comindex{Require}}

This command looks in the loadpath for a file containing
module {\qualid} and adds the corresponding module to the environment
of {\Coq}. As library files have dependencies in other library files,
the command {\tt Require {\qualid}} recursively requires all library
files the module {\qualid} depends on and adds the corresponding modules to the
environment of {\Coq} too. {\Coq} assumes that the compiled files have
been produced by a valid {\Coq} compiler and their contents are then not
replayed nor rechecked.

To locate the file in the file system, {\qualid} is decomposed under
the form {\dirpath}{\tt .}{\textsl{ident}} and the file {\ident}{\tt
.vo} is searched in the physical directory of the file system that is
mapped in {\Coq} loadpath to the logical path {\dirpath} (see
Section~\ref{loadpath}). The mapping between physical directories and
logical names at the time of requiring the file must be consistent
with the mapping used to compile the file.

\begin{Variants}
\item {\tt Require Import {\qualid}.} \comindex{Require} 

  This loads and declares the module {\qualid} and its dependencies
  then imports the contents of {\qualid} as described in
  Section~\ref{Import}.

  It does not import the modules on which {\qualid} depends unless
  these modules were itself required in module {\qualid} using {\tt
  Require Export}, as described below, or recursively required through
  a sequence of {\tt Require Export}.

  If the module required has already been loaded, {\tt Require Import
  {\qualid}} simply imports it, as {\tt Import {\qualid}} would.

\item {\tt Require Export {\qualid}.}
  \comindex{Require Export}

  This command acts as {\tt Require Import} {\qualid}, but if a
  further module, say {\it A}, contains a command {\tt Require
  Export} {\it B}, then the command {\tt Require Import} {\it A}
  also imports the module {\it B}.

\item {\tt Require \zeroone{Import {\sl |} Export}} {\qualid}$_1$ {\ldots} {\qualid}$_n${\tt .}

  This loads the modules {\qualid}$_1$, \ldots, {\qualid}$_n$ and
  their recursive dependencies. If {\tt Import} or {\tt Export} is
  given, it also imports {\qualid}$_1$, \ldots, {\qualid}$_n$ and all
  the recursive dependencies that were marked or transitively marked
  as {\tt Export}.

\item {\tt Require \zeroone{Import {\sl |} Export} {\str}.}

  This shortcuts the resolution of the qualified name into a library
  file name by directly requiring the module to be found in file
  {\str}.vo.
\end{Variants}

\begin{ErrMsgs}

\item \errindex{Cannot load {\qualid}: no physical path bound to {\dirpath}}

\item \errindex{Cannot find library foo in loadpath}

  The command did not find the file {\tt foo.vo}. Either {\tt
    foo.v} exists but is not compiled or {\tt foo.vo} is in a directory
  which is not in your {\tt LoadPath} (see Section~\ref{loadpath}).

\item \errindex{Compiled library {\ident}.vo makes inconsistent assumptions over library {\qualid}}

  The command tried to load library file {\ident}.vo that depends on
  some specific version of library {\qualid} which is not the one
  already loaded in the current {\Coq} session. Probably {\ident}.v
  was not properly recompiled with the last version of the file
  containing module {\qualid}.

\item \errindex{Bad magic number}

  \index{Bad-magic-number@{\tt Bad Magic Number}}
  The file {\tt{\ident}.vo} was found but either it is not a \Coq\
  compiled module, or it was compiled with an older and incompatible
  version of \Coq.

\item \errindex{The file {\ident}.vo contains library {\dirpath} and not
  library {\dirpath'}}

  The library file {\dirpath'} is indirectly required by the {\tt
  Require} command but it is bound in the current loadpath to the file
  {\ident}.vo which was bound to a different library name {\dirpath}
  at the time it was compiled.

\end{ErrMsgs}

\SeeAlso Chapter~\ref{Addoc-coqc}

\subsection[\tt Print Libraries.]{\tt Print Libraries.\comindex{Print Libraries}}

This command displays the list of library files loaded in the current
{\Coq} session. For each of these libraries, it also tells if it is
imported.

\subsection[\tt Declare ML Module {\str$_1$} .. {\str$_n$}.]{\tt Declare ML Module {\str$_1$} .. {\str$_n$}.\comindex{Declare ML Module}}
This commands loads the Objective Caml compiled files {\str$_1$} {\ldots}
{\str$_n$} (dynamic link). It is mainly used to load tactics
dynamically.
% (see Chapter~\ref{WritingTactics}).
 The files are
searched into the current Objective Caml loadpath (see the command {\tt
Add ML Path} in the Section~\ref{loadpath}).  Loading of Objective Caml
files is only possible under the bytecode version of {\tt coqtop}
(i.e. {\tt coqtop} called with options {\tt -byte}, see chapter 
\ref{Addoc-coqc}), or when Coq has been compiled with a version of
Objective Caml that supports native {\tt Dynlink} ($\ge$ 3.11).

\begin{Variants}
\item {\tt Local Declare ML Module {\str$_1$} .. {\str$_n$}.}\\
  This variant is not exported to the modules that import the module
  where they occur, even if outside a section.
\end{Variants}

\begin{ErrMsgs}
\item \errindex{File not found on loadpath : \str}
\item \errindex{Loading of ML object file forbidden in a native Coq}
\end{ErrMsgs}

\subsection[\tt Print ML Modules.]{\tt Print ML Modules.\comindex{Print ML Modules}}
This print the name of all \ocaml{} modules loaded with \texttt{Declare
  ML Module}. To know from where these module were loaded, the user
should use the command \texttt{Locate File} (see Section~\ref{Locate File})

\section[Loadpath]{Loadpath\label{loadpath}\index{Loadpath}}

There are currently two loadpaths in \Coq. A loadpath where seeking
{\Coq} files (extensions {\tt .v} or {\tt .vo} or {\tt .vi}) and one where
seeking Objective Caml files. The default loadpath contains the
directory ``\texttt{.}'' denoting the current directory and mapped to the empty logical path (see Section~\ref{LongNames}).

\subsection[\tt Pwd.]{\tt Pwd.\comindex{Pwd}\label{Pwd}}
This command displays the current working directory.

\subsection[\tt Cd {\str}.]{\tt Cd {\str}.\comindex{Cd}}
This command changes the current directory according to {\str} 
which can be any valid path.

\begin{Variants}
\item {\tt Cd.}\\
  Is equivalent to {\tt Pwd.}
\end{Variants}

\subsection[\tt Add LoadPath {\str} as {\dirpath}.]{\tt Add LoadPath {\str} as {\dirpath}.\comindex{Add LoadPath}\label{AddLoadPath}}

This command adds the physical directory {\str} to the current {\Coq}
loadpath and maps it to the logical directory {\dirpath}, which means
that every file \textrm{\textsl{dirname}}/\textrm{\textsl{basename.v}}
physically lying in subdirectory {\str}/\textrm{\textsl{dirname}}
becomes accessible in {\Coq} through absolute logical name
{\dirpath}{\tt .}\textrm{\textsl{dirname}}{\tt
.}\textrm{\textsl{basename}}.

\Rem {\tt Add LoadPath} also adds {\str} to the current ML loadpath.

\begin{Variants}
\item {\tt Add LoadPath {\str}.}\\
Performs as {\tt Add LoadPath {\str} as {\dirpath}} but for the empty directory path.
\end{Variants}

\subsection[\tt Add Rec LoadPath {\str} as {\dirpath}.]{\tt Add Rec LoadPath {\str} as {\dirpath}.\comindex{Add Rec LoadPath}\label{AddRecLoadPath}}
This command adds the physical directory {\str} and all its subdirectories to
the current \Coq\ loadpath. The top directory {\str} is mapped to the
logical directory {\dirpath} and any subdirectory {\textsl{pdir}} of it is
mapped to logical name {\dirpath}{\tt .}\textsl{pdir} and
recursively. Subdirectories corresponding to invalid {\Coq}
identifiers are skipped, and, by convention, subdirectories named {\tt
CVS} or {\tt \_darcs} are skipped too.

Otherwise, said, {\tt Add Rec LoadPath {\str} as {\dirpath}} behaves
as {\tt Add LoadPath {\str} as {\dirpath}} excepts that files lying in
validly named subdirectories of {\str} need not be qualified to be
found.

In case of files with identical base name, files lying in most recently
declared {\dirpath} are found first and explicit qualification is
required to refer to the other files of same base name.

If several files with identical base name are present in different
subdirectories of a recursive loadpath declared via a single instance of
{\tt Add Rec LoadPath}, which of these files is found first is
system-dependent and explicit qualification is recommended.

\Rem {\tt Add Rec LoadPath} also recursively adds {\str} to the current ML loadpath.

\begin{Variants}
\item {\tt Add Rec LoadPath {\str}.}\\
Works as {\tt Add Rec LoadPath {\str} as {\dirpath}} but for the empty logical directory path.
\end{Variants}

\subsection[\tt Remove LoadPath {\str}.]{\tt Remove LoadPath {\str}.\comindex{Remove LoadPath}}
This command removes the path {\str} from the current \Coq\ loadpath.

\subsection[\tt Print LoadPath.]{\tt Print LoadPath.\comindex{Print LoadPath}}
This command displays the current \Coq\ loadpath.

\begin{Variants}
\item {\tt Print LoadPath {\dirpath}.}\\
Works as {\tt Print LoadPath} but displays only the paths that extend the {\dirpath} prefix.
\end{Variants}

\subsection[\tt Add ML Path {\str}.]{\tt Add ML Path {\str}.\comindex{Add ML Path}}
This command adds the path {\str} to the current Objective Caml loadpath (see
the command {\tt Declare ML Module} in the Section~\ref{compiled}).

\Rem This command is implied by {\tt Add LoadPath {\str} as {\dirpath}}.

\subsection[\tt Add Rec ML Path {\str}.]{\tt Add Rec ML Path {\str}.\comindex{Add Rec ML Path}}
This command adds the directory {\str} and all its subdirectories 
to the current Objective Caml loadpath (see
the command {\tt Declare ML Module} in the Section~\ref{compiled}).

\Rem This command is implied by {\tt Add Rec LoadPath {\str} as {\dirpath}}.

\subsection[\tt Print ML Path {\str}.]{\tt Print ML Path {\str}.\comindex{Print ML Path}}
This command displays the current Objective Caml loadpath.
This command makes sense only under the bytecode version of {\tt
coqtop}, i.e. using option {\tt -byte} (see the
command {\tt Declare ML Module} in the section
\ref{compiled}).

\subsection[\tt Locate File {\str}.]{\tt Locate File {\str}.\comindex{Locate
  File}\label{Locate File}}
This command displays the location of file {\str} in the current loadpath.
Typically, {\str} is a \texttt{.cmo} or \texttt{.vo} or \texttt{.v} file.

\subsection[\tt Locate Library {\dirpath}.]{\tt Locate Library {\dirpath}.\comindex{Locate Library}\label{Locate Library}}
This command gives the status of the \Coq\ module {\dirpath}. It tells if the
module is loaded and if not searches in the load path for a module
of logical name {\dirpath}.

\section{Backtracking}

The backtracking commands described in this section can only be used
interactively, they cannot be part of a vernacular file loaded via
{\tt Load} or compiled by {\tt coqc}.

\subsection[\tt Reset \ident.]{\tt Reset \ident.\comindex{Reset}}
This command removes all the objects in the environment since \ident\ 
was introduced, including \ident. \ident\ may be the name of a defined
or declared object as well as the name of a section. One cannot reset
over the name of a module or of an object inside a module.

\begin{ErrMsgs}
\item \ident: \errindex{no such entry}
\end{ErrMsgs}

\begin{Variants}
 \item {\tt Reset Initial.}\comindex{Reset Initial}\\
   Goes back to the initial state, just after the start of the
   interactive session.
\end{Variants}

\subsection[\tt Back.]{\tt Back.\comindex{Back}}

This commands undoes all the effects of the last vernacular
command. Commands read from a vernacular file via a {\tt Load} are
considered as a single command. Proof managment commands
are also handled by this command (see Chapter~\ref{Proof-handling}).
For that, {\tt Back} may have to undo more than one command in order
to reach a state where the proof managment information is available.
For instance, when the last command is a {\tt Qed}, the managment
information about the closed proof has been discarded. In this case,
{\tt Back} will then undo all the proof steps up to the statement of
this proof.

\begin{Variants}
\item {\tt Back $n$} \\
  Undoes $n$ vernacular commands. As for {\tt Back}, some extra
  commands may be undone in order to reach an adequate state.
  For instance {\tt Back n} will not re-enter a closed proof,
  but rather go just before that proof.
\end{Variants}

\begin{ErrMsgs}
\item \errindex{Invalid backtrack} \\
  The user wants to undo more commands than available in the history.
\end{ErrMsgs}

\subsection[\tt BackTo $\num$.]{\tt BackTo $\num$.\comindex{BackTo}}
\label{sec:statenums}

This command brings back the system to the state labelled $\num$,
forgetting the effect of all commands executed after this state.
The state label is an integer which grows after each successful command.
It is displayed in the prompt when in \texttt{-emacs} mode.
Just as {\tt Back} (see above), the {\tt BackTo} command now handles
proof states. For that, it may have to undo some
extra commands and end on a state $\num' \leq \num$ if necessary.

\begin{Variants}
\item {\tt Backtrack $\num_1$ $\num_2$ $\num_3$}.\comindex{Backtrack}\\
  {\tt Backtrack} is a \emph{deprecated} form of {\tt BackTo} which
  allows to explicitely manipulate the proof environment. The three
  numbers $\num_1$, $\num_2$ and $\num_3$ represent the following:
\begin{itemize}
\item $\num_3$: Number of \texttt{Abort} to perform, i.e. the number
  of currently opened nested proofs that must be canceled (see
  Chapter~\ref{Proof-handling}).
\item $\num_2$: \emph{Proof state number} to unbury once aborts have
  been done. Coq will compute the number of \texttt{Undo} to perform
  (see Chapter~\ref{Proof-handling}).
\item $\num_1$: State label to reach, as for {\tt BackTo}.
\end{itemize}
\end{Variants}

\begin{ErrMsgs}
\item \errindex{Invalid backtrack} \\
  The destination state label is unknown.
\end{ErrMsgs}

\section{State files}

\subsection[\tt Write State \str.]{\tt Write State \str.\comindex{Write State}}
Writes the current state into a file \str{} for
use in a further session. This file can be given as the {\tt
  inputstate} argument of the commands {\tt coqtop} and {\tt coqc}.

\begin{Variants}
\item {\tt Write State \ident}\\
 Equivalent to {\tt Write State "}{\ident}{\tt .coq"}.
 The state is saved in the current directory (see Section~\ref{Pwd}).
\end{Variants}

\subsection[\tt Restore State \str.]{\tt Restore State \str.\comindex{Restore State}}
  Restores the state contained in the file \str.

\begin{Variants}
\item {\tt Restore State \ident}\\
 Equivalent to {\tt Restore State "}{\ident}{\tt .coq"}.
\end{Variants}

\section{Quitting and debugging}

\subsection[\tt Quit.]{\tt Quit.\comindex{Quit}}
This command permits to quit \Coq.

\subsection[\tt Drop.]{\tt Drop.\comindex{Drop}\label{Drop}}

This is used mostly as a debug facility by \Coq's implementors
and does not concern the casual user.
This command permits to leave {\Coq} temporarily and enter the
Objective Caml toplevel. The Objective Caml command:

\begin{flushleft}
\begin{verbatim}
#use "include";;
\end{verbatim}
\end{flushleft}

\noindent add the right loadpaths and loads some toplevel printers for
all abstract types of \Coq - section\_path, identifiers, terms, judgments,
\dots. You can also use the file \texttt{base\_include} instead,
that loads only the pretty-printers for section\_paths and
identifiers.
% See Section~\ref{test-and-debug} more information on the
% usage of the toplevel.
You can return back to \Coq{} with the command: 

\begin{flushleft}
\begin{verbatim}
go();;
\end{verbatim}
\end{flushleft}

\begin{Warnings}
\item It only works with the bytecode version of {\Coq} (i.e. {\tt coqtop} called with option {\tt -byte}, see the contents of Section~\ref{binary-images}).
\item You must have compiled {\Coq} from the source package and set the
  environment variable \texttt{COQTOP} to the root of your copy of the sources (see Section~\ref{EnvVariables}).
\end{Warnings}

\subsection[\tt Time \textrm{\textsl{command}}.]{\tt Time \textrm{\textsl{command}}.\comindex{Time}
\label{time}}
This command executes the vernacular command \textrm{\textsl{command}}
and display the time needed to execute it.


\subsection[\tt Timeout \textrm{\textsl{int}} \textrm{\textsl{command}}.]{\tt Timeout \textrm{\textsl{int}} \textrm{\textsl{command}}.\comindex{Timeout}
\label{timeout}}

This command executes the vernacular command \textrm{\textsl{command}}. If
the command has not terminated after the time specified by the integer
(time expressed in seconds), then it is interrupted and an error message
is displayed.

\subsection[\tt Set Default Timeout \textrm{\textsl{int}}.]{\tt Set
  Default Timeout \textrm{\textsl{int}}.\comindex{Set Default Timeout}}

After using this command, all subsequent commands behave as if they
were passed to a {\tt Timeout} command. Commands already starting by
a {\tt Timeout} are unaffected.

\subsection[\tt Unset Default Timeout.]{\tt Unset Default Timeout.\comindex{Unset Default Timeout}}

This command turns off the use of a default timeout.

\subsection[\tt Test Default Timeout.]{\tt Test Default Timeout.\comindex{Test Default Timeout}}

This command displays whether some default timeout has be set or not.

\section{Controlling display}

\subsection[\tt Set Silent.]{\tt Set Silent.\comindex{Set Silent}
\label{Begin-Silent}
\index{Silent mode}}
This command turns off the normal displaying.

\subsection[\tt Unset Silent.]{\tt Unset Silent.\comindex{Unset Silent}}
This command turns the normal display on.

\subsection[\tt Set Printing Width {\integer}.]{\tt Set Printing Width {\integer}.\comindex{Set Printing Width}}
This command sets which left-aligned part of the width of the screen
is used for display. 

\subsection[\tt Unset Printing Width.]{\tt Unset Printing Width.\comindex{Unset Printing Width}}
This command resets the width of the screen used for display to its
default value (which is 78 at the time of writing this documentation).

\subsection[\tt Test Printing Width.]{\tt Test Printing Width.\comindex{Test Printing Width}}
This command displays the current screen width used for display.

\subsection[\tt Set Printing Depth {\integer}.]{\tt Set Printing Depth {\integer}.\comindex{Set Printing Depth}}
This command sets the nesting depth of the formatter used for
pretty-printing. Beyond this depth, display of subterms is replaced by
dots. 

\subsection[\tt Unset Printing Depth.]{\tt Unset Printing Depth.\comindex{Unset Printing Depth}}
This command resets the nesting depth of the formatter used for
pretty-printing to its default value (at the
time of writing this documentation, the default value is 50).

\subsection[\tt Test Printing Depth.]{\tt Test Printing Depth.\comindex{Test Printing Depth}}
This command displays the current nesting depth used for display.

%\subsection{\tt Abstraction ...}
%Not yet documented.

\section{Controlling the reduction strategies and the conversion algorithm}
\label{Controlling reduction strategy}

{\Coq} provides reduction strategies that the tactics can invoke and
two different algorithms to check the convertibility of types.
The first conversion algorithm lazily
compares applicative terms while the other is a brute-force but efficient
algorithm that first normalizes the terms before comparing them.  The
second algorithm is based on a bytecode representation of terms
similar to the bytecode representation used in the ZINC virtual
machine~\cite{Leroy90}. It is especially useful for intensive
computation of algebraic values, such as numbers, and for reflexion-based
tactics. The commands to fine-tune the reduction strategies and the
lazy conversion algorithm are described first.

\subsection[{\tt Opaque} \qualid$_1$ {\ldots} \qualid$_n${\tt .}]{{\tt Opaque} \qualid$_1$ {\ldots} \qualid$_n${\tt .}\comindex{Opaque}\label{Opaque}}
This command has an effect on unfoldable constants, i.e. 
on constants defined by {\tt Definition} or {\tt Let} (with an explicit
body), or by a command assimilated to a definition such as {\tt
Fixpoint}, {\tt Program Definition}, etc, or by a proof ended by {\tt
Defined}. The command tells not to unfold
the constants {\qualid$_1$} {\ldots} {\qualid$_n$} in tactics using
$\delta$-conversion (unfolding a constant is replacing it by its
definition).

{\tt Opaque} has also on effect on the conversion algorithm of {\Coq},
telling to delay the unfolding of a constant as later as possible in
case {\Coq} has to check the conversion (see Section~\ref{conv-rules})
of two distinct applied constants.

The scope of {\tt Opaque} is limited to the current section, or
current file, unless the variant {\tt Global Opaque \qualid$_1$ {\ldots}
\qualid$_n$} is used.

\SeeAlso sections \ref{Conversion-tactics}, \ref{Automatizing},
\ref{Theorem}

\begin{ErrMsgs}
\item \errindex{The reference \qualid\ was not found in the current
environment}\\
    There is no constant referred by {\qualid} in the environment.
    Nevertheless, if you asked \texttt{Opaque foo bar}
    and if \texttt{bar} does not exist, \texttt{foo} is set opaque.
\end{ErrMsgs}

\subsection[{\tt Transparent} \qualid$_1$ {\ldots} \qualid$_n${\tt .}]{{\tt Transparent} \qualid$_1$ {\ldots} \qualid$_n${\tt .}\comindex{Transparent}\label{Transparent}}
This command is the converse of {\tt Opaque} and it applies on
unfoldable constants to restore their unfoldability after an {\tt
Opaque} command.

Note in particular that constants defined by a proof ended by {\tt
Qed} are not unfoldable and {\tt Transparent} has no effect on
them. This is to keep with the usual mathematical practice of {\em
proof irrelevance}: what matters in a mathematical development is the
sequence of lemma statements, not their actual proofs. This
distinguishes lemmas from the usual defined constants, whose actual
values are of course relevant in general.

The scope of {\tt Transparent} is limited to the current section, or
current file, unless the variant {\tt Global Transparent} \qualid$_1$
{\ldots} \qualid$_n$ is used.

\begin{ErrMsgs}
% \item \errindex{Can not set transparent.}\\
%     It is a constant from a required module or a parameter.
\item \errindex{The reference \qualid\ was not found in the current
environment}\\
    There is no constant referred by {\qualid} in the environment.
\end{ErrMsgs}

\SeeAlso sections \ref{Conversion-tactics}, \ref{Automatizing},
\ref{Theorem}

\subsection{{\tt Strategy} {\it level} {\tt [} \qualid$_1$ {\ldots} \qualid$_n$
  {\tt ].}\comindex{Strategy}\comindex{Local Strategy}\label{Strategy}}
This command generalizes the behavior of {\tt Opaque} and {\tt
  Transparent} commands. It is used to fine-tune the strategy for
unfolding constants, both at the tactic level and at the kernel
level. This command associates a level to \qualid$_1$ {\ldots}
\qualid$_n$. Whenever two expressions with two distinct head
constants are compared (for instance, this comparison can be triggered
by a type cast), the one with lower level is expanded first. In case
of a tie, the second one (appearing in the cast type) is expanded.

Levels can be one of the following (higher to lower):
\begin{description}
\item[opaque]: level of opaque constants. They cannot be expanded by
  tactics (behaves like $+\infty$, see next item).
\item[\num]: levels indexed by an integer. Level $0$ corresponds
  to the default behavior, which corresponds to transparent
  constants. This level can also be referred to as {\bf transparent}.
  Negative levels correspond to constants to be expanded before normal
  transparent constants, while positive levels correspond to constants
  to be expanded after normal transparent constants.
\item[expand]: level of constants that should be expanded first
  (behaves like $-\infty$)
\end{description}

These directives survive section and module closure, unless the
command is prefixed by {\tt Local}. In the latter case, the behavior
regarding sections and modules is the same as for the {\tt
  Transparent} and {\tt Opaque} commands.

\subsection{\tt Declare Reduction \ident\ := {\rm\sl convtactic}.}

This command allows to give a short name to a reduction expression,
for instance {\tt lazy beta delta [foo bar]}. This short name can
then be used in {\tt Eval \ident\ in ...} or {\tt eval} directives.
This command accepts the {\tt Local} modifier, for discarding
this reduction name at the end of the file or module. For the moment
the name cannot be qualified. In particular declaring the same name
in several modules or in several functor applications will be refused
if these declarations are not local. The name \ident\ cannot be used
directly as an Ltac tactic, but nothing prevent the user to also
perform a {\tt Ltac \ident\ := {\rm\sl convtactic}}.

\SeeAlso sections \ref{Conversion-tactics}

\subsection{\tt Set Virtual Machine
\label{SetVirtualMachine}
\comindex{Set Virtual Machine}}

This activates the bytecode-based conversion algorithm.

\subsection{\tt Unset Virtual Machine
\comindex{Unset Virtual Machine}}

This deactivates the bytecode-based conversion algorithm.

\subsection{\tt Test Virtual Machine
\comindex{Test Virtual Machine}}

This tells if the bytecode-based conversion algorithm is
activated. The default behavior is to have the bytecode-based
conversion algorithm deactivated.

\SeeAlso sections~\ref{vmcompute} and~\ref{vmoption}.

\section{Controlling the locality of commands}

\subsection{{\tt Local}, {\tt Global}
\comindex{Local}
\comindex{Global}
}

Some commands support a {\tt Local} or {\tt Global} prefix modifier to
control the scope of their effect. There are four kinds of commands:

\begin{itemize}
\item Commands whose default is to extend their effect both outside the
  section and the module or library file they occur in.

  For these commands, the {\tt Local} modifier limits the effect of
  the command to the current section or module it occurs in.

  As an example, the {\tt Coercion} (see Section~\ref{Coercions})
  and {\tt Strategy} (see Section~\ref{Strategy})
  commands belong to this category.

\item Commands whose default behavior is to stop their effect at the
  end of the section they occur in but to extent their effect outside
  the module or library file they occur in.

  For these commands, the {\tt Local} modifier limits the effect of
  the command to the current module if the command does not occur in a
  section and the {\tt Global} modifier extends the effect outside the
  current sections and current module if the command occurs in a
  section.

  As an example, the {\tt Implicit Arguments} (see
  Section~\ref{Implicit Arguments}), {\tt Ltac} (see
  Chapter~\ref{TacticLanguage}) or {\tt Notation} (see
  Section~\ref{Notation}) commands belong to this category.

  Notice that a subclass of these commands do not support extension of
  their scope outside sections at all and the {\tt Global} is not
  applicable to them.

\item Commands whose default behavior is to stop their effect at the
  end of the section or module they occur in.

  For these commands, the {\tt Global} modifier extends their effect
  outside the sections and modules they occurs in.

  The {\tt Transparent} and {\tt Opaque} (see
  Section~\ref{Controlling reduction strategy}) commands belong to
  this category.

\item Commands whose default behavior is to extend their effect
  outside sections but not outside modules when they occur in a
  section and to extend their effect outside the module or library
  file they occur in when no section contains them.

  For these commands, the {\tt Local} modifier limits the effect to
  the current section or module while the {\tt Global} modifier extends
  the effect outside the module even when the command occurs in a section.

  The {\tt Set} and {\tt Unset} commands belong to this category.
\end{itemize}


%%% Local Variables: 
%%% mode: latex
%%% TeX-master: "Reference-Manual"
%%% End: 
