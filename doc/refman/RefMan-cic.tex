\chapter[Calculus of Inductive Constructions]{Calculus of Inductive Constructions
\label{Cic}
\index{Cic@\textsc{CIC}}
\index{pCic@p\textsc{CIC}}
\index{Calculus of (Co)Inductive Constructions}}

The underlying formal language of {\Coq} is a {\em Calculus of
  Constructions} with {\em Inductive Definitions}.  It is presented in
this chapter.  
For {\Coq} version V7, this Calculus was known as the
{\em Calculus of (Co)Inductive Constructions}\index{Calculus of
  (Co)Inductive Constructions} (\iCIC\ in short).
The underlying calculus of {\Coq} version V8.0 and up is a weaker
  calculus where the sort \Set{} satisfies predicative rules. 
We call this calculus the 
{\em Predicative Calculus of (Co)Inductive
  Constructions}\index{Predicative Calculus of
  (Co)Inductive Constructions} (\pCIC\ in short).
In section~\ref{impredicativity} we give the extra-rules for \iCIC. A
  compiling option of \Coq{} allows to type-check theories in this
  extended system.

In \CIC\, all objects have a {\em type}. There are types for functions (or
programs), there are atomic types (especially datatypes)... but also
types for proofs and types for the types themselves.
Especially, any object handled in the formalism must belong to a
type.  For instance, the statement {\it ``for all x, P''} is not
allowed in type theory; you must say instead: {\it ``for all x
belonging to T, P''}. The expression {\it ``x belonging to T''} is
written {\it ``x:T''}. One also says: {\it ``x has type T''}.
The terms of {\CIC} are detailed in section \ref{Terms}.

In \CIC\, there is an internal reduction mechanism. In particular, it
allows to decide if two programs are {\em intentionally} equal (one
says {\em convertible}). Convertibility is presented in section 
\ref{convertibility}.

The remaining sections are concerned with the type-checking of terms.
The beginner can skip them.

The reader seeking a background on the Calculus of Inductive
Constructions may read several papers. Gim�nez and Cast�ran~\cite{GimCas05} 
provide
an introduction to inductive and coinductive definitions in Coq. In
their book~\cite{CoqArt}, Bertot and Cast�ran give a precise
description of the \CIC{} based on numerous practical examples.
Barras~\cite{Bar99}, Werner~\cite{Wer94} and
Paulin-Mohring~\cite{Moh97} are the most recent theses dealing with
Inductive Definitions. Coquand-Huet~\cite{CoHu85a,CoHu85b,CoHu86}
introduces the Calculus of Constructions. Coquand-Paulin~\cite{CoPa89}
extended this calculus to inductive definitions. The {\CIC} is a
formulation of type theory including the possibility of inductive
constructions, Barendregt~\cite{Bar91} studies the modern form of type
theory.

\section[The terms]{The terms\label{Terms}}

In most type theories, one usually makes a syntactic distinction
between types and terms. This is not the case for \CIC\ which defines
both types and terms in the same syntactical structure. This is
because the type-theory itself forces terms and types to be defined in
a mutual recursive way and also because similar constructions can be
applied to both terms and types and consequently can share the same
syntactic structure.

Consider for instance the $\ra$ constructor and assume \nat\ is the
type of natural numbers. Then $\ra$ is used both to denote
$\nat\ra\nat$ which is the type of functions from \nat\ to \nat, and
to denote $\nat \ra \Prop$ which is the type of unary predicates over
the natural numbers. Consider abstraction which builds functions. It
serves to build ``ordinary'' functions as $\kw{fun}~x:\nat \Ra ({\tt mult} ~x~x)$ (assuming {\tt mult} is already defined) but may build also 
predicates over the natural numbers. For instance $\kw{fun}~x:\nat \Ra
(x=x)$ will
represent a predicate $P$, informally written in mathematics
$P(x)\equiv x=x$. If $P$ has type $\nat \ra \Prop$, $(P~x)$ is a
proposition, furthermore $\kw{forall}~x:\nat,(P~x)$ will represent the type of
functions which associate to each natural number $n$ an object of
type $(P~n)$ and consequently represent proofs of the formula
``$\forall x.P(x)$''.

\subsection[Sorts]{Sorts\label{Sorts}
\index{Sorts}}
Types are seen as terms of the language and then should belong to
another type. The type of a type is always a constant of the language
called a {\em sort}.

The two basic sorts in the language of \CIC\ are \Set\ and \Prop.

The sort \Prop\ intends to be the type of logical propositions.  If
$M$ is a logical proposition then it denotes a class, namely the class
of terms representing proofs of $M$. An object $m$ belonging to $M$
witnesses the fact that $M$ is true.  An object of type \Prop\ is
called a {\em proposition}.

The sort \Set\ intends to be the type of specifications. This includes
programs and the usual sets such as booleans, naturals, lists
etc.

These sorts themselves can be manipulated as ordinary terms.
Consequently sorts also should be given a type.  Because assuming
simply that \Set\ has type \Set\ leads to an inconsistent theory, we
have infinitely many sorts in the language of \CIC. These are, in
addition to \Set\ and \Prop\, a hierarchy of universes \Type$(i)$
for any integer $i$.  We call \Sort\ the set of sorts
which is defined by:
\[\Sort \equiv \{\Prop,\Set,\Type(i)| i \in \NN\} \]
\index{Type@{\Type}}
\index{Prop@{\Prop}}
\index{Set@{\Set}}
The sorts enjoy the following properties: {\Prop:\Type(0)}, {\Set:\Type(0)} and
  {\Type$(i)$:\Type$(i+1)$}.

The user will never mention explicitly the index $i$ when referring to
the universe \Type$(i)$. One only writes \Type. The
system itself generates for each instance of \Type\ a new
index for the universe and checks that the constraints between these
indexes can be solved. From the user point of view we consequently
have {\sf Type :Type}.

We shall make precise in the typing rules the constraints between the
indexes. 

\paragraph{Implementation issues}
In practice, the {\Type} hierarchy is implemented using algebraic
universes. An algebraic universe $u$ is either a variable (a qualified
identifier with a number) or a successor of an algebraic universe (an
expression $u+1$), or an upper bound of algebraic universes (an
expression $max(u_1,...,u_n)$), or the base universe (the expression
$0$) which corresponds, in the arity of sort-polymorphic inductive
types, to the predicative sort {\Set}. A graph of constraints between
the universe variables is maintained globally. To ensure the existence
of a mapping of the universes to the positive integers, the graph of
constraints must remain acyclic.  Typing expressions that violate the
acyclicity of the graph of constraints results in a \errindex{Universe
inconsistency} error (see also section~\ref{PrintingUniverses}).

\subsection{Constants}
Besides the sorts, the language also contains constants denoting
objects in the environment. These constants may denote previously
defined objects but also objects related to inductive definitions
(either the type itself or one of its constructors or destructors).

\medskip\noindent {\bf Remark. } In other presentations of \CIC, 
the inductive objects are not seen as
external declarations but as first-class terms. Usually the
definitions are also completely ignored.  This is a nice theoretical
point of view but not so practical. An inductive definition is
specified by a possibly huge set of declarations, clearly we want to
share this specification among the various inductive objects and not
to duplicate it. So the specification should exist somewhere and the
various objects should refer to it.  We choose one more level of
indirection where the objects are just represented as constants and
the environment gives the information on the kind of object the
constant refers to.

\medskip
Our inductive objects will be manipulated as constants declared in the
environment. This roughly corresponds to the way they are actually
implemented in the \Coq\ system. It is simple to map this presentation
in a theory where inductive objects are represented by terms.

\subsection{Terms}

Terms are built from variables, global names, constructors,
abstraction, application, local declarations bindings (``let-in''
expressions) and product.

From a syntactic point of view, types cannot be distinguished from terms,
except that they cannot start by an abstraction, and that if a term is
a sort or a product, it should be a type.

More precisely the language of the {\em Calculus of Inductive
  Constructions} is built from the following rules:

\begin{enumerate}
\item the sorts {\sf Set, Prop, Type} are terms.
\item names for global constants of the environment are terms.
\item variables are terms.
\item if $x$ is a variable and $T$, $U$ are terms then $\forall~x:T,U$
  ($\kw{forall}~x:T,U$ in \Coq{} concrete syntax) is a term. If $x$
  occurs in $U$, $\forall~x:T,U$ reads as {\it ``for all x of type T,
    U''}. As $U$ depends on $x$, one says that $\forall~x:T,U$ is a
  {\em dependent product}. If $x$ doesn't occurs in $U$ then
  $\forall~x:T,U$ reads as {\it ``if T then U''}. A non dependent
  product can be written: $T \rightarrow U$.
\item if $x$ is a variable and $T$, $U$ are terms then $\lb~x:T \mto U$
  ($\kw{fun}~x:T\Ra U$ in \Coq{} concrete syntax) is a term. This is a
  notation for the $\lambda$-abstraction of
  $\lambda$-calculus\index{lambda-calculus@$\lambda$-calculus}
  \cite{Bar81}. The term $\lb~x:T \mto U$ is a function which maps
  elements of $T$ to $U$.
\item if $T$ and $U$ are terms then $(T\ U)$ is a term  
 ($T~U$ in \Coq{} concrete syntax).  The term $(T\ 
  U)$ reads as {\it ``T applied to U''}.
\item if $x$ is a variable, and $T$, $U$ are terms then
  $\kw{let}~x:=T~\kw{in}~U$ is a
  term which denotes the term $U$ where the variable $x$ is locally
  bound to $T$. This stands for the common ``let-in'' construction of
  functional programs such as ML or Scheme.
\end{enumerate}

\paragraph{Notations.} Application associates to the left such that
$(t~t_1\ldots t_n)$ represents $(\ldots (t~t_1)\ldots t_n)$. The
products and arrows associate to the right such that $\forall~x:A,B\ra C\ra
D$ represents $\forall~x:A,(B\ra (C\ra D))$.  One uses sometimes
$\forall~x~y:A,B$ or
$\lb~x~y:A\mto B$ to denote the abstraction or product of several variables
of the same type. The equivalent formulation is $\forall~x:A, \forall y:A,B$ or
$\lb~x:A \mto \lb y:A \mto B$

\paragraph{Free variables.}
The notion of free variables is defined as usual.  In the expressions
$\lb~x:T\mto U$ and $\forall x:T, U$ the occurrences of $x$ in $U$
are bound.  They are represented by de Bruijn indexes in the internal
structure of terms.

\paragraph[Substitution.]{Substitution.\index{Substitution}}
The notion of substituting a term $t$ to free occurrences of a
variable $x$ in a term $u$ is defined as usual. The resulting term
is written $\subst{u}{x}{t}$.


\section[Typed terms]{Typed terms\label{Typed-terms}}

As objects of type theory, terms are subjected to {\em type
discipline}. The well typing of a term depends on an environment which
consists in a global environment (see below) and a local context.

\paragraph{Local context.}
A {\em local context} (or shortly context) is an ordered list of
declarations of variables. The declaration of some variable $x$ is
either an assumption, written $x:T$ ($T$ is a type) or a definition,
written $x:=t:T$.  We use brackets to write contexts. A
typical example is $[x:T;y:=u:U;z:V]$.  Notice that the variables
declared in a context must be distinct. If $\Gamma$ declares some $x$,
we write $x \in \Gamma$. By writing $(x:T) \in \Gamma$ we mean that
either $x:T$ is an assumption in $\Gamma$ or that there exists some $t$ such
that $x:=t:T$ is a definition in $\Gamma$. If $\Gamma$ defines some
$x:=t:T$, we also write $(x:=t:T) \in \Gamma$.  Contexts must be
themselves {\em well formed}.  For the rest of the chapter, the
notation $\Gamma::(y:T)$ (resp. $\Gamma::(y:=t:T)$) denotes the context
$\Gamma$ enriched with the declaration $y:T$ (resp. $y:=t:T$). The
notation $[]$ denotes the empty context.  \index{Context}

% Does not seem to be used further...
% Si dans l'explication WF(E)[Gamma] concernant les constantes
% definies ds un contexte

We define the inclusion of two contexts $\Gamma$ and $\Delta$ (written
as $\Gamma \subset \Delta$) as the property, for all variable $x$,
type $T$ and term $t$, if $(x:T) \in \Gamma$ then $(x:T) \in \Delta$
and if $(x:=t:T) \in \Gamma$ then $(x:=t:T) \in \Delta$.  
%We write
% $|\Delta|$ for the length of the context $\Delta$, that is for the number
% of declarations (assumptions or definitions) in $\Delta$.

A variable $x$ is said to be free in $\Gamma$ if $\Gamma$ contains a
declaration $y:T$ such that $x$ is free in $T$.

\paragraph[Environment.]{Environment.\index{Environment}}
Because we are manipulating global declarations (constants and global
assumptions), we also need to consider a global environment $E$.

An environment is an ordered list of declarations of global
names. Declarations are either assumptions or ``standard''
definitions, that is abbreviations for well-formed terms
but also definitions of inductive objects. In the latter
case, an object in the environment will define one or more constants
(that is types and constructors, see section \ref{Cic-inductive-definitions}).

An assumption will be represented in the environment as
\Assum{\Gamma}{c}{T} which means that $c$ is assumed of some type $T$
well-defined in some context $\Gamma$. An (ordinary) definition will
be represented in the environment as \Def{\Gamma}{c}{t}{T} which means
that $c$ is a constant which is valid in some context $\Gamma$ whose
value is $t$ and type is $T$.

The rules for inductive definitions (see section
\ref{Cic-inductive-definitions}) have to be considered as assumption
rules to which the following definitions apply: if the name $c$ is
declared in $E$, we write $c \in E$ and if $c:T$ or $c:=t:T$ is
declared in $E$, we write $(c : T) \in E$.

\paragraph[Typing rules.]{Typing rules.\label{Typing-rules}\index{Typing rules}}
In the following, we assume $E$ is a valid environment wrt to
inductive definitions.  We define simultaneously two
judgments.  The first one \WTEG{t}{T} means the term $t$ is well-typed
and has type $T$ in the environment $E$ and context $\Gamma$.  The
second judgment \WFE{\Gamma} means that the environment $E$ is
well-formed and the context $\Gamma$ is a valid context in this
environment.  It also means a third property which makes sure that any
constant in $E$ was defined in an environment which is included in
$\Gamma$
\footnote{This requirement could be relaxed if we instead introduced
  an explicit mechanism for instantiating constants. At the external
  level, the Coq engine works accordingly to this view that all the
  definitions in the environment were built in a sub-context of the
  current context.}.

A term $t$ is well typed in an environment $E$ iff there exists a
context $\Gamma$ and a term $T$ such that the judgment \WTEG{t}{T} can
be derived from the following rules.
\begin{description}
\item[W-E] \inference{\WF{[]}{[]}}
\item[W-S]  % Ce n'est pas vrai : x peut apparaitre plusieurs fois dans Gamma
\inference{\frac{\WTEG{T}{s}~~~~s \in \Sort~~~~x \not\in
      \Gamma           % \cup E
      }
      {\WFE{\Gamma::(x:T)}}~~~~~
    \frac{\WTEG{t}{T}~~~~x \not\in
      \Gamma           % \cup E
     }{\WFE{\Gamma::(x:=t:T)}}}
\item[Def] \inference{\frac{\WTEG{t}{T}~~~c \notin E \cup \Gamma}
                      {\WF{E;\Def{\Gamma}{c}{t}{T}}{\Gamma}}}
\item[Assum] \inference{\frac{\WTEG{T}{s}~~~~s \in \Sort~~~~c \notin E \cup \Gamma}
                      {\WF{E;\Assum{\Gamma}{c}{T}}{\Gamma}}}
\item[Ax] \index{Typing rules!Ax}
\inference{\frac{\WFE{\Gamma}}{\WTEG{\Prop}{\Type(p)}}~~~~~
\frac{\WFE{\Gamma}}{\WTEG{\Set}{\Type(q)}}}
\inference{\frac{\WFE{\Gamma}~~~~i<j}{\WTEG{\Type(i)}{\Type(j)}}}
\item[Var]\index{Typing rules!Var}
 \inference{\frac{ \WFE{\Gamma}~~~~~(x:T) \in \Gamma~~\mbox{or}~~(x:=t:T) \in \Gamma~\mbox{for some $t$}}{\WTEG{x}{T}}}
\item[Const]  \index{Typing rules!Const}
\inference{\frac{\WFE{\Gamma}~~~~(c:T) \in E~~\mbox{or}~~(c:=t:T) \in E~\mbox{for some $t$} }{\WTEG{c}{T}}}
\item[Prod]  \index{Typing rules!Prod}
\inference{\frac{\WTEG{T}{s}~~~~s \in \Sort~~~
    \WTE{\Gamma::(x:T)}{U}{\Prop}}
      { \WTEG{\forall~x:T,U}{\Prop}}} 
\inference{\frac{\WTEG{T}{s}~~~~s \in\{\Prop, \Set\}~~~~~~
    \WTE{\Gamma::(x:T)}{U}{\Set}}
      { \WTEG{\forall~x:T,U}{\Set}}} 
\inference{\frac{\WTEG{T}{\Type(i)}~~~~i\leq k~~~
    \WTE{\Gamma::(x:T)}{U}{\Type(j)}~~~j \leq k}
    {\WTEG{\forall~x:T,U}{\Type(k)}}}
\item[Lam]\index{Typing rules!Lam} 
\inference{\frac{\WTEG{\forall~x:T,U}{s}~~~~ \WTE{\Gamma::(x:T)}{t}{U}}
        {\WTEG{\lb~x:T\mto t}{\forall x:T, U}}}
\item[App]\index{Typing rules!App}
 \inference{\frac{\WTEG{t}{\forall~x:U,T}~~~~\WTEG{u}{U}}
                 {\WTEG{(t\ u)}{\subst{T}{x}{u}}}}
\item[Let]\index{Typing rules!Let} 
\inference{\frac{\WTEG{t}{T}~~~~ \WTE{\Gamma::(x:=t:T)}{u}{U}}
        {\WTEG{\kw{let}~x:=t~\kw{in}~u}{\subst{U}{x}{t}}}}
\end{description}
    
\Rem We may have $\kw{let}~x:=t~\kw{in}~u$
well-typed without having $((\lb~x:T\mto u)~t)$ well-typed (where
$T$ is a type of $t$). This is because the value $t$ associated to $x$
may be used in a conversion rule (see section \ref{conv-rules}).

\section[Conversion rules]{Conversion rules\index{Conversion rules}
\label{conv-rules}}
\paragraph[$\beta$-reduction.]{$\beta$-reduction.\label{beta}\index{beta-reduction@$\beta$-reduction}}

We want to be able to identify some terms as we can identify the
application of a function to a given argument with its result. For
instance the identity function over a given type $T$ can be written
$\lb~x:T\mto x$. In any environment $E$ and context $\Gamma$, we want to identify any object $a$ (of type $T$) with the
application $((\lb~x:T\mto x)~a)$. We define for this a {\em reduction} (or a
{\em conversion}) rule we call $\beta$:
\[ \WTEGRED{((\lb~x:T\mto
  t)~u)}{\triangleright_{\beta}}{\subst{t}{x}{u}} \] 
We say that $\subst{t}{x}{u}$ is the {\em $\beta$-contraction} of
$((\lb~x:T\mto t)~u)$ and, conversely, that $((\lb~x:T\mto t)~u)$
is the {\em $\beta$-expansion} of $\subst{t}{x}{u}$.

According to $\beta$-reduction, terms of the {\em Calculus of
  Inductive Constructions} enjoy some fundamental properties such as
confluence, strong normalization, subject reduction. These results are
theoretically of great importance but we will not detail them here and
refer the interested reader to \cite{Coq85}.

\paragraph[$\iota$-reduction.]{$\iota$-reduction.\label{iota}\index{iota-reduction@$\iota$-reduction}}
A specific conversion rule is associated to the inductive objects in
the environment.  We shall give later on (section \ref{iotared}) the
precise rules but it just says that a destructor applied to an object
built from a constructor behaves as expected.  This reduction is
called $\iota$-reduction and is more precisely studied in
\cite{Moh93,Wer94}.


\paragraph[$\delta$-reduction.]{$\delta$-reduction.\label{delta}\index{delta-reduction@$\delta$-reduction}}

We may have defined variables in contexts or constants in the global
environment. It is legal to identify such a reference with its value,
that is to expand (or unfold) it into its value. This
reduction is called $\delta$-reduction and shows as follows.

$$\WTEGRED{x}{\triangleright_{\delta}}{t}~~~~~\mbox{if $(x:=t:T) \in \Gamma$}~~~~~~~~~\WTEGRED{c}{\triangleright_{\delta}}{t}~~~~~\mbox{if $(c:=t:T) \in E$}$$


\paragraph[$\zeta$-reduction.]{$\zeta$-reduction.\label{zeta}\index{zeta-reduction@$\zeta$-reduction}}

Coq allows also to remove local definitions occurring in terms by
replacing the defined variable by its value. The declaration being
destroyed, this reduction differs from $\delta$-reduction. It is
called $\zeta$-reduction and shows as follows.

$$\WTEGRED{\kw{let}~x:=u~\kw{in}~t}{\triangleright_{\zeta}}{\subst{t}{x}{u}}$$

\paragraph[Convertibility.]{Convertibility.\label{convertibility}
\index{beta-reduction@$\beta$-reduction}\index{iota-reduction@$\iota$-reduction}\index{delta-reduction@$\delta$-reduction}\index{zeta-reduction@$\zeta$-reduction}}

Let us write $\WTEGRED{t}{\triangleright}{u}$ for the contextual closure of the relation $t$ reduces to $u$ in the environment $E$ and context $\Gamma$ with one of the previous reduction $\beta$, $\iota$, $\delta$ or $\zeta$.

We say that two terms $t_1$ and $t_2$ are {\em convertible} (or {\em
  equivalent)} in the environment $E$ and context $\Gamma$ iff there exists a term $u$ such that $\WTEGRED{t_1}{\triangleright \ldots \triangleright}{u}$
and $\WTEGRED{t_2}{\triangleright \ldots \triangleright}{u}$.
We then write $\WTEGCONV{t_1}{t_2}$.

The convertibility relation allows to introduce a new typing rule
which says that two convertible well-formed types have the same
inhabitants.

At the moment, we did not take into account one rule between universes
which says that any term in a universe of index $i$ is also a term in
the universe of index $i+1$. This property is included into the
conversion rule by extending the equivalence relation of
convertibility into an order inductively defined by:
\begin{enumerate}
\item if $\WTEGCONV{t}{u}$ then $\WTEGLECONV{t}{u}$,
\item if $i \leq j$ then $\WTEGLECONV{\Type(i)}{\Type(j)}$,
\item for any $i$, $\WTEGLECONV{\Prop}{\Type(i)}$,
\item for any $i$, $\WTEGLECONV{\Set}{\Type(i)}$,
\item if $\WTEGCONV{T}{U}$ and $\WTELECONV{\Gamma::(x:T)}{T'}{U'}$ then $\WTEGLECONV{\forall~x:T,T'}{\forall~x:U,U'}$.
\end{enumerate}

The conversion rule is now exactly:

\begin{description}\label{Conv}
\item[Conv]\index{Typing rules!Conv}
 \inference{
      \frac{\WTEG{U}{s}~~~~\WTEG{t}{T}~~~~\WTEGLECONV{T}{U}}{\WTEG{t}{U}}}
  \end{description}


\paragraph{$\eta$-conversion.
\label{eta}
\index{eta-conversion@$\eta$-conversion}
\index{eta-reduction@$\eta$-reduction}}

An other important rule is the $\eta$-conversion. It is to identify
terms over a dummy abstraction of a variable followed by an
application of this variable. Let $T$ be a type, $t$ be a term in
which the variable $x$ doesn't occurs free. We have
\[ \WTEGRED{\lb~x:T\mto (t\ x)}{\triangleright}{t} \]
Indeed, as $x$ doesn't occur free in $t$, for any $u$ one
applies to $\lb~x:T\mto (t\ x)$, it $\beta$-reduces to $(t\ u)$. So
$\lb~x:T\mto (t\ x)$ and $t$ can be identified.

\Rem The $\eta$-reduction is not taken into account in the
convertibility rule of \Coq.

\paragraph[Normal form.]{Normal form.\index{Normal form}\label{Normal-form}\label{Head-normal-form}\index{Head normal form}}
A term which cannot be any more reduced is said to be in {\em normal
  form}. There are several ways (or strategies) to apply the reduction
rule. Among them, we have to mention the {\em head reduction} which
will play an important role (see chapter \ref{Tactics}). Any term can
be written as $\lb~x_1:T_1\mto \ldots \lb x_k:T_k \mto 
(t_0\ t_1\ldots t_n)$ where
$t_0$ is not an application. We say then that $t_0$ is the {\em head
  of $t$}. If we assume that $t_0$ is $\lb~x:T\mto u_0$ then one step of
$\beta$-head reduction of $t$ is:
\[\lb~x_1:T_1\mto \ldots \lb x_k:T_k\mto (\lb~x:T\mto u_0\ t_1\ldots t_n) 
~\triangleright ~ \lb~(x_1:T_1)\ldots(x_k:T_k)\mto 
(\subst{u_0}{x}{t_1}\ t_2 \ldots t_n)\]
Iterating the process of head reduction until the head of the reduced
term is no more an abstraction leads to the {\em $\beta$-head normal
  form} of $t$:
\[ t \triangleright \ldots \triangleright
\lb~x_1:T_1\mto \ldots\lb x_k:T_k\mto (v\ u_1
\ldots u_m)\]
where $v$ is not an abstraction (nor an application).  Note that the
head normal form must not be confused with the normal form since some
$u_i$ can be reducible.

Similar notions of head-normal forms involving $\delta$, $\iota$ and $\zeta$
reductions or any combination of those can also be defined.

\section{Derived rules for environments}

From the original rules of the type system, one can derive new rules
which change the context of definition of objects in the environment.
Because these rules correspond to elementary operations in the \Coq\ 
engine used in the discharge mechanism at the end of a section, we
state them explicitly.

\paragraph{Mechanism of substitution.}

One rule which can be proved valid, is to replace a term $c$ by its
value in the environment. As we defined the substitution of a term for
a variable in a term, one can define the substitution of a term for a
constant. One easily extends this substitution to contexts and
environments.

\paragraph{Substitution Property:} 
\inference{\frac{\WF{E;\Def{\Gamma}{c}{t}{T}; F}{\Delta}}
           {\WF{E; \subst{F}{c}{t}}{\subst{\Delta}{c}{t}}}}


\paragraph{Abstraction.}

One can modify the context of definition of a constant $c$ by
abstracting a constant with respect to the last variable $x$ of its
defining context. For doing that, we need to check that the constants
appearing in the body of the declaration do not depend on $x$, we need
also to modify the reference to the constant $c$ in the environment
and context by explicitly applying this constant to the variable $x$.
Because of the rules for building environments and terms we know the
variable $x$ is available at each stage where $c$ is mentioned.

\paragraph{Abstracting property:} 
 \inference{\frac{\WF{E; \Def{\Gamma::(x:U)}{c}{t}{T};
       F}{\Delta}~~~~\WFE{\Gamma}}
           {\WF{E;\Def{\Gamma}{c}{\lb~x:U\mto t}{\forall~x:U,T};
  \subst{F}{c}{(c~x)}}{\subst{\Delta}{c}{(c~x)}}}}

\paragraph{Pruning the context.} 
We said the judgment \WFE{\Gamma} means that the defining contexts of
constants in $E$ are included in $\Gamma$. If one abstracts or
substitutes the constants with the above rules then it may happen
that the context $\Gamma$ is now bigger than the one needed for
defining the constants in $E$. Because defining contexts are growing
in $E$, the minimum context needed for defining the constants in $E$
is the same as the one for the last constant. One can consequently
derive the following property.

\paragraph{Pruning property:}
\inference{\frac{\WF{E; \Def{\Delta}{c}{t}{T}}{\Gamma}}
                {\WF{E;\Def{\Delta}{c}{t}{T}}{\Delta}}}


\section[Inductive Definitions]{Inductive Definitions\label{Cic-inductive-definitions}}

A (possibly mutual) inductive definition is specified by giving the
names and the type of the inductive sets or families to be
defined and the names and types of the constructors of the inductive
predicates.  An inductive declaration in the environment can
consequently be represented with two contexts (one for inductive
definitions, one for constructors).

Stating the rules for inductive definitions in their general form
needs quite tedious definitions. We shall try to give a concrete
understanding of the rules by precising them on running examples.  We
take as examples the type of natural numbers, the type of
parameterized lists over a type $A$, the relation which states that
a list has some given length and the mutual inductive definition of trees and
forests. 

\subsection{Representing an inductive definition}
\subsubsection{Inductive definitions without parameters}
As for constants, inductive definitions can be defined in a non-empty
context. \\
We write \NInd{\Gamma}{\Gamma_I}{\Gamma_C} an inductive
definition valid in a context $\Gamma$, a
context of definitions $\Gamma_I$ and a context of constructors
$\Gamma_C$.
\paragraph{Examples.}
The inductive declaration for the type of natural numbers will be:
\[\NInd{}{\nat:\Set}{\nO:\nat,\nS:\nat\ra\nat}\]
In a context with a variable $A:\Set$, the lists of elements in $A$ is
represented by:
\[\NInd{A:\Set}{\List:\Set}{\Nil:\List,\cons : A \ra \List \ra
  \List}\]
 Assuming 
  $\Gamma_I$ is $[I_1:A_1;\ldots;I_k:A_k]$, and $\Gamma_C$ is
  $[c_1:C_1;\ldots;c_n:C_n]$, the general typing rules are, 
  for $1\leq j\leq k$ and $1\leq i\leq n$:

\bigskip
\inference{\frac{\NInd{\Gamma}{\Gamma_I}{\Gamma_C} \in E}{(I_j:A_j) \in E}}

\inference{\frac{\NInd{\Gamma}{\Gamma_I}{\Gamma_C} \in E}{(c_i:C_i) \in E}}

\subsubsection{Inductive definitions with parameters}

We have to slightly complicate the representation above in order to handle
the delicate problem of parameters. 
Let us explain that on the example of \List. As they were defined
above, the type \List\ can only be used in an environment where we
have a variable $A:\Set$. Generally one want to consider lists of
elements in different types. For constants this is easily done by abstracting
the value over the parameter. In the case of inductive definitions we
have to handle the abstraction over several objects.

One possible way to do that would be to define the type \List\
inductively as being an inductive family of type $\Set\ra\Set$:
\[\NInd{}{\List:\Set\ra\Set}{\Nil:(A:\Set)(\List~A),\cons : (A:\Set)A
  \ra (\List~A) \ra (\List~A)}\]
There are drawbacks to this point of view. The
information which says that for any $A$, $(\List~A)$ is an inductively defined
\Set\ has been lost.
So we introduce two important definitions.

\paragraph{Inductive parameters, real arguments.}
An inductive definition $\NInd{\Gamma}{\Gamma_I}{\Gamma_C}$ admits 
$r$ inductive parameters if each type of constructors $(c:C)$ in
$\Gamma_C$ is such that 
\[C\equiv \forall
p_1:P_1,\ldots,\forall p_r:P_r,\forall a_1:A_1, \ldots \forall a_n:A_n,
(I~p_1~\ldots p_r~t_1\ldots t_q)\]
with $I$ one of the inductive definitions in $\Gamma_I$. 
We say that $n$ is the number of real arguments of the constructor
$c$. 
\paragraph{Context of parameters.}
If an inductive definition $\NInd{\Gamma}{\Gamma_I}{\Gamma_C}$ admits 
$r$ inductive parameters, then there exists a context $\Gamma_P$ of
size $r$, such that $\Gamma_P=p_1:P_1;\ldots;p_r:P_r$ and 
if $(t:A) \in \Gamma_I,\Gamma_C$ then $A$ can be written as 
$\forall p_1:P_1,\ldots \forall p_r:P_r,A'$. 
We call $\Gamma_P$ the context of parameters of the inductive
definition and use the notation $\forall \Gamma_P,A'$ for the term $A$.
\paragraph{Remark.}
If we have a term $t$ in an instance of an
inductive definition $I$ which starts with a constructor $c$, then the
$r$ first arguments of $c$ (the parameters) can be deduced from the
type $T$ of $t$: these are exactly the $r$ first arguments of $I$ in
the head normal form of $T$.
\paragraph{Examples.}
The \List{} definition has $1$ parameter:
\[\NInd{}{\List:\Set\ra\Set}{\Nil:(A:\Set)(\List~A),\cons : (A:\Set)A
  \ra (\List~A) \ra (\List~A)}\]
This is also the case for this more complex definition where there is
a recursive argument on a different instance of \List: 
\[\NInd{}{\List:\Set\ra\Set}{\Nil:(A:\Set)(\List~A),\cons : (A:\Set)A
           \ra (\List~A\ra A) \ra (\List~A)}\]
But the following definition has $0$ parameters:
\[\NInd{}{\List:\Set\ra\Set}{\Nil:(A:\Set)(\List~A),\cons : (A:\Set)A
           \ra (\List~A) \ra (\List~A*A)}\]

%\footnote{
%The interested reader may compare the above definition with the two 
%following ones which have very different logical meaning:\\
%$\NInd{}{\List:\Set}{\Nil:\List,\cons : (A:\Set)A
%  \ra \List \ra \List}$ \\
%$\NInd{}{\List:\Set\ra\Set}{\Nil:(A:\Set)(\List~A),\cons : (A:\Set)A
%  \ra (\List~A\ra A) \ra (\List~A)}$.}
\paragraph{Concrete syntax.}
In the Coq system, the context of parameters is given explicitly
after the name of the inductive definitions and is shared between the
arities and the type of constructors.
% The vernacular declaration of polymorphic trees and forests will be:\\
% \begin{coq_example*}
% Inductive Tree (A:Set) : Set := 
%    Node : A -> Forest A -> Tree A
% with Forest (A : Set) : Set := 
%    Empty : Forest A
%  | Cons  : Tree A -> Forest A -> Forest A
% \end{coq_example*}
% will correspond in our formalism to:
% \[\NInd{}{{\tt Tree}:\Set\ra\Set;{\tt Forest}:\Set\ra \Set}
%   {{\tt Node} : \forall A:\Set, A \ra {\tt Forest}~A \ra {\tt Tree}~A,
%    {\tt Empty} : \forall A:\Set, {\tt Forest}~A,
%    {\tt Cons} : \forall A:\Set, {\tt Tree}~A \ra {\tt Forest}~A \ra
%    {\tt Forest}~A}\]
We keep track in the syntax of the number of
parameters. 

Formally the representation of an inductive declaration
will be 
\Ind{\Gamma}{p}{\Gamma_I}{\Gamma_C} for an inductive
definition valid in a context $\Gamma$ with $p$ parameters, a
context of definitions $\Gamma_I$ and a context of constructors
$\Gamma_C$.

The definition \Ind{\Gamma}{p}{\Gamma_I}{\Gamma_C} will be
well-formed exactly when \NInd{\Gamma}{\Gamma_I}{\Gamma_C} is and 
when $p$ is (less or equal than) the number of parameters in
\NInd{\Gamma}{\Gamma_I}{\Gamma_C}. 

\paragraph{Examples}
The declaration for parameterized lists is:
\[\Ind{}{1}{\List:\Set\ra\Set}{\Nil:\forall A:\Set,\List~A,\cons : \forall
  A:\Set, A \ra \List~A \ra   \List~A}\]

The declaration for the length of lists is:
\[\Ind{}{1}{\Length:\forall A:\Set, (\List~A)\ra \nat\ra\Prop}
      {\LNil:\forall A:\Set, \Length~A~(\Nil~A)~\nO,\\ 
     \LCons :\forall A:\Set,\forall a:A, \forall l:(\List~A),\forall n:\nat, (\Length~A~l~n)\ra (\Length~A~(\cons~A~a~l)~(\nS~n))}\]

The declaration for a mutual inductive definition of forests and trees is:
\[\NInd{}{\tree:\Set,\forest:\Set}
      {\\~~\node:\forest \ra \tree,
       \emptyf:\forest,\consf:\tree \ra \forest \ra \forest\-}\]
  
These representations are the ones obtained as the result of the \Coq\ 
declaration:
\begin{coq_example*}
Inductive nat : Set :=
  | O : nat
  | S : nat -> nat.
Inductive list (A:Set) : Set :=
  | nil : list A
  | cons : A -> list A -> list A.
\end{coq_example*}
\begin{coq_example*}
Inductive Length (A:Set) : list A -> nat -> Prop :=
  | Lnil : Length A (nil A) O
  | Lcons :
      forall (a:A) (l:list A) (n:nat),
        Length A l n -> Length A (cons A a l) (S n).
Inductive tree : Set :=
    node : forest -> tree
with forest : Set :=
  | emptyf : forest
  | consf : tree -> forest -> forest.
\end{coq_example*}
% The inductive declaration in \Coq\ is slightly different from the one
% we described theoretically. The difference is that in the type of
% constructors the inductive definition is explicitly applied to the
% parameters variables.
The \Coq\ type-checker verifies that all
parameters are applied in the correct manner in the conclusion of the
type of each constructors~:

In particular, the following definition will not be accepted because 
there is an occurrence of \List\ which is not applied to the parameter
variable in the conclusion of the type of {\tt cons'}:
\begin{coq_eval}
Set Printing Depth 50.
(********** The following is not correct and should produce **********)
(********* Error: The 1st argument of list' must be A in ... *********)
\end{coq_eval}
\begin{coq_example}
Inductive list' (A:Set) : Set :=
  | nil' : list' A
  | cons' : A -> list' A -> list' (A*A).
\end{coq_example}
Since \Coq{} version 8.1, there is no restriction about parameters in
the types of arguments of constructors. The following definition is
valid:
\begin{coq_example}
Inductive list' (A:Set) : Set :=
  | nil' : list' A
  | cons' : A -> list' (A->A) -> list' A.
\end{coq_example}


\subsection{Types of inductive objects}
We have to give the type of constants in an environment $E$ which
contains an inductive declaration.

\begin{description}
\item[Ind-Const] Assuming 
  $\Gamma_I$ is $[I_1:A_1;\ldots;I_k:A_k]$, and $\Gamma_C$ is
  $[c_1:C_1;\ldots;c_n:C_n]$,
   
\inference{\frac{\Ind{\Gamma}{p}{\Gamma_I}{\Gamma_C} \in E
    ~~j=1\ldots k}{(I_j:A_j) \in E}}

\inference{\frac{\Ind{\Gamma}{p}{\Gamma_I}{\Gamma_C} \in E
    ~~~~i=1.. n}
   {(c_i:C_i) \in E}}
\end{description}

\paragraph{Example.}
We have $(\List:\Set \ra \Set), (\cons:\forall~A:\Set,A\ra(\List~A)\ra
(\List~A))$, \\ 
$(\Length:\forall~A:\Set, (\List~A)\ra\nat\ra\Prop)$, $\tree:\Set$ and $\forest:\Set$.

From now on, we write $\ListA$ instead of $(\List~A)$ and $\LengthA$
for $(\Length~A)$.

%\paragraph{Parameters.}
%%The parameters introduce a distortion between the inside specification
%%of the inductive declaration where parameters are supposed to be
%%instantiated (this representation is appropriate for checking the
%%correctness or deriving the destructor principle) and the outside
%%typing rules where the inductive objects are seen as objects
%%abstracted with respect to the parameters.

%In the definition of \List\ or \Length\, $A$ is a parameter because
%what is effectively inductively defined is $\ListA$ or $\LengthA$ for
%a given $A$ which is constant in the type of constructors.  But when
%we define $(\LengthA~l~n)$, $l$ and $n$ are not parameters because the
%constructors manipulate different instances of this family.

\subsection{Well-formed inductive definitions}
We cannot accept any inductive declaration because some of them lead
to inconsistent systems. We restrict ourselves to definitions which
satisfy a syntactic criterion of positivity. Before giving the formal
rules, we need a few definitions:

\paragraph[Definitions]{Definitions\index{Positivity}\label{Positivity}}

A type $T$ is an {\em arity of sort $s$}\index{Arity} if it converts
to the sort $s$ or to a product $\forall~x:T,U$ with $U$ an arity
of sort $s$. (For instance $A\ra \Set$ or $\forall~A:\Prop,A\ra
\Prop$ are arities of sort respectively \Set\ and \Prop).  A {\em type
  of constructor of $I$}\index{Type of constructor} is either a term
$(I~t_1\ldots ~t_n)$ or $\fa x:T,C$ with $C$ a {\em type of constructor
  of $I$}.

\smallskip

The type of constructor $T$ will be said to {\em satisfy the positivity
condition} for a constant $X$ in the following cases:

\begin{itemize}
\item $T=(X~t_1\ldots ~t_n)$ and $X$ does not occur free in
any $t_i$
\item $T=\forall~x:U,V$ and $X$ occurs only strictly positively in $U$ and
the type $V$ satisfies the positivity condition for $X$
\end{itemize}

The constant $X$ {\em occurs strictly positively} in $T$ in the
following cases:

\begin{itemize}
\item $X$ does not occur in $T$
\item $T$ converts to $(X~t_1 \ldots ~t_n)$ and $X$ does not occur in
  any of $t_i$
\item $T$ converts to $\forall~x:U,V$ and $X$ does not occur in
  type $U$ but occurs strictly positively in type $V$
\item $T$ converts to $(I~a_1 \ldots ~a_m ~ t_1 \ldots ~t_p)$ where
  $I$ is the name of an inductive declaration of the form
  $\Ind{\Gamma}{m}{I:A}{c_1:\forall p_1:P_1,\ldots \forall
    p_m:P_m,C_1;\ldots;c_n:\forall p_1:P_1,\ldots \forall
    p_m:P_m,C_n}$ 
  (in particular, it is not mutually defined and it has $m$
  parameters) and $X$ does not occur in any of the $t_i$, and the
  (instantiated) types of constructor $C_i\{p_j/a_j\}_{j=1\ldots m}$
  of $I$ satisfy 
  the nested positivity condition for $X$
%\item more generally, when $T$ is not a type, $X$ occurs strictly
%positively in $T[x:U]u$ if $X$ does not occur in $U$ but occurs
%strictly positively in $u$
\end{itemize}

The type of constructor $T$ of $I$ {\em satisfies the nested
positivity condition} for a constant $X$ in the following
cases:

\begin{itemize}
\item $T=(I~b_1\ldots b_m~u_1\ldots ~u_{p})$, $I$ is an inductive
  definition with $m$ parameters and $X$ does not occur in
any $u_i$
\item $T=\forall~x:U,V$ and $X$ occurs only strictly positively in $U$ and
the type $V$ satisfies the nested positivity condition for $X$
\end{itemize}

\paragraph{Example}

$X$ occurs strictly positively in $A\ra X$ or $X*A$ or $({\tt list}~
X)$ but not in $X \ra A$ or $(X \ra A)\ra A$ nor $({\tt neg}~A)$
assuming the notion of product and lists were already defined and {\tt
  neg} is an inductive definition with declaration \Ind{}{A:\Set}{{\tt
    neg}:\Set}{{\tt neg}:(A\ra{\tt False}) \ra {\tt neg}}.  Assuming
$X$ has arity ${\tt nat \ra Prop}$ and {\tt ex} is the inductively
defined existential quantifier, the occurrence of $X$ in ${\tt (ex~
  nat~ \lb~n:nat\mto (X~ n))}$ is also strictly positive.

\paragraph{Correctness rules.}
We shall now describe the rules allowing the introduction of a new
inductive definition.

\begin{description}
\item[W-Ind] Let $E$ be an environment and
  $\Gamma,\Gamma_P,\Gamma_I,\Gamma_C$ are contexts such that
  $\Gamma_I$ is $[I_1:\forall \Gamma_P,A_1;\ldots;I_k:\forall
  \Gamma_P,A_k]$ and $\Gamma_C$ is 
  $[c_1:\forall \Gamma_P,C_1;\ldots;c_n:\forall \Gamma_P,C_n]$. 
\inference{
  \frac{
  (\WTE{\Gamma;\Gamma_P}{A_j}{s'_j})_{j=1\ldots  k}
  ~~ (\WTE{\Gamma;\Gamma_I;\Gamma_P}{C_i}{s_{p_i}})_{i=1\ldots  n}
}
  {\WF{E;\Ind{\Gamma}{p}{\Gamma_I}{\Gamma_C}}{\Gamma}}}
provided that the following side conditions hold:
\begin{itemize}
\item $k>0$, $I_j$, $c_i$ are different names for $j=1\ldots  k$ and $i=1\ldots  n$,
\item $p$ is the number of parameters of \NInd{\Gamma}{\Gamma_I}{\Gamma_C}
  and $\Gamma_P$ is the context of parameters, 
\item for $j=1\ldots  k$ we have $A_j$ is an arity of sort $s_j$ and $I_j
  \notin \Gamma \cup E$,
\item for $i=1\ldots  n$ we have $C_i$ is a type of constructor of
  $I_{p_i}$ which satisfies the positivity condition for $I_1 \ldots  I_k$
  and $c_i \notin \Gamma \cup E$.
\end{itemize}
\end{description}
One can remark that there is a constraint between the sort of the
arity of the inductive type and the sort of the type of its
constructors which will always be satisfied for the impredicative sort
(\Prop) but may fail to define inductive definition 
on sort \Set{} and generate constraints between universes for
inductive definitions in the {\Type} hierarchy.

\paragraph{Examples.}
It is well known that existential quantifier can be encoded as an
inductive definition.
The following declaration introduces the second-order existential
quantifier $\exists X.P(X)$.
\begin{coq_example*}
Inductive exProp (P:Prop->Prop) : Prop 
  := exP_intro : forall X:Prop, P X -> exProp P.
\end{coq_example*}
The same definition on \Set{} is not allowed and fails~:
\begin{coq_eval}
(********** The following is not correct and should produce **********)
(*** Error: Large non-propositional inductive types must be in Type***)
\end{coq_eval}
\begin{coq_example}
Inductive exSet (P:Set->Prop) : Set 
  := exS_intro : forall X:Set, P X -> exSet P.
\end{coq_example}
It is possible to declare the same inductive definition in the
universe \Type. 
The \texttt{exType} inductive definition has type  $(\Type_i \ra\Prop)\ra
\Type_j$ with the constraint that the parameter \texttt{X} of \texttt{exT\_intro} has type $\Type_k$ with $k<j$ and $k\leq i$.
\begin{coq_example*}
Inductive exType (P:Type->Prop) : Type
  := exT_intro : forall X:Type, P X -> exType P.
\end{coq_example*}
%We shall assume for the following definitions that, if necessary, we
%annotated the type of constructors such that we know if the argument
%is recursive or not.  We shall write the type $(x:_R T)C$ if it is 
%a recursive argument and $(x:_P T)C$ if the argument is not recursive.

\paragraph[Sort-polymorphism of inductive families.]{Sort-polymorphism of inductive families.\index{Sort-polymorphism of inductive families}}

From {\Coq} version 8.1, inductive families declared in {\Type} are
polymorphic over their arguments in {\Type}.

If $A$ is an arity and $s$ a sort, we write $A_{/s}$ for the arity
obtained from $A$ by replacing its sort with $s$. Especially, if $A$
is well-typed in some environment and context, then $A_{/s}$ is typable
by typability of all products in the Calculus of Inductive Constructions.
The following typing rule is added to the theory.

\begin{description}
\item[Ind-Family] Let $\Gamma_P$ be a context of parameters
$[p_1:P_1;\ldots;p_{m'}:P_{m'}]$ and $m\leq m'$ be the length of the
initial prefix of parameters that occur unchanged in the recursive
occurrences of the constructor types. Assume that $\Gamma_I$ is
$[I_1:\forall \Gamma_P,A_1;\ldots;I_k:\forall \Gamma_P,A_k]$ and
$\Gamma_C$ is $[c_1:\forall \Gamma_P,C_1;\ldots;c_n:\forall
\Gamma_P,C_n]$.

Let $q_1$, \ldots, $q_r$, with $0\leq r\leq m$, be a possibly partial
instantiation of the parameters in $\Gamma_P$. We have:

\inference{\frac
{\left\{\begin{array}{l}
\Ind{\Gamma}{p}{\Gamma_I}{\Gamma_C} \in E\\
(E[\Gamma] \vdash q_s : P'_s)_{s=1\ldots r}\\
(E[\Gamma] \vdash \WTEGLECONV{P'_s}{\subst{P_s}{x_u}{q_u}_{u=1\ldots s-1}})_{s=1\ldots r}\\
1 \leq j \leq k
\end{array}
\right.}
{(I_j\,q_1\,\ldots\,q_r:\forall \Gamma^{r+1}_p, (A_j)_{/s})}
}

provided that the following side conditions hold:

\begin{itemize}
\item $\Gamma_{P'}$ is the context obtained from $\Gamma_P$ by
replacing, each $P_s$ that is an arity with the
sort of $P'_s$, as soon as $1\leq s \leq r$ (notice that
$P_s$ arity implies $P'_s$ arity since $E[\Gamma]
\vdash \WTEGLECONV{P'_s}{ \subst{P_s}{x_u}{q_u}_{u=1\ldots s-1}}$);
\item there are sorts $s_i$, for $1 \leq i \leq k$ such that, for
 $\Gamma_{I'}$ obtained from $\Gamma_I$ by changing each $A_i$ by $(A_i)_{/s_i}$,
we have $(\WTE{\Gamma;\Gamma_{I'};\Gamma_{P'}}{C_i}{s_{p_i}})_{i=1\ldots  n}$;
\item the sorts are such that all elimination are allowed (see
section~\ref{elimdep}).
\end{itemize}
\end{description}

Notice that if $I_j\,q_1\,\ldots\,q_r$ is typable using the rules {\bf
Ind-Const} and {\bf App}, then it is typable using the rule {\bf
Ind-Family}. Conversely, the extended theory is not stronger than the
theory without {\bf Ind-Family}. We get an equiconsistency result by
mapping each $\Ind{\Gamma}{p}{\Gamma_I}{\Gamma_C}$ occurring into a
given derivation into as many fresh inductive types and constructors
as the number of different (partial) replacements of sorts, needed for
this derivation, in the parameters that are arities. That is, the
changes in the types of each partial instance $q_1\,\ldots\,q_r$ can
be characterized by the ordered sets of arity sorts among the types of
parameters, and to each signature is associated a new inductive
definition with fresh names. Conversion is preserved as any (partial)
instance $I_j\,q_1\,\ldots\,q_r$ or $C_i\,q_1\,\ldots\,q_r$ is mapped
to the names chosen in the specific instance of
$\Ind{\Gamma}{p}{\Gamma_I}{\Gamma_C}$.

\newcommand{\Single}{\mbox{\textsf{Set}}}

In practice, the rule is used by {\Coq} only with in case the
inductive type is declared with an arity of a sort in the $\Type$
hierarchy, and, then, the polymorphism is over the parameters whose
type is an arity in the {\Type} hierarchy. The sort $s_j$ are then
chosen canonically so that each $s_j$ is minimal with respect to the
hierarchy ${\Prop_u}\subset{\Set_p}\subset\Type$ where $\Set_p$ is
predicative {\Set}, and ${\Prop_u}$ is the sort of small singleton
inductive types (i.e. of inductive types with one single constructor
and that contains either proofs or inhabitants of singleton types
only). More precisely, a small singleton inductive family is set in
{\Prop}, a small non singleton inductive family is set in {\Set} (even
in case {\Set} is impredicative -- see section~\ref{impredicativity}),
and otherwise in the {\Type} hierarchy.
% TODO: clarify the case of a partial application ??

Note that the side-condition about allowed elimination sorts in the
rule~{\bf Ind-Family} is just to avoid to recompute the allowed
elimination sorts at each instance of a pattern-matching (see
section~\ref{elimdep}).

As an example, let us consider the following definition:
\begin{coq_example*}
Inductive option (A:Type) : Type := 
| None : option A 
| Some : A -> option A.
\end{coq_example*}

As the definition is set in the {\Type} hierarchy, it is used
polymorphically over its parameters whose types are arities of a sort
in the {\Type} hierarchy. Here, the parameter $A$ has this property,
hence, if \texttt{option} is applied to a type in {\Set}, the result is
in {\Set}. Note that if \texttt{option} is applied to a type in {\Prop},
then, the result is not set in \texttt{Prop} but in \texttt{Set}
still. This is because \texttt{option} is not a singleton type (see
section~\ref{singleton}) and it would loose the elimination to {\Set} and
{\Type} if set in {\Prop}.

\begin{coq_example}
Check (fun A:Set => option A).
Check (fun A:Prop => option A).
\end{coq_example}

Here is another example.

\begin{coq_example*}
Inductive prod (A B:Type) : Type := pair : A -> B -> prod A B.
\end{coq_example*}

As \texttt{prod} is a singleton type, it will be in {\Prop} if applied
twice to propositions, in {\Set} if applied twice to at least one type
in {\Set} and none in {\Type}, and in {\Type} otherwise. In all cases,
the three kind of eliminations schemes are allowed.

\begin{coq_example}
Check (fun A:Set => prod A).
Check (fun A:Prop => prod A A).
Check (fun (A:Prop) (B:Set) => prod A B).
Check (fun (A:Type) (B:Prop) => prod A B).
\end{coq_example}

\subsection{Destructors}
The specification of inductive definitions with arities and
constructors is quite natural.  But we still have to say how to use an
object in an inductive type.

This problem is rather delicate. There are actually several different
ways to do that. Some of them are logically equivalent but not always
equivalent from the computational point of view or from the user point
of view.

From the computational point of view, we want to be able to define a
function whose domain is an inductively defined type by using a
combination of case analysis over the possible constructors of the
object and recursion.

Because we need to keep a consistent theory and also we prefer to keep
a strongly normalizing reduction, we cannot accept any sort of
recursion (even terminating). So the basic idea is to restrict
ourselves to primitive recursive functions and functionals.

For instance, assuming a parameter $A:\Set$ exists in the context, we
want to build a function \length\ of type $\ListA\ra \nat$ which
computes the length of the list, so such that $(\length~(\Nil~A)) = \nO$
and $(\length~(\cons~A~a~l)) = (\nS~(\length~l))$.  We want these
equalities to be recognized implicitly and taken into account in the
conversion rule.

From the logical point of view, we have built a type family by giving
a set of constructors.  We want to capture the fact that we do not
have any other way to build an object in this type. So when trying to
prove a property $(P~m)$ for $m$ in an inductive definition it is
enough to enumerate all the cases where $m$ starts with a different
constructor.

In case the inductive definition is effectively a recursive one, we
want to capture the extra property that we have built the smallest
fixed point of this recursive equation.  This says that we are only
manipulating finite objects. This analysis provides induction
principles.

For instance, in order to prove $\forall l:\ListA,(\LengthA~l~(\length~l))$
it is enough to prove:

\noindent $(\LengthA~(\Nil~A)~(\length~(\Nil~A)))$ and

\smallskip
$\forall a:A, \forall l:\ListA, (\LengthA~l~(\length~l)) \ra
(\LengthA~(\cons~A~a~l)~(\length~(\cons~A~a~l)))$.
\smallskip

\noindent which given the conversion equalities satisfied by \length\ is the
same as proving:
$(\LengthA~(\Nil~A)~\nO)$ and $\forall a:A, \forall l:\ListA, 
(\LengthA~l~(\length~l)) \ra
(\LengthA~(\cons~A~a~l)~(\nS~(\length~l)))$.

One conceptually simple way to do that, following the basic scheme
proposed by Martin-L\"of in his Intuitionistic Type Theory, is to
introduce for each inductive definition an elimination operator. At
the logical level it is a proof of the usual induction principle and
at the computational level it implements a generic operator for doing
primitive recursion over the structure.

But this operator is rather tedious to implement and use. We choose in
this version of Coq to factorize the operator for primitive recursion
into two more primitive operations as was first suggested by Th. Coquand
in~\cite{Coq92}.  One is the definition by pattern-matching. The second one is a definition by guarded fixpoints. 

\subsubsection[The {\tt match\ldots with \ldots end} construction.]{The {\tt match\ldots with \ldots end} construction.\label{Caseexpr}
\index{match@{\tt match\ldots with\ldots end}}}

The basic idea of this destructor operation is that we have an object
$m$ in an inductive type $I$ and we want to prove a property $(P~m)$
which in general depends on $m$. For this, it is enough to prove the
property for $m = (c_i~u_1\ldots  u_{p_i})$ for each constructor of $I$.

The \Coq{} term for this proof will be written~:
\[\kw{match}~m~\kw{with}~ (c_1~x_{11}~...~x_{1p_1}) \Ra f_1 ~|~\ldots~|~
  (c_n~x_{n1}...x_{np_n}) \Ra f_n~ \kw{end}\]
In this expression, if
$m$ is a term built from a constructor $(c_i~u_1\ldots u_{p_i})$ then
the expression will behave as it is specified with $i$-th branch and
will reduce to $f_i$ where the $x_{i1}$\ldots $x_{ip_i}$ are replaced
by the $u_1\ldots u_p$ according to the $\iota$-reduction.

Actually, for type-checking a \kw{match\ldots with\ldots end}
expression we also need to know the predicate $P$ to be proved by case
analysis. In the general case where $I$ is an inductively defined
$n$-ary relation, $P$ is a $n+1$-ary relation: the $n$ first arguments
correspond to the arguments of $I$ (parameters excluded), and the last
one corresponds to object $m$. \Coq{} can sometimes infer this
predicate but sometimes not. The concrete syntax for describing this
predicate uses the \kw{as\ldots in\ldots return} construction. For
instance, let us assume that $I$ is an unary predicate with one
parameter. The predicate is made explicit using the syntax~:
\[\kw{match}~m~\kw{as}~ x~ \kw{in}~ I~\verb!_!~a~ \kw{return}~ (P~ x)
 ~\kw{with}~ (c_1~x_{11}~...~x_{1p_1}) \Ra f_1 ~|~\ldots~|~
  (c_n~x_{n1}...x_{np_n}) \Ra f_n \kw{end}\]
The \kw{as} part can be omitted if either the result type does not
depend on $m$ (non-dependent elimination) or $m$ is a variable (in
this case, the result type can depend on $m$). The \kw{in} part can be
omitted if the result type does not depend on the arguments of
$I$. Note that the arguments of $I$ corresponding to parameters
\emph{must} be \verb!_!, because the result type is not generalized to
all possible values of the parameters. The expression after \kw{in}
must be seen as an \emph{inductive type pattern}. As a final remark,
expansion of implicit arguments and notations apply to this pattern.

For the purpose of presenting the inference rules, we use a more
compact notation~:
\[ \Case{(\lb a x \mto P)}{m}{ \lb x_{11}~...~x_{1p_1} \mto f_1 ~|~\ldots~|~
  \lb x_{n1}...x_{np_n} \mto f_n}\]

%% CP 06/06 Obsolete avec la nouvelle syntaxe et incompatible avec la
%% presentation theorique qui suit
% \paragraph{Non-dependent elimination.}
%
% When defining a function of codomain $C$ by case analysis over an
% object in an inductive type $I$, we build an object of type $I
% \ra C$. The minimality principle on an inductively defined logical
% predicate $I$ of type $A \ra \Prop$ is often used to prove a property
% $\forall x:A,(I~x)\ra (C~x)$.  These are particular cases of the dependent
% principle that we stated before with a predicate which does not depend
% explicitly on the object in the inductive definition.

% For instance, a function testing whether a list is empty 
% can be
% defined as:
% \[\kw{fun} l:\ListA \Ra \kw{match}~l~\kw{with}~ \Nil \Ra \true~
% |~(\cons~a~m) \Ra \false \kw{end}\]
% represented by
% \[\lb~l:\ListA \mto\Case{\bool}{l}{\true~ |~ \lb a~m,~\false}\]
%\noindent {\bf Remark. } 

% In the system \Coq\ the expression above, can be
% written without mentioning
% the dummy abstraction:
% \Case{\bool}{l}{\Nil~ \mbox{\tt =>}~\true~ |~ (\cons~a~m)~
%  \mbox{\tt =>}~ \false}

\paragraph[Allowed elimination sorts.]{Allowed elimination sorts.\index{Elimination sorts}}

An important question for building the typing rule for \kw{match} is
what can be the type of $P$ with respect to the type of the inductive
definitions.

We define now a relation \compat{I:A}{B} between an inductive
definition $I$ of type $A$ and an arity $B$. This relation states that
an object in the inductive definition $I$ can be eliminated for
proving a property $P$ of type $B$.

The case of inductive definitions in sorts \Set\ or \Type{} is simple.
There is no restriction on the sort of the predicate to be
eliminated. 

\paragraph{Notations.}
The \compat{I:A}{B} is defined as the smallest relation satisfying the
following rules:
We write \compat{I}{B} for \compat{I:A}{B} where $A$ is the type of
$I$.

\begin{description}
\item[Prod] \inference{\frac{\compat{(I~x):A'}{B'}}
                      {\compat{I:(x:A)A'}{(x:A)B'}}}
\item[\Set \& \Type] \inference{\frac{
    s_1 \in \{\Set,\Type(j)\}, 
    s_2 \in \Sort}{\compat{I:s_1}{I\ra s_2}}}
\end{description}

The case of Inductive definitions of sort \Prop{} is a bit more
complicated, because of our interpretation of this sort. The only
harmless allowed elimination, is the one when predicate $P$ is also of
sort \Prop.
\begin{description}
\item[\Prop] \inference{\compat{I:\Prop}{I\ra\Prop}}
\end{description}
\Prop{} is the type of logical propositions, the proofs of properties
$P$ in \Prop{} could not be used for computation and are consequently
ignored by the extraction mechanism.
Assume $A$ and $B$ are two propositions, and the logical disjunction
$A\vee B$ is defined inductively by~:
\begin{coq_example*}
Inductive or (A B:Prop) : Prop :=
  lintro : A -> or A B | rintro : B -> or A B.
\end{coq_example*}
The following definition which computes a boolean value by case over
the proof of \texttt{or A B} is not accepted~:
\begin{coq_eval}
(***************************************************************)
(*** This example should fail with ``Incorrect elimination'' ***)
\end{coq_eval}
\begin{coq_example}
Definition choice (A B: Prop) (x:or A B) := 
   match x with lintro a => true | rintro b => false end.
\end{coq_example}
From the computational point of view, the structure of the proof of
\texttt{(or A B)} in this term is needed for computing the boolean
value.

In general, if $I$ has type \Prop\ then $P$ cannot have type $I\ra
\Set$, because it will mean to build an informative proof of type
$(P~m)$ doing a case analysis over a non-computational object that
will disappear in the extracted program.  But the other way is safe
with respect to our interpretation we can have $I$ a computational
object and $P$ a non-computational one, it just corresponds to proving
a logical property of a computational object.

% Also if $I$ is in one of the sorts \{\Prop, \Set\}, one cannot in
% general allow an elimination over a bigger sort such as \Type.  But
% this operation is safe whenever $I$ is a {\em small inductive} type,
% which means that all the types of constructors of
% $I$ are small with the following definition:\\
% $(I~t_1\ldots t_s)$ is a {\em small type of constructor} and
% $\forall~x:T,C$ is a small type of constructor if $C$ is and if $T$
% has type \Prop\ or \Set.  \index{Small inductive type}

% We call this particular elimination which gives the possibility to
% compute a type by induction on the structure of a term, a {\em strong
%   elimination}\index{Strong elimination}.

In the same spirit, elimination on $P$ of type $I\ra
\Type$ cannot be allowed because it trivially implies the elimination
on $P$ of type $I\ra \Set$ by cumulativity. It also implies that there
is two proofs of the same property which are provably different,
contradicting the proof-irrelevance property which is sometimes a
useful axiom~:
\begin{coq_example}
Axiom proof_irrelevance : forall (P : Prop) (x y : P), x=y.
\end{coq_example}
\begin{coq_eval}
Reset proof_irrelevance.
\end{coq_eval}
The elimination of an inductive definition of type \Prop\ on a
predicate $P$ of type $I\ra \Type$ leads to a paradox when applied to 
impredicative inductive definition like the second-order existential
quantifier \texttt{exProp} defined above, because it give access to
the two projections on this type.

%\paragraph{Warning: strong elimination}
%\index{Elimination!Strong elimination}
%In previous versions of Coq, for a small inductive definition, only the
%non-informative strong elimination on \Type\ was allowed, because
%strong elimination on \Typeset\ was not compatible with the current
%extraction procedure. In this version, strong elimination on \Typeset\
%is accepted but a dummy element is extracted from it and may generate
%problems if extracted terms are explicitly used such as in the 
%{\tt Program} tactic or when extracting ML programs.

\paragraph[Empty and singleton elimination]{Empty and singleton elimination\label{singleton}
\index{Elimination!Singleton elimination}
\index{Elimination!Empty elimination}}

There are special inductive definitions in \Prop\ for which more
eliminations are allowed. 
\begin{description}
\item[\Prop-extended] 
\inference{
   \frac{I \mbox{~is an empty or singleton
       definition}~~~s \in \Sort}{\compat{I:\Prop}{I\ra s}}
}
\end{description}

% A {\em singleton definition} has always an informative content,
% even if it is a proposition.

A {\em singleton
definition} has only one constructor and all the arguments of this
constructor have type \Prop. In that case, there is a canonical
way to interpret the informative extraction on an object in that type,
such that the elimination on any sort $s$ is legal.  Typical examples are
the conjunction of non-informative propositions and the equality. 
If there is an hypothesis $h:a=b$ in the context, it can be used for
rewriting not only in logical propositions but also in any type.
% In that case, the term \verb!eq_rec! which was defined as an axiom, is
% now a term of the calculus.
\begin{coq_example}
Print eq_rec.
Extraction eq_rec.
\end{coq_example}
An empty definition has no constructors, in that case also,
elimination on any sort is allowed.

\paragraph{Type of branches.}
Let $c$ be a term of type $C$, we assume $C$ is a type of constructor
for an inductive definition $I$. Let $P$ be a term that represents the
property to be proved.
We assume $r$ is the number of parameters.

We define a new type \CI{c:C}{P} which represents the type of the
branch corresponding to the $c:C$ constructor.
\[
\begin{array}{ll}
\CI{c:(I_i~p_1\ldots p_r\ t_1 \ldots t_p)}{P} &\equiv (P~t_1\ldots ~t_p~c) \\[2mm]
\CI{c:\forall~x:T,C}{P} &\equiv \forall~x:T,\CI{(c~x):C}{P} 
\end{array}
\]
We write \CI{c}{P} for \CI{c:C}{P} with $C$ the type of $c$.

\paragraph{Examples.}
For $\ListA$ the type of $P$ will be $\ListA\ra s$ for $s \in \Sort$. \\ 
$ \CI{(\cons~A)}{P} \equiv
\forall a:A, \forall l:\ListA,(P~(\cons~A~a~l))$.

For $\LengthA$, the type of $P$ will be
$\forall l:\ListA,\forall n:\nat, (\LengthA~l~n)\ra \Prop$ and the expression
\CI{(\LCons~A)}{P} is defined as:\\ 
$\forall a:A, \forall l:\ListA, \forall n:\nat, \forall
h:(\LengthA~l~n), (P~(\cons~A~a~l)~(\nS~n)~(\LCons~A~a~l~n~l))$.\\ 
If $P$ does not depend on its third argument, we find the more natural
expression:\\ 
$\forall a:A, \forall l:\ListA, \forall n:\nat,
(\LengthA~l~n)\ra(P~(\cons~A~a~l)~(\nS~n))$.

\paragraph{Typing rule.}

Our very general destructor for inductive definition enjoys the
following typing rule
% , where we write 
% \[
% \Case{P}{c}{[x_{11}:T_{11}]\ldots[x_{1p_1}:T_{1p_1}]g_1\ldots
%   [x_{n1}:T_{n1}]\ldots[x_{np_n}:T_{np_n}]g_n}
% \]
% for 
% \[
% \Case{P}{c}{(c_1~x_{11}~...~x_{1p_1}) \Ra g_1 ~|~\ldots~|~
% (c_n~x_{n1}...x_{np_n}) \Ra g_n }
% \]

\begin{description}
\item[match] \label{elimdep} \index{Typing rules!match}
\inference{
\frac{\WTEG{c}{(I~q_1\ldots  q_r~t_1\ldots  t_s)}~~
  \WTEG{P}{B}~~\compat{(I~q_1\ldots  q_r)}{B}
 ~~
(\WTEG{f_i}{\CI{(c_{p_i}~q_1\ldots  q_r)}{P}})_{i=1\ldots  l}}
{\WTEG{\Case{P}{c}{f_1|\ldots  |f_l}}{(P\ t_1\ldots  t_s\ c)}}}%\\[3mm]

provided $I$ is an inductive type in a declaration
\Ind{\Delta}{r}{\Gamma_I}{\Gamma_C} with 
$\Gamma_C = [c_1:C_1;\ldots;c_n:C_n]$ and $c_{p_1}\ldots  c_{p_l}$ are the
only constructors of $I$.
\end{description}

\paragraph{Example.}
For \List\ and \Length\ the typing rules for the {\tt match} expression
are (writing just $t:M$ instead of \WTEG{t}{M}, the environment and
context being the same in all the judgments).

\[\frac{l:\ListA~~P:\ListA\ra s~~~f_1:(P~(\Nil~A))~~
      f_2:\forall a:A, \forall l:\ListA, (P~(\cons~A~a~l))}
    {\Case{P}{l}{f_1~|~f_2}:(P~l)}\]

\[\frac{
     \begin{array}[b]{@{}c@{}} 
H:(\LengthA~L~N) \\ P:\forall l:\ListA, \forall n:\nat, (\LengthA~l~n)\ra
  \Prop\\
 f_1:(P~(\Nil~A)~\nO~\LNil) \\
   f_2:\forall a:A, \forall l:\ListA, \forall n:\nat, \forall
  h:(\LengthA~l~n), (P~(\cons~A~a~n)~(\nS~n)~(\LCons~A~a~l~n~h)) 
      \end{array}}
    {\Case{P}{H}{f_1~|~f_2}:(P~L~N~H)}\]

\paragraph[Definition of $\iota$-reduction.]{Definition of $\iota$-reduction.\label{iotared}
\index{iota-reduction@$\iota$-reduction}}
We still have to define the $\iota$-reduction in the general case.

A $\iota$-redex is a term of the following form:
\[\Case{P}{(c_{p_i}~q_1\ldots  q_r~a_1\ldots  a_m)}{f_1|\ldots |
    f_l}\]
with $c_{p_i}$ the $i$-th constructor of the inductive type $I$ with $r$
parameters.

The $\iota$-contraction of this term is $(f_i~a_1\ldots a_m)$ leading
to the general reduction rule:
\[ \Case{P}{(c_{p_i}~q_1\ldots  q_r~a_1\ldots  a_m)}{f_1|\ldots |
    f_n} \triangleright_{\iota} (f_i~a_1\ldots a_m) \]

\subsection[Fixpoint definitions]{Fixpoint definitions\label{Fix-term} \index{Fix@{\tt Fix}}}
The second operator for elimination is fixpoint definition. 
This fixpoint may involve several mutually recursive definitions.
The basic concrete syntax for a recursive set of mutually recursive 
declarations is (with $\Gamma_i$ contexts)~: 
\[\kw{fix}~f_1 (\Gamma_1) :A_1:=t_1~\kw{with} \ldots \kw{with}~ f_n
(\Gamma_n) :A_n:=t_n\]
The terms are obtained by projections from this set of declarations
and are written 
\[\kw{fix}~f_1 (\Gamma_1) :A_1:=t_1~\kw{with} \ldots \kw{with}~ f_n
(\Gamma_n) :A_n:=t_n~\kw{for}~f_i\]
In the inference rules, we represent such a
term by 
\[\Fix{f_i}{f_1:A_1':=t_1' \ldots f_n:A_n':=t_n'}\]
with $t_i'$ (resp. $A_i'$) representing the term $t_i$ abstracted
(resp. generalized) with
respect to the bindings in the context $\Gamma_i$, namely
$t_i'=\lb \Gamma_i \mto t_i$ and $A_i'=\forall \Gamma_i, A_i$.

\subsubsection{Typing rule}
The typing rule is the expected one for a fixpoint.

\begin{description}
\item[Fix] \index{Typing rules!Fix}
\inference{\frac{(\WTEG{A_i}{s_i})_{i=1\ldots n}~~~~
            (\WTE{\Gamma,f_1:A_1,\ldots,f_n:A_n}{t_i}{A_i})_{i=1\ldots n}}
         {\WTEG{\Fix{f_i}{f_1:A_1:=t_1 \ldots f_n:A_n:=t_n}}{A_i}}}
\end{description}

Any fixpoint definition cannot be accepted because non-normalizing terms
will lead to proofs of absurdity.

The basic scheme of recursion that should be allowed is the one needed for 
defining primitive
recursive functionals. In that case the fixpoint enjoys a special
syntactic restriction, namely one of the arguments belongs to an
inductive type, the function starts with a case analysis and recursive
calls are done on variables coming from patterns and representing subterms.

For instance in the case of natural numbers, a proof of the induction
principle of type 
\[\forall P:\nat\ra\Prop, (P~\nO)\ra(\forall n:\nat, (P~n)\ra(P~(\nS~n)))\ra
\forall n:\nat, (P~n)\]
can be represented by the term:
\[\begin{array}{l}
\lb P:\nat\ra\Prop\mto\lb f:(P~\nO)\mto \lb g:(\forall n:\nat,
(P~n)\ra(P~(\nS~n))) \mto\\
\Fix{h}{h:\forall n:\nat, (P~n):=\lb n:\nat\mto \Case{P}{n}{f~|~\lb
    p:\nat\mto (g~p~(h~p))}}
\end{array}
\]

Before accepting a fixpoint definition as being correctly typed, we
check that the definition is ``guarded''. A precise analysis of this
notion can be found in~\cite{Gim94}.

The first stage is to precise on which argument the fixpoint will be
decreasing. The type of this argument should be an inductive
definition.

For doing this the syntax of fixpoints is extended and becomes 
 \[\Fix{f_i}{f_1/k_1:A_1:=t_1 \ldots f_n/k_n:A_n:=t_n}\]
where $k_i$ are positive integers.
Each $A_i$ should be a type (reducible to a term) starting with at least
$k_i$ products $\forall y_1:B_1,\ldots \forall y_{k_i}:B_{k_i}, A'_i$ 
and $B_{k_i}$
being an instance of an inductive definition.

Now in the definition $t_i$, if $f_j$ occurs then it should be applied
to at least $k_j$ arguments and the $k_j$-th argument should be
syntactically recognized as structurally smaller than $y_{k_i}$


The definition of being structurally smaller is a bit technical.
One needs first to  define the notion of 
{\em recursive arguments of a constructor}\index{Recursive arguments}.
For an inductive definition \Ind{\Gamma}{r}{\Gamma_I}{\Gamma_C},
the type of a constructor $c$ has the form
$\forall p_1:P_1,\ldots \forall p_r:P_r, 
\forall x_1:T_1, \ldots \forall x_r:T_r, (I_j~p_1\ldots 
p_r~t_1\ldots  t_s)$ the recursive arguments will correspond to $T_i$ in
which one of the $I_l$ occurs.


The main rules for being structurally smaller are the following:\\
Given a variable $y$ of type an inductive
definition in a declaration 
\Ind{\Gamma}{r}{\Gamma_I}{\Gamma_C}
where $\Gamma_I$ is $[I_1:A_1;\ldots;I_k:A_k]$, and $\Gamma_C$ is
  $[c_1:C_1;\ldots;c_n:C_n]$.
The terms structurally smaller than $y$ are:
\begin{itemize}
\item $(t~u), \lb x:u \mto t$ when $t$ is structurally smaller than $y$ .
\item \Case{P}{c}{f_1\ldots f_n} when each $f_i$ is structurally
  smaller than $y$. \\
  If $c$ is $y$ or is structurally smaller than $y$, its type is an inductive
  definition $I_p$ part of the inductive
  declaration corresponding to $y$. 
  Each $f_i$ corresponds to a type of constructor $C_q \equiv
  \forall p_1:P_1,\ldots,\forall p_r:P_r, \forall y_1:B_1, \ldots \forall y_k:B_k, (I~a_1\ldots a_k)$ 
  and can consequently be
  written $\lb y_1:B'_1\mto \ldots \lb y_k:B'_k\mto g_i$.
  ($B'_i$ is obtained from $B_i$ by substituting parameters variables)
  the variables $y_j$ occurring
  in $g_i$ corresponding to recursive arguments $B_i$ (the ones in
  which one of the $I_l$ occurs) are structurally smaller than $y$.
\end{itemize}
The following definitions are correct, we enter them using the
{\tt Fixpoint} command as described in section~\ref{Fixpoint} and show
the internal representation.
\begin{coq_example}
Fixpoint plus (n m:nat) {struct n} : nat :=
  match n with
  | O => m
  | S p => S (plus p m)
  end.
Print plus.
Fixpoint lgth (A:Set) (l:list A) {struct l} : nat :=
  match l with
  | nil => O
  | cons a l' => S (lgth A l')
  end.
Print lgth.
Fixpoint sizet (t:tree) : nat := let (f) := t in S (sizef f)
 with sizef (f:forest) : nat :=
  match f with
  | emptyf => O
  | consf t f => plus (sizet t) (sizef f)
  end.
Print sizet.
\end{coq_example}


\subsubsection[Reduction rule]{Reduction rule\index{iota-reduction@$\iota$-reduction}}
Let $F$ be the set of declarations: $f_1/k_1:A_1:=t_1 \ldots
f_n/k_n:A_n:=t_n$.
The reduction for fixpoints is:
\[ (\Fix{f_i}{F}~a_1\ldots
a_{k_i}) \triangleright_{\iota} \substs{t_i}{f_k}{\Fix{f_k}{F}}{k=1\ldots n}\]
when $a_{k_i}$ starts with a constructor.
This last restriction is needed in order to keep strong normalization
and corresponds to the reduction for primitive recursive operators.

We can illustrate this behavior on examples.
\begin{coq_example}
Goal forall n m:nat, plus (S n) m = S (plus n m).
reflexivity.
Abort.
Goal forall f:forest, sizet (node f) = S (sizef f).
reflexivity.
Abort.
\end{coq_example}
But assuming the definition of a son function from \tree\ to \forest:
\begin{coq_example}
Definition sont (t:tree) : forest 
   := let (f) := t in f.
\end{coq_example}
The following is not a conversion but can be proved after a case analysis.
\begin{coq_eval}
(******************************************************************)
(** Error: Impossible to unify ....                              **)
\end{coq_eval}
\begin{coq_example}
Goal forall t:tree, sizet t = S (sizef (sont t)).
reflexivity. (** this one fails **)
destruct t.
reflexivity.
\end{coq_example}
\begin{coq_eval}
Abort.
\end{coq_eval}

% La disparition de Program devrait rendre la construction Match obsolete
% \subsubsection{The {\tt Match \ldots with \ldots end} expression}
% \label{Matchexpr}
% %\paragraph{A unary {\tt Match\ldots with \ldots end}.}
% \index{Match...with...end@{\tt Match \ldots with \ldots end}}
% The {\tt Match} operator which was a primitive notion in older
% presentations of the Calculus of Inductive Constructions is now just a
% macro definition which generates the good combination of {\tt Case}
% and {\tt Fix} operators in order to generate an operator for primitive
% recursive definitions. It always considers an inductive definition as 
% a single inductive definition.

% The following examples illustrates this feature.
% \begin{coq_example}
% Definition nat_pr : (C:Set)C->(nat->C->C)->nat->C 
%   :=[C,x,g,n]Match n with x g end.
% Print nat_pr.
% \end{coq_example}
% \begin{coq_example}
% Definition forest_pr 
%   : (C:Set)C->(tree->forest->C->C)->forest->C
%   := [C,x,g,n]Match n with x g end.
% \end{coq_example}

% Cet exemple faisait error (HH le 12/12/96), j'ai change pour une
% version plus simple
%\begin{coq_example}
%Definition forest_pr 
%  : (P:forest->Set)(P emptyf)->((t:tree)(f:forest)(P f)->(P (consf t f)))
%    ->(f:forest)(P f)
%  := [C,x,g,n]Match n with x g end.
%\end{coq_example}

\subsubsection{Mutual induction}

The principles of mutual induction can be automatically generated 
using the {\tt Scheme} command described in section~\ref{Scheme}.

\section{Coinductive types}
The implementation contains also coinductive definitions, which are
types inhabited by infinite objects. 
More information on coinductive definitions can be found
in~\cite{Gimenez95b,Gim98,GimCas05}.
%They are described in chapter~\ref{Coinductives}.

\section[\iCIC : the Calculus of Inductive Construction with
  impredicative \Set]{\iCIC : the Calculus of Inductive Construction with
  impredicative \Set\label{impredicativity}}

\Coq{} can be used as a type-checker for \iCIC{}, the original 
Calculus of Inductive Constructions with an impredicative sort \Set{}
by using the compiler option \texttt{-impredicative-set}.

For example, using the ordinary \texttt{coqtop} command, the following
is rejected.
\begin{coq_eval}
(** This example should fail *******************************
    Error: The term forall X:Set, X -> X has type Type
    while it is expected to have type Set
***)
\end{coq_eval}
\begin{coq_example}
Definition id: Set := forall X:Set,X->X.
\end{coq_example}
while it will type-check, if one use instead the \texttt{coqtop
  -impredicative-set} command.

The major change in the theory concerns the rule for product formation
in the sort \Set, which is extended to a domain in any sort~:
\begin{description}
\item [Prod]  \index{Typing rules!Prod (impredicative Set)}
\inference{\frac{\WTEG{T}{s}~~~~s \in \Sort~~~~~~
    \WTE{\Gamma::(x:T)}{U}{\Set}}
      { \WTEG{\forall~x:T,U}{\Set}}} 
\end{description}
This extension has consequences on the inductive definitions which are
allowed. 
In the impredicative system, one can build so-called {\em large inductive
  definitions} like the example of second-order existential
quantifier (\texttt{exSet}).

There should be restrictions on the eliminations which can be
performed on such definitions. The eliminations rules in the
impredicative system for sort \Set{} become~:
\begin{description}
\item[\Set] \inference{\frac{s \in
      \{\Prop, \Set\}}{\compat{I:\Set}{I\ra s}}
~~~~\frac{I \mbox{~is a small inductive definition}~~~~s \in
      \{\Type(i)\}}
         {\compat{I:\Set}{I\ra s}}}
\end{description}
     


% $Id$ 

%%% Local Variables: 
%%% mode: latex
%%% TeX-master: "Reference-Manual"
%%% End: 


