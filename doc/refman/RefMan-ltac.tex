\chapter[The tactic language]{The tactic language\label{TacticLanguage}}

%\geometry{a4paper,body={5in,8in}}

This chapter gives a compact documentation of Ltac, the tactic
language available in {\Coq}. We start by giving the syntax, and next,
we present the informal semantics. If you want to know more regarding
this language and especially about its foundations, you can refer
to~\cite{Del00}. Chapter~\ref{Tactics-examples} is devoted to giving
examples of use of this language on small but also with non-trivial
problems.


\section{Syntax}

\def\tacexpr{\textrm{\textsl{expr}}}
\def\tacexprlow{\textrm{\textsl{tacexpr$_1$}}}
\def\tacexprinf{\textrm{\textsl{tacexpr$_2$}}}
\def\tacexprpref{\textrm{\textsl{tacexpr$_3$}}}
\def\atom{\textrm{\textsl{atom}}}
%%\def\recclause{\textrm{\textsl{rec\_clause}}}
\def\letclause{\textrm{\textsl{let\_clause}}}
\def\matchrule{\textrm{\textsl{match\_rule}}}
\def\contextrule{\textrm{\textsl{context\_rule}}}
\def\contexthyp{\textrm{\textsl{context\_hyp}}}
\def\tacarg{\nterm{tacarg}}
\def\cpattern{\nterm{cpattern}}
\def\selector{\textrm{\textsl{selector}}}
\def\toplevelselector{\textrm{\textsl{toplevel\_selector}}}

The syntax of the tactic language is given Figures~\ref{ltac}
and~\ref{ltac-aux}. See Chapter~\ref{BNF-syntax} for a description of
the BNF metasyntax used in these grammar rules. Various already
defined entries will be used in this chapter: entries
{\naturalnumber}, {\integer}, {\ident}, {\qualid}, {\term},
{\cpattern} and {\atomictac} represent respectively the natural and
integer numbers, the authorized identificators and qualified names,
{\Coq}'s terms and patterns and all the atomic tactics described in
Chapter~\ref{Tactics}. The syntax of {\cpattern} is the same as that
of terms, but it is extended with pattern matching metavariables. In
{\cpattern}, a pattern-matching metavariable is represented with the
syntax {\tt ?id} where {\tt id} is an {\ident}. The notation {\tt \_}
can also be used to denote metavariable whose instance is
irrelevant. In the notation {\tt ?id}, the identifier allows us to
keep instantiations and to make constraints whereas {\tt \_} shows
that we are not interested in what will be matched. On the right hand
side of pattern-matching clauses, the named metavariable are used
without the question mark prefix. There is also a special notation for
second-order pattern-matching problems: in an applicative pattern of
the form {\tt @?id id$_1$ \ldots id$_n$}, the variable {\tt id}
matches any complex expression with (possible) dependencies in the
variables {\tt id$_1$ \ldots id$_n$} and returns a functional term of
the form {\tt fun id$_1$ \ldots id$_n$ => {\term}}.


The main entry of the grammar is {\tacexpr}. This language is used in
proof mode but it can also be used in toplevel definitions as shown in
Figure~\ref{ltactop}.

\begin{Remarks}
\item The infix tacticals ``\dots\ {\tt ||} \dots'', ``\dots\ {\tt +}
  \dots'', and ``\dots\ {\tt ;} \dots'' are associative.

\item In {\tacarg}, there is an overlap between {\qualid} as a
direct tactic argument and {\qualid} as a particular case of
{\term}. The resolution is done by first looking for a reference of
the tactic language and if it fails, for a reference to a term. To
force the resolution as a reference of the tactic language, use the
form {\tt ltac :} {\qualid}. To force the resolution as a reference to
a term, use the syntax {\tt ({\qualid})}.

\item As shown by the figure, tactical {\tt ||} binds more than the
prefix tacticals {\tt try}, {\tt repeat}, {\tt do} and
{\tt abstract} which themselves bind more than the postfix tactical
``{\tt \dots\ ;[ \dots\ ]}'' which binds more than ``\dots\ {\tt ;}
\dots''.

For instance
\begin{quote}
{\tt try repeat \tac$_1$ ||
  \tac$_2$;\tac$_3$;[\tac$_{31}$|\dots|\tac$_{3n}$];\tac$_4$.}
\end{quote}
is understood as
\begin{quote}
{\tt (try (repeat (\tac$_1$ || \tac$_2$)));} \\
{\tt ((\tac$_3$;[\tac$_{31}$|\dots|\tac$_{3n}$]);\tac$_4$).}
\end{quote}
\end{Remarks}


\begin{figure}[htbp]
\begin{centerframe}
\begin{tabular}{lcl}
{\tacexpr} & ::= &
           {\tacexpr} {\tt ;} {\tacexpr}\\
& | & {\tt [>} \nelist{\tacexpr}{|} {\tt ]}\\
& | & {\tacexpr} {\tt ; [} \nelist{\tacexpr}{|} {\tt ]}\\
& | & {\tacexprpref}\\
\\
{\tacexprpref} & ::= &
           {\tt do} {\it (}{\naturalnumber} {\it |} {\ident}{\it )} {\tacexprpref}\\
& | & {\tt progress} {\tacexprpref}\\
& | & {\tt repeat} {\tacexprpref}\\
& | & {\tt try} {\tacexprpref}\\
& | & {\tt once} {\tacexprpref}\\
& | & {\tt exactly\_once} {\tacexprpref}\\
& | & {\tt timeout} {\it (}{\naturalnumber} {\it |} {\ident}{\it )} {\tacexprpref}\\
& | & {\tt time} \zeroone{\qstring} {\tacexprpref}\\
& | & {\tt only} {\selector} {\tt :} {\tacexprpref}\\
& | & {\tacexprinf} \\
\\
{\tacexprinf} & ::= &
           {\tacexprlow} {\tt ||} {\tacexprpref}\\
& | &      {\tacexprlow} {\tt +} {\tacexprpref}\\
& | &      {\tt tryif} {\tacexprlow} {\tt then} {\tacexprlow} {\tt else} {\tacexprlow}\\
& | & {\tacexprlow}\\
\\
{\tacexprlow} & ::= &
{\tt fun} \nelist{\name}{} {\tt =>} {\atom}\\
& | &
{\tt let} \zeroone{\tt rec} \nelist{\letclause}{\tt with} {\tt in}
{\atom}\\
& | &
{\tt match goal with} \nelist{\contextrule}{\tt |} {\tt end}\\
& | &
{\tt match reverse goal with} \nelist{\contextrule}{\tt |} {\tt end}\\
& | &
{\tt match} {\tacexpr} {\tt with} \nelist{\matchrule}{\tt |} {\tt end}\\
& | &
{\tt lazymatch goal with} \nelist{\contextrule}{\tt |} {\tt end}\\
& | &
{\tt lazymatch reverse goal with} \nelist{\contextrule}{\tt |} {\tt end}\\
& | &
{\tt lazymatch} {\tacexpr} {\tt with} \nelist{\matchrule}{\tt |} {\tt end}\\
& | &
{\tt multimatch goal with} \nelist{\contextrule}{\tt |} {\tt end}\\
& | &
{\tt multimatch reverse goal with} \nelist{\contextrule}{\tt |} {\tt end}\\
& | &
{\tt multimatch} {\tacexpr} {\tt with} \nelist{\matchrule}{\tt |} {\tt end}\\
& | & {\tt abstract} {\atom}\\
& | & {\tt abstract} {\atom} {\tt using} {\ident} \\
& | & {\tt first [} \nelist{\tacexpr}{\tt |} {\tt ]}\\
& | & {\tt solve [} \nelist{\tacexpr}{\tt |} {\tt ]}\\
& | & {\tt idtac} \sequence{\messagetoken}{}\\
& | & {\tt fail} \zeroone{\naturalnumber} \sequence{\messagetoken}{}\\
& | & {\tt gfail} \zeroone{\naturalnumber} \sequence{\messagetoken}{}\\
& | & {\tt fresh} ~|~ {\tt fresh} {\qstring}|~ {\tt fresh} {\qualid}\\
& | & {\tt context} {\ident} {\tt [} {\term} {\tt ]}\\
& | & {\tt eval} {\nterm{redexpr}} {\tt in} {\term}\\
& | & {\tt type of} {\term}\\
& | & {\tt external} {\qstring} {\qstring} \nelist{\tacarg}{}\\
& | & {\tt constr :} {\term}\\
& | & {\tt uconstr :} {\term}\\
& | & {\tt type\_term} {\term}\\
& | & {\tt numgoals} \\
& | & {\tt guard} {\it test}\\
& | & \atomictac\\
& | & {\qualid} \nelist{\tacarg}{}\\
& | & {\atom}
\end{tabular}
\end{centerframe}
\caption{Syntax of the tactic language}
\label{ltac}
\end{figure}



\begin{figure}[htbp]
\begin{centerframe}
\begin{tabular}{lcl}
{\atom} & ::= &
           {\qualid} \\
& | & ()\\
& | & {\integer}\\
& | & {\tt (} {\tacexpr} {\tt )}\\
\\
{\messagetoken}\!\!\!\!\!\! & ::= & {\qstring} ~|~ {\ident} ~|~ {\integer} \\
\\
\tacarg & ::= &
        {\qualid}\\
& $|$ & {\tt ()} \\
& $|$ & {\tt ltac :} {\atom}\\
& $|$ & {\term}\\
\\
\letclause & ::= & {\ident} \sequence{\name}{} {\tt :=} {\tacexpr}\\
\\
\contextrule & ::= &
  \nelist{\contexthyp}{\tt ,} {\tt |-}{\cpattern} {\tt =>} {\tacexpr}\\
& $|$ & {\tt |-} {\cpattern} {\tt =>} {\tacexpr}\\
& $|$ & {\tt \_ =>} {\tacexpr}\\
\\
\contexthyp & ::= & {\name} {\tt :} {\cpattern}\\
             & $|$ & {\name} {\tt :=} {\cpattern} \zeroone{{\tt :} {\cpattern}}\\
\\
\matchrule & ::= &
           {\cpattern} {\tt =>} {\tacexpr}\\
& $|$ & {\tt context} {\zeroone{\ident}} {\tt [} {\cpattern} {\tt ]}
           {\tt =>} {\tacexpr}\\
& $|$ & {\tt appcontext} {\zeroone{\ident}} {\tt [} {\cpattern} {\tt ]}
           {\tt =>} {\tacexpr}\\
& $|$ & {\tt \_ =>} {\tacexpr}\\
\\
{\it test} & ::= &
        {\integer} {\tt \,=\,} {\integer}\\
& $|$ & {\integer} {\tt \,<\,} {\integer}\\
& $|$ & {\integer} {\tt <=} {\integer}\\
& $|$ & {\integer} {\tt \,>\,} {\integer}\\
& $|$ & {\integer} {\tt >=} {\integer}\\
\\
\selector & ::= &
        [{\ident}]\\
& $|$ & {\integer}\\
& $|$ & \nelist{{\it (}{\integer} {\it |} {\integer} {\tt -} {\integer}{\it )}}
          {\tt ,}\\
\\
\toplevelselector & ::= &
        \selector\\
& $|$ & {\tt all}\\
& $|$ & {\tt par}
\end{tabular}
\end{centerframe}
\caption{Syntax of the tactic language (continued)}
\label{ltac-aux}
\end{figure}

\begin{figure}[ht]
\begin{centerframe}
\begin{tabular}{lcl}
\nterm{top} & ::= & \zeroone{\tt Local} {\tt Ltac} \nelist{\nterm{ltac\_def}} {\tt with} \\
\\
\nterm{ltac\_def} & ::= & {\ident} \sequence{\ident}{} {\tt :=}
{\tacexpr}\\
& $|$ &{\qualid} \sequence{\ident}{} {\tt ::=}{\tacexpr}
\end{tabular}
\end{centerframe}
\caption{Tactic toplevel definitions}
\label{ltactop}
\end{figure}


%%
%% Semantics
%%
\section{Semantics}
%\index[tactic]{Tacticals}
\index{Tacticals}
%\label{Tacticals}

Tactic expressions can only be applied in the context of a proof.  The
evaluation yields either a term, an integer or a tactic. Intermediary
results can be terms or integers but the final result must be a tactic
which is then applied to the focused goals.

There is a special case for {\tt match goal} expressions of which
the clauses evaluate to tactics. Such expressions can only be used as
end result of a tactic expression (never as argument of a non recursive local
definition or of an application).

The rest of this section explains the semantics of every construction
of Ltac.


%% \subsection{Values}

%% Values are given by Figure~\ref{ltacval}. All these values are tactic values,
%% i.e. to be applied to a goal, except {\tt Fun}, {\tt Rec} and $arg$ values.

%% \begin{figure}[ht]
%% \noindent{}\framebox[6in][l]
%% {\parbox{6in}
%% {\begin{center}
%% \begin{tabular}{lp{0.1in}l}
%% $vexpr$ & ::= & $vexpr$ {\tt ;} $vexpr$\\
%% & | & $vexpr$ {\tt ; [} {\it (}$vexpr$ {\tt |}{\it )}$^*$ $vexpr$ {\tt
%% ]}\\
%% & | & $vatom$\\
%% \\
%% $vatom$ & ::= & {\tt Fun} \nelist{\inputfun}{}  {\tt ->} {\tacexpr}\\
%% %& | & {\tt Rec} \recclause\\
%% & | &
%% {\tt Rec} \nelist{\recclause}{\tt And} {\tt In}
%% {\tacexpr}\\
%% & | &
%% {\tt Match Context With} {\it (}$context\_rule$ {\tt |}{\it )}$^*$
%% $context\_rule$\\
%% & | & {\tt (} $vexpr$ {\tt )}\\
%% & | & $vatom$ {\tt Orelse} $vatom$\\
%% & | & {\tt Do} {\it (}{\naturalnumber} {\it |} {\ident}{\it )} $vatom$\\
%% & | & {\tt Repeat} $vatom$\\
%% & | & {\tt Try} $vatom$\\
%% & | & {\tt First [} {\it (}$vexpr$ {\tt |}{\it )}$^*$ $vexpr$ {\tt ]}\\
%% & | & {\tt Solve [} {\it (}$vexpr$ {\tt |}{\it )}$^*$ $vexpr$ {\tt ]}\\
%% & | & {\tt Idtac}\\
%% & | & {\tt Fail}\\
%% & | & {\primitivetactic}\\
%% & | & $arg$
%% \end{tabular}
%% \end{center}}}
%% \caption{Values of ${\cal L}_{tac}$}
%% \label{ltacval}
%% \end{figure}

%% \subsection{Evaluation}

\subsubsection[Sequence]{Sequence\tacindex{;}
\index{Tacticals!;@{\tt {\tac$_1$};\tac$_2$}}}

A sequence is an expression of the following form:
\begin{quote}
{\tacexpr}$_1$ {\tt ;} {\tacexpr}$_2$
\end{quote}
The expressions {\tacexpr}$_1$ and {\tacexpr}$_2$ are evaluated
to $v_1$ and $v_2$ which have to be tactic values. The tactic $v_1$ is
then applied and $v_2$ is applied to the goals generated by the
application of $v_1$. Sequence is left-associative.

\subsubsection[Local application of tactics]{Local application of tactics\tacindex{[>\ldots$\mid$\ldots$\mid$\ldots]}\tacindex{;[\ldots$\mid$\ldots$\mid$\ldots]}\index{Tacticals![> \mid ]@{\tt {\tac$_0$};[{\tac$_1$}$\mid$\ldots$\mid$\tac$_n$]}}\index{Tacticals!; [ \mid ]@{\tt {\tac$_0$};[{\tac$_1$}$\mid$\ldots$\mid$\tac$_n$]}}}
%\tacindex{; [ | ]}
%\index{; [ | ]@{\tt ;[\ldots$\mid$\ldots$\mid$\ldots]}}

Different tactics can be applied to the different goals using the following form:
\begin{quote}
{\tt [ >} {\tacexpr}$_1$ {\tt |} $...$ {\tt |} {\tacexpr}$_n$ {\tt ]}
\end{quote}
The expressions {\tacexpr}$_i$ are evaluated to $v_i$, for $i=0,...,n$
and all have to be tactics. The $v_i$ is applied to the $i$-th goal,
for $=1,...,n$. It fails if the number of focused goals is not exactly $n$.

\begin{Variants}
  \item If no tactic is given for the $i$-th goal, it behaves as if
    the tactic {\tt idtac} were given. For instance, {\tt [~> | auto
    ]} is a shortcut for {\tt [ > idtac | auto ]}.

  \item {\tt [ >} {\tacexpr}$_1$ {\tt |} $...$ {\tt |}
    {\tacexpr}$_i$ {\tt |} {\tacexpr} {\tt ..} {\tt |}
    {\tacexpr}$_{i+1+j}$ {\tt |} $...$ {\tt |} {\tacexpr}$_n$ {\tt ]}

  In this variant, {\tt expr} is used for each goal numbered from
  $i+1$ to $i+j$ (assuming $n$ is the number of goals).

  Note that {\tt ..} is part of the syntax, while $...$ is the meta-symbol used
  to describe a list of {\tacexpr} of arbitrary length.
  goals numbered from $i+1$ to $i+j$.

  \item {\tt [ >} {\tacexpr}$_1$ {\tt |} $...$ {\tt |}
    {\tacexpr}$_i$ {\tt |} {\tt ..} {\tt |} {\tacexpr}$_{i+1+j}$ {\tt |}
    $...$ {\tt |} {\tacexpr}$_n$ {\tt ]}

  In this variant, {\tt idtac} is used for the goals numbered from
  $i+1$ to $i+j$.

  \item {\tt [ >} {\tacexpr} {\tt ..} {\tt ]}

    In this variant, the tactic {\tacexpr} is applied independently to
    each of the goals, rather than globally. In particular, if there
    are no goal, the tactic is not run at all. A tactic which
    expects multiple goals, such as {\tt swap}, would act as if a single
    goal is focused.

  \item {\tacexpr} {\tt ; [ } {\tacexpr}$_1$ {\tt |} $...$ {\tt |} {\tacexpr}$_n$ {\tt ]}

    This variant of local tactic application is paired with a
    sequence.  In this variant, $n$ must be the number of goals
    generated by the application of {\tacexpr} to each of the
    individual goals independently. All the above variants work in
    this form too. Formally, {\tacexpr} {\tt ; [} $...$ {\tt ]} is
    equivalent to
    \begin{quote}
    {\tt [ >} {\tacexpr} {\tt ; [ >} $...$ {\tt ]} {\tt ..} {\tt ]}
    \end{quote}

\end{Variants}

\subsubsection[Goal selectors]{Goal selectors\label{ltac:selector}
\tacindex{\tt :}\index{Tacticals!:@{\tt :}}}

We can restrict the application of a tactic to a subset of
the currently focused goals with:
\begin{quote}
  {\toplevelselector} {\tt :} {\tacexpr}
\end{quote}
We can also use selectors as a tactical, which allows to use them nested in
a tactic expression, by using the keyword {\tt only}:
\begin{quote}
  {\tt only} {\selector} {\tt :} {\tacexpr}
\end{quote}
When selecting several goals, the tactic {\tacexpr} is applied globally to
all selected goals.

\begin{Variants}
  \item{} [{\ident}] {\tt :} {\tacexpr}

    In this variant, {\tacexpr} is applied locally to a goal
    previously named by the user.

  \item {\num} {\tt :} {\tacexpr}

    In this variant, {\tacexpr} is applied locally to the
    {\num}-th goal.

  \item $n_1$-$m_1$, \dots, $n_k$-$m_k$ {\tt :} {\tacexpr}

    In this variant, {\tacexpr} is applied globally to the subset
    of goals described by the given ranges. You can write a single
    $n$ as a shortcut for $n$-$n$ when specifying multiple ranges.

  \item {\tt all:} {\tacexpr}

    In this variant, {\tacexpr} is applied to all focused goals.
    {\tt all:} can only be used at the toplevel of a tactic expression.

  \item {\tt par:} {\tacexpr}

    In this variant, {\tacexpr} is applied to all focused goals
    in parallel.  The number of workers can be controlled via the
    command line option {\tt -async-proofs-tac-j} taking as argument
    the desired number of workers.  Limitations:  {\tt par: } only works
    on goals containing no existential variables and {\tacexpr} must
    either solve the goal completely or do nothing (i.e. it cannot make
    some progress).
    {\tt par:} can only be used at the toplevel of a tactic expression.

\end{Variants}

\ErrMsg \errindex{No such goal}

\subsubsection[For loop]{For loop\tacindex{do}
\index{Tacticals!do@{\tt do}}}

There is a for loop that repeats a tactic {\num} times:
\begin{quote}
{\tt do} {\num} {\tacexpr}
\end{quote}
{\tacexpr} is evaluated to $v$ which must be a tactic value.
This tactic value $v$ is
applied {\num} times. Supposing ${\num}>1$, after the first
application of $v$, $v$ is applied, at least once, to the generated
subgoals and so on. It fails if the application of $v$ fails before
the {\num} applications have been completed.

\subsubsection[Repeat loop]{Repeat loop\tacindex{repeat}
\index{Tacticals!repeat@{\tt repeat}}}

We have a repeat loop with:
\begin{quote}
{\tt repeat} {\tacexpr}
\end{quote}
{\tacexpr} is evaluated to $v$. If $v$ denotes a tactic, this tactic
is applied to each focused goal independently. If the application
succeeds, the tactic is applied recursively to all the generated subgoals
until it eventually fails.  The recursion stops in a subgoal when the
tactic has failed \emph{to make progress}.  The tactic {\tt repeat
  {\tacexpr}} itself never fails.

\subsubsection[Error catching]{Error catching\tacindex{try}
\index{Tacticals!try@{\tt try}}}

We can catch the tactic errors with:
\begin{quote}
{\tt try} {\tacexpr}
\end{quote}
{\tacexpr} is evaluated to $v$ which must be a tactic value.
The tactic value $v$ is
applied to each focused goal independently. If the application of $v$
fails in a goal, it catches the error and leaves the goal
unchanged. If the level of the exception is positive, then the
exception is re-raised with its level decremented.

\subsubsection[Detecting progress]{Detecting progress\tacindex{progress}}

We can check if a tactic made progress with:
\begin{quote}
{\tt progress} {\tacexpr}
\end{quote}
{\tacexpr} is evaluated to $v$ which must be a tactic value.
The tactic value $v$ is
applied to each focued subgoal independently. If the application of
$v$ to one of the focused subgoal produced subgoals equal to the
initial goals (up to syntactical equality), then an error of level 0
is raised.

\ErrMsg \errindex{Failed to progress}

\subsubsection[Backtracking branching]{Backtracking branching\tacindex{$+$}
\index{Tacticals!or@{\tt $+$}}}

We can branch with the following structure:
\begin{quote}
{\tacexpr}$_1$ {\tt +} {\tacexpr}$_2$
\end{quote}
{\tacexpr}$_1$ and {\tacexpr}$_2$ are evaluated to $v_1$ and
$v_2$ which must be tactic values. The tactic value $v_1$ is applied to each
focused goal independently and if it fails or a later tactic fails,
then the proof backtracks to the current goal and $v_2$ is applied.

Tactics can be seen as having several successes. When a tactic fails
it asks for more successes of the prior tactics. {\tacexpr}$_1$ {\tt
  +} {\tacexpr}$_2$ has all the successes of $v_1$ followed by all the
successes of $v_2$. Algebraically, ({\tacexpr}$_1$ {\tt +}
{\tacexpr}$_2$);{\tacexpr}$_3$ $=$ ({\tacexpr}$_1$;{\tacexpr}$_3$)
{\tt +} ({\tacexpr}$_2$;{\tacexpr}$_3$).

Branching is left-associative.

\subsubsection[First tactic to work]{First tactic to work\tacindex{first}
\index{Tacticals!first@{\tt first}}}

Backtracking branching may be too expensive. In this case we may
restrict to a local, left biased, branching and consider the first
tactic to work (i.e. which does not fail) among a panel of tactics:
\begin{quote}
{\tt first [} {\tacexpr}$_1$ {\tt |} $...$ {\tt |} {\tacexpr}$_n$ {\tt ]}
\end{quote}
{\tacexpr}$_i$ are evaluated to $v_i$ and $v_i$ must be tactic values,
for $i=1,...,n$. Supposing $n>1$, it applies, in each focused goal
independently, $v_1$, if it works, it stops otherwise it tries to
apply $v_2$ and so on. It fails when there is no applicable tactic. In
other words, {\tt first [} {\tacexpr}$_1$ {\tt |} $...$ {\tt |}
  {\tacexpr}$_n$ {\tt ]} behaves, in each goal, as the the first $v_i$
to have \emph{at least} one success.

\ErrMsg \errindex{No applicable tactic}

\subsubsection[Left-biased branching]{Left-biased branching\tacindex{$\mid\mid$}
\index{Tacticals!orelse@{\tt $\mid\mid$}}}

Yet another way of branching without backtracking is the following structure:
\begin{quote}
{\tacexpr}$_1$ {\tt ||} {\tacexpr}$_2$
\end{quote}
{\tacexpr}$_1$ and {\tacexpr}$_2$ are evaluated to $v_1$ and
$v_2$ which must be tactic values. The tactic value $v_1$ is applied in each
subgoal independently and if it fails \emph{to progress} then $v_2$ is
applied. {\tacexpr}$_1$ {\tt ||} {\tacexpr}$_2$ is equivalent to {\tt
  first [} {\tt progress} {\tacexpr}$_1$ {\tt |} {\tt progress}
  {\tacexpr}$_2$ {\tt ]} (except that if it fails, it fails like
$v_2$). Branching is left-associative.

\subsubsection[Generalized biased branching]{Generalized biased branching\tacindex{tryif}
\index{Tacticals!tryif@{\tt tryif}}}

The tactic
\begin{quote}
{\tt tryif {\tacexpr}$_1$ then {\tacexpr}$_2$ else {\tacexpr}$_3$}
\end{quote}
is a generalization of the biased-branching tactics above. The
expression {\tacexpr}$_1$ is evaluated to $v_1$, which is then applied
to each subgoal independently. For each goal where $v_1$ succeeds at
least once, {tacexpr}$_2$ is evaluated to $v_2$ which is then applied
collectively to the generated subgoals. The $v_2$ tactic can trigger
backtracking points in $v_1$: where $v_1$ succeeds at least once, {\tt
  tryif {\tacexpr}$_1$ then {\tacexpr}$_2$ else {\tacexpr}$_3$} is
equivalent to $v_1;v_2$. In each of the goals where $v_1$ does not
succeed at least once, {\tacexpr}$_3$ is evaluated in $v_3$ which is
is then applied to the goal.

\subsubsection[Soft cut]{Soft cut\tacindex{once}\index{Tacticals!once@{\tt once}}}

Another way of restricting backtracking is to restrict a tactic to a
single success \emph{a posteriori}:
\begin{quote}
{\tt once} {\tacexpr}
\end{quote}
{\tacexpr} is evaluated to $v$ which must be a tactic value.
The tactic value $v$ is
applied but only its first success is used. If $v$ fails, {\tt once}
{\tacexpr} fails like $v$. If $v$ has a least one success, {\tt once}
{\tacexpr} succeeds once, but cannot produce more successes.

\subsubsection[Checking the successes]{Checking the successes\tacindex{exactly\_once}\index{Tacticals!exactly\_once@{\tt exactly\_once}}}

Coq provides an experimental way to check that a tactic has \emph{exactly one} success:
\begin{quote}
{\tt exactly\_once} {\tacexpr}
\end{quote}
{\tacexpr} is evaluated to $v$ which must be a tactic value.
The tactic value $v$ is
applied if it has at most one success. If $v$ fails, {\tt
  exactly\_once} {\tacexpr} fails like $v$. If $v$ has a exactly one
success, {\tt exactly\_once} {\tacexpr} succeeds like $v$. If $v$ has
two or more successes, {\tt exactly\_once} {\tacexpr} fails.

The experimental status of this tactic pertains to the fact if $v$ performs side effects, they may occur in a unpredictable way. Indeed, normally $v$ would only be executed up to the first success until backtracking is needed, however {\tt exactly\_once} needs to look ahead to see whether a second success exists, and may run further effects immediately.

\ErrMsg \errindex{This tactic has more than one success}

\subsubsection[Solving]{Solving\tacindex{solve}
\index{Tacticals!solve@{\tt solve}}}

We may consider the first to solve (i.e. which generates no subgoal) among a
panel of tactics:
\begin{quote}
{\tt solve [} {\tacexpr}$_1$ {\tt |} $...$ {\tt |} {\tacexpr}$_n$ {\tt ]}
\end{quote}
{\tacexpr}$_i$ are evaluated to $v_i$ and $v_i$ must be tactic values,
for $i=1,...,n$. Supposing $n>1$, it applies $v_1$ to each goal
independently, if it doesn't solve the goal then it tries to apply
$v_2$ and so on. It fails if there is no solving tactic.

\ErrMsg \errindex{Cannot solve the goal}

\subsubsection[Identity]{Identity\label{ltac:idtac}\tacindex{idtac}
\index{Tacticals!idtac@{\tt idtac}}}

The constant {\tt idtac} is the identity tactic: it leaves any goal
unchanged but it appears in the proof script.

\variant {\tt idtac \nelist{\messagetoken}{}}

This prints the given tokens. Strings and integers are printed
literally. If a (term) variable is given, its contents are printed.


\subsubsection[Failing]{Failing\tacindex{fail}
\index{Tacticals!fail@{\tt fail}}
\tacindex{gfail}\index{Tacticals!gfail@{\tt gfail}}}

The tactic {\tt fail} is the always-failing tactic: it does not solve
any goal. It is useful for defining other tacticals since it can be
caught by {\tt try}, {\tt repeat}, {\tt match goal}, or the branching
tacticals. The {\tt fail} tactic will, however, succeed if all the
goals have already been solved.

\begin{Variants}
\item {\tt fail $n$}\\ The number $n$ is the failure level. If no
  level is specified, it defaults to $0$.  The level is used by {\tt
    try}, {\tt repeat}, {\tt match goal} and the branching tacticals.
  If $0$, it makes {\tt match goal} considering the next clause
  (backtracking). If non zero, the current {\tt match goal} block,
  {\tt try}, {\tt repeat}, or branching command is aborted and the
  level is decremented. In the case of {\tt +}, a non-zero level skips
  the first backtrack point, even if the call to {\tt fail $n$} is not
  enclosed in a {\tt +} command, respecting the algebraic identity.

\item {\tt fail \nelist{\messagetoken}{}}\\
The given tokens are used for printing the failure message.

\item {\tt fail $n$ \nelist{\messagetoken}{}}\\
This is a combination of the previous variants.

\item {\tt gfail}\\
This variant fails even if there are no goals left.

\item {\tt gfail \nelist{\messagetoken}{}}\\
{\tt gfail $n$ \nelist{\messagetoken}{}}\\
These variants fail with an error message or an error level even if
there are no goals left. Be careful however if Coq terms have to be
printed as part of the failure: term construction always forces the
tactic into the goals, meaning that if there are no goals when it is
evaluated, a tactic call like {\tt let x:=H in fail 0 x} will succeed.

\end{Variants}

\ErrMsg \errindex{Tactic Failure {\it message} (level $n$)}.

\subsubsection[Timeout]{Timeout\tacindex{timeout}
\index{Tacticals!timeout@{\tt timeout}}}

We can force a tactic to stop if it has not finished after a certain
amount of time:
\begin{quote}
{\tt timeout} {\num} {\tacexpr}
\end{quote}
{\tacexpr} is evaluated to $v$ which must be a tactic value.
The tactic value $v$ is
applied normally, except that it is interrupted after ${\num}$ seconds
if it is still running. In this case the outcome is a failure.

Warning: For the moment, {\tt timeout} is based on elapsed time in
seconds, which is very
machine-dependent: a script that works on a quick machine may fail
on a slow one. The converse is even possible if you combine a
{\tt timeout} with some other tacticals. This tactical is hence
proposed only for convenience during debug or other development
phases, we strongly advise you to not leave any {\tt timeout} in
final scripts. Note also that this tactical isn't available on
the native Windows port of Coq.

\subsubsection{Timing a tactic\tacindex{time}
\index{Tacticals!time@{\tt time}}}

A tactic execution can be timed:
\begin{quote}
 {\tt time} {\qstring} {\tacexpr}
\end{quote}
evaluates {\tacexpr}
and displays the time the tactic expression ran, whether it fails or
successes. In case of several successes, the time for each successive
runs is displayed. Time is in seconds and is machine-dependent. The
{\qstring} argument is optional. When provided, it is used to identify
this particular occurrence of {\tt time}.

\subsubsection[Local definitions]{Local definitions\index{Ltac!let@\texttt{let}}
\index{Ltac!let rec@\texttt{let rec}}
\index{let@\texttt{let}!in Ltac}
\index{let rec@\texttt{let rec}!in Ltac}}

Local definitions can be done as follows:
\begin{quote}
{\tt let} {\ident}$_1$ {\tt :=} {\tacexpr}$_1$\\
{\tt with} {\ident}$_2$ {\tt :=} {\tacexpr}$_2$\\
...\\
{\tt with} {\ident}$_n$ {\tt :=} {\tacexpr}$_n$ {\tt in}\\
{\tacexpr}
\end{quote}
each {\tacexpr}$_i$ is evaluated to $v_i$, then, {\tacexpr} is
evaluated by substituting $v_i$ to each occurrence of {\ident}$_i$,
for $i=1,...,n$. There is no dependencies between the {\tacexpr}$_i$
and the {\ident}$_i$.

Local definitions can be recursive by using {\tt let rec} instead of
{\tt let}. In this latter case, the definitions are evaluated lazily
so that the {\tt rec} keyword can be used also in non recursive cases
so as to avoid the eager evaluation of local definitions.

\subsubsection{Application}

An application is an expression of the following form:
\begin{quote}
{\qualid} {\tacarg}$_1$ ... {\tacarg}$_n$
\end{quote}
The reference {\qualid} must be bound to some defined tactic
definition expecting at least $n$ arguments.  The expressions
{\tacexpr}$_i$ are evaluated to $v_i$, for $i=1,...,n$.
%If {\tacexpr} is a {\tt Fun} or {\tt Rec} value then the body is evaluated by
%substituting $v_i$ to the formal parameters, for $i=1,...,n$. For recursive
%clauses, the bodies are lazily substituted (when an identifier to be evaluated
%is the name of a recursive clause).

%\subsection{Application of tactic values}

\subsubsection[Function construction]{Function construction\index{fun@\texttt{fun}!in Ltac}
\index{Ltac!fun@\texttt{fun}}}

A parameterized tactic can be built anonymously (without resorting to
local definitions) with:
\begin{quote}
{\tt fun} {\ident${}_1$} ... {\ident${}_n$} {\tt =>} {\tacexpr}
\end{quote}
Indeed, local definitions of functions are a syntactic sugar for
binding a {\tt fun} tactic to an identifier.

\subsubsection[Pattern matching on terms]{Pattern matching on terms\index{Ltac!match@\texttt{match}}
\index{match@\texttt{match}!in Ltac}}

We can carry out pattern matching on terms with:
\begin{quote}
{\tt match} {\tacexpr} {\tt with}\\
~~~{\cpattern}$_1$ {\tt =>} {\tacexpr}$_1$\\
~{\tt |} {\cpattern}$_2$ {\tt =>} {\tacexpr}$_2$\\
~...\\
~{\tt |} {\cpattern}$_n$ {\tt =>} {\tacexpr}$_n$\\
~{\tt |} {\tt \_} {\tt =>} {\tacexpr}$_{n+1}$\\
{\tt end}
\end{quote}
The expression {\tacexpr} is evaluated and should yield a term which
is matched against {\cpattern}$_1$. The matching is non-linear: if a
metavariable occurs more than once, it should match the same
expression every time. It is first-order except on the
variables of the form {\tt @?id} that occur in head position of an
application. For these variables, the matching is second-order and
returns a functional term.

Alternatively, when a metavariable of the form {\tt ?id} occurs under
binders, say $x_1$, \ldots, $x_n$ and the expression matches, the
metavariable is instantiated by a term which can then be used in any
context which also binds the variables $x_1$, \ldots, $x_n$ with
same types. This provides with a primitive form of matching
under context which does not require manipulating a functional term.

If the matching with {\cpattern}$_1$ succeeds, then {\tacexpr}$_1$ is
evaluated into some value by substituting the pattern matching
instantiations to the metavariables. If {\tacexpr}$_1$ evaluates to a
tactic and the {\tt match} expression is in position to be applied to
a goal (e.g. it is not bound to a variable by a {\tt let in}), then
this tactic is applied. If the tactic succeeds, the list of resulting
subgoals is the result of the {\tt match} expression. If
{\tacexpr}$_1$ does not evaluate to a tactic or if the {\tt match}
expression is not in position to be applied to a goal, then the result
of the evaluation of {\tacexpr}$_1$ is the result of the {\tt match}
expression.

If the matching with {\cpattern}$_1$ fails, or if it succeeds but the
evaluation of {\tacexpr}$_1$ fails, or if the evaluation of
{\tacexpr}$_1$ succeeds but returns a tactic in execution position
whose execution fails, then {\cpattern}$_2$ is used and so on.  The
pattern {\_} matches any term and shunts all remaining patterns if
any. If all clauses fail (in particular, there is no pattern {\_})
then a no-matching-clause error is raised.

Failures in subsequent tactics do not cause backtracking to select new
branches or inside the right-hand side of the selected branch even if
it has backtracking points.

\begin{ErrMsgs}

\item \errindex{No matching clauses for match}

  No pattern can be used and, in particular, there is no {\tt \_} pattern.

\item \errindex{Argument of match does not evaluate to a term}

  This happens when {\tacexpr} does not denote a term.

\end{ErrMsgs}

\begin{Variants}

\item \index{multimatch@\texttt{multimatch}!in Ltac}
\index{Ltac!multimatch@\texttt{multimatch}}
Using {\tt multimatch} instead of {\tt match} will allow subsequent
tactics to backtrack into a right-hand side tactic which has
backtracking points left and trigger the selection of a new matching
branch when all the backtracking points of the right-hand side have
been consumed.

The syntax {\tt match \ldots} is, in fact, a shorthand for
{\tt once multimatch \ldots}.

\item \index{lazymatch@\texttt{lazymatch}!in Ltac}
\index{Ltac!lazymatch@\texttt{lazymatch}}
Using {\tt lazymatch} instead of {\tt match} will perform the same
pattern matching procedure but will commit to the first matching
branch rather than trying a new matching if the right-hand side
fails. If the right-hand side of the selected branch is a tactic with
backtracking points, then subsequent failures cause this tactic to
backtrack.

\item \index{context@\texttt{context}!in pattern}
There is a special form of patterns to match a subterm against the
pattern:
\begin{quote}
{\tt context} {\ident} {\tt [} {\cpattern} {\tt ]}
\end{quote}
It matches any term with a subterm matching {\cpattern}. If there is
a match, the optional {\ident} is assigned the ``matched context'', i.e.
the initial term where the matched subterm is replaced by a
hole. The example below will show how to use such term contexts.

If the evaluation of the right-hand-side of a valid match fails, the
next matching subterm is tried. If no further subterm matches, the
next clause is tried. Matching subterms are considered top-bottom and
from left to right (with respect to the raw printing obtained by
setting option {\tt Printing All}, see Section~\ref{SetPrintingAll}).

\begin{coq_example}
Ltac f x :=
  match x with
    context f [S ?X] =>
    idtac X;                    (* To display the evaluation order *)
    assert (p := eq_refl 1 : X=1);    (* To filter the case X=1 *)
    let x:= context f[O] in assert (x=O) (* To observe the context *)
  end.
Goal True.
f (3+4).
\end{coq_example}

\item \index{appcontext@\texttt{appcontext}!in pattern}
  \optindex{Tactic Compat Context}
For historical reasons, {\tt context} used to consider $n$-ary applications
such as {\tt (f 1 2)} as a whole, and not as a sequence of unary
applications {\tt ((f 1) 2)}. Hence {\tt context [f ?x]} would fail
to find a matching subterm in {\tt (f 1 2)}: if the pattern was a partial
application, the matched subterms would have necessarily been
applications with exactly the same number of arguments.
As a workaround, one could use the following variant of {\tt context}:
\begin{quote}
{\tt appcontext} {\ident} {\tt [} {\cpattern} {\tt ]}
\end{quote}
This syntax is now deprecated, as {\tt context} behaves as intended. The former
behavior can be retrieved with the {\tt Tactic Compat Context} flag.

\end{Variants}

\subsubsection[Pattern matching on goals]{Pattern matching on goals\index{Ltac!match goal@\texttt{match goal}}
\index{Ltac!match reverse goal@\texttt{match reverse goal}}
\index{match goal@\texttt{match goal}!in Ltac}
\index{match reverse goal@\texttt{match reverse goal}!in Ltac}}

We can make pattern matching on goals using the following expression:
\begin{quote}
\begin{tabbing}
{\tt match goal with}\\
~~\={\tt |} $hyp_{1,1}${\tt ,}...{\tt ,}$hyp_{1,m_1}$
   ~~{\tt |-}{\cpattern}$_1${\tt =>} {\tacexpr}$_1$\\
  \>{\tt |} $hyp_{2,1}${\tt ,}...{\tt ,}$hyp_{2,m_2}$
   ~~{\tt |-}{\cpattern}$_2${\tt =>} {\tacexpr}$_2$\\
~~...\\
  \>{\tt |} $hyp_{n,1}${\tt ,}...{\tt ,}$hyp_{n,m_n}$
   ~~{\tt |-}{\cpattern}$_n${\tt =>} {\tacexpr}$_n$\\
  \>{\tt |\_}~~~~{\tt =>} {\tacexpr}$_{n+1}$\\
{\tt end}
\end{tabbing}
\end{quote}

If each hypothesis pattern $hyp_{1,i}$, with $i=1,...,m_1$
is matched (non-linear first-order unification) by an hypothesis of
the goal and if {\cpattern}$_1$ is matched by the conclusion of the
goal, then {\tacexpr}$_1$ is evaluated to $v_1$ by substituting the
pattern matching to the metavariables and the real hypothesis names
bound to the possible hypothesis names occurring in the hypothesis
patterns. If $v_1$ is a tactic value, then it is applied to the
goal. If this application fails, then another combination of
hypotheses is tried with the same proof context pattern. If there is
no other combination of hypotheses then the second proof context
pattern is tried and so on. If the next to last proof context pattern
fails then {\tacexpr}$_{n+1}$ is evaluated to $v_{n+1}$ and $v_{n+1}$
is applied. Note also that matching against subterms (using the {\tt
context} {\ident} {\tt [} {\cpattern} {\tt ]}) is available and is
also subject to yielding several matchings.

Failures in subsequent tactics do not cause backtracking to select new
branches or combinations of hypotheses, or inside the right-hand side
of the selected branch even if it has backtracking points.

\ErrMsg \errindex{No matching clauses for match goal}

No clause succeeds, i.e. all matching patterns, if any,
fail at the application of the right-hand-side.

\medskip

It is important to know that each hypothesis of the goal can be
matched by at most one hypothesis pattern. The order of matching is
the following: hypothesis patterns are examined from the right to the
left (i.e. $hyp_{i,m_i}$ before $hyp_{i,1}$). For each hypothesis
pattern, the goal hypothesis are matched in order (fresher hypothesis
first), but it possible to reverse this order (older first) with
the {\tt match reverse goal with} variant.

\variant

\index{multimatch goal@\texttt{multimatch goal}!in Ltac}
\index{Ltac!multimatch goal@\texttt{multimatch goal}}
\index{multimatch reverse goal@\texttt{multimatch reverse goal}!in Ltac}
\index{Ltac!multimatch reverse goal@\texttt{multimatch reverse goal}}

Using {\tt multimatch} instead of {\tt match} will allow subsequent
tactics to backtrack into a right-hand side tactic which has
backtracking points left and trigger the selection of a new matching
branch or combination of hypotheses when all the backtracking points
of the right-hand side have been consumed.

The syntax {\tt match [reverse] goal \ldots} is, in fact, a shorthand for
{\tt once multimatch [reverse] goal \ldots}.

\index{lazymatch goal@\texttt{lazymatch goal}!in Ltac}
\index{Ltac!lazymatch goal@\texttt{lazymatch goal}}
\index{lazymatch reverse goal@\texttt{lazymatch reverse goal}!in Ltac}
\index{Ltac!lazymatch reverse goal@\texttt{lazymatch reverse goal}}
Using {\tt lazymatch} instead of {\tt match} will perform the same
pattern matching procedure but will commit to the first matching
branch with the first matching combination of hypotheses rather than
trying a new matching if the right-hand side fails. If the right-hand
side of the selected branch is a tactic with backtracking points, then
subsequent failures cause this tactic to backtrack.

\subsubsection[Filling a term context]{Filling a term context\index{context@\texttt{context}!in expression}}

The following expression is not a tactic in the sense that it does not
produce subgoals but generates a term to be used in tactic
expressions:
\begin{quote}
{\tt context} {\ident} {\tt [} {\tacexpr} {\tt ]}
\end{quote}
{\ident} must denote a context variable bound by a {\tt context}
pattern of a {\tt match} expression. This expression evaluates
replaces the hole of the value of {\ident} by the value of
{\tacexpr}.

\ErrMsg \errindex{not a context variable}


\subsubsection[Generating fresh hypothesis names]{Generating fresh hypothesis names\index{Ltac!fresh@\texttt{fresh}}
\index{fresh@\texttt{fresh}!in Ltac}}

Tactics sometimes have to generate new names for hypothesis. Letting
the system decide a name with the {\tt intro} tactic is not so good
since it is very awkward to retrieve the name the system gave.
The following expression returns an identifier:
\begin{quote}
{\tt fresh} \nelist{\textrm{\textsl{component}}}{}
\end{quote}
It evaluates to an identifier unbound in the goal. This fresh
identifier is obtained by concatenating the value of the
\textrm{\textsl{component}}'s (each of them is, either an {\qualid} which
has to refer to a (unqualified) name, or directly a name denoted by a
{\qstring}). If the resulting name is already used, it is padded
with a number so that it becomes fresh. If no component is
given, the name is a fresh derivative of the name {\tt H}.

\subsubsection[Computing in a constr]{Computing in a constr\index{Ltac!eval@\texttt{eval}}
\index{eval@\texttt{eval}!in Ltac}}

Evaluation of a term can be performed with:
\begin{quote}
{\tt eval} {\nterm{redexpr}} {\tt in} {\term}
\end{quote}
where \nterm{redexpr} is a reduction tactic among {\tt red}, {\tt
hnf}, {\tt compute}, {\tt simpl}, {\tt cbv}, {\tt lazy}, {\tt unfold},
{\tt fold}, {\tt pattern}.

\subsubsection{Recovering the type of a term}
%\tacindex{type of}
\index{Ltac!type of@\texttt{type of}}
\index{type of@\texttt{type of}!in Ltac}

The following returns the type of {\term}:

\begin{quote}
{\tt type of} {\term}
\end{quote}

\subsubsection[Manipulating untyped terms]{Manipulating untyped terms\index{Ltac!uconstr@\texttt{uconstr}}
\index{uconstr@\texttt{uconstr}!in Ltac}
\index{Ltac!type\_term@\texttt{type\_term}}
\index{type\_term@\texttt{type\_term}!in Ltac}}

The terms built in Ltac are well-typed by default. It may not be
appropriate for building large terms using a recursive Ltac function:
the term has to be entirely type checked at each step, resulting in
potentially very slow behavior. It is possible to build untyped terms
using Ltac with the syntax

\begin{quote}
{\tt uconstr :} {\term}
\end{quote}

An untyped term, in Ltac, can contain references to hypotheses or to
Ltac variables containing typed or untyped terms. An untyped term can
be type-checked using the function {\tt type\_term} whose argument is
parsed as an untyped term and returns a well-typed term which can be
used in tactics.

\begin{quote}
{\tt type\_term} {\term}
\end{quote}

Untyped terms built using {\tt uconstr :} can also be used as
arguments to the {\tt refine} tactic~\ref{refine}. In that case the
untyped term is type checked against the conclusion of the goal, and
the holes which are not solved by the typing procedure are turned into
new subgoals.

\subsubsection[Counting the goals]{Counting the goals\index{Ltac!numgoals@\texttt{numgoals}}\index{numgoals@\texttt{numgoals}!in Ltac}}

The number of goals under focus can be recovered using the {\tt
  numgoals} function. Combined with the {\tt guard} command below, it
can be used to branch over the number of goals produced by previous tactics.

\begin{coq_example*}
Ltac pr_numgoals := let n := numgoals in idtac "There are" n "goals".

Goal True /\ True /\ True.
split;[|split].
\end{coq_example*}
\begin{coq_example}
all:pr_numgoals.
\end{coq_example}

\subsubsection[Testing boolean expressions]{Testing boolean expressions\index{Ltac!guard@\texttt{guard}}\index{guard@\texttt{guard}!in Ltac}}

The {\tt guard} tactic tests a boolean expression, and fails if the expression evaluates to false. If the expression evaluates to true, it succeeds without affecting the proof.

\begin{quote}
{\tt guard} {\it test}
\end{quote}

The accepted tests are simple integer comparisons.

\begin{coq_example*}
Goal True /\ True /\ True.
split;[|split].
\end{coq_example*}
\begin{coq_example}
all:let n:= numgoals in guard n<4.
Fail all:let n:= numgoals in guard n=2.
\end{coq_example}
\begin{ErrMsgs}

\item \errindex{Condition not satisfied}

\end{ErrMsgs}

\begin{coq_eval}
Reset Initial.
\end{coq_eval}

\subsubsection[Proving a subgoal as a separate lemma]{Proving a subgoal as a separate lemma\tacindex{abstract}\tacindex{transparent\_abstract}\comindex{Qed exporting}
\index{Tacticals!abstract@{\tt abstract}}\index{Tacticals!transparent\_abstract@{\tt transparent\_abstract}}}

From the outside ``\texttt{abstract \tacexpr}'' is the same as
{\tt solve \tacexpr}. Internally it saves an auxiliary lemma called
{\ident}\texttt{\_subproof}\textit{n} where {\ident} is the name of the
current goal and \textit{n} is chosen so that this is a fresh name.
Such auxiliary lemma is inlined in the final proof term
unless the proof is ended with ``\texttt{Qed exporting}''.  In such
case the lemma is preserved.  The syntax
``\texttt{Qed exporting }\ident$_1$\texttt{, ..., }\ident$_n$''
is also supported.  In such case the system checks that the names given by the
user actually exist when the proof is ended.

This tactical is useful with tactics such as \texttt{omega} or
\texttt{discriminate} that generate huge proof terms. With that tool
the user can avoid the explosion at time of the \texttt{Save} command
without having to cut manually the proof in smaller lemmas.

It may be useful to generate lemmas minimal w.r.t. the assumptions they depend
on. This can be obtained thanks to the option below.

\begin{quote}
\optindex{Shrink Abstract}
{\tt Set Shrink Abstract}
\end{quote}

When set, all lemmas generated through \texttt{abstract {\tacexpr}}
and \texttt{transparent\_abstract {\tacexpr}} are quantified only over the
variables that appear in the term constructed by \texttt{\tacexpr}.

\begin{Variants}
\item \texttt{abstract {\tacexpr} using {\ident}}.\\
  Give explicitly the name of the auxiliary lemma.
  Use this feature at your own risk; explicitly named and reused subterms
  don't play well with asynchronous proofs.
\item \texttt{transparent\_abstract {\tacexpr}}.\\
  Save the subproof in a transparent lemma rather than an opaque one.
  Use this feature at your own risk; building computationally relevant terms
  with tactics is fragile.
\item \texttt{transparent\_abstract {\tacexpr} using {\ident}}.\\
  Give explicitly the name of the auxiliary transparent lemma.
  Use this feature at your own risk; building computationally relevant terms
  with tactics is fragile, and explicitly named and reused subterms
  don't play well with asynchronous proofs.
\end{Variants}

\ErrMsg \errindex{Proof is not complete}

\section[Tactic toplevel definitions]{Tactic toplevel definitions\comindex{Ltac}}

\subsection{Defining {\ltac} functions}

Basically, {\ltac} toplevel definitions are made as follows:
%{\tt Tactic Definition} {\ident} {\tt :=} {\tacexpr}\\
%
%{\tacexpr} is evaluated to $v$ and $v$ is associated to {\ident}. Next, every
%script is evaluated by substituting $v$ to {\ident}.
%
%We can define functional definitions by:\\
\begin{quote}
{\tt Ltac} {\ident} {\ident}$_1$ ... {\ident}$_n$ {\tt :=}
{\tacexpr}
\end{quote}
This defines a new {\ltac} function that can be used in any tactic
script or new {\ltac} toplevel definition.

\Rem The preceding definition can equivalently be written:
\begin{quote}
{\tt Ltac} {\ident} {\tt := fun} {\ident}$_1$ ... {\ident}$_n$
{\tt =>} {\tacexpr}
\end{quote}
Recursive and mutual recursive function definitions are also
possible with the syntax:
\begin{quote}
{\tt Ltac} {\ident}$_1$ {\ident}$_{1,1}$ ...
{\ident}$_{1,m_1}$~~{\tt :=} {\tacexpr}$_1$\\
{\tt with} {\ident}$_2$ {\ident}$_{2,1}$ ... {\ident}$_{2,m_2}$~~{\tt :=}
{\tacexpr}$_2$\\
...\\
{\tt with} {\ident}$_n$ {\ident}$_{n,1}$ ... {\ident}$_{n,m_n}$~~{\tt :=}
{\tacexpr}$_n$
\end{quote}
\medskip
It is also possible to \emph{redefine} an existing user-defined tactic
using the syntax:
\begin{quote}
{\tt Ltac} {\qualid} {\ident}$_1$ ... {\ident}$_n$ {\tt ::=}
{\tacexpr}
\end{quote}
A previous definition of {\qualid} must exist in the environment.
The new definition will always be used instead of the old one and
it goes across module boundaries.

If preceded by the keyword {\tt Local} the tactic definition will not
be exported outside the current module.

\subsection[Printing {\ltac} tactics]{Printing {\ltac} tactics\comindex{Print Ltac}}

Defined {\ltac} functions can be displayed using the command

\begin{quote}
{\tt Print Ltac {\qualid}.}
\end{quote}

The command {\tt Print Ltac Signatures\comindex{Print Ltac Signatures}} displays a list of all user-defined tactics, with their arguments.

\section{Debugging {\ltac} tactics}

\subsection[Info trace]{Info trace\comindex{Info}\optindex{Info Level}}

It is possible to print the trace of the path eventually taken by an {\ltac} script. That is, the list of executed tactics, discarding all the branches which have failed. To that end the {\tt Info} command can be used with the following syntax.

\begin{quote}
{\tt Info} {\num} {\tacexpr}.
\end{quote}

The number {\num} is the unfolding level of tactics in the trace. At level $0$, the trace contains a sequence of tactics in the actual script, at level $1$, the trace will be the concatenation of the traces of these tactics, etc\ldots

\begin{coq_eval}
Reset Initial.
\end{coq_eval}
\begin{coq_example*}
Ltac t x := exists x; reflexivity.

Goal exists n, n=0.
\end{coq_example*}
\begin{coq_example}
Info 0 t 1||t 0.
\end{coq_example}
\begin{coq_example*}
Undo.
\end{coq_example*}
\begin{coq_example}
Info 1 t 1||t 0.
\end{coq_example}

The trace produced by {\tt Info} tries its best to be a reparsable {\ltac} script, but this goal is not achievable in all generality. So some of the output traces will contain oddities.

As an additional help for debugging, the trace produced by {\tt Info} contains (in comments) the messages produced by the {\tt idtac} tacticals~\ref{ltac:idtac} at the right possition in the script. In particular, the calls to {\tt idtac} in branches which failed are not printed.

An alternative to the {\tt Info} command is to use the {\tt Info Level} option as follows:

\begin{quote}
{\tt Set Info Level} \num.
\end{quote}

This will automatically print the same trace as {\tt Info \num} at each tactic call. The unfolding level can be overridden by a call to the {\tt Info} command. And this option can be turned off with:

\begin{quote}
{\tt Unset Info Level} \num.
\end{quote}

The current value for the {\tt Info Level} option can be checked using the {\tt Test Info Level} command.

\subsection[Interactive debugger]{Interactive debugger\optindex{Ltac Debug}\optindex{Ltac Batch Debug}}

The {\ltac} interpreter comes with a step-by-step debugger. The
debugger can be activated using the command

\begin{quote}
{\tt Set Ltac Debug.}
\end{quote}

\noindent and deactivated using the command

\begin{quote}
{\tt Unset Ltac Debug.}
\end{quote}

To know if the debugger is on, use the command \texttt{Test Ltac Debug}.
When the debugger is activated, it stops at every step of the
evaluation of the current {\ltac} expression and it prints information
on what it is doing. The debugger stops, prompting for a command which
can be one of the following:

\medskip
\begin{tabular}{ll}
simple newline: & go to the next step\\
h: & get help\\
x: & exit current evaluation\\
s: & continue current evaluation without stopping\\
r $n$: & advance $n$ steps further\\
r {\qstring}: & advance up to the next call to ``{\tt idtac} {\qstring}''\\
\end{tabular}

A non-interactive mode for the debugger is available via the command

\begin{quote}
{\tt Set Ltac Batch Debug.}
\end{quote}

This option has the effect of presenting a newline at every prompt,
when the debugger is on.  The debug log thus created, which does not
require user input to generate when this option is set, can then be
run through external tools such as \texttt{diff}.

\subsection[Profiling {\ltac} tactics]{Profiling {\ltac} tactics\optindex{Ltac Profiling}\comindex{Show Ltac Profile}\comindex{Reset Ltac Profile}}

It is possible to measure the time spent in invocations of primitive tactics as well as tactics defined in {\ltac} and their inner invocations. The primary use is the development of complex tactics, which can sometimes be so slow as to impede interactive usage. The reasons for the performence degradation can be intricate, like a slowly performing {\ltac} match or a sub-tactic whose performance only degrades in certain situations. The profiler generates a call tree and indicates the time spent in a tactic depending its calling context. Thus it allows to locate the part of a tactic definition that contains the performance bug.

\begin{quote}
{\tt Set Ltac Profiling}.
\end{quote}
Enables the profiler

\begin{quote}
{\tt Unset Ltac Profiling}.
\end{quote}
Disables the profiler

\begin{quote}
{\tt Show Ltac Profile}.
\end{quote}
Prints the profile

\begin{quote}
{\tt Show Ltac Profile} {\qstring}.
\end{quote}
Prints a profile for all tactics that start with {\qstring}. Append a period (.) to the string if you only want exactly that name.

\begin{quote}
{\tt Reset Ltac Profile}.
\end{quote}
Resets the profile, that is, deletes all accumulated information.  Note that backtracking across a {\tt Reset Ltac Profile} will not restore the information.

\begin{coq_eval}
Reset Initial.
\end{coq_eval}
\begin{coq_example*}
Require Import Coq.omega.Omega.

Ltac mytauto := tauto.
Ltac tac := intros; repeat split; omega || mytauto.

Notation max x y := (x + (y - x)) (only parsing).
\end{coq_example*}
\begin{coq_example*}
Goal forall x y z A B C D E F G H I J K L M N O P Q R S T U V W X Y Z,
    max x (max y z) = max (max x y) z /\ max x (max y z) = max (max x y) z
    /\ (A /\ B /\ C /\ D /\ E /\ F /\ G /\ H /\ I /\ J /\ K /\ L /\ M /\ N /\ O /\ P /\ Q /\ R /\ S /\ T /\ U /\ V /\ W /\ X /\ Y /\ Z
        -> Z /\ Y /\ X /\ W /\ V /\ U /\ T /\ S /\ R /\ Q /\ P /\ O /\ N /\ M /\ L /\ K /\ J /\ I /\ H /\ G /\ F /\ E /\ D /\ C /\ B /\ A).
Proof.
\end{coq_example*}
\begin{coq_example}
  Set Ltac Profiling.
  tac.
\end{coq_example}
{\let\textit\texttt% use tt mode for the output of ltacprof
\begin{coq_example}
  Show Ltac Profile.
\end{coq_example}
\begin{coq_example}
  Show Ltac Profile "omega".
\end{coq_example}
}
\begin{coq_example*}
Abort.
Unset Ltac Profiling.
\end{coq_example*}

\tacindex{start ltac profiling}\tacindex{stop ltac profiling}
The following two tactics behave like {\tt idtac} but enable and disable the profiling. They allow you to exclude parts of a proof script from profiling.

\begin{quote}
{\tt start ltac profiling}.
\end{quote}

\begin{quote}
{\tt stop ltac profiling}.
\end{quote}

You can also pass the {\tt -profile-ltac} command line option to {\tt coqc}, which performs a {\tt Set Ltac Profiling} at the beginning of each document, and a {\tt Show Ltac Profile} at the end.

Note that the profiler currently does not handle backtracking into multi-success tactics, and issues a warning to this effect in many cases when such backtracking occurs.

\endinput

\subsection{Permutation on closed lists}

\begin{figure}[b]
\begin{center}
\fbox{\begin{minipage}{0.95\textwidth}
\begin{coq_eval}
Reset Initial.
\end{coq_eval}
\begin{coq_example*}
Require Import List.
Section Sort.
Variable A : Set.
Inductive permut : list A -> list A -> Prop :=
  | permut_refl   : forall l, permut l l
  | permut_cons   :
      forall a l0 l1, permut l0 l1 -> permut (a :: l0) (a :: l1)
  | permut_append : forall a l, permut (a :: l) (l ++ a :: nil)
  | permut_trans  :
      forall l0 l1 l2, permut l0 l1 -> permut l1 l2 -> permut l0 l2.
End Sort.
\end{coq_example*}
\end{center}
\caption{Definition of the permutation predicate}
\label{permutpred}
\end{figure}


Another more complex example is the problem of permutation on closed
lists. The aim is to show that a closed list is a permutation of
another one.  First, we define the permutation predicate as shown on
Figure~\ref{permutpred}.

\begin{figure}[p]
\begin{center}
\fbox{\begin{minipage}{0.95\textwidth}
\begin{coq_example}
Ltac Permut n :=
  match goal with
  | |- (permut _ ?l ?l) => apply permut_refl
  | |- (permut _ (?a :: ?l1) (?a :: ?l2)) =>
      let newn := eval compute in (length l1) in
      (apply permut_cons; Permut newn)
  | |- (permut ?A (?a :: ?l1) ?l2) =>
      match eval compute in n with
      | 1 => fail
      | _ =>
          let l1' := constr:(l1 ++ a :: nil) in
          (apply (permut_trans A (a :: l1) l1' l2);
            [ apply permut_append | compute; Permut (pred n) ])
      end
  end.
Ltac PermutProve :=
  match goal with
  | |- (permut _ ?l1 ?l2) =>
      match eval compute in (length l1 = length l2) with
      | (?n = ?n) => Permut n
      end
  end.
\end{coq_example}
\end{minipage}}
\end{center}
\caption{Permutation tactic}
\label{permutltac}
\end{figure}

\begin{figure}[p]
\begin{center}
\fbox{\begin{minipage}{0.95\textwidth}
\begin{coq_example*}
Lemma permut_ex1 :
  permut nat (1 :: 2 :: 3 :: nil) (3 :: 2 :: 1 :: nil).
Proof.
PermutProve.
Qed.

Lemma permut_ex2 :
  permut nat
    (0 :: 1 :: 2 :: 3 :: 4 :: 5 :: 6 :: 7 :: 8 :: 9 :: nil)
    (0 :: 2 :: 4 :: 6 :: 8 :: 9 :: 7 :: 5 :: 3 :: 1 :: nil).
Proof.
PermutProve.
Qed.
\end{coq_example*}
\end{minipage}}
\end{center}
\caption{Examples of {\tt PermutProve} use}
\label{permutlem}
\end{figure}

Next, we can write naturally the tactic and the result can be seen on
Figure~\ref{permutltac}. We can notice that we use two toplevel
definitions {\tt PermutProve} and {\tt Permut}. The function to be
called is {\tt PermutProve} which computes the lengths of the two
lists and calls {\tt Permut} with the length if the two lists have the
same length. {\tt Permut} works as expected.  If the two lists are
equal, it concludes. Otherwise, if the lists have identical first
elements, it applies {\tt Permut} on the tail of the lists.  Finally,
if the lists have different first elements, it puts the first element
of one of the lists (here the second one which appears in the {\tt
  permut} predicate) at the end if that is possible, i.e., if the new
first element has been at this place previously. To verify that all
rotations have been done for a list, we use the length of the list as
an argument for {\tt Permut} and this length is decremented for each
rotation down to, but not including, 1 because for a list of length
$n$, we can make exactly $n-1$ rotations to generate at most $n$
distinct lists. Here, it must be noticed that we use the natural
numbers of {\Coq} for the rotation counter. On Figure~\ref{ltac}, we
can see that it is possible to use usual natural numbers but they are
only used as arguments for primitive tactics and they cannot be
handled, in particular, we cannot make computations with them. So, a
natural choice is to use {\Coq} data structures so that {\Coq} makes
the computations (reductions) by {\tt eval compute in} and we can get
the terms back by {\tt match}.

With {\tt PermutProve}, we can now prove lemmas such those shown on
Figure~\ref{permutlem}.


\subsection{Deciding intuitionistic propositional logic}

\begin{figure}[tbp]
\begin{center}
\fbox{\begin{minipage}{0.95\textwidth}
\begin{coq_example}
Ltac Axioms :=
  match goal with
  | |- True => trivial
  | _:False |- _  => elimtype False; assumption
  | _:?A |- ?A  => auto
  end.
Ltac DSimplif :=
  repeat
   (intros;
    match goal with
     | id:(~ _) |- _ => red in id
     | id:(_ /\ _) |- _ =>
         elim id; do 2 intro; clear id
     | id:(_ \/ _) |- _ =>
         elim id; intro; clear id
     | id:(?A /\ ?B -> ?C) |- _ =>
         cut (A -> B -> C);
          [ intro | intros; apply id; split; assumption ]
     | id:(?A \/ ?B -> ?C) |- _ =>
         cut (B -> C);
          [ cut (A -> C);
             [ intros; clear id
             | intro; apply id; left; assumption ]
          | intro; apply id; right; assumption ]
     | id0:(?A -> ?B),id1:?A |- _ =>
         cut B; [ intro; clear id0 | apply id0; assumption ]
     | |- (_ /\ _) => split
     | |- (~ _) => red
     end).
\end{coq_example}
\end{minipage}}
\end{center}
\caption{Deciding intuitionistic propositions (1)}
\label{tautoltaca}
\end{figure}

\begin{figure}
\begin{center}
\fbox{\begin{minipage}{0.95\textwidth}
\begin{coq_example}
Ltac TautoProp :=
  DSimplif;
   Axioms ||
     match goal with
     | id:((?A -> ?B) -> ?C) |- _ =>
          cut (B -> C);
          [ intro; cut (A -> B);
             [ intro; cut C;
                [ intro; clear id | apply id; assumption ]
             | clear id ]
          | intro; apply id; intro; assumption ]; TautoProp
     | id:(~ ?A -> ?B) |- _ =>
         cut (False -> B);
          [ intro; cut (A -> False);
             [ intro; cut B;
                [ intro; clear id | apply id; assumption ]
             | clear id ]
          | intro; apply id; red; intro; assumption ]; TautoProp
     | |- (_ \/ _) => (left; TautoProp) || (right; TautoProp)
     end.
\end{coq_example}
\end{minipage}}
\end{center}
\caption{Deciding intuitionistic propositions (2)}
\label{tautoltacb}
\end{figure}

The pattern matching on goals allows a complete and so a powerful
backtracking when returning tactic values. An interesting application
is the problem of deciding intuitionistic propositional logic.
Considering the contraction-free sequent calculi {\tt LJT*} of
Roy~Dyckhoff (\cite{Dyc92}), it is quite natural to code such a tactic
using the tactic language. On Figure~\ref{tautoltaca}, the tactic {\tt
  Axioms} tries to conclude using usual axioms. The {\tt DSimplif}
tactic applies all the reversible rules of Dyckhoff's system.
Finally, on Figure~\ref{tautoltacb}, the {\tt TautoProp} tactic (the
main tactic to be called) simplifies with {\tt DSimplif}, tries to
conclude with {\tt Axioms} and tries several paths using the
backtracking rules (one of the four Dyckhoff's rules for the left
implication to get rid of the contraction and the right or).

\begin{figure}[tb]
\begin{center}
\fbox{\begin{minipage}{0.95\textwidth}
\begin{coq_example*}
Lemma tauto_ex1 : forall A B:Prop, A /\ B -> A \/ B.
Proof.
TautoProp.
Qed.

Lemma tauto_ex2 :
   forall A B:Prop, (~ ~ B -> B) -> (A -> B) -> ~ ~ A -> B.
Proof.
TautoProp.
Qed.
\end{coq_example*}
\end{minipage}}
\end{center}
\caption{Proofs of tautologies with {\tt TautoProp}}
\label{tautolem}
\end{figure}

For example, with {\tt TautoProp}, we can prove tautologies like those of
Figure~\ref{tautolem}.


\subsection{Deciding type isomorphisms}

A more tricky problem is to decide equalities between types and modulo
isomorphisms. Here, we choose to use the isomorphisms of the simply typed
$\lb{}$-calculus with Cartesian product and $unit$ type (see, for example,
\cite{RC95}). The axioms of this $\lb{}$-calculus are given by
Figure~\ref{isosax}.

\begin{figure}
\begin{center}
\fbox{\begin{minipage}{0.95\textwidth}
\begin{coq_eval}
Reset Initial.
\end{coq_eval}
\begin{coq_example*}
Open Scope type_scope.
Section Iso_axioms.
Variables A B C : Set.
Axiom Com : A * B = B * A.
Axiom Ass : A * (B * C) = A * B * C.
Axiom Cur : (A * B -> C) = (A -> B -> C).
Axiom Dis : (A -> B * C) = (A -> B) * (A -> C).
Axiom P_unit : A * unit = A.
Axiom AR_unit : (A -> unit) = unit.
Axiom AL_unit : (unit -> A) = A.
Lemma Cons : B = C -> A * B = A * C.
Proof.
intro Heq; rewrite Heq; reflexivity.
Qed.
End Iso_axioms.
\end{coq_example*}
\end{minipage}}
\end{center}
\caption{Type isomorphism axioms}
\label{isosax}
\end{figure}

The tactic to judge equalities modulo this axiomatization can be written as
shown on Figures~\ref{isosltac1} and~\ref{isosltac2}. The algorithm is quite
simple. Types are reduced using axioms that can be oriented (this done by {\tt
MainSimplif}). The normal forms are sequences of Cartesian
products without Cartesian product in the left component. These normal forms
are then compared modulo permutation of the components (this is done by {\tt
CompareStruct}). The main tactic to be called and realizing this algorithm is
{\tt IsoProve}.

\begin{figure}
\begin{center}
\fbox{\begin{minipage}{0.95\textwidth}
\begin{coq_example}
Ltac DSimplif trm :=
  match trm with
  | (?A * ?B * ?C) =>
      rewrite <- (Ass A B C); try MainSimplif
  | (?A * ?B -> ?C) =>
      rewrite (Cur A B C); try MainSimplif
  | (?A -> ?B * ?C) =>
      rewrite (Dis A B C); try MainSimplif
  | (?A * unit) =>
      rewrite (P_unit A); try MainSimplif
  | (unit * ?B) =>
      rewrite (Com unit B); try MainSimplif
  | (?A -> unit) =>
      rewrite (AR_unit A); try MainSimplif
  | (unit -> ?B) =>
      rewrite (AL_unit B); try MainSimplif
  | (?A * ?B) =>
      (DSimplif A; try MainSimplif) || (DSimplif B; try MainSimplif)
  | (?A -> ?B) =>
      (DSimplif A; try MainSimplif) || (DSimplif B; try MainSimplif)
  end
 with MainSimplif :=
  match goal with
  | |- (?A = ?B) => try DSimplif A; try DSimplif B
  end.
Ltac Length trm :=
  match trm with
  | (_ * ?B) => let succ := Length B in constr:(S succ)
  | _ => constr:1
  end.
Ltac assoc := repeat rewrite <- Ass.
\end{coq_example}
\end{minipage}}
\end{center}
\caption{Type isomorphism tactic (1)}
\label{isosltac1}
\end{figure}

\begin{figure}
\begin{center}
\fbox{\begin{minipage}{0.95\textwidth}
\begin{coq_example}
Ltac DoCompare n :=
  match goal with
  | [ |- (?A = ?A) ] => reflexivity
  | [ |- (?A * ?B = ?A * ?C) ] =>
    apply Cons; let newn := Length B in DoCompare newn
  | [ |- (?A * ?B = ?C) ] =>
    match eval compute in n with
    | 1 => fail
    | _ =>
      pattern (A * B) at 1; rewrite Com; assoc; DoCompare (pred n)
    end
  end.
Ltac CompareStruct :=
  match goal with
  | [ |- (?A = ?B) ] =>
      let l1 := Length A
      with l2 := Length B in
      match eval compute in (l1 = l2) with
      | (?n = ?n) => DoCompare n
      end
  end.
Ltac IsoProve := MainSimplif; CompareStruct.
\end{coq_example}
\end{minipage}}
\end{center}
\caption{Type isomorphism tactic (2)}
\label{isosltac2}
\end{figure}

Figure~\ref{isoslem} gives examples of what can be solved by {\tt IsoProve}.

\begin{figure}
\begin{center}
\fbox{\begin{minipage}{0.95\textwidth}
\begin{coq_example*}
Lemma isos_ex1 :
  forall A B:Set, A * unit * B = B * (unit * A).
Proof.
intros; IsoProve.
Qed.

Lemma isos_ex2 :
  forall A B C:Set,
    (A * unit -> B * (C * unit)) =
    (A * unit -> (C -> unit) * C) * (unit -> A -> B).
Proof.
intros; IsoProve.
Qed.
\end{coq_example*}
\end{minipage}}
\end{center}
\caption{Type equalities solved by {\tt IsoProve}}
\label{isoslem}
\end{figure}

%%% Local Variables:
%%% mode: latex
%%% TeX-master: "Reference-Manual"
%%% End:
